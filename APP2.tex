
\chapter{课外实验活动}
\section{自制指南针}
如图10.10所示,用硬纸板、大头针、按扣、缝衣针自制一
个指南针。

用磁铁的一端在缝衣针上
朝一个方向擦几下,缝衣针就
有了磁性。为了使缝衣针能顺
利地穿过按扣(取按扣中较薄
的一扇)的两个小孔,可用钳子
把按扣的边缘向下夹一下,当
自制的指南针静下来后,记住针的哪一端指北。

\begin{figure}[htp]\centering
    \includegraphics[scale=.6]{fig/10-10.png}
    \caption{}
    \end{figure}


\section{验证环形电流的磁场}
这个实验是用自制的指南针来验证环形电流的磁场方向
(图10.11),在一个瓶子(或硬纸筒)上用漆包线绕一个10至15
匝的线圈,把绕好的线圈从瓶子上取下来,再用胶布把线圆竖
直固定在一块六板上,将你自制的指南针放在图10.11所示
的位置,转动木板使磁针处在线圈平面内,用学过的环形电
流磁场的知识判断一下,如果线圈的两端接上电池,指南针将
怎样偏转,然后再给线圈通电,看一看实验结果跟你的判断
是否一致。
\begin{figure}[htp]\centering
    \includegraphics[scale=.6]{fig/10-11.png}
    \caption{}
    \end{figure}

    \section{验证通电螺线管的南北极}
把漆包线绕在一支铅笔上,然后抽出铅笔,做成一个螺线
管。用学过的通电螺线管磁场的知识判断一下,如果给螺线
管通电,通电螺线管哪端是南极,哪端是北极。然后把自制的
指南针放在螺线管的两端,给螺线管通电,看看实验结果跟你
的判断是否一致。

\section{观察磁化现象}
取一个条形磁铁和一个大铁钉,把铁钉插入铁屑,并把
条形磁铁的一个磁极靠近钉子头,然后同时提起磁铁和铁钉,
你将看到一些铁屑粘到钉子上。将磁铁移去,铁钉上的大部
分铁屑将掉下来,但仍有一部分铁屑粘在钉子上。再用磁铁
的另一个磁极靠近钉子头,剩在钉子上的铁屑就会掉下来。

解释上述现象。

\section{判断指南针的偏转方向}
在一个铅笔刀或一个大些的铁钉上,用漆包线绕上两个
线圈$A$和$B$,将线圈$B$的两端接在一起,并把$CD$段漆包线放
在静止的自制指南针的上方(图10.12)。试判断当用于电池
给线圈$A$通电的一瞬间,指南针偏转的方向。做这个实验,看
一看你判断的指南针偏转方向与实验是否一致。
\begin{figure}[htp]\centering
    \includegraphics[scale=.6]{fig/10-12.png}
    \caption{}
    \end{figure}

\section{自制测电笔}
准备一个小氖灯,一个小弹簧,再找一个装中药片的小
玻璃瓶,两个瓶盖,两个铁钉,一个0.25瓦、2—5兆欧的电阻。
在稍粗糙的水泥砖上把玻璃瓶底磨掉,做成一个玻璃圆筒,让
铁钉穿过瓶盖,盖上瓶盖后使钉帽在瓶里,把电阻的两根引
线齐根去掉,并把电阻两端的绝缘漆去掉。照图10.13那样
把上述器材安装起来,就做成了一个测电笔。
\begin{figure}[htp]\centering
    \includegraphics[scale=.6]{fig/10-13.png}
    \caption{自制测电笔}
    \end{figure}

用这个自制的测电笔可以辨别照明电路的火线和地线。
用拇指和食指拿住玻璃瓶,前面的钉子接触待辨别的导线,后
面的钉子接触手。当前面的钉子接触的是火线时,小氖灯发
光;接触的是地线时,小氖灯不发光。这样就可以辨别出火线
和地线。

要注意:\textit{手的任何部位都不要接触前面的钉子},因为它
接触可能是火线,会使人触电。

\section{测定水的折射率}
找一个广口瓶,在瓶内盛
满水,照图10.14那样把直尺$AB$
紧挨着瓶口的$C$点竖直插入瓶
内,从尺的对面一点$P$观察水
面,可以同时看到直尺在水中
的部分和露出水面的部分在水
中的像。读出你看到的直尺水
下部分最低点的刻度$S_1$,以及
跟这个刻度相重合的、水上部
分刻度$S_2$的像$S'_2$。记下$CS_1$和
$CS_2$的长度,量出广口瓶瓶口的内径$d$,就能算出水的折射率。
你用这种方法求出的水的折射率为多少?
\begin{figure}[htp]\centering
    \includegraphics[scale=.6]{fig/10-14.png}
    \caption{}
    \end{figure}

如果你能同时读出直尺在水下的两个刻度$S_1$和$S_3$,以
及跟它们相重合的、两个水上刻度$S_2$和$S_4$在水中的像$S'_2$和
$S'_4$,就可以不必测量瓶口的内径,直接用从直尺上读出的两
组数据求出水的折射率来。比较这两种方法测量的结果,看
哪种方法测得的折射率更准确?

用后一种方法进行测量,瓶中的水不一定非盛满不可,竖
直插入水中的直尺也不一定要紧挨瓶口,做起来更简便。

\section{测定凹透镜的焦距}
凹透镜所成的虚像不能在像屏上显示出来,因此它的焦
距不可能象凸透镜那样直接利用焦点或成像方法来测量,
面介绍一种测量凹透镜焦距的简便方法。
\begin{figure}[htp]\centering
    \includegraphics[scale=.6]{fig/10-15.png}
    \caption{}
    \end{figure}


在凹透镜的中心贴一个半径为$R$的黑色圆纸片$A$,另取
一张白纸$B$,在$B$上画一个半径为$2R$的圆。把白纸和凹透镜
平行地放在太阳光下(图10.15),让透镜对着太阳,调节透镜
和白纸间的距离,使黑色圆纸片的影恰好跟白纸上的圆圈重
合。这时透镜和白纸间的距离就等于凹透镜的焦距。想想看,
为什么?做这个实验,并将测得的焦距跟已知的焦距相比较,
看相差多少。















































































\chapter{常用电磁学量的国际制单位}

电磁学的单位制是一个比较复杂的问题,长期以来存在
着多种单位制,本书采用的是国际单位制。在国际单位制中,
所有的电磁学量,都是由长度、质量、时间、电流强度这四个基
本量导出的。因此,米、千克、秒、安培是电磁学里的基本单位。
下表列出了常用的电磁学量的国际制单位。

\begin{center}
    \begin{tabular}{cc|cc|c}
  \hline
\multicolumn{2}{c|}{物理量} & \multicolumn{2}{c|}{单位} & 量纲式\\
名称 & 符号 & 名称 & 国际符号 \\
  \hline
电流   &  $I$ & 安培 & A & $[I]$\\
电量    &  $Q$  &  库仑  & C   & $[TI]$   \\
电场强度    &$E$    &  伏特每米  & V/m   &  $[LMT^{-3}I^{-1}]$  \\
电势、电势差、电压& $U$ ($V$)   &  伏特  &  V  &  $[L^2MT^{-3}I^{-1}]$  \\
电容    &  $C$  &  法拉  & F   &  $[L^{-2}M^{-1}T^{4}I^{2}]$   \\
电阻    & $R$  & 欧姆  &  $\Omega$ & $[L^2MT^{-3}I^{-2}]$  \\
电阻率    & $\rho$  & 欧姆米  &  $\Omega\cdot {\rm m}$ & $[L^3MT^{-3}I^{-2}]$  \\
磁感应强度    & $B$  & 特斯拉  & T  &  $[MT^{-2}I^{-1}]$ \\
磁通量    & $\phi$  & 韦伯  & Wb  & $[L^2MT^{-2}I^{-1}]$  \\
电感    & $L$  & 亨利  & H  &  $[L^2MT^{-2}I^{-2}]$ \\
  \hline      
    \end{tabular}
\end{center}

\chapter{常用的物理恒量}
\begin{center}
    \begin{tabular}{ll}
        万有引力恒量&    $G=6.67\x10^{-11}\; {\rm N}\cdot {\rm m^2}/{\rm kg}^2$\\
        摩尔气体恒量&    $R=8.31\;  {\rm J}/({\rm mol}\cdot {\rm K})$\\
        阿伏伽德罗常数&    $N=6.02\x 10^{23}\; {\rm mol}^{-1}$\\
        静电力恒量&    $k=9.0\x10^9\;  {\rm N}\cdot {\rm m^2}/{\rm C}$\\
        法拉第恒量&    $F=9.65\x10^4\;  {\rm C}/{\rm mol}$\\
        基本电荷&    $e=1.60\x10^{-19}\;  {\rm C}$\\
        电子的质量&    $m_e=0.91\x10^{-30}\; {\rm kg}$\\
        质子的质量&    $m_p=1.67\x10^{-27}\;   {\rm kg}$\\
        中子的质量&    $m_n=1.67\x10^{-27} \;   {\rm kg}$\\
        $\alpha$粒子的质量&    $m_{\alpha}=6.64\x10^{-27} \;  {\rm kg}$\\
        原子质量单位&    $1{\rm u}=1.66\x10^{-27} \;   {\rm kg}$\\
        真空中光速&    $c=3.00\x10^8\; \ms$\\
        电子的荷质比&    $e/m=1.76\x10^{11}\; {\rm C}/{\rm kg}$ \\
        氢原子的半径&    $a_0=0.53\x10^{-10}\; {\rm m}$   \\
        普朗克恒量&    $h=6.63\x10^{-34}\;  {\rm J}\cdot {\rm s}$   \\
        里德伯恒量&    $R=1.097\x10^7\;  {\rm m}^{-1}$\\        
    \end{tabular}
\end{center}

\chapter{用于构成十进倍数和分数单位的词头}

\begin{center}
    \begin{tabular}{cccc}
        \hline
所表示的因数&词头名称(中文)&词头名称(英文)&词头符号\\
\hline
$10^{18}$ &艾[可萨]&exa-&E\\
$10^{15}$ &拍[它]&peta-&P\\
$10^{12}$ &太[拉]&tera-&T\\
$10^9$ &吉[咖] &giga-&G\\
$10^6$ &兆 &mega-&M\\
$10^3$ & 千 &kilo-& k\\
$10^2$ & 百 &hecto-&h\\
$10^1$ & 十 &deca-&da\\
$10^{-1}$ & 分 &deci-&d\\
$10^{-2}$ &厘 &centi-&c\\
$10^{-3}$ &毫 &milli-&m\\
$10^{-6}$&微&micro-&$\mu$\\
$10^{-9}$&纳[诺]  &nano-&n\\
$10^{-12}$ &皮[可]&pico-&p\\
$10^{-15}$&飞[母托]&femto-&f\\
$10^{-18}$&阿[托]&atto-&a\\
\hline
    \end{tabular}
\end{center}















































