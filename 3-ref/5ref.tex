\chapter{光的反射和折射}
\minitoc[n]
\section{教学要求}
    本章讲述几何光学知识,包括两部分内容:一部分是几何
光学的基本规律,讲述光的反射定律和折射定律;另一部分是
运用基本规律来研究平面镜、球面镜、棱镜和透镜等基本光学
元件,这些元件在成像和控制光路方面各有其自身的规律和
特点.学习这些知识,可以便我们认识自然界和生活中的许
多光现象,理解基本光学元件的作用和光学仪器的原理.因
此,这一章知识具有实际意义.

全章教材分为五个单元.一、二节为第一单元,讲述光的
直线传播和光速.三、四节为第二单元,讲述光的反射和平面
镜、球面镜.五—八节为第三单元,讲述光的折射、全反射和
棱镜.九—十二节为第四单元,讲述透镜知识.十三、十四节
为第五单元,讲述眼睛、显微镜和望远镜的构造和成像原理.

本章教材相当大一部分内容是在复习初中知识的基础
上,进一步扩大、加深和提高的.光的直线传播、光的反射等
知识,是复习性的教材。光的折射则在初中学过的知识基础
上予以提高,引入了折射定律和折射率的概念,是进一步学习
棱镜、色散、透镜的知识基础。透镜是许多光学仪器的主要元
件,掌握透镜成像的规律,可以理解一些助视仪器的成像原
理.所以,光的折射和透镜成像是本章教学的重点.

这一章的教学要求是:
\begin{enumerate}
\item 了解光在同一媒质中沿直线传播的性质,会用光的
直线传播解释有关的现象。知道光在真空中的速度。
\item 掌握光的反射定律,会用反射定律解释有关的现象.
掌握平面镜成像的原理和规律。了解凹镜对光的会聚作用和
凸镜对光的发散作用,了解它们的成像规律,
\item 掌握光的折射定律和折射率的概念,了解媒质的折射
率与光速的关系;会用折射定律解释简单的现象。
\item 理解光的全反射现象,掌握临界角的概念和发生全反
射的条件,了解全反射现象的应用。
\item 了解棱镜对光的偏折作用和光通过棱镜后的色散
现象。
\item 了解凸透镜对光的会聚作用和凹透镜对光的发散作
用。了解透镜的主轴、光心、焦点和焦距,掌握透镜成像规律、
成像作图法和成像公式。
\item 了解眼睛、显微镜和望远镜的基本构造和成像原理,
了解近视眼和远视眼以及眼镜的作用。
\end{enumerate}

虽然学生每天都接触光现象,但是却不能直接看到光的
传播路径,讲解光的反射、折射和全反射,要做好演示,让学
生看到光的传播路径,以利于揭示规律,加深印象,同时也可
以提高学生的学习兴趣。

在几何光学的教学中,研究成像是一个重要的问题,要让
学生了解成像规律和所成像的虚实。但是要注意,不要机械
地记忆什么镜成什么像,而要着重让学生从根本上弄清实像
和虚像的含义,掌握分析成像的规律和成像性质的方法。

在反射和折射现象中光路的可逆性对分析实际问题很有
用。教学中要通过具体例子让学生领会到这一点。

媒质的折射率是个重要的概念,要让学生掌握绝对折射
率的概念,为了使学生了解媒质的折射率总是相对于另一种
媒质而言的,教材介绍了相对折射率,教学中不要求把相对
折射率与绝对折射率的关系作为讲述重点,也不要求利用相
对折射率进行计算。

学生对于全反射现象是比较生疏的。教学中应通过演示
让学生认识这个现象,理解产生全反射现象的条件。在演示
时,还应引导学生注意观察光强分布的变化情况,这对于正确
理解全反射现象是有帮助的。光导纤维是全反射现象的应
用,教学中可根据实际情况,向学生介绍这一新技术的发展情
况,以开阔学生的眼界。

透镜是许多光学仪器的重要元件。在透镜成像的教学
中,要求学生会用三条特殊光线求出像的位置、大小和倒正,
判断像的虚实。对于透镜成像公式,应使学生能独立地把它
推导出来,知道什么情况下$f$取正值,什么情况下$f$取负值,
并能根据$v$的正负,判断出像的虚实、倒正和位置。

最后一个单元介绍眼睛、显微镜和望远镜。讲眼睛时,主
要从物理方面讲解眼睛的成像原理,眼睛的调节以及近视眼、
远视眼和眼镜的作用,不要过多地牵涉与生理机制有关的内
容,对于显微镜和望远镜,主要学习它们的基本构造和成像
原理。要求学生能看懂光路图,了解视角放大的情况。如果
有条件,最好在教学中能结合望远镜和显微镜的实物来讲解,
并介绍一些使用和维护方面的知识。

\section{教学建议}
\subsection{光的直线传播和光的速度}
光的直线传播是几何光学的基础,应通过演示实验让
学生看到光传播的路径。演示内容应包括:
\begin{enumerate}
\item 光在同一种均
匀媒质中沿直线传播;    \item 光到达两种媒质的界面,发生反射和
折射;    \item 条件较好的学校还可演示光在非均匀媒质中传播时
光线发生弯曲的现象(为阅读教材“海市蜃楼”做准备)。
\end{enumerate}


光线是几何光学的最基本的概念之一,它为应用几何
学的方法研究光的传播奠定了基础,光的反射定律和折射定
律等都是应用光线的概念来表述的,要使学生理解:
\begin{enumerate}
\item 光线
的概念建立在光的直线传播的基础上,所以,作图时画光线必
须要有表示光传播方向的头。    
\item 光线和光是有区别的,光
线只代表了光的传播方向,并不是实际存在的东西,而光是实
际存在的。    
\item 有时用光线代表很窄的平行光束的传播路径,
从这个意义上讲,光线是光束的抽象;有时又用两条光线代表
一束光束,这两条光线表示光束的边缘,也表示出光束的会
聚、发散或平行的性质。
\end{enumerate}

要注意应用几何学的方法研究与光的传播有关的实
际问题。在对影、本影和半影等问题进行分析时可参考下面
五点逐步深入:
\begin{enumerate}

\item 点光源发出的光照到不透明物体上,在物
体背后的光屏上可以看到光线照不到的黑暗区域——影子。
作出光路图,说明影子的范围;
\item 如果物体大小是已知的,应
用相似形的关系可以说明影子的大小决定于点光源、物体和光屏的相对位置;\item 利用发光面较大的面光源代替点光源,
说明本影和半影的概念;
\item 面光源的线度大于物体时,移动
光屏的位置,讨论屏上影的情况,说明本影、半影等概念的关
系.
\item 观察者处于图5.1中$A$、$B$、$C$、$D$等不同位置时看面
光源$PQ$, 见到的现象会不同。通过以上分析,可以为学生认
识日蚀或月蚀等现象、解答课本练习一第2题打下基
础,教学时应着重引导学生体会几何光学的研究方法:从光
传播的基本定律出发,做出光路图,运用几何学的方法研究
有关线段、角度等量的关系。
\end{enumerate}

\begin{figure}[htp]
    \centering
    \includegraphics[scale=.6]{fig/5-1.png}
    \caption{}
\end{figure}

光的速度是物理学中的重要基本常数,为了测量光
速,从十七世纪伽利略开始一直到今天,经过了许多科学家
的努力,测量结果越来越精确。讲述光的速度要结合物理学
史介绍科学家百折不挠、精益求精的探索精神,要重点讲述光
的传播速度虽然很大,但是有限的,分析测量光速实验的设计
思想。伽利略根据$v=s/t$
的原理进行了历史上第一次测光速
的实验,尽管没能测出光的速度来,但是他的实验设计思想却
给人们指出了努力方向,由于光速是很大的,早期测量光速
的方法或者是增大光传播的距离,或者是利用更精巧的办法
准确地测量光传播的时间。

罗默观测木星的卫星蚀,第一次证实了光以有限的速度
传播,用的就是前一种方法。学生理解罗默方法的原理是有
困难的,教材并不要求掌握具体细节,但应使学生理解:虽然
卫星绕木星运动的周期是固定的,但在地球上观测木星的两
次卫星蚀的时间间隔却是不同的,这是由于地球位于公转轨
道上的不同位置时,与木星的卫星间的距离不同。而光以有
限速度传播,从卫星发出的光到达地球时经过的距离不等,因
此,观测到的卫星蚀的时间间隔也就不等。

迈克耳逊用旋转棱镜法测光速,是采用精巧的办法准确
地测量出光传播的时间。他是早期测量光速最准确的科学家。
讲述旋转棱镜法测光速的实验设计思想时,要着重理解光由
光源$S$发出最后到达望远镜$C$中(课本图5.5)所经过
的路径和旋转棱镜能精巧地测量时间的原理,调节八面棱镜
的转速,使光由凹镜反射返回来时八面镜恰好转过$1/8$转,在
望远镜$C$中能看到$S$的像(还可进一步引导学生讨论转速增
加到转过$\frac{2}{8},\frac{3}{8},\ldots$
转的情况)。测出转速便可求得光在两
个山峰之间往返一次所用的时间。

近代科学家还提出用其他的实验原理来测量光速。例如
测量激光的频率$\nu$,和波长$\lambda$, 根据公式$c=\lambda\cdot \nu$计算光速,大
大提高了测量的精度。应该要求学生记住真空中光速的近似
值$c=3.00\x10^8\ms$.

\subsection{光的反射}
\subsubsection{反射定律}

反射定律、镜面反射和漫反射在初中讲得
都比较清楚,教材中这部分内容是按照复习旧知识安排的。可
以通过演示光的反射光路、镜面反射和漫反射,再现这些现
象,再进一步采用讨论的方法加深对反射定律的认识。选择
讨论题目要针对学生的具体情况,下列问题可供参考:
\begin{enumerate}
\item 在什么条件下发生光的反射现象?
\item 反射定律的内容为什么一定要包括课本中讲的两条?
    \item 怎样确定反射界面的法线?如
果界面是球面,它的法线是什么?
    \item 发生镜面反射时,反射光
线和入射光线之间的夹角取值范围可能多大?
    \item 漫反射和镜面反射有什么区别和联系?
\end{enumerate}



\subsubsection{光路的可逆性}

光路的可逆性是几何光学的重要原
理之一。要结合反射定律(包括以后的折射定律)理解反射时
的光路(包括折射时的光路)是可逆的。几何光学中的光路(面
镜、棱镜和透镜的光路和成像)都建立在反射定律和折射定
律的基础之上,并且几何光学中的所有光路都是可逆的,应
该注意,物理光学中的光路是不可逆的。

做为一种方法,光路的可逆性可以用来解决一些具体问
题。例如,人眼的视野问题。可以设想在人眼所在的位置处
放一个点光源,求出由该点光源发出的光所能照亮的区域,根
据光路可逆性,在这个区域内的物体发出的光都能到达人眼,
被人眼看到。因此,这个区域是人眼的视野范围。

\subsubsection{平面镜}

平面镜的内容也是在初中知识的基础上讲
述的。讲述的重点放在虚像的概念、虚像的形成以及平面镜
成像的特点上。

虚像的概念及其形成分三个层次讲解:
\begin{enumerate}
\item 在光的直
线传播一节中讲述了根据光的直线传播确定发光点的位置,
这为讲解虚像做了准备、由发光点发出的光向各个方向传播,
可以将其中的任意两条光线反向延长,它们的交点就是发光
点的位置。应该注意,这两条光线必须是同一发光点发出的,
由不同发光点发出的两条光线的交点,在成像问题上没有意
义;
\item 由某一物点$S$发出的、经平面镜反射后的光线进入了眼
睛,人们根据光的直线传播的经验认为这些光线都是由它们
的反向延长线的交点$S'$射来的,好象在$S'$点有一个发光点
一样,$S'$点就是物点$S$的虚像。这里说明了虚像不是光线的
实际交点,是“虚”的,需要借助其他光学仪器(如眼睛)才能观
察到;
\item 组成物体的每一个物点都在平面镜里产生一个虚像
点,这些虚像点的集合就是物体的虚像。教学中要注意培养
学生抽象思维的能力,不仅要从实验现象中看到物体通过平
面镜可以成像,还要学会从理论上分析推理得出像的概念,
再根据反射定律进一步推出物体通过平面镜所成的像是等
大、正立的虚像,物和像对于镜面是对称的。
\end{enumerate}

平面镜的应用包括应用平面镜成像,应用平面镜控
制光路等,在课文中提到了利用平面镜将微小效应放大、潜望
镜的光路图等实例,在习题中安排了平面镜成像,利用激光测
量月球和地球间的距离以及人眼通过平面镜观察的视野范围
等问题。在分析实际问题时要画好光路图,使学生了解几何光
学的研究方法,培养他们灵活应用物理知识解决实际问题的
能力。

\subsubsection{球面镜}

通过演示实验复习凹镜的会聚作用和凸镜
的发散作用,注意从实验观察到的光路出发讲述球面镜的顶
点、主轴、焦点(实焦点和虚焦点)和焦距等概念。进一步理解
课本193页讲述的凹镜和凸镜的实际应用的例子.

近轴光线是球面镜和透镜能成理想像的前提条件,教材
中给出了在近轴条件下,球面镜的焦距$f=r/2$
是为了使学
生对球面镜焦距的大小有一个具体的了解,并不要求深入展
开讨论和证明这个公式。根据反射定律,可以分别画出平行
于主轴的近轴光线和远轴光线经过凹镜反射后与主轴的交
点,如图5.2中的$F$和$E$点,说明球面镜的非近轴光线不能会
聚于一点。实验给出的几条光线能够在焦点会聚仅是近似
的;近似的条件就是入射光线都是近轴光线。为了使远轴的
平行光线也能会聚于焦点,实际应用的大口径凹镜都用抛物
面镜而不用球面镜。
\begin{figure}[htp]
    \centering
    \includegraphics[scale=.6]{fig/5-2.png}
    \caption{}
\end{figure}

球面镜成像是从实验得出的,教材不要求画成像的光路
图,也不要求讲述球面镜成像公式,只是从实验得出物体通过
凹镜可以成放大实像、缩小实像和放大虚像,通过凸镜只能成
缩小虚像的情况 演示实验中要让学生注意观察用光屏能够
接收到实像,而无论把光屏放在什么位置都不能接收到虚像。
这表明实像是反射光线实际会聚而成的;虚像不是反射光线
的实际会聚点,而是光线的反向延长线的交点,这是实像和虚
像的主要区别。


\subsection{光的折射}

折射定律和折射率是本章的重点内容,可以按下述
几个层次安排教学。
\begin{enumerate}
\item 通过演示实验观察光在两种媒质的界面上的折射现
象,观察折射角随入射角的增大而增大的规律,激发学生深入
研究折射角和入射角的关系的学习动机。
\item 分析课本197页表中的实验数据,得出折射定律.
\item 折射定律中的常数决定于两种媒质的性质,引入相
对折射率$n_{21}$·
\item 引入绝对折射率,进而求出相对折射率和绝对折射
率的关系。
\end{enumerate}

\subsubsection{折射定律}

课本结合物理学史讲述了历史上经历了
一千多年,由托勒密到斯涅耳经过许多科学家的努力,从大量
的实验数据中归纳总结出折射定律,教学过程也要从演示折
射现象开始,演示的内容有:
\begin{enumerate}
\item 光达到两种媒质的界面上发生
折射的现象;
\item 折射角随入射角增大而增大的规律;
\item 光由光
疏媒质进入光密媒质向靠近法线方向偏折,光由光密媒质进
入光疏媒质向远离法线方向偏折。
\end{enumerate}
通过演示,使学生获得光
的折射现象的感性认识。在定性实验的基础上,再分析课本
197页表中所给出的实验数据,说明托勒密最初得出的折射
角与入射角成正比的结论是片面的,给出斯涅耳经过深入的
分析得出的折射定律内容。(折射定律现代表述的形式是由
笛卡儿提出的。)

和反射定律一样,掌握折射定律也要注意培养学生的想
象力,要使学生建立起光由媒质I进入媒质II的折射图景。
使学生在理解界面及其法线、入射光线和折射光线,入射角和
折射角等概念的基础上,进一步掌握折射定律。折射定律的
第一点指出了入射光线、折射光线、法线在平面内的大致方
位,第二点定量地确定折射角,确定折射光线的确切方向。

\subsubsection{折射率}

媒质的折射率是由折射定律引入的.教材
从光在两种媒质界面上发生折射的现象人手,引人相对折射
率,并且直接给出了相对折射率和光在两种媒质中传播速度
的关系,然后从光从真空进入媒质发生折射的特殊情况,引
入绝对折射率的概念,并通过折射率和光速的关系,求出相对
折射率和绝对折射率的关系。这部分内容,特别是绝对折射
率和相对折射率的关系,涉及的物理量比较多,初学者较难
掌握,容易造成混乱。教学中应强调$n_{21}$下标的表述及其意
义.讲解公式$n_{21}=\dfrac{n_2}{n_1}=\frac{v_1}{v_2}$时,要和具体的光疏、光密媒质以
及光线的偏折情况相联系。

教学中要注意培养学生的想象力,建立光由媒质I射入
媒质II发生折射的图景,理解其中的平面、直线、角度之间的
关系,进一步寻求入射角和折射角、媒质的折射率及光速之
间的关系,例如,已知水中的光速是真空中光速的$3/4$,光由
空气进入水中,以多大的入射角射入时,反射光线和折射光线
垂直,学生在头脑中建立了光在空气和水的界面上发生反射
和折射的图景之后,就能找到入射角、反射角和折射角间的关
系,求出入射角来。

课本201页讨论的光通过两面平行的玻璃砖发生侧移的
例题,既应用了折射定律分析解决具体问题,又为“测定玻璃
折射率”的学生实验做准备。可以先演示光通过玻璃砖发生平
行侧移的现象,再提出如下问题:①光线穿过玻璃砖后,为什
么只发生平移而不改变传播方向?②光线平行侧移的距离、方
向决定于什么条件?让学生自己通过讨论去研究解释这个现
象,寻求例题的解。

\subsubsection{全反射}
 由于学生先接受了光到达两种媒质的界面
上既要发生反射,又要发生折射的现象,要进一步认识在一定
条件下,光到达界面上只发生反射不发生折射的全反射现象
比较困难,学生往往对于发生全反射的条件理解不深刻。教
学中,要使学生在实验的基础上,认识全反射现象,了解发生
全反射的条件。

做演示实验时,要使学生看到光从光密媒质射入光疏媒
质时,折射角大于入射角。随着入射角的增大,折射角也逐渐
增大.当入射角达到某一角度时,折射角达到$90^{\circ}$. 再增大
入射角,就只有反射光线,没有折射光线。通过现象的观察,
使学生了解全反射现象是光由光密媒质射入光疏媒质时,入
射角达到某一特定值以后产生的现象。这有助于使学生弄清
楚“临界角”的含义。

还要使学生理解全反射现象并不违背折射定律。当光从
光密媒质进入光疏媒质时,有$\sin r=\dfrac{\sin i}{n_{21}}$, 而光疏媒质对
光密媒质的相对折射率$n_{21}=\dfrac{v_1}{v_2}<1$, 所以折射角$r$大于入射
角$i$. 当折射角$r=90^{\circ}$时,即使再增大入射角,也不会存
在大于$90^{\circ}$的折射角$r$, 这时就发生了全反射现象。

教学时,可结合实验观察使学生理解临界角的概念,掌
握计算临界角的方法。

为使学生获得全面的认识,在讨论全反射现象时,也可以
分析光由光疏媒质进入光密媒质的情况。通过比较,加深学
生对产生全反射条件的认识。

在演示时,还要注意引导学生观察入射光的能量在反射
光和折射光中的分配随入射角的大小而变化。使学生认识到
发生全反射时,入射光线的能量几乎都被反射回原来的媒质
中,没有光能量进入另一种媒质,因而反射光最强。由此可以
认识生活中的全反射现象,如课本205页讲述的实例,以及使
用光导纤维和全反射棱镜的意义。

\subsubsection{棱镜}
三棱镜对光的偏折作用和色散现象是折射定律的具体应
用,设计如下的教学过程可供参考。

演示单色光通过三棱镜发生偏折的光路,分别演示
棱镜的底面在下方和底面在上方的两种情况,让学生观察现
象,归纳出光过三棱镜总是向底面偏折的结论。

根据折射定律讨论光在三棱镜两个侧面的折射,定
性地说明光通过光密三棱镜向底面偏折的光路。也要提醒学
生注意,如果是光疏三棱镜,光的偏折情况将会怎样?让学生
在做课本练习六第1题时回答.

定性说明偏折角度$\theta$与棱镜的折射率有关。在保持
入射角$i$和棱镜顶角$A$不变的条件下,棱镜材料的折射率越
大,偏折角度也越大。这是由于棱镜的折射率大,在$AB$面
上光的偏折角度也大,也就是$AC$面上的入射角变大。在$AC$
面,由于入射角和棱镜材料的折射率都变大,偏折角度也要变
大。因此总的偏折角度随折射率增大而增大。

演示白光过三棱镜的色散现象,可观察到由色散
产生的彩色光带(连续光谱)。要注意引导学生观察色散时光线
的偏折方向以及光谱中不同颜色光的偏折角度。演示时还应
在三棱镜的入射侧面前加红、黄、蓝等不同颜色的滤色片,显示
单色光通过棱镜的偏折情况。可以看到红光偏折最小,蓝光
偏折最大。由此揭示白光包含有各种颜色的色光,不同色光通
过棱镜的偏折角度不同,进而可以解释产生色散现象的原因。

组成白光的各种单色光,对三棱镜的入射角都相同,
但出射时各色光偏折角度不同,这是由于棱镜对于不同色光
的折射率不同造成的,红光的偏折最小,表明棱镜材料对于红
光的折射率最小;蓝光的偏折大,表明棱镜对于蓝光的折射率
较大,要让学生了解媒质对于不同颜色的光的折射率不同,说
明不同色光在该媒质中的传播速度不一样,使学生进一步加
深对折射率的认识。

可结合练习五中的题目讲一讲全反射棱镜.常用的
全反射棱镜是等腰直角三棱镜。在光学仪器中常用它来控制
光路,改变光线的传播方向或使像倒转等。例如,潜望镜中用
全反射棱镜代替平面反光镜,改变光线的传播方向;望远镜中
常用一对全反射棱镜使像的上下和左右都反转,观察者能够
直接看到正立的像。

常用一对全反射棱镜使像的上下和左右都反转,观察者能够
直接看到正立的像。

\subsection{透镜}
这一单元讲述透镜的成像规律,成像作图法和成像公式。
这些内容是本章中重要的基础知识,也是教学的重点。

实验是讲解透镜知识的基础.凸透镜对于光线的会
聚作用,凹透镜对于光线的发散作用,物体通过透镜可以成像
(包括实像和虚像),都是通过实验得出结论的。通过实验建
立起物和像的一一对应关系,为进一步用作图法和公式法研
究物像的关系打好基础,要做好下述演示实验:

平行于主轴的光线通过凸透镜后都会聚于主轴上的
一点;平行于主轴的光线通过凹透镜后变得发散,借助于直尺
看出出射光线的延长线交于主轴上的一点。通过实验的光路
讲述主轴、光心、焦点(包括实焦点和虚焦点)和焦距等概念。

从主轴、光心、焦点的概念分别引出凸透镜和凹透镜
的三条主要光线,再用实验复现这三条主要光线,强化学生的
认识。

通过演示实验复习物体通过凸透镜成像的五种情况
($u=f$时不成像)和凹透镜成像的一种情况。演示过程要注
意:
\begin{enumerate}
\item 利用光屏接收实像,确定实像的位置要注意实验操作的
示范,让学生看到光屏在成像位置附近移动时,像由模糊到清
晰再到模糊的过程。
\item 改变物距的大小应由物距较大逐渐变
为较小,使学生对于凸透镜和凹透镜成像的几种情况有比较
完整的认识,体会研究方法,切忌只孤立地演示成像的几种情
况,而不体现物距(或像距)连续变化的过程。
\end{enumerate}

\subsubsection{成像作图}

透镜成像作图建立在物体通过透镜可以
成像的实验基础之上,物体上的每一个发光点都对应着一个
像点,即由物点发出的光线经过透镜折射后,所有出射光线,包
括三条主要光线,都汇交于它对应的像点.课本安排练
习九第1题的目的就是为了加深这个认识,为了确定像点的
位置,可以在这些光线中任意选取两条出射光线求得交点,选
用三条主要光线中的两条比较方便。同样,确定物点的位置,
也可求出与像点相应的任意两条入射光线的交点。这样,可以
使学生认识物点和像点的意义,认识成像作图的基本出发点,
正确理解作图方法,避免得出只有三条主要光线才能成像的
错误印象。

教学中要注意示范,使学生养成作图规范的良好习惯,作
图规范包括:每条光线在透镜折射前后都应标出表示光线行
进方向的箭头;正确绘出透镜的符号,标注光心、焦点以及物
和像的位置和正倒;分清图中的实线和虚线。

教学中还要注意:
\begin{enumerate}
\item 采用对比的方法区别凸透镜和凹透镜三条主要光线
的不同点。学生对凹透镜的三条主要光线往往掌握不好,其
主要原因是对于虚焦点的概念和位置理解不深刻,混淆了凹
透镜和凸透镜对光线的作用。
\item 强调物点是透镜的入射光线的出射点,像点则是上
述人射光线经透镜折射后的出射光线或其延长线的交点,不
是随意两条光线的交点,学生往往不注意这点而造成错误。
\item 利用作图法确定物像关系,要和物体通过透镜成像
的情况相联系,许多实际问题往往是在分析成像情况做出判
断后才能正确做图。
\end{enumerate}

































