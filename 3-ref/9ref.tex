\chapter{原子核}
\minitoc[n]

\section{教学要求}

本章讲述原子核的初步知识,主要包括原子核的组成、核
能及其应用两方面的内容。这些知识对于学生进一步认识微
观世界,了解研究微观世界的方法,都有重要意义。

这章内容分为三个单元,第一单元讲述原子核组成方面
的知识,包括第一节到第四节;第二单元讲述关于核能方面的
知识,包括第五节到第七节。最后一节,作为一个单元,讲述
基本粒子,是选学内容。

原子核的人工转变和原子核的组成是本章教学的重点。
在核能正逐渐广泛地被和平利用的当今社会中,了解核能知
识是十分必要的。由于原子核的结合能是讨论裂变、聚变中
释放原子能的依据,是了解核能的基础,因此,原子核的结合
能也是教学的重点内容。

天然放射现象的教学,应着重介绍三种射线的性质,确定
三种射线的本质,和放射性元素的衰变规律,说明原子核不仅
有复杂的结构,而且可以发生转变。应使学生了解衰变过程
都遵守质量数守恒和电荷数守恒的规律。学生应能根据这两
条规律得出位移定则,确定衰变后的产物。

通过教学,不仅要使学生了解原子核的组成,而且要使学
生知道人们是怎样确定原子核的组成的。通过介绍查得威克
确定中子质量的分析,要使学生理解守恒定律在物理研究中
所起的作用。

在介绍原子核的结合能时,除了要求学生会计算,还要让
学生理解核子平均结合能随质量数变化的曲线的意义。爱因
斯坦的质能方程是在说明怎样计算结合能时直接给出的。教
学中不要求论证,只要求学生会使用。

通过基本粒子的教学,应使学生了解基本粒子并不是组
成物质的最小单元,特别是强子,已发现它们仍有内部结构,
许多物理学家倾向于不再用“基本粒子”这个名称,而把它改
称为粒子。探索强子结构的夸克(我国叫层子)模型,是一种比
较成功的理论。


本章的教学要求是:
\begin{enumerate}
\item 了解天然放射现象,知道三种射线的性质以及放射性
元素衰变的规律;了解探测放射线的方法。
\item 了解原子能的人工转变,知道原子核的组成,会写核
反应方程,了解放射性同位素的应用。
\item 理解原子核的结合能的概念,会用质能方程进行计
算,了解释放核能的两种途径——裂变和聚变,以及它们的
应用。
\end{enumerate}

\section{教学建议}
\subsection{原子核的组成}
\subsubsection{天然放射现象}

通过这一节的教学应该使学生了解,天然放射现象
的发现,打开了人们认识原子内部世界的窗口,它不仅使人
类认识到原子核也是具有结构的,而且告诉人们,一种原子核
可以自发地转变为另一种原子核。这一发现,揭开了原子核物
理的新篇章,至于原子核的内部组成如何,在这里暂不涉及,在
讲第三节原子核的人工转变和原子核的组成等内容时,再作
统一处理。这一节可着重讲述三种放射线的本质、它们的特性
和放射性元素的衰变规律,还应该使学生注意,天然放射性并
不是少数元素才具有的.原子序数大于83的天然元素都具有
放射性,原子序数小于83的天然元素,也有一些具有放射性。

三种射线的本质和特性,根据教材中的叙述学生不
难掌握。教学中可通过电场中偏转的实验,使学生加深对射
线带电性质的理解。对于三种射线的特性(贯穿本领、电离作
用等),可以简要地列表对比加以说明。三种粒子的符号、质
量数和电荷数、也应该让学生正确掌握:教学中还须告诉学
生,$\alpha$、$\beta$、$\gamma$粒子都是从原子核里放射出来的,但不能认为这
三种粒子就是原子核的组成部分。

放射性元素的衰变,放射性元素的原子核发生衰变
时,要放出一个$\alpha$粒子或$\beta$粒子,使原子核转变为新核。要让
学生了解,衰变过程遵守质量数守恒和电荷数守恒的规律。
衰变后的产物(新核)可由这两个守恒定律来确定。根据放射
性元素的衰变过程可写出衰变方程。教学中,要训练学生会
正确地书写核符号和衰变时的核反应方程。

教材从衰变方程中根据质量数守恒和电荷数守恒定律得
出了位移定则,但在实际教学中,不要强调机械地记忆这个定
则 学生只要理解了衰变的物理过程,独立推导出位移定则
是并不困难的。

半衰期是研究衰变过程的一个重要概念.应使学生
明白,放射性元素的衰变规律是统计规律,只适用于含有大量
原子的样品,半衰期是表示放射性元素的大量原子核半数发
生衰变所需要的时间,表示大量原子核衰变的快慢,当样品
中的原子数目减少到统计规律不再起作用的时候,我们就不
能肯定在某一时间里这些原子核会有多少发生衰变了。因
此也就无法肯定某一放射性样品的全部原子完全衰变所需的
时间,有的同学认为,可以由半衰期推算出放射性样品完
全衰变的寿命期,当然是不正确的,在学生明白了半衰期
的概念后,再引导他们求放射性元素经过$n$个半衰期后的剩
余质量,可启发学生自己得出
\[m_n=m_0\left(\frac{1}{2}\right)^n\]
的结论($m_0$为放
射性元素的初始质量,$m_n$为$n$个半衰期后的质量)。


\subsubsection{探测放射线的方法}

这一节里讲的云室、计数器和乳
胶照相,都是很早就使用了的最起码的核物理的实验手段,跟
原子核的人工转变等教材都有关系,应该让学生了解这部分
内容。教材主要讲述了三种探测方法的原理,目的是使学生
知道研究原子核变化中的微观现象,可以根据各种粒子产生
的次级效应来进行观察和判断,进而体会到研究微观现象的
规律并不神秘。因此,凡是有条件的学校都应该做好演示,让
学生观察,以加深印象,云室设备如果没有,也可以自制(参看
实验指导部分)。

云室实验由于粒子径迹呈现时间较短,云室又较小,许多
学生同时观察会有困难。教学时可先将各种粒子的径迹图样
向同学讲清楚,然后让学生分组进行观察,也可以利用投影
幻灯进行观察。

本节教材较多地涉及以前学过的知识,如气体的绝热膨
胀、过饱和汽、气体的电离和气体放电等内容。教学中要注意
新旧知识之间的联系,及时提示学生回忆过去学过的知识,加
深对三种探测方法的原理的理解。

\subsubsection{原子核的人工转变和原子核的组成}
学生已经知道了原子核是由质子和中子组成的,但并不了解是根据什么得
到这种结论的。本节的教学,应该使学生认识人们是怎样通
变革原子核的实验弄清了原子核的组成的。

关于原子核的人工转变,首先要注意讲清卢瑟福用
$\alpha$粒子轰击氮核的实验装置以及怎样确定实验中产生的新粒
子就是质子。还可以结合这个实验讲一讲卢瑟福的科学态
度,早在1915年,卢瑟福的学生马斯登就观察到了用$\alpha$粒子
轰击空气时会产生不寻常的长射程粒子。一种可能的解释是
这种粒子是氢核,因为这是用$\alpha$粒子轰击氢时常常出现的现
象。卢瑟福没有轻易作出结论,而是耐心地进行实验研究,以
便弄清那些粒子到底是氮核、氦核还是氢核,实验要在荧光
屏前观察和统计微弱的闪烁,条件是相当艰苦的,经过了三年
多的时间,在1919年夏,他才总结了$\alpha$粒子与轻原子的碰撞
现象,对氮原子核的人工转变作出了无可置疑的结论。

关于中子的发现,在教学中可以讲一讲守恒定律对
物理学的意义,在中子发现之前,摆在物理学家们面前的问题
是:要么$\alpha$粒子轰击铍发出的是$\gamma$光子,它在跟质子的碰撞中
能量和动量不再守恒;要么$\alpha$粒子轰击铍发出的射线不是$\gamma$
光子,而是一种新粒子。查德威克运用了能量和动量守恒定
律,科学地分析了实验结果,终于发现了中子。还可以讨论一
下,在约里奥·居里夫妇的实验中中子已经出现了,但他们不
能识别它,一项划时代的发现,就这样从他们手中溜走了,
我们应该由此得到什么教训。

在讲过上述两部分内容的基础上,再来讨论原子核
的组成,学生就可以认识到确定原子核的组成并不是一件很
容易的事情.从1911年提出的原子的核式结构起到1932年
完成对原子核组成的认识,经历了21年.很多科学家为此付
出了辛勤劳动。

本节的教学要注意培养学生综合运用知识的能力.
还可以让学生进一步练习写核反应方程。要让学生知道核反
应方程也和以前讲过核衰变方程一样,遵守电荷数守恒和质
量数守恒的规律,不过应该注意,核反应方程反映的是客观
的物理过程,是不能反过来根据上述两条规律任意编造的。

\subsubsection{放射性同位素及其应用}

人工放射性同位素的发现
是核研究的又一项重要成果,这一发现使放射性同位素获得
了广泛的应用。

关于同位素,要使学生了解以下几点:
\begin{enumerate}
\item 质子数相同而中子数不同的原子互称同位素;
\item 同种元素的不同的同位素在元素周期表上具有相同
的位置(原子序数相同),它们的核电荷数相同,具有相同的化
学性质;
\item 同种元素的各种同位素的中子数不同,因此它们的
物理性质有差异;
\item 同一种元素的多种同位素中,有稳定的,也有不稳定
的。不稳定的同位素会自发地放出$\alpha$粒子或$\beta$粒子衰变为别
种元素。这种不稳定的同位素就叫放射性同位素。四十多种
元素具有天然放射性同位素,各种元素都有人工放射性同
位素。
\end{enumerate}

人工放射性的教学中还要注意讲清$\beta^+$衰变和什么是正
电子,说明它的性质,并让学生掌握它的符号。

关于放射性同位素的产生及其应用,应使学生了解以下
几点:
\begin{enumerate}
\item 用中子、质子、氘核、$\alpha$粒子或$\gamma$光子轰击原子核都
可制取放射性同位素,重核裂变的产物也有放射性同位素。现
在大量使用的放射性同位素有许多是利用核反应堆产生的。

\item 放射性同位素的应用主要分为两个方面:
\begin{itemize}
\item 利用它
的射线,    \item 用作示踪原子。
\end{itemize}
在这两个方面各有其在工业、农业、
医疗、科研等方面的许多应用实例。这些实例不要求学生过
多的记忆,只要理解是怎样利用射线的特性和示踪原子来解
决实际问题的就可以了。
\item 
我国在这方面取得的新成就,可以适当地补充与介
绍,以充实这部分内容,进行爱国主义教育。
\end{enumerate}


放射性同位素的应用尽管都是一般性的介绍,但它涉及
的知识原理,有些也是学生较难理解的,可适当运用挂图幻灯
等教学手段配合教学。

\subsection{核能}
\subsubsection{原子核的结合能}

原子核的结合能是讨论裂变和聚
变时释放原子能的根据,是本章的一个重要概念。

核力的概念是讲解结合能的基础.关于核力的本质
问题目前尚未完全弄清楚,仍在进一步深入研究中。已经确
定的核的主要特性有:核力比电磁力强100多倍;核力是
短程力,只有距离接近到$10^{-15}$米的数量级时才发生作用;每
个核子只跟它相邻的核子间有核力的作用,而不是跟原子中
所有的核子有核力的作用。

讲解结合能的概念时,为了便于学生接受,可以按教
材中的思路讲述,即先说明要把原子核拆散成核子,需要克服
核子间的核力做功,因而需要巨大的能量。在讲解克服核力做
功需要能量时,如果跟学生熟知的宏观的力学现象作对比,更
便于学生理解,例如把宇宙飞船发射到地球引力范围之外,相
当于把地球和飞船拆开,需要克服地球引力做功,因而需要提
供能量,然后再讲反过来,核子结合成原子核时,就要释放能
量。核子结合成原子核时放出的能量或原子核分解为核子时
吸收的能量,都叫原子核的结合能。

根据核反应时的质量亏损和爱因斯坦的质能方程可
以计算原子核的结合能$E=\Delta mc^2$. 在进行计算时,常常用u
表示原子质量单位.书上331页的$1u=931.5$兆电子伏,指
的是1个原子质量单位所对应的能量,即$1u\x c^2=931.5$兆电
子伏。教学时要讲清这个式子的意义,以免引起学生的误解。

质能方程是爱因斯坦从相对论得出的。在教学时,要引
导学生不要把质能方程理解为质量就是能量或质量可以变成
能量。质能方程指出的是质量和能量之间的联系。即物体的
质量和它具有的能量之间保持着严格的正比关系。物体质量
不但会因为它吸收或放出能量而增减,还会由于机械运动状
态的改变而发生变化,只不过由于$c^2$是一个非常大的恒量,
通常的能量变化只引起微不足道的质量变化,在中学阶段只
要求学生初步理解和会用质能方程,不要对该方程作进一步
的讨论。

讲平均结合能时,应明确以下几点:
\begin{enumerate}
\item 平均结合能反
映了核子结合成不同原子核时平均每个核子所释放的能量,
即原子核结合能对每一个核子的平均值;    \item 不同原子核的平
均结合能不相同,平均结合能的大小表征原子核的稳定程度,
平均结合能越大,原子核越稳定;    \item 轻核和重核的平均结合能
都比较小,中等质量的原子核平均结合能较大,质量数为50
—60的原子核平均结合能最大,表示这部分核最稳定.这些
内容都可以引导学生观察平均结合能随质量数变化的曲线得
出来.
\end{enumerate}
练习三第5题让学生根据平均结合能计算裂变中释
放的能量,可以巩固平均结合能的概念,同时为讲解裂变作了
准备。

\subsubsection{重核的裂变}

在讲解重核裂变时,可以先根据平
均结合能曲线说明为什么重核裂变时要放出能量,从图中可
以看出,重核的核子平均结合能小于中等质量的核子的平均
结合能,因此重核分裂成中等质量的核时,会有一部分结合能
放出来。这种由核结构发生变化而放出的能叫做核能,也叫
做原子能。

讲述裂变的过程时,还可补充下面的核裂变反应方
程
\[\atom{U}{235}{92}+\atom{n}{1}{0}\longrightarrow \atom{Xe}{139}{54}+\atom{Sr}{95}{38}+2\atom{n}{1}{0}\]

铀核裂变产物$\atom{Xe}{139}{54}$(氙)和$\atom{Sr}{95}{38}$(锶)都是有放射性的,例
如$\atom{Xe}{139}{54}$可以经过一系列的$\beta$衰变而变成$\atom{La}{139}{57}$:
\[\atom{Xe}{139}{54} \mathop{\longrightarrow}^{\beta}   \atom{Cs}{139}{55}  \mathop{\longrightarrow}^{\beta}    \atom{Ba}{139}{56}  \mathop{\longrightarrow}^{\beta}    \atom{La}{139}{57}\]
所以在重核裂变过程中可以得到多种放射性同位素,这与前
面讲述的放射性同位素有关知识可以呼应。

根据质能方程算出释放的能量跟用平均结合能计算
的结果是一致的,因为结合能本身也是用质能方程计算出来
的.教学时,可把计算出来释放的能量201兆电子伏,补充写
到课本334页的核反应方程中去,强调一下能量守恒。

讲铀核裂变的链式反应时,应着重说明铀235与铀
238跟中子的作用的区别,为讲核反应堆作准备.
最后可说明铀核裂变时还可能分裂成三部分或四部分,
这是我国科学家钱三强、何泽慧夫妇于1946年在法国首先
发现的。

核反应堆是用人工控制链式反应的装置,在教学时,
可通过挂图或幻灯片,讲清楚反应堆的构造、各部分的名称和
作用。可以把反应堆分成五部分来讲述:
\begin{enumerate}
\item 铀棒,是天然铀或浓缩铀。作为原子燃料,提供原子
能,在裂变时释放大量能量。
\item 减速剂,它的作用是使裂变时产生的快中子减速,变
成慢中子,慢中子容易被铀235俘获而引起裂变,维持链式
反应。
\item 控制棒。它的作用是调节中子数目以控制链式反应
速度,镉吸收中子的能力很强,所以用它作控制棒,当反应过
于激烈时,使镉棒插入深一些,让它多吸收一些中子,链式反
应的速度就会慢一些;当反应过于缓慢,达不到所需功率时,
使镉棒插入浅一些,让它少吸收一些中子,链式反应速度就可
增大,用电子仪器自动地调节镉棒的升降,就能使反应堆保
持一定功率。
\item 冷却剂。用水、液体金属钠或空气等,在反应堆内外
循环流动,不断地带走反应堆放出的热量,同时用来输出
热能。
\item 水泥防护层。作用是防止铀核裂变物放出的各种射
线对人体的危害,用来屏蔽射线,不让它们透射出来。
\end{enumerate}

还应说明原子能反应堆不仅可以提供强大的原子能,而
且它产生的大量中子,还可以用来进行各种原子核物理实验,
制造各种放射性同位素,利用反应堆还可以生产新的核燃料。

最后,可以介绍一下我国第一座大高通量原子反应堆
的简单情况,进行爱国主义教育。我国自行设计建造的第一
座大型高通量原子反应堆,于1978年12月安装完毕,于
1980年12月达到高功率运行,它的主体及其配套工程的设
备,全部都是我国自行设计制造的。

这座高通量反应堆,热功率设计额为12.5万千瓦,最大
热中子通量是$6.2\x10^{14}\text{个}/{\rm cm^2\cdot s}$,据不完全统计,目前
世界各国共有400座研究堆,其中中子通量在$3\x10^{14}\text{个}/{\rm cm^2\cdot s}$以上的约20座左右,这从一个侧面反映我国核科学技
术的发展水平。

这座高通量反应堆是一座试验研究反应堆。它具有一堆
多用的特点,可以同时产生多种放射性同位素和超钚元素。它
的建成有助于提高我国许多科学领域和工业部门的技术
水平。

\subsubsection{轻核的聚变}

与讲裂变相似,在教学时可先据
平均结合能曲线说明轻核的平均结合能很小,因此,当某些轻
核结合成质量较大的核时,能释放出更多的结合能,一个氘核
和一个氚核结合成一个氦核时,能释放17.6兆电子伏的能
量,也可以把这个能量,补充写到核反应方程中去。轻核结合
成质量较大的核叫做聚变。还可让学生计算一下上述聚变反
应中平均每个核子释放出来的能量为:
\[\frac{17.6}{5}=3.52{\rm MeV}\]
而铀核裂变时平均每个核子放出的能量约为1兆电子伏,说
明轻核聚变时每个核子放出的能量比重核裂变时所释放的能
量还要大几倍。关于轻核聚变的例子,可以再提供以下两个
核反应方程:
\[\begin{split}
    \atom{Li}{6}{3}+\atom{H}{2}{1}&\longrightarrow 2\atom{He}{4}{2}+22.4{\rm MeV}\\
    \atom{H}{3}{1}+\atom{H}{1}{1}&\longrightarrow \atom{He}{4}{2}+19.2{\rm MeV}\\  
\end{split}\]

关于聚变发生的条件,要着重说明聚变为什么要在
高温下才能发生。因为要使轻核接近到$10^{-15}$米,由于原子
核都是带正电的,这样就必须克服电荷之间很大的斥力作
用,这就要使核具有很大的动能,必须把它们加热到很高的
温度。因此聚变反应又叫做热核反应,这里还可以复习核力
为短程力这样一些旧知识。

教学时,应该使学生了解热核反应可分为爆炸式热
核反应和可控热核反应。现在人们已经掌握并利用爆炸式热
核反应,例如氢弹的爆炸,对可控热核反应,至今尚在研之
中,还有许多困难需要克服,这些困难主要是把几千万度以
上的高温聚变物质控制在一定足够长的时间,我国自行设计
和制造的可控热核反应试验装置“中国环流器一号”已于
1984年9月顺利起动,它标志着我国可控核聚变的研究有了
新的发展和提高,必将为人类探求新能源作出贡献。

\subsection{基本粒子}
本章第八节介绍基本粒子,这是选讲教材。编入这一节
教材,目的是使学生知道现代物理的前沿,了解人类对物质结
构的认识是不断深入的,是无穷尽的。

\subsubsection{基本粒子的概念和种类}

随着人类对基本粒子的不
断深入的认识,基本粒子的概念也是不断发展的,当人类知道
了质子和中子组成原子核,原子核和电子组成原子时,人们把
电子、质子和中子叫做基本粒子。后来又发现正电子、$\mu$介
子、$K$介子和$\pi$介子等。人们把电子、正电子称为轻子,质
子、中子称为重子,质量介于质子和电子之间的粒子叫做介
子。后来又发现了质量比质子大的粒子,名叫超子,超子也属
于重子。后来又发现了反粒子等。现在发现的基本粒子已达
几百种。按照基本粒子之间的相互作用,可以把它们分为三
类:强子、轻子和媒介子,所有强子都参与强相互作用,轻子
都不参与强相互作用,媒介子是传递粒子间相互作用的粒子,
例如光子就是其中的一种,是传递电磁相互作用的。原来按
照粒子质量所做的分类,已不能恰当地反映粒子的性质。例
如,重子和介子都属于强子,但$\mu$介子只参与弱相互作用,应
属于轻子,现已改称为$\mu$子。

应使学生明确绝大多数基本粒子都是不稳定的,在很
短时间内就发生衰变,并且能互相转化,课本上列举四个转
化的例子,前两个例子是中子转化为质子和质子转化为中子,
可以概括为实物粒子之间的转化;后两个例子是正、负电子转
化为光子和光子转化为正、负电子,这可以概括为实物粒子与
场粒子之间的转化。这些事实表明,实物粒子(如电子)与场
粒子(如光子)虽然有基本的区别,但是仍然有着深刻的联系。

应使学生明确基本粒子不是组成物质的最基本的、最
小的单元,它们也有复杂的结构。夸克(我国称为层子)模型
就是一个强子结构的理论模型,这个理论模型可以解释许多
实验现象,强子的性质已明显表现出有内部结构,在这以后,
“基本粒子”这个名称就名不副实了,于是近来又把它们改称
为“粒子”。研究粒子的种类、性质、运动规律以及它们的内部
结构的学科叫粒子物理学。由于许多科学工作者的努力,粒
子物理学已经取得了辉煌的成果,然而,就在得到这些新成果
的同时,客观世界也把新的疑问呈现在人们面前,等待人们去
研究、认识。

这一章的内容,大都只要求学生了解,以扩大学生的视
野,因此在教学过程中,可以较多地采用培养学生阅读能力为
主的“自学、讨论、总结”式的教学方法,即在堂上由学生自学
课文,然后学生自己归纳课文的基本内容,提出问题,通过讨
论,最后总结出应该掌握的几个方面的知识。教师也可以提
出一些启发性的问题,引导学生讨论,或补充讲解一些课文上
没有的内容,最后小结知识要点。

\section{实验指导}
\subsection{演示实验}
\subsubsection{用云室观察$\alpha$粒子径迹}

课本图9.2的云室,是活塞式云室,它是用杠杆机构
牵动云室底上下运动,使云室内的饱和汽因绝热膨胀而达到
过饱和。

实验前,用长约150毫米的玻璃移液管,从云室壁上的孔
中将酒精均匀地滴洒在云室内的呢子上,约10—20滴左右;
把放射源插在握子上后送入云室,拧紧螺丝不使漏气。然后
把200—300伏左右的直流电源接在云室的“$+$”“$-$”接线柱
上,不要接错。演示时可轻快迅速地向下按压杠杆,使云室骤
然膨胀,在膨胀的一刹那可看到很不清楚的$\alpha$粒子的径迹如
课本图9.3的左图所示,这时不要松手,径迹可保留一小段
时间.只要一松手,径迹立即消失,必须间隔30秒乃至1分
钟,才可进行下一次的膨胀与观察。为了让较多的学生观察,
可以在云室的上方放置倾斜$45^{\circ}$角的平面镜,让学生从镜里
观察;也可让学生分组观察(每组4—5人),配合以较强光源
从云室壁上的透明窗把$\alpha$粒子径迹照亮,观察效果更好。

演示实验中可能出现的故障:
\begin{enumerate}
    \item 密封不严密.如果在按下杠杆时,能听到轻微的嘶嘶
漏气声,需要检查一下上盖压圈上的螺钉是否松动,应再紧一
紧;或者放射源握子的螺旋没有拧紧,也应再紧一下。务使云
室密闭严密可靠。
\item 径迹模糊不清.假如云室在膨胀后,充满白雾,径迹
模糊不清,可以降低膨胀比,也就是把限制器的柱头向上拧,
然后再试验几次,直到径迹清晰为止。假如这样仍无效,应检
查一下电场电压是否确已加上,如果因电压未加而云室内离
子扫除不尽,实验效果就差。有时把电压提高一些也有帮助。
\item 演示径迹不直。由于微量漏气,使云室内的气体在数
次膨胀之后,压强发生变化,引起a粒子径迹畸变,为此,在经
过几次膨胀后,可以扭开放射源螺旋,略微放气,使云室内气
体压强恢复,再拧紧握子螺旋,即能继续演示实验。
\item 云室上盖玻璃板上结成雾珠.当滴入酒精过多或房
间温度较高时,往往会出现雾珠,使上盖不透明,影响观察。消
除雾珠的方法是把放射源握子拧下,按几次杠杆,使云室连续
进行吸气和排气动作,若干次后就可以使雾珠消失,玻璃盖恢
复透明。这时应将纱布遮住结合套的孔,防止灰尘进入云室。
同时要在通风的地方排气,以免云室内排出的射气散布在教
室内,污染教室空气。
\end{enumerate}

在实际的教学过程中,本实验还是比较容易成功的,上面
这些故障很少发生,即使出现了也不难解决。如果不是使用
杠杆式的云室,而是用老式的“橡皮球”式的云室,实验成功率
就差一些,参照以上的方法,摸索实验的规律,也可以把演示
实验做好。

云室装置也可以自制,有两种自制的方案:一种是用干冰
致冷,制作容易,但需要使用干冰;另一种还是利用绝热膨胀
致冷,演示比较困难,需要有一定技巧,教师可于课前多做几
次,以掌握演示中的一些技巧,保证课堂演示成功。现将两种
方案分述如下:

\paragraph{干冰式云室} 将一个带盖的透明塑料圆筒或玻璃圆筒
放在一块干冰上,用一条浸透
酒精的黑布,围绕在圆筒内壁
靠近上部的地方。在圆筒盖子
上的软木塞下面插一根针,把
放射源装在针眼里,这个云室
的结构如图9.1所示.

在云室中,从布条上蒸发出来的酒精,到达容器底部的冷
区域里遇冷要重新凝结,如果靠近云室底部的区域内没有能
使酒精蒸汽形成液滴的其他凝结核(例如灰尘)存在,则酒精
蒸汽附着的凝结核就只能是带电粒子或射线经过时产生的离
子。当顶部的蒸汽向低温处扩散时,在容器底部达到过饱和
状态,并以离子为凝结核形成雾珠。因此酒精液滴的径迹反
映的是粒子或射线经过的路径。这种径迹在暗底衬托下是可
见的,并可以照相。

演示时,应使云室内形成
高压电场。为便于观察,使光从
侧面沿水平方向照射云室内靠
近底部的区域。

\begin{figure}[htp]\centering
    \begin{minipage}[t]{0.48\textwidth}
    \centering
\includegraphics[scale=.6]{fig/9-1.png}
    \caption{}
    \end{minipage}
    \begin{minipage}[t]{0.48\textwidth}
    \centering
\includegraphics[scale=.6]{fig/9-2.png}
    \caption{}
    \end{minipage}
    \end{figure}

\paragraph{绝热式云室} 这种云
室与威尔逊云室原理相同。云
室的结构如图9.2所示.在一
个广口瓶上加一个中间有孔的
橡皮塞。橡皮塞中间的孔中插一根玻璃管,用橡皮管将玻璃
管连接到自行车的打气筒嘴上。瓶塞上穿过两根裸导线,瓶
内的一端焊接一条铜片电极。实验时,穿出橡皮塞的一端接
电源。放射源倒插在橡皮塞上。在瓶内要先放好浸透酒精的
脱脂棉。这种云室的结构简单,易于操作,径迹显示时间长。

演示时,在两电源间加上200—300伏的直流电压,用手
把打气筒的活塞慢慢地向下压,直到感觉到有一定压力时,突
然松手。压入瓶内的空气突然作绝热膨胀,使瓶内温度降低,
瓶中酒精蒸汽达到过饱和状态。由于放射源放出的射线粒子
使沿途的气体电离,这时过饱和的酒精蒸汽便以这些离子为
核迅速凝结成液滴,显示出放射线的径迹。演示本实验时要
注意:
\begin{enumerate}
    \item 仪器各部分应密封,不能漏气。在压气时漏气,放气
    时就不能获得绝热膨胀致冷的效果。
    \item 压气时,要缓慢地把打气筒的活塞压下,以防止压缩
    气体冲开瓶塞。瓶塞要塞紧。
    \item 加在电极上的直流高压,可以从电子管收音机取出,
    也可由自行装置的直流高压电源提供。要注意这时整流器处
    于空载情况,输出电压较高,所以应在整流器输出端并联一只
    2瓦、100千欧的电阻,以防止击穿整流器的滤波电容.
\end{enumerate}


\subsubsection{用盖革计数器进行演示}
盖革计数器是由盖革计数管、放大和显示装置组成,是利
用放射线的电离作用制成的粒子探测仪器(如图9.3)。
\begin{figure}[htp]
    \centering
    \includegraphics[scale=.6]{fig/9-3.png}
    \caption{}
\end{figure}


当一个放射性粒子进入计数管时,就使计数管发生一次
短暂的放电,从而得到一个脉冲电流。把这个脉冲电流用电
子电路加以放大,送入显示装置就可以显示出来。显示的方
法可以用扬声器发声、氖泡发光和数字显示等。把计数管用
开有窗口的金属套筒保护起来,加上手柄,再用电缆跟放大显
示装置连接起来就成为盖革计数器,又叫探测器,例如目前
医学及工业用的$\beta$、$\gamma$射线探测器。

用探测器可以进行一些演示。只是有的探测器是用耳机
听的,声音不够大。演示时,可以通过合适的线路把它接到扩
音机上,使全班学生都能听到。

利用计数器可以探测放射线。做法如下:把几个火柴盒
放在讲台上,其中一个装有放射源。调节探测器,使扩音机发
出间隔约两秒钟左右的“叭”“叭”声。告诉学生,这响声是宇
宙射线进入计数管产生的。然后把探头分别接近各只火柴
盒。当探头接近装有放射源的火柴盒时,扩音机里“叭”“叭”
声响的次数马上大量增加,这表示了有放射性粒子进入了探
测器。

利用计数器还可以演示原子辐射的防护。将探头和放射
源放在一定的距离上,让学生注意扩音机中“叭”“叭”声每分
钟响的次数,然后分别在探头与放射源之间放上硬纸板、木
板、铜片、铁片和铅片,让学生注意扩音机响声间隔的变化。从
实验可知哪种材料对射线有较大的防护作用,即阻止射线的
本领较大。

\subsubsection{放射源的制作}

取一个烧坏了的汽灯纱罩,将它研成粉末,用胶水粘成直
径2—3毫米的小球,插在一根火柴杆上,干后即成为放射源。
放射源应放进试管或玻璃瓶里保存。

汽灯纱罩是用浸过硝酸钍$\rm Th(NO_3)_4$的苎麻做成的,灼
烧后的灰烬含有99\%的二氧化钍$\rm ThO_2$。二氧化钍具有放
射性,所以能作为放射源使用。

还可以用搜集到的旧夜光表指针(或文具商店出售的罗
盘指针)上的磷光粉做成放射源。


\section{习题解答}


\subsection{练习一}
\begin{enumerate}
    \item 钍230是$\alpha$放射性的,它放出一个$\alpha$粒子后变成了
什么?写出衰变方程.


\begin{solution}
    钍230经过$\alpha$衰变后变成镭226, 其衰变方程为
    \[\atom{Th}{230}{90}\longrightarrow \atom{Ra}{226}{88}+\atom{He}{4}{2}\]
\end{solution}
\item 钫223是$\beta$放射性的,它放出一个$\beta$粒子后变成了
什么?写出衰变方程.


\begin{solution}
    钫223经过$\beta$衰变后变成镭223, 其衰变方程为
    \[\atom{Fr}{223}{87}\longrightarrow \atom{Ra}{223}{88}+\atom{e}{0}{-1}\]
\end{solution}
\item 钍232经过六次$\alpha$衰变和四次$\beta$衰变后变成一种稳
定的元素.这种元素是什么?它的原子量是多少?它的原子序
数是多少?


\begin{solution}
    经过六次$\alpha$衰变,质量数减少$6\x4$, 电荷数减少
    $6\x2$; 经过四次$\beta$衰变,质量数不变,电核数增加$4\x1$. 故新
    核的质量数:$232-6\x4=208$, 
    电荷数:$90-6\x2+4\x1=82$. 
    所以新元素为$\atom{Pb}{208}{82}$(铅).
\end{solution}
\item 
$\atom{U}{238}{92}$变成$\atom{Pb}{206}{82}$,要经过几次$\alpha$衰变和几次$\beta$衰变?

\begin{solution}
    衰变过程中质量数减少为$238-206=32$, 因为$\beta$衰
    变不影响质量数,可见要经过八次$\alpha$衰变;而经过八次$\alpha$衰
    变,电荷数应减少$8\x2=16$, 现在电荷数只减少了$92-82
    =10$. 因为每经过一次$\beta$衰变,电荷数增加1, 可见经过了六
    次$\beta$衰变。
\end{solution}
\item 
$\atom{Bi}{210}{83}$的半衰期是5天.10克的铋210经过20天后
还剩下多少?


\begin{solution}
    20天为4个半衰期,故20天后剩下铋210的质量
\[m=m_0\left(\frac{1}{2}\right)^4=10\x \left(\frac{1}{2}\right)^4=0.625{\rm g}\]
\end{solution}
\item 放射性元素$\atom{Na}{24}{11}$经过6小时后只剩下1/8没有衰
变,它的半衰期是多少?


\begin{solution}
因为$\dfrac{1}{8}=\left(\dfrac{1}{2}\right)^3$,所以是经过了3个半衰期,半衰期为
$6/3=2$小时。
\end{solution}
\end{enumerate}




\subsection{练习二}

\begin{enumerate}
    \item 用$\alpha$粒子轰击氮核使它发生转变.从云室的照片中
    为什么可以确定细而长的径迹是质子产生的,粗而短的径迹
    是反冲氧核产生的.


    \begin{solution}
    质子与氧核相比,质子的质量小,氧核的质量大。根
据动量守恒定律,当质子从复核中射出时,就具有比氧核大的
速度。由于质子速度大,在气体中能够通过较长的路程。但质
子的电量少,电离本领小,所以在云室中的径迹细而长;而反
冲氧核的质量大,速度小,在气体中通过的路程短,它的带电
量大,电离本领也大,因此它的径迹粗而短。
    \end{solution}
    \item 用$\alpha$粒子轰击氩40,复核衰变时产生一个中子和一
    个反冲核,这反冲核是什么?写出核反应方程.


    \begin{solution}
        反冲核是钙43, 其核反应方程为:
    \[\atom{He}{4}{2}+\atom{Ar}{40}{18}\longrightarrow \atom{Ca}{43}{20}+\atom{n}{1}{0}\]
    \end{solution}
    \item 用$\alpha$粒子轰击硼10,产生一个中子和一个具有放射
    性的核,它是什么?这个核能放出正电子,它衰变后变成什
    么?写出核反应方程.


    \begin{solution}
        具有放射性的核为$\atom{N}{13}{7}$(氮),经过衰变后变成$\atom{C}{13}{6}$(碳),其核反应方程分别为:
\[\begin{split}
    \atom{He}{4}{2}+\atom{B}{10}{5}&\longrightarrow \atom{N}{13}{7}+\atom{n}{1}{0}\\
    \atom{N}{13}{7} &\longrightarrow \atom{C}{13}{6}+\atom{e}{0}{1}
\end{split}\]
    \end{solution}
    \item 用中子轰击氮14,产生碳14,碳14具有$\beta$放射性,
    它放出一个$\beta$粒子后衰变成什么?写出核反应方程.


    \begin{solution}
        最后衰变为氮14, 其核反应方程为:
\[\begin{split}
    \atom{N}{14}{7}+\atom{n}{1}{0}&\longrightarrow \atom{C}{14}{6}+\atom{H}{1}{1}\\
    \atom{C}{14}{6} &\longrightarrow \atom{N}{14}{7}+\atom{e}{0}{-1}
\end{split}\]    
    \end{solution}
    \item 用中子轰击铝27,产生钠24,写出核反应方程.钠
    24是具有放射性的,衰变后变成镁24,写出核反应方程.


    \begin{solution}
        它们的核反应方程分别是:
\[\begin{split}
    \atom{Al}{27}{13}+\atom{n}{1}{0}&\longrightarrow \atom{Na}{24}{11}+\atom{He}{4}{2}\\
    \atom{Na}{24}{11} &\longrightarrow \atom{Mg}{24}{12}+\atom{e}{0}{-1}
\end{split}\]  
    \end{solution}
\end{enumerate}





\subsection{练习三}

\begin{enumerate}
    \item 氘核的质量是2.013553u,根据质量亏损,计算氘核的结合能.

    \begin{solution}
氘核由一个质子和一个中子组成,质子、中子和氘核
的质量分别为:
\[m_p=1.007277u,\qquad m_n=1.008665u,\qquad m_D=2.013553u\]
质量亏损 
\[\Delta m=m_p-m_n-m_D=0.002389u\]
氘核的结合能
\[\Delta E=\Delta mc^2=0.002389\x 931.5=2.225{\rm MeV}\]
    \end{solution}
    \item 碳原子的质量是12.000000u,可以看做是由6个氢
原子(质量是1.007825u)和6个中子组成的.求碳原子核的
结合能.(在计算中可以用碳原子的质量代替碳原子核的质
量,用氢原子的质量代替质子的质量,因为电子的质量可以在
相减过程中消去.)

\begin{solution}
    6个氢原子的质量
\[M_H=6m_H=6\x 1.007825u=6.046950u\]
6个中子的质量
\[M_n=6m_n=6\x1.008665u=6.051990u\]
质量总和:
\[M_H+M_n=6.046950u+6.051990u=12.098940u\]
质量亏损
\[\Delta m=12.098940u-12.000000u=0.098940u\]
碳原子的结合能
\[\Delta E=\Delta mc^2=0.098940\x931.5=92.16{\rm MeV}\]
\end{solution}
\item  在$\atom{He}{4}{2}$,$\atom{Kr}{82}{36}$,$\atom{U}{238}{92}$等原子核中核子的平均结合能个最大?哪个最小?原子核的结合能哪个最大?哪个最小?(根据平均结合能曲线进行比较)


\begin{solution}
    根据平均结合能曲线所示,$\atom{Kr}{82}{36}$的平均结合能最
    大,$\atom{He}{4}{2}$的平均结合能最小。原子核的结合能应为平均结合能乘以核子数,故$\atom{U}{238}{92}$的原子核的结合能最大,而$\atom{He}{4}{2}$的原
    子核的结合能最小。
\end{solution}
\item 如果要把$\atom{O}{16}{8}$分成8个质子和8个中子,要给它多
少能量?要把它分成$\atom{He}{4}{2}$,要给它多少能量?已知$\atom{O}{16}{8}$的核
子平均结合能是7.98MeV,$\atom{He}{4}{2}$的核子平均结合能是
7.07MeV.

\begin{solution}
要把氧16分成8个质子和8个中子,需要的能量就
是氧16原子核的结合能,即氧原子核的核子的平均结合能乘
以核子数:
\[E=16\x7.98=128{\rm MeV}\]
要把氧16分成4个氦核,所需能量应为上述能量减去4
个氦原子核的结合能($E'$)。因为
\[E'=4\x4\x7.07=113{\rm MeV}\]
所以
\[E-E'=128-113=15{\rm MeV}\]
计算把氧16分成4个氦核所需能量,还可用核子的平均
结合能之差乘以核子总数计算:
\[(7.98-7.07)\x16=14.6{\rm MeV}\]
\end{solution}
\item 在一次核反应中,铀核$\atom{U}{235}{92}$变成了氙核$\atom{Xe}{136}{54}$和锶
核$\atom{Sr}{90}{38}$(同时放出了若干中子).铀核的核子平均结合能约为
7.6MeV,氙核的核子平均结合能约为8.4MeV,锶核
的核子平均结合能约为8.7MeV.
\begin{enumerate}
    \item 把U235分解为核子,要吸收多少能量?
    \item 再使相应的核子分别结合成Xe136和Sr90,要放出多少能量?
    \item 在这个核反应中是吸收还是放出能量?这个能量大
约是多大?
\end{enumerate}

\begin{solution}
\begin{enumerate}
    \item 把U235分解为核子时吸收的能量等于U235原子
    核的结合能
\[E=7.6\x235=1.79\x10^3{\rm MeV}\]
\item 核子结合成Xe136和Sr90放出的能量,等于该原子
核的结合能
\[\begin{split}
    E_1&=8.4\x136=1.14\x10^3{\rm MeV}\\
E_2&=8.7\x90=0.78\x10^3{\rm MeV}
\end{split}\]
\item 在这个核反应中,是放出能量。这个能量为
\[E_1+E_2-E=(1.14+0.783-1.79)\x 10^3=1.3\x 10^2 {\rm MeV}\]
\end{enumerate}
\end{solution}
\end{enumerate}



\subsection{习题}
\begin{enumerate}
    \item 铀238的半衰期是$4.5\x 10^9$年,假设一块矿石中含
有1千克的铀238,经过45亿年(相当于地球的年龄)以后,
还剩有多少铀238?假设发生衰变的铀238都变成了铅206,
矿石中会有多少铅?这时铀铅的比例是多少?再经过45亿年,
矿石中的铀铅比例将变成多少? 根据这种铀铅比例能不能判
断出矿石的年龄?


\begin{solution}
经过45亿年即经过1个半衰期,剩下质量应为原
来的一半,即剩下铀238的质量为0.5千克.

因为有一半铀衰变为铅,所以铀、铅的原子核数相等,它
们的质量之比等于原子核质量数之比,即
\[238:206=0.5:x\]
所以铅的质量$x=0.433$千克.铀、铅比例为:$0.5:0.433
=1.15:1$.

再经过45亿年,即又经过一个半衰期,0.5千克铀衰变后
质量剩下一半,还有0.25千克铀,又有0.216千克铅产生。这时铀、铅比例为$0.25:(0.433+0.216)=0.385:1$.

由于铀、铅质量的比例与矿石的年龄有关,因此我们可以
根据铀、铅的比例粗略判断矿石的年龄。
\end{solution}
\item 镭核在$\alpha$衰变中放出能量为4.78MeV的$\alpha$粒
子和能量为0.19MeV的$\gamma$粒子.如果1克镭每秒钟有
$3.7\x10^{10}$个原子核发生$\alpha$衰变,算出它每秒钟释放多少能量.

\begin{solution}
1个镭原子核发生衰变时放出的能量
\[\Delta E=(4.78+0.19)=4.97{\rm MeV}\]
$3.7\x10^{10}$个原子核发生衰变时放出能量
\[E=3.7\x10^{10}\Delta E=3.7\x10^{10}\x4.97=1.8\x10^{11}{\rm MeV}\]
\end{solution}
\item 静止状态的放射性原子核镭($\atom{Ra}{226}{88}$)进行$\alpha$衰变.为
了测量$\alpha$粒子的动能$E$,让$\alpha$粒子垂直飞进$B=1$特的匀强磁
场,测得$\alpha$粒子的轨道半径$r=0.2$米.
\begin{enumerate}
    \item 写出Ra的$\alpha$衰变方程.
    \item 试计算$\alpha$粒子的动能$E$.
\end{enumerate}

\begin{solution}
\begin{enumerate}
    \item 镭的$\alpha$衰变方程
\[\atom{Ra}{226}{88}\longrightarrow \atom{Rn}{222}{86}+\atom{He}{4}{2}\]
    \item 在匀强磁场中,$\alpha$粒子在洛仑兹力作用下作圆周运
    动,洛仑兹力作为向心力,有
    \[qvB=m\frac{v^2}{R}\]
    所以 \[v=\frac{qRB}{m}\]
    $\alpha$粒子的动能
\[\begin{split}
    E=\frac{1}{2}mv^2=\frac{1}{2}m\left(\frac{qRB}{m}\right)^2    &=\frac{q^2R^2B^2}{2m}\\
&=\frac{(2\x 1.6\x 10^{-19})^2\x 0.2^2\x 1^2}{2\x 6.64\x 10^{-27}\x 1.6\x 10^{-19}}=2{\rm MeV}
\end{split}\]
\end{enumerate}


\end{solution}
\item 在某些恒星内,三个$\alpha$粒子结合成一个$\atom{C}{12}{6}$核.$\atom{C}{12}{6}$
的质量是12.0000u,$\atom{He}{4}{2}$的质量是4.0026u.这个反应中放出
多少能量?

\begin{solution}
    三个$\atom{He}{4}{2}$的质量
 \[   m=3\x4.0026u=12.0078u\]
    三个$\alpha$粒子结合成碳12核时质量亏损
\[\Delta m=12.0078u-12.0000u=0.0078u\]
    放出的能量
\[\Delta E=\Delta mc^2=0.0078\x931.5=7.3{\rm MeV}\]
\end{solution}
\item 已知$\atom{Ra}{226}{88}$,$\atom{Rn}{222}{86}$,$\atom{He}{4}{2}$的原子量分别是226.0254,
222.0175,4.0026.求在$\atom{Ra}{226}{88}$衰变
\[\atom{Ra}{226}{88}\longrightarrow \atom{Rn}{222}{86}+\atom{He}{4}{2}\]
中放出的能量是多少电子伏?如果这些能量都以Rn核和He
核的动能形式释放出来,放出的$\alpha$粒子的速度有多大?

提示:能量和动量都守恒.

\begin{solution}
    衰变过程中质量亏损
\[\Delta m=226.0254u-(222.0175u+4.0026u)=0.0053u\]
放出的能量
\[\Delta E=\Delta mc^2=0.0053\x931.5=4.9{\rm MeV}=7.9\x10^{-13}{\rm J}\]
设$\alpha$粒子质量为$m$, 速率为$v$, Rn222的质量为$M$, 速率
为$V$, 由于衰变过程中能量和动量守恒.则有
\begin{align}
    mv+MV&=0\\
    \frac{1}{2}mv^2+\frac{1}{2}MV^2&=\Delta E
\end{align}
由(9.1)可得
\[V=-\frac{m}{M}v\]
代入(9.2)式,可得
\[\begin{split}
    \frac{1}{2}mv^2\left(\frac{m}{M}+1\right)&=\Delta E\\
    \frac{1}{2}mv^2&=\frac{M}{M+m}\Delta E\\
v&=\sqrt{\frac{2M\Delta E}{(M+m)m}}\\
&=\sqrt{\frac{2\x222\x7.9\x10^{-13}}{226\x4\x1.66\x10^{-27}}}=1.5\x10^7\ms
\end{split}\]
\end{solution}
\item 在一原子反应堆中,用石墨(碳)作减速剂使快中子
减速.已知碳核的质量是中子的12倍,假设把中子与碳核的
每次碰撞都看作是弹性正碰,而且认为碰撞前碳核都是静止
的.
\begin{enumerate}
    \item 设碰撞前中子的动能是$E_0$,经过一次碰撞,中子损失的能量是多少?
    \item 至少经过多少次碰撞,中子的动能才能小于$10^{-6}E_0$?
\end{enumerate}

\begin{solution}
    设中子质量$m_1$, 初速度为$v_0$, 碳核质量为$m_2$, 初速度
    为0. 一次碰撞后,中子速度为$-v_1$, 碳核速度为$v_2$. 
    根据动量守恒
\begin{equation}
    m_1v_0=m_2v_2-m_1v_1
\end{equation}
根据动能守恒
\begin{equation}
    \frac{1}{2}m_1v_0^2=\frac{1}{2}m_2v_2^2+\frac{1}{2}m_1v_1^2
\end{equation}
由题意设$m_2=Am_1$, $A=12$. 
所以
\begin{align}
    v_0&=Av_2-v_1\\
    v^2_0&=Av^2_2+v_1^2
\end{align}
解(9.5)和(9.6)可得
\begin{equation}
v_1=\frac{A-1}{A+1}v_0
\end{equation}
\begin{enumerate}
\item 经过一次碰撞后中子损失的能量
\[\Delta E=\frac{1}{2}m_1v_0^2-\frac{1}{2}m_1v_1^2=\frac{1}{2}m_1v^2_0\left[1-\left(\frac{A-1}{A+1}\right)^2\right]=\frac{48}{169}E_0\]

\item 经过第$n$次碰撞后,中子速度为$v_n$, 则有
\[v_n=\left(\frac{A-1}{A+1}\right)^n v_0\]
由题意要求,有
\[ \frac{\frac{1}{2}m_1v_n^2}{\frac{1}{2}m_1v_0^2}<10^{-6} \]
即
\[\left(\frac{A-1}{A+1}\right)^{2n}<10^{-6}\quad \Rightarrow\quad \left(\frac{11}{13}\right)^{2n}<10^{-6}\]
由对数计算,可得$n\ge 42$. 

所以至少经过42次碰撞,中子动能才能小于$10^{-6}E_0$。
\end{enumerate}
\end{solution}
\end{enumerate}


\section{参考资料}
\subsection{正电子的发现}
威尔逊云室用于粒子物理研究,由于可以“抓拍”在短暂
瞬间内发生的现象,不仅为已知粒子留了影,也记录了原先不
为人知的新现象,帮助人们发现新的粒子。第一个被云室摄
到的新粒子就是正电子,这也是人们所发现的第一个反粒子,
它是从宇宙线中拍摄下来的。

宇宙线是来自外层空间的射线,奥地利科学家亥斯(V. 
F. Hess, 1883—1965)在研究地球上和大气中的放射性时发
现了这种射线。在分析宇宙线的成分和性质时,威尔逊云室
是一个很好的工具,密立根是位多年致力于宇宙线研究的专
家,他把云室放到强磁体的磁极中间,不间断地每15秒钟对
云室拍一次照。由于飞行的带电粒子在磁场中发生偏转,就
可根据照片中径迹的偏转方向确定其带电的状况。带有相反
电荷的粒子偏转方向也相反,在同一磁场强度下,速度的快慢
也与偏转有关。另外,在粒子径迹的单位长度上水滴的个数
与粒子的质量有一定关系。所以,磁场中的云室的照片可以
提供径迹的多种信息,1932年,密立根的同事年轻的安德孙
(C. D. Anderson, 生于1905年)在认真察看几千张云室照片
时,发现其中有一种径迹是陌生的。这种径迹的弯曲方向表
明它是一个带正电的粒子,但是与已知的带正电的粒子如质
子相比,它的雾珠密度完全不一样。经过仔细安排的进一步
实验,确认了这是一种新粒子,它的质量与电子质量相等,但
电荷相反,称为正电子。

在此以前,狄拉克从关于电子的相对论量子理论预见到,
一定有一种质量与电子完全相等,电荷大小也相等,只是符号
相反的粒子。安德孙发现的正电子恰好具有这种性质,因而,
这是狄拉克理论的一个有力支持,狄拉克的理论还预言电
子和正电子可以同时成对地由光子从真空中产生出来;相反
地,如果一个电子和正电子相撞,他们便同时湮灭,转化为光
子。这两种现象也为实验证实了。另外,在发现正电子后测
得它在物质中平均只能存在百万分之一秒的时间,但正电子
与电子一样是不会自行衰变的,它在物质中存在时间短正是
因为它十分容易与物质中的电子发生湮灭反应的缘故。这也
说明了为什么只有在云室发明后才能从宇宙线中发现正
电子。

由于狄拉克理论的成功,人们从此认识到了微观世界的
几个特征。首先,从狄拉克理论导致一个结论:每一种微观粒
子都有其反粒子。虽然正粒子和其反粒子相遇时即发生湮
灭,所幸我们生存的宇宙空间内,正粒子的数量与反粒子的数
量并不相等,例如地球上只有极少数来自外层空间的反质子
和正电子,因此不用担心,地球上的物质是不会被湮灭掉的。

\subsection{放射性的应用}
\subsubsection{工农业方面}
放射性测井。如天然$\gamma$射线测井,只需测量钻井内各
个深度的岩层天然放射的$\gamma$射线强度,加以比较,就能够了解
岩层的构造。放射性测井不必分析取出的岩心,方法简便可
靠。我国的石油勘探和煤碳勘探等已经应用了这种方法。

土壤水分的测定。可以应用吸收射线的原理来测量
土壤水分,通常采用钴60的土壤水分测量仪。这种仪器是
双管型的,一管装放射源,另一管装盖革计数管。两管按一定
的距离插入土中,计数管所测到的计数率和标准曲线相对照,
即可测出土壤的水分。

农作物的储藏。利用射线的照射可以抑制马铃薯的
发芽和杀死谷仓里的各种害虫。

\subsubsection{医疗方面}
诊断上的应用,放射性碘(碘131)已被应用来诊断甲
状腺的机能是否正常,当病人食用一定量的放射性碘后,从
小便中的含碘量可以看出正常人和甲状腺失常的人有显著差
异。还可以用来检查脑瘤的位置和测量血液循环的速度,诊
断血管有无阻塞或硬化。

治疗上的应用,放射线有生理效应,当放射性辐射作
用到活的细胞上,可以造成死亡,这种作用的机理跟快速的带
电粒子通过细胞对细胞内原子的电离和分子的分解有关。正
在迅速地成长和繁殖的细胞,对辐射的作用特别敏感,这种
作用可以用来治疗癌肿。

\subsubsection{科学研究方面}

半导体的扩散研究.把放射性元素锌(锌65)电镀到
半导体锗的表面,然后在电炉内加热,锌即从表面向里扩散。
扩散分子的浓度跟距离表层的深度有关系,有了放射性锌作
标记就容易测定这个关系,方法是把锗片一层层磨下,收集
每一层磨下的粉末,测量其放射性强度,即可求出扩散分子的
浓度。

制造发光剂。发光物质可以在放射性的作用下发光。
在发光物质(例如硫化锌)中加很少量的镭盐,就制成了一种
经常发光的漆。这种漆要是涂在仪表的表盘、指针或瞄准装
置上,就可以使它们在黑暗中也能被看见。

测定地球的年龄。从不含放射性元素的矿中采出的
通常的铅,原子量是207.20。因铀238的衰变而形成的铅,
原子量是206.含在某些矿中的铅,原子量很接近206。由此
可见,在铀矿形成的时候(从熔融体或液体中结晶出来),其中
并不含有铅,这些矿物中现含的铅都是由铀的衰变积聚起来
的。利用衰变定律,根据矿物中的含铅量与含铀量的比例,就
可以决定矿物的年龄。应用这种方法确定出来的矿物年龄,要
以几亿年来计量,最老的矿物年龄超过15亿年,通常是把硬
地形成后所经过的时间,当作地球的年龄,根据放射性测定
出地球的年龄约为45亿年.

\subsection{核电及安全}
核电站以铀为燃料,在反应堆中裂变产生热,再用冷却剂
带出,传给汽轮发电机,生产电能。这种以高温高压水为冷却
剂的核电站称为压水堆型核电站。

压水堆型核电站的安全性好。铀燃料以及产生的大量放
射性产物被包盖在耐高温、耐腐蚀的锆金属管中,形成燃料元
件。管子被称为防止放射释放的第一道屏障,燃料元件组成
堆芯装在直径约四米、壁厚约二百毫米的低合金钢压力壳中,
压力壳被称为第二道屏障,压力壳及有关设备又被完全封闭
在安全壳中,安全壳又被称为第三道屏障。

对压水堆型核电站进行事故发生可能性的分析表明,堆
芯熔化最大假想事故的可能性为$10^{-4}$—$10^{-8}$/堆年,即一个反
应堆运行一万年到一百万年,才可能产生一次这样的事故。而
同时由安全壳受损造成大量放射性释放的可能性就更小了。

在全世界已经运行的三百七十四座核电站中,压水堆型
占总数的50\%; 正在建设的核电站一百七十六座中,压水堆
型占63\%. 我国正在建设和计划建设的核电站,都是压水堆
型核电站。

在工业化、高科技社会对能源的需求日益增加的同时,环
境污染和生态问题也日益尖锐,电能是常规能源,常规电站
(尤其是燃煤电站)会排出大量的废渣与废气,煤燃料中的
20\%成为灰渣和飞尘.飞尘形成煤烟引起大气污染,烟气中
含有的大量氧化硫、氧化氮,是产生酸雨的主要因素。科学家
们预言,酸雨和二氧化碳产生的“温室效应”将成为今后严重
的环境问题。

核能是一种清洁、安全、经济的能源。压水堆型核电站在
正常运行时,由于燃料元件包壳、压力系统和安全壳三层屏障
的包容,以及放射性废气、废液排放前的一系列处理,加上严
格的排放标准和管理,最终的排放量是很小的。一座大型的压
水堆电站,允许的排放量对附近居民的最大照射仅为天然放
射性本底的5\%, 世界各国多年的运行经验表明,实际排放
量很容易被控制在允许排放量以下。对于废气、废液处理过
程中的浓缩物,经固化后运至厂外贮存库作永久贮存。在核
电站正常运行时,排出的废物很少,有利于保护环境。

现在,发展核电是世界能源发展的共同趋势。据联合国
国际原子能机构统计,全世界1985年的核发电量占总发电量
的15\%. 

\subsection{贝克勒耳}
贝克勒耳(1852—1908),法国物理学家。
1852年11月15日诞生于法国巴黎的一个科
学世家,祖父、父亲都是法国科学院院士,贝克勒耳幼年偏
爱自然科学.1872年中学毕业后,曾先后在巴黎工业大学和
土木工程学院学习,1876年到工业大学任教.1888年以光
的吸收方面的论文获得博士学位,1892年任巴黎自然博物馆
和巴黎工业大学教授,1903年贝克勒耳因为发现放射性现
象和皮埃尔·居里夫妇同获诺贝尔物理学奖。

贝克勒耳早年从事磷光和荧光现象的研究已取得不少成
就.1895年德国物理学家伦琴发现了X射线.这一发现引
起了贝克勒耳的很大兴趣。贝克勒耳试图研究X射线与可
见光的联系,他把一张照相底片用黑纸包得严严实实,再把硫
酸铀钾晶体放在上面,在太阳光下照射。几个小时后,把底片
冲出来一看,发现底片上有浅浅的黑斑,贝克勒耳认为,这是
硫酸铀钾晶体受阳光照射激发出荧光和X射线,但是后来的
实验,由于几天连阴雨而不能用阳光照射,他只好扫兴地把铀
盐晶体和用黑纸包好的照相底片一起放入写字台的抽屉里,
结果惊奇地发现:这些未经阳光照射的铀盐也能使底片感光,
冲出来的底片上清清楚楚地显现出黑斑。他作了认真的分析,
认为这种现象同荧光、阳光都无关,可能是铀盐晶体自身发出
的一种神秘的射线的结果,接着他又对铀的穿透辐射性质进
行了试验,证明这种辐射在一定距离内具有放电体的性质,即
放射性。这是科学实验中认识放射性的开端。在3月2日科
学院例会上,贝克勒耳激动地宣布了这个新发现。

会后贝克勒耳精心地做了一系列实验,分别对铀盐晶体
加热、冷冻、研成粉末、溶解在酸里等物理、化学加工,发现只
要有铀元素,就有这种神奇的贯穿辐射。他还试验了纯金属
铀,发现它所产生的贯穿辐射要比用硫酸铀钾强,在五月十
八日科学院例会上,贝克勒耳宣布,铀盐会自发放射出射线,
这是一种新的、由原子自身产生的射线,后人就把这种射线
叫做贝克勒耳射线。

贝克勒耳刚年过半百,身体健康状况就很差,医生劝他休
养,他却舍不得离开实验室,对医生说:“除非把我的实验室搬
到我疗养的地方,否则我决不离开!”1908年8月24日,贝克
勒耳离开了人世,终年56岁.后人为纪念这位放射性研究的先
驱者,把放射性活度的单位命名为“贝克勒耳”,简称“贝克”。

\subsection{玛丽·居里}
玛丽·居里(1867—1934),法国物理学家、化学家.是放射性学说的奠基人之一,原籍波兰,出生
于华沙.1884年在华沙中学毕业时得到了金质奖章,1891
年至1894年,在巴黎大自然科学系学习,毕业时得到物理
学和数学两个硕士学位,1903年在巴黎大学完成了博士论
文“放射性物质的研究”的答辩。从1906年起,为巴黎大学教
授和教研室主任,1914年起兼任镭学研究院院长。

玛丽·居里和皮埃尔·居里对铀、钍等矿物的放射现象进
行了研究,并从大量的沥青铀矿中分离出两种具有更强的放
射性物质,于1898年发现了两种元素:钋和镭。

1902年玛丽·居里分离出了零点几克纯净的镭盐,而在
1910年她和法国化学家德别爱而诺一起得到了金属镭,她
确定了镭的原子量和它在化学元素周期表中的位置.1903年
由于放射现象的研究,居里夫妇和贝克勒耳一起得到了物理
学诺贝尔奖金,而在1911年,玛丽·居里由于得到了金属状态
的镭和对放射性元素性质的研究而荣获化学诺贝尔奖金。

玛丽·居里试验了很多种放射性元素、研究它们的性质,
详细分析了放射性测量的方法,研究了放射性辐射对人体细
胞的影响,第一个在医学上利用放射性。居里夫妇的杰出发
现,开创了利用原子能的新纪元。

1914年巴黎镭学研究院成立,居里夫人任理事,后任院
长,1922年当选为巴黎医学科学院院士。同年出任国际文
化合作委员会委员,后任副主席。居里夫人除两次获得诺贝
尔奖金外,还获得各国科学奖章十六枚,获得二十五个国家的
荣誉头衔一百多个,被人们誉为“镭的母亲”。

居里夫人为人类作出了巨大的贡献,深受各国人民的爱
戴和崇敬。1920年,美国记者麦朗宁夫人获悉居里夫人很需
要一克镭继续作放射性研究,就在全美国妇女中展开了募捐
运动。她在不到一年的时间里募集到十万美元,买了一克镭
准备献给居里夫人。1921年5月21日,美国第三十四任
总统哈定代表美国妇女,向居里夫人馈赠了这一克镭。

由于长期受放射线照射,居里夫人晚年患了恶性贫血症,
双目也几乎失明。但她经常隐瞒病情,坚持进行研究,她常
说:“我的生活是不能离开实验室的”.1934年7月4日,居
里夫人在法国萨瓦去世,终年67岁。

\subsection{约里奥·居里}
约里奥·居里(1900—1958),法国物理学家,
是居里夫妇的女婿,对原子核物理有重要
贡献。

约里奥·居里1900年3月19日诞生在法国巴黎的一个
商人家庭。他从小喜欢读书,尤其喜欢钻研自然科学。他非
常崇敬法国大科学家巴斯德(1822—1895)和居里夫妇,不但
阅读他们的生平传记,还模仿他们的科学生活.约里奥1918年
考取居里夫妇发现镭的巴黎理化学院,每门功课都是第一。不
久,约里奥应召入伍服役。战后,他回巴黎理化学院一边工
作,一边在朗之万教授指导下学习,他的兴趣在物理、化学方
面.1925年约里奥作居里夫人的助手,并且结识了伊丽芙·居里。

1930年约里奥以论述放射性元素钋的电化学的论文获
得博士学位。1932年约里奥和他的妻子伊丽芙·居里(1897
—1956)合作,用放射性钋所产生的$\alpha$射线轰击铍、锂、硼等元
素,发现了前所未见的穿透性强的辐射。后经查德威克的
研究,确定为中子,1934年他俩在用$\alpha$粒子轰击铝硼时,
首次产生了人工放射性物质。由于这一重大发现,两人于
1935年获得诺贝尔奖金。约里奥·居里于1937年任法兰西学
院教授,法国国家科学研究中心原子能研究所所长。1946年
任法国科学研究中心主任,主持法国原子能委员会的领导
工作。

从本世纪四十年代中期起,约里奥·居里为了维护世界和
平,致力于和平利用原子能的事业,极力反对法国生产和发展
原子武器.为世界和平事业做出了重大贡献,1958年8月14
日,约里奥·居里在巴黎去世,终年58岁。

































































