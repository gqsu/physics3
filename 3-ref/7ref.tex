\chapter{光的粒子性}\minitoc[n]
\section{教学要求}

本章讲述光电效应、爱因斯坦的光子说的主要内容,还简要讲述光的波粒二象性以及物质波的概念。通过本章的教学,使学生进一步了解光的本性,初步接触一些量子观念,对微观粒子的特性有一个初步印象,本章知识也为学习原子物理作了准备。

本章知识大都只需定性了解。但由于微观粒子的性质不同于宏观物体,学生往往习惯于用观察宏观现象时形成的观念去认识微观粒子,因而在理解微观粒子的性质时会遇到困难,教学中要特别注意使学生了解微观现象与宏观现象有着本质上的差别,以逐步适应这一章内容的教学。

本章可分为两个单元。第一、二、三节为第一单元,讲述光电效应-光的粒子性;第四、五节为第二单元,讲述微观粒子的波粒二象性。

光电效应现象及其规律,是认识光的粒子性的基础,最好是做好紫外线照射锌板时发生光电效应的演示,以加深学生对现象的认识,但是,教材中介绍的研究光电效应规律的实验,比较复杂,不易做准确。因此教学中只要讲清实验现象,使学生能在了解实验现象的基础上,理解光电效应的规律就可以了,不要求一定在课堂上做演示。

光的波动说不能解释光电效应的规律,为了理解这一点要用到下述知识:按照波动理论,光的能量由光的强度决定,而光的强度又由光的振幅决定,与频率无关。这一知识学生过去没有学过,可联系声波的知识(振幅越大声音越强)给学生讲一讲,以帮助学生理解光电效应与光的波动说之间的矛盾。爱因斯坦对光电效应的解释是在普朗克量子说的基础上作出的,教学中也应对普朗克量子说的主要内容作些简单的介绍。

爱因斯坦的光子说及其对光电效应的解释,是本章的重点内容,讲述爱因斯坦对光电效应的解释时,要抓住入射光与金属中自由电子间能量接受关系的线索,着重说明光电效应的极限频率,要让学生理解逸出功的概念,知道不同金属的逸出功不同。

光的波粒二象性是光的两大对立学说的历史性的综合。学生在分别认识了光的波动性和粒子性后,要把这两种认识统一起来仍然是有困难的。教学时要提醒学生,不能用机械模型来想象波粒二象性,爱因斯坦的光子能量公式和动量公式与表征波的特征的频率和波长相联系,表明他的光子说与牛顿的微粒说有着本质的区别,对于公式$p=E/c$, 学生知道就行,不要求进行讲解。

物质波一节教材,是选讲内容,对学生不作要求,但这节内容扩大了对波粒二象性的认识,说明了一切微观粒子都具有波粒二象性。如果时间允许,讲一讲还是必要的和有益的。

本章的教学要求是:
\begin{enumerate}
\item 了解光电效应的基本规律和光电效应的应用.
\item 知道光子说的基本内容及其对光电效应的解释,掌握
爱因斯坦光电效应方程。
\item 了解光的波粒二象性的初步概念.
\end{enumerate}


\section{教学建议}
\subsection{光电效应——光的粒子性}

\subsubsection{光电效应}

光电效应是学生认识光的粒子性的实验基础,所以最好是做好课本图7.1的演示.实验时,要注意验电器金箔闭合或张开的现象都不能简单地说明光电效应,需要对实验现象进行分析后才能得出结论。用紫外线照射带负电的锌板,验电器的金箔立即闭合的实验,需用紫外线照射带正电的锌板,验电器的金箔不闭合的实验做对比,表明紫外光照射锌板只能打出负电荷;用紫外线照射不带电的锌板时,验电器的金箔张开,应补充检验锌板上带正电荷而不是负电荷,这样,通过对这些实验现象的深入分析,才能归纳出现象的共同本质,即光照射下物体中能够发射电子,进而引出光电效应的概念。

光电效应规律的实验,有条件的学校也可以做,没有条件的学校可按照课本274页的装置简图描述实验的过程及现象,使学生理解怎样通过实验得出光电效应的实验规律。

有的学生不理解课本图7.2电路中的光电流的回路,光电子到达极板$A$后,经电流表再通过电路(包括滑动变阻器和
电池组两个支路)到达极板$K$形成回路,这是由于极板$K$发射光电子后带正电荷,光电子到达极板$A$后,极板$A$带负电荷。只有光电子从极板$A$经光电管以外的电路到达极板$K$形成回路,才能持续不断地维持光电流。

光电效应的教学中,应强调它的主要规律,即极限频率的存在以及光电子的最大初动能只与入射光的频率有关而与人射光的强度无关,这是波动说无法解释的实验事实,也是引入光子说的主要依据。把这两点讲清楚了,也就为讲述本单元的核心内容——爱因斯坦的光子说打下了基础。

\subsubsection{光子和光子说}
学生接受光子说的主要困难在于建立光量子的观念。教学中要强调光子说与波动说不同,光子说认为光的能量不是连续分布的,而是一份一份地集中在一个个粒子——光子上。光的能量既跟光子的数目有关,也跟每个光子的能量有关。每个光子的能量$E=h\nu$, 与光波的频率成正比。

在介绍爱因斯坦的光子说时,可联系普朗克的量子假设,说明理论的继承和发展的关系,学生没有学过黑体辐射的规律,这里也不必补充这方面的知识,只需简要介绍普朗克提出电磁辐射的发射和吸收时,能量是不连续的、一份一份进行的。爱因斯坦则进一步推广了普朗克的假设,认为自由电子与电磁辐射相互作用时,吸收的能量也是不连续的、一份一份的。光就是由这种不连续的粒子组成的。

帮助学生按照光子说建立光的粒子模型:光的能量分布在一个个光子上,光束可以看作光子流,每个光子都以光速运动着。光和其他物质相互作用,就是光和其他物质交换
一份一份的光子的能量。

在运用光电效应解释光电效应的规律时,要抓住爱因斯坦的光电效应方程$\frac{1}{2}mv^2_m=h\nu-W$, 着重说明极限频率的存在和入射光子的频率决定光子的最大初动能等实验规律。要让学生了解光电效应方程中各个量所代表的物理意义,知道这个方程是对单个光子而言的,并能从方程上看出,如果入射光的频率很低,$h\nu$小于金属的逸出功$W$, 自由电子就不会从金属中逸出。只有当光子的能量达到或超过金属的逸出功时,才能产生光电效应。

逸出功是个很重要的概念,应使学生掌握它的意义和计算式。学生还应该知道,不同金属的逸出功不同,它们的极限频率也不同。

教学时要注意能量的单位焦耳与电子伏的换算关系。

在运用光电方程解释光电效应规律时,还可就粒子说和波动说对应的能量、光的强度的认识进行对比,以加深对光子说的认识。

\subsubsection{光电效应的应用} 

这一节教材讲述的重点是光电管
的作用。因此,教学时要使学生清楚地了解光电管的工作原理。

在讲述具体应用时,应使学生了解光控继电器和有声电影是怎样利用光电管来工作的。对光控继电器,应了解电路的基本组成,应清楚是光电管将光信号变为电信号达到自动控制电路的作用的。在有声电影的放映中,是利用光电管把声音信号还原为声音的。对于装置中的其他问题,教学中不必
详细介绍。

\subsection{波粒二象性}
\subsubsection{光的波粒二象性}

光的波粒二象性是十七世纪以来关于光的本性的微粒说和波动说两大对立学说的历史的综合,是对光的本性进行长期研究得出的结论。通过光的波粒二象性的教学,一方面使学生对光的本性有比较完整的了解;另一方面使学生对于描述微观粒子的量子概念有初步的了解。

光的波粒二象性的观点,用经典物理观念是无法理解的。因此只有破除旧观念的影响,才能接受近代物理对微观粒子运动规律的认识,才能理解波粒二象性,但是限于学生的知识基础,在高中物理课中不可能以丰富的事实和有很强说服力的论证去破除学生头脑中已有的经典物理观念,所以对于学生对波粒二象性的了解不能要求过高。教学中只要学生对课文中所述内容大体知道即可,不宜再补充内容。

\subsubsection{物质波}

把光的波粒二象性推广到一切微观粒子上,就是德布罗意提出的物质波的假说。电子衍射现象证明了电子的波动性,说明了物质波的理论是正确的。在教学中,要注意介绍电子衍射实验,使学生了解物质波理论的实验验证,并说明物质波也是一种几率波。同时,还要使学生了解,虽然一切微观粒子具有波粒二象性,但光子跟其他微观粒子还是有区别的,光子是永远以光速运动的、没有静止质量的粒子。

这一节的教学,对扩展学生的知识面是有益的。通过对物质波假说的提出和检验过程的了解,还可以对学生进行科
学的方法论和辩证唯物主义认识论的教育。

\section{实验指导}
\subsection{演示实验}
\subsubsection{光电效应现象}

被紫外线照射的锌板能发射出电子,紫外线光源可用弧光灯(摘掉聚光透镜)、紫外线灯或用荧光高压汞灯(除去涂有荧光粉的外壳,从灯芯直接射出的光中含有紫外线)。 实验室中高压汞灯的接线电路如图7.1所示.图中的镇流器可用万用变压器的初级0—110伏的线圈代替.
\begin{figure}[htp]
    \centering
\includegraphics[scale=.6]{fig/7-1.png}
    \caption{}
\end{figure}

图7.2中的锌板必须用细砂纸打磨,除去表面的氧化层,
把表面打磨干净的锌板用细尼龙线直接捆绑在指针验电器的小球旁边。

用毛皮摩擦过的橡胶棒使锌板带上负电,验电器的指针偏开一定的角度。用紫外线照射锌板时,见到验电器的指针逐渐回到0处,表明在紫外线照射下锌板所带的负电荷消失了。

若使锌板带上正电,用紫外线照射时,验电器的指针的偏角几乎保持不变,表明锌板上的正电荷并不因紫外线的照射而消失。

只用紫外线照射不带电的锌板时,验电器的指针不偏转。若在锌板附近放置带正电的玻璃棒,可以看到验电器的指针发生偏转。进一步用带正电的玻璃棒或带负电的橡胶棒靠近锌板,可以用实验检验出锌板上带的是正电荷。若用带负电的橡胶棒靠近锌板时,验电器的指针也不偏转。

这是由于,锌板在紫外线作用下能够放出电子,锌板带正电。释放出的电子聚集在锌板周围形成空间电荷区,使锌板和周围空间电荷区之间形成反向电压,阻碍光电子的发射,只有用带正电的物体吸附了锌板周围的负电荷,才能使锌板发射出较多的光电子,验电器的指针才能发生偏转。

实验时要注意防止紫外线伤害眼睛。

\begin{figure}[htp]\centering
    \begin{minipage}[t]{0.48\textwidth}
    \centering
    \includegraphics[scale=.6]{fig/7-2.png}
    \caption{}
    \end{minipage}
    \begin{minipage}[t]{0.48\textwidth}
    \centering
    \includegraphics[scale=.6]{fig/7-3.png}
    \caption{}
    \end{minipage}
    \end{figure}

\subsubsection{研究光电效应的规律}

研究光电效应的规律的实验线路如图7.3, 其中光电管的型号是GD-28,它的阴极材料为铯锑(CsSb), 极限波长约为0. 6500微米,灵敏度大干10微安/流明,工作电压24伏,光源可用低压白炽灯,如双缝干涉仪(J2515型)的光源,电表可用一般的演示用电表,如J0401型等.


(1)光电效应的产生

按图连好电路,光电管两极加正向电压约15伏左右,这时电路中没有电流。接通照明电路,当光照到光电管的光入射窗口时,电路中有了光电流。用黑纸板遮挡入射窗口,电流随即消失;移开挡板,又有光电流。

(2)光电管的饱和光电流随入射光强度增大而增大

接通电路,微安表中有光电流通过。移动滑动变阻器的滑动触头,使加在光电管$A$, $C$两极间的正向电压增大,光电流也增大,直到光电流达到饱和值,照到光电管阴极的光强度要适当(入射光不能过强,否则会影响光电管的使用寿命),要使光电流的饱和值小于80微安,正向工作电压小于24伏.提高入射光的强度(如把光源移近光电管或提高白炽灯的供电电压),光电流的值继续变大,饱和光电流的值也随之增大。

(3)将光电管的两极改接反向电压

电压较小时,电路中仍有光电流,逐渐提高反向电压的值,光电流逐渐减小,直到电压达到某一值时,光电流变为0. 再提高入射光的强度,光电流仍为零,在光入射窗口换上不同颜色的滤色片,可以见到,对于紫光阻止光电流的反向电压最大。

(4)光电管阴极的极限波长,极限效率

电路接正向电压,在入射窗口放蓝色(或绿色)滤色片时,电路中有光电流,若加红色滤色片时,即使是增大入射光的强度,电路中也没有光电流。

\subsubsection{光电效应的应用}

光电管、电流放大器(光电效应演示器中的附件)、演示用继电器(J2413型)和小灯泡按图7.4连好电路.用灯泡代表
计数器,接继电器的常闭触点,继电路的原线圈和电流放大器相连,光照在光电管上,电路中有电流,继电器动作,常闭触点打开,电灯不亮,代表计数器不工作,用手挡住射入光电管的光流,继电器原线圈中的电流消失,常闭触点接通,灯泡亮代
表计数器工作一次.若没有电流放大器,可按图7.5电路自制。

\begin{figure}[htp]\centering
    \begin{minipage}[t]{0.48\textwidth}
    \centering
    \includegraphics[scale=.6]{fig/7-4.png}
    \caption{}
    \end{minipage}
    \begin{minipage}[t]{0.48\textwidth}
    \centering
    \includegraphics[scale=.6]{fig/7-5.png}
    \caption{}
    \end{minipage}
    \end{figure}


\section{习题解答}


\subsection{练习一}

\begin{enumerate}
    \item 使锌板产生光电效应的光子的最长波长是0.37微
米,这种光子的能量是多少电子伏?锌的逸出功是多少?

\begin{solution}
光子的能量
\[E=h\nu=\frac{hc}{\lambda}=\frac{6.63\x 10^{-34}\x 3.0\x 10^8}{0.37\x 10^{-6}\x 1.6\x 10^{-19}}=3.4{\rm eV}\]
锌的逸出功$W=3.4{\rm eV}$
\end{solution}
\item 可见光的光子,能量范围是多大(用电子伏表示)?为
什么用可见光不能使锌板产生光电效应?

\begin{solution}
可见光的频率范围是$3.9\x10^{14}$—$7.7\x10^{14}$赫.光
子的能量
\[\begin{split}
E_1&=h\nu_1=\frac{6.63\x 10^{-34}\x 3.9\x10^{14}}{1.60\x10^{-19}}=1.6{\rm eV}\\
E_2&=h\nu_2=\frac{6.63\x 10^{-34}\x 7.7\x10^{14}}{1.60\x10^{-19}}=3.2{\rm eV}\\
\end{split}\]
可见光光子的能量范围为$1.6$—$3.2$电子伏,都小于锌的逸出功,因此不能使锌板产生光电效应。
\end{solution}
\item 铯的逸出功是$3.0\x10^{-19}$焦,用波长是0.59微米的
黄光照射铯,电子从铯表面飞出的最大初动能是多大?

\begin{solution}
最大初动能
\[\begin{split}
    \frac{1}{2}mv^2_m &=\frac{hc}{\lambda}-W\\
    &=\frac{6.63\x 10^{-34}\x 3.0\x 10^8}{0.59\x 10^{-6}}-3.0\x 10^{-19}\\
    &=3.7\x 10^{-20}{\rm J}
\end{split}\]
\end{solution}
\item 钨的逸出功是4.52电子伏,使钨产生光电效应的最
长波长是多少?这种波长是可见光吗?

\begin{solution}
因为$W=\dfrac{hc}{\lambda}$,所以极限波长是$\lambda_0=\dfrac{hc}{W}$,
\[\lambda_0=\frac{6.63\x 10^{-34}\x 3.0\x 10^8}{4.52\x 1.6\x 10^{-19}}{\rm m}=0.275\mu{\rm m}\]
这种波长的光不是可见光。
\end{solution}
\end{enumerate}



\subsection{习题}
\begin{enumerate}
    \item 功率为1瓦的手电筒灯泡大约有5\%的电能转化为
可见光,试估算它1秒钟能释放出多少个可见光的光子.

\begin{solution}
 可见光的频率按$6\x10^{14}$赫来估算,光子的能量$E=h\nu$, 一秒钟释放的光子数
\[n=\frac{Pt\x 5\%}{h\nu}=\frac{1\x 1\x 5\%}{6.63\x 10^{-34}\x 6\x 10^{14}}=10^{17}\text{(个)}\]
\end{solution}
\item 使铜产生光电效应的最低频率是$1.1\x10^{15}$
赫,用频率为$1.5\x10^{15}$赫的紫外线照射铜时,它发射出的光电子的
最大速度是多大?

\begin{solution}
\[\frac{1}{2}mv^2_m=h\nu-h\nu_0\]
最大速度
\[\begin{split}
    v_m&=\sqrt{\frac{2h(\nu-nu_0)}{m}}\\
    &=\sqrt{\frac{2\x 6.63\x 10^{-34}\x (1.5-1.1)\x 10^{15}}{9.1\x 10^{-31}}}\\
    &=7.6\x 10^5\ms
\end{split}\]
\end{solution}
\item 一个质量为$4\x10^{-4}$克的尘埃颗粒,以$1{\rm cm}/{\rm s}$
的速度在空气中下落,计算它的德布罗意波长.

\begin{solution}
    德布罗意波长
\[\lambda=\frac{h}{mv}=\frac{6.63\x 10^{-34}}{4\x 10^{-7}\x 1\x 10^{-2}}=1.7\x 10^{-25}{\rm m}\]
\end{solution}
\item 计算速度为$10^3\ms$的中子的德布罗意波长.这
个波长跟$\gamma$射线的波长相比如何?中子的质量是$1.67\x10^{-27}$
kg.

\begin{solution}
中子的德布罗意波长
\[\lambda=\frac{h}{mv}=\frac{6.63\x 10^{-34}}{1.67\x 10^{-27}\x 10^{3}}=4\x 10^{-10}{\rm m}\]
在$\gamma$射线的波长范围内。
\end{solution}
\end{enumerate}



\section{参考资料}
\subsection{早期量子论和光电效应}

量子的概念是德国物理学家普朗克研究黑体辐射时最先提出的。

在任何温度下都能全部吸收落在它上面的一切电磁辐射的物体叫做绝对黑体,简称黑体,一个不透射任何辐射的器壁围住带有一个小孔的空腔,可以作为绝对黑体的模型,实
验表明,黑体辐射的能谱(即光谱的能量分布曲线)与组成黑体的材料无关,只与黑体的温度有关,图7.6就是黑体辐射的能谱图.图中的$E_{\lambda}$表示波长在$\lambda$和$\lambda+\dd \lambda$之间时辐射的能量密度。可以看出,能量的分布有极其显著的最大值,这个最大值对应的波长$\lambda_m$随着黑体温度的升高而向短波方向移动。
\begin{figure}[htp]
    \centering
\includegraphics[scale=.6]{fig/7-6.png}
    \caption{}
\end{figure}

许多物理学家试图从经典理论推导黑体辐射的能量分布公式,结果都失败了。例如,英国物理学家瑞利和琼斯根据经典物理的电磁场理论和统计物理推得黑体辐射的能量分布公式,只在波长相当长的部分才与实验曲线相符合,而随着波长的减小,辐射通量趋于无限大,而被称为黑体辐射的“紫外灾难”。

1900年,德国物理学家普朗克假设黑体的腔壁由无数带电谐振子组成,并且假设这些谐振子的能量不能连续变化,只能取一些分立值:$0,e,2e,\ldots$这些分立值的能量是$e=h\nu$的整数倍。从而推出黑体辐射的普朗克公式,这个公式与实验曲线吻合得很好。

普朗克能量量子假设是对经典物理学的一个巨大的突破,1918年,普朗克由于对量子理论的贡献而荣获诺贝尔物理学奖。

爱因斯坦认识到普朗克量子概念带来的将是物理学理论基础的根本变革,无论是经典力学或经典电动力学都不能应用到微观世界所发生的过程,他进一步发展了能量子的假设,提出了光量子假说,普朗克理论认为空腔中的辐射场(电磁场)本质上仍然是连续的,只是当它们与腔壁谐振子发生能量交
换时才显示出量子性,爱因斯坦进一步认为光不仅是一份一份地被辐射或吸收,而且它的能量也是聚集成一份一份地在空间传播,光是由光子组成的粒子流,每个光子的能量$e=h\nu$.

光子学说最初是以假说的形式提出的,这是由于当时的实验事实还不充分,光电效应实验也是非常原始的,是在真空条件不很好的条件下做出的。爱因斯坦根据光子假说得出的光电效应方程成功地解释了光电效应的实验规律,但在当时并未被物理学家广泛承认,他们不赞成光子说,因为它违背了光的波动理论。

美国物理学家密立根花了十年的时间做光电效应实验,开始他不相信光量子理论,想用实验否定它,结果恰恰相反,他于1915年宣布:证实了爱因斯坦的光电效应方程,并根据该方程用实验测得普朗克常数$h$与普朗克公式提出的$h$值完全一致,其他的一些实验,如光压,康普顿效应等都证明了光量子假设的正确性,1922年爱因斯坦由于理论物理方面的贡献和光电效应方程而荣获诺贝尔物理奖,密立根也于1923年由于他在基本电荷和光电效应方面的研究工作荣获诺贝尔物理奖。


\subsection{外光电效应和内光电效应}

通常所说的光电效应是指外光电效应,即物体在光的照射下光电子飞到物体外部的现象。另一种光电效应叫内光电效应,它是物体在光的照射下,内部原子中的一部分束缚电子变为自由电子,这些电子仍留在物体内部,使物体的导电性
加强。

利用内光电效应可以制成光敏电阻、光敏二极管和光电池。光敏电阻的阻值随着光照射的强弱而明显地变化。光敏二极管的顶部有玻璃透镜,通常给光敏二极管接上反向电压,如果没有光照,反向电阻很大,电路里几乎没有电流;有光照时,反向电流就随着增大,经三极管放大后,可推动继电器工作。

\begin{figure}[htp]
    \centering
\includegraphics[scale=.6]{fig/7-7.png}
    \caption{}
\end{figure}

光电池的种类很多,早期有氧化铜光电池和硒光电池,现在又有硅、砷化镓、硫化镉、磷化铟等光电池。硅光电池是由单晶硅材料制成的、具有大面积PN结的半导体器件.图7.7
是硅光电池的结构示意图,PN结两边各有一个引出电极,就是光电池的正负极。光电池的表面镀有一层增透膜,其作用是为了减小光的反射,提高光电池的转换效率。光电池之所以能把光能转变为能是应用了半导体PN结的光生伏打效应。在光照射下,物体内部原子的束缚电子变成自由电子,
形成自由电子和空穴。在结电场作用下,自由电子向N区运动,空穴向P区运动,于是在N区和P区分别聚集了电子和空穴,产生了电动势。接有负载时,负载中就有电流通过。现在的硅光电池在强光照射下,能产生0.5伏的电动势,每平方厘米工作面积输出24毫安的光电流,相当输出功率10—12毫瓦。使用时,可把大量的硅光电池串联和并联起来,以得到所需要的电压和电流。光电池作为电源应用在人造地球卫星和灯塔、无人气象站等处。









