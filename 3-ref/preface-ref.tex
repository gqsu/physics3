\chapter{前言}
为了帮助教师使用好高中物理(甲种本)第三册教材,我
们编写了这本教学参考书。内容包括全书的说明,以及各章
的教学说明和资料。

“高中物理(甲种本)第三册说明”对这教材的内容安排
及编写这册课本的一些主要想法,作了概括的说明。

各章的教学说明和资料,包括教学要求、教学建议、实验
指导、习题解答、参考资料五项内容,在“教学要求”中,对教
学内容提出了具体的要求和说明,在“教学建议”中,对怎样
进行教学提了参考性意见。在“实验指导”中,提出了演示实
验,学生实验以及课外实验活动中应当注意的问题,还提供了
简单仪器的自制方法和不同的实验方法,补充了一些实验内
容,供教师选用。在“习题解答”中,给出了课本中全部练习和
习题的解答,供教师参考。在“参考资料”中提供了一些可供
教学参考的材料,有些材料也可在教学中引用。

全书的“教学建议”、“习题解答”、“参考资料”分别由北京
工业学院附中王杏村(一、二章)、北京海淀区教师进修学校蒋
宏涵(三、四章)、北京和平街一中王天谡(五—七章)、北京朝
阳中学邵醒凌(八、九章)编写,“实验指导”分别由蒋宏涵(一—
四章)、王天谡(五—七章)、邵醒凌(八、九章)编写。人民教
育出版社杜敏编写其余部分并统稿,刘克桓复审。

欢迎教师对本书提出宝贵意见。

\begin{flushright}
    编者
\end{flushright}


\chapter{高中物理甲种本第三册的说明}

高中物理课本(甲种本)第三册,同前两册一样,是按
照高中物理教学纲要(草案)的较高要求编写的。这册共有九
章教材,第一章至第四章讲电磁学知识,第五章至第七章
讲光学知识,第八章和第九章讲原子和原子核物理方面的
知识。

这册教材的内容,同试用本比较,增加了几何光学和三相
交流电的知识,调整了“电子技术基础”一章的内容,其他内容
与试用本基本相同。

为使本册教材内容要求更适应学生的接受能力,使
他们把基础知识掌握得更好些,本册教材对一些非主干知识
放低了教学要求。

关于直线电流的磁场,只要求学生对这个磁场的磁感应
强度的大小由哪些因素决定有所了解,而不要求用直线电流
的磁场公式$B=kI/r$
进行计算,以免增大习题难度,加重学生
负担。

关于电感和电容对交流电相位的影响,现象虽然容易演
示,但道理不易说清。为便于教学,将这两项内容改为选讲。
教学中只要求通过实验介绍现象,而不要求说明道理,使学生
对这方面的知识有个一般的了解即可。

光的衍射。在中学阶段,很难讲清衍射条纹是如何产生
的,这是因为多数学生理解不了为什么在干涉中把狭缝看作
是单一光源,而在衍射中却把狭缝看作是无数光源的集合。本
册教材只介绍了衍射现象及其产生的条件,而没有用波的叠
加来分析衍射现象。

为便于教学和有助于学生能力的培养,本册教材在
讲法和叙述上努力做到循序渐进,思路清楚。概念和规律的
得出尽可能比较平易,易于为学生接受。例如,用位移电流的
概念讲述电磁场,教学上困难较多,本册教材不再提出位移电
流和传导电流的概念,力求使讲法深入浅出一些,变压器的
输入电流随着输出电流而增大的道理,在中学很难讲清楚,而
中学阶段讲变压器,主要是讲它的基本原理,没有必要深究这
个问题,本册教材将这个内容删去。

在讲述电磁感应的知识时,紧紧把握磁通量变化这个线
索。讲述感生电流产生的条件,注意启发和引导学生一步一
步地分析问题。得出结论为了突出磁通量变化这个主要线索,
楞次定律不再用能量守恒定律推导的办法,直接用磁通量变
化的观点分析现象。对法拉第电磁感应定律先用磁通量
变化率的观点表达一般结论,然后再讲切割磁力线的特殊
情况。

用能量的观点分析现象,是研究物理问题时常用的方法
之一。为了培养学生较深入地体会能量守恒定律在电磁现象
中的应用,本册教材改变了过去将能量守恒问题夹在各节中
分析的做法,单独设立一节专门分析电磁感应现象中能量的
转化和守恒。

本册教材在讲解知识的同时,注意介绍认识客观事
物的方法,以开阔学生的眼界,提高他们分析问题、解决问题
的能力。

人类认识客观世界总是从片面到全面,从现象到本质,从
宏观世界到微观世界,经过实践,认识,再实践,再认识而
步步深入的。教材在介绍光的本性时,介绍了在认识上经历
的辩证发展过程,使学生对光的波动性和粒子性都有明确的
印象。原子结构和原子核内容的介绍,也都按历史线索叙述,
使学生一步步体会人类认识微观世界的方法和途径,理解所
学的知识。

物理实验对物理学的发展起着很重要的作用。本册教材
对在物理学史上起了重要作用的一些著名实验作了介绍,介
绍的实验有:罗兰实验、赫兹实验、$\alpha$粒子散射实验。还介绍
了人们测量光速的几种方法。希望通过这些介绍,使学生了
解实验在物理学中地位的重要,并通过学习前人设计实验
时的巧妙构思,培养他们灵活运用知识,发展他们的思维
能力。

本册教材的内容涉及近代物理学中的一些重大发
现,讲述这些内容,使学生了解近代物理学中的科学观点,可
以增长他们的见识,有助于培养辩证唯物主义的世界观。

“场”是物理学的基本概念。引入场的概念,使物理学获
得很大的成就。教材中注意了培养学生用场的观点分析电磁
现象,知道场的客观存在和这个概念的重要性,教材从稳定的
电场和磁场到变化的电磁场,再到电磁波,强调了电磁波可以
脱离电荷而独立存在,不需要别的物质做媒介而传播,并且具
有能量,通过这样的讲解,逐步扩展学生对场的认识,使他们
获得场是客观存在的具体印象。

客观世界是相互联系的统一整体。将表面上不同的现象
联系起来,揭示出它们的共同本质,是物理学的伟大成果,物
理学发展中已完成了几次大的统一。教材中介绍了电现象和
磁现象的统一,光现象与电磁现象的统一,还介绍了把微观世
界统一起来的波粒二象性,学生对这些内容有所认识,将有
助于他们把知识联系起来,培养他们的综合能力。

质量守恒、电荷守恒、动量守恒和能量守恒是自然界普遍
遵守的规律,对微观世界和宏观世界都适用。教材注意介绍
在宏观、微观领域中这些规律的应用。人们对微观粒子相互
作用的认识的发展,很大一部分归功于人们能自觉地用守恒
定律来分析物理现象。通过教学,要使学生逐步体会在千变
万化的物理世界中,在现象背后存在着规律。守恒定律就体
现出变化的规律,要使学生体会到这一点,并培养学生重视
用守恒定律来处理问题。

这册教材介绍的波粒二象性、能量量子化等,反映了微观
世界的特有规律,体现了微观世界特有规律与宏观世界的不
同。教学中要注意告诉学生,观念必须与认识对象相适应,不
能用宏观世界得出的一般观念来看微观世界。

本册教材注意理论联系实际,联系现代科学技术.
讲好联系实际的内容,可以巩固基本知识,开阔思路和眼
界,提高运用知识的志趣和能力。根据实际情况,这册教材联
系实际的内容有三种情况:
\begin{enumerate}
    \item 基本知识的实际运用,对这部分内容,要讲清原理,弄清把基本知识应用于实际中去的思路。这类内容有:电流
    表工作原理,荷质比的测定和质谱仪,回旋加速器,自感现象
    的应用,变压器,电能的输送,感应电动机,光电效应等。教学
    中要注意不深究技术细节和枝叉问题。
    \item 一般性说明原理(不宜过细分析)。这部分内容有:涡
    流,无线电波的发送和接收原理,光导纤维,显微镜和望远镜,
    薄膜干涉及其应用,红外线、紫外线、伦琴射线的应用,光谱分
    析,激光,放射性同位素及其应用,核反应堆,可控热核反
    应等。
    \item 介绍性,这部分内容,大部分是阅读材料,如寻找磁
    单极子,直流输电,直线电机和悬浮列车,传真、电视和雷
    达,电子显微镜和射电望远镜,全息照相,偏振光和立体电
    影等。
\end{enumerate}

这册教材的习题,难度同试用本大体相当,根据这
册内容的特点,加强了联系实际的题目。为加强能力培养,安
排了少量的改进或设计实际装置的题目,还安排了一些综合
运用知识的题目。

本册教材同前两册一样,安排了演示实验、学生实验
和课外实验。高中物理教学仍然以实验为基础,对实验教学
应予以足够的重视。

对演示实验,要尽量做给学生看。在实验设备不足的情
况下,教师应尽可能地自制一些仪器设备,以保证教学效果。
在学校的仪器设备允许的情况下,可以将一些演示实验让学
生自己动手在课堂上做,以增加学生动手的机会。

对课外实验,虽不作要求,但应该鼓励学生做。有的课外
实验活动,需要教师给予必要的指导和帮助。

高中物理甲种本第三册的教学内容可按每周4课
时,全年共112课时讲授完.其中,第一章13课时,第二章10
课时,第三章16课时,第四章12课时,第五章17课时,第六
章11课时,第七章4课时,第八章5课时,第九章9课时;平
时复习和机动时间15课时。












