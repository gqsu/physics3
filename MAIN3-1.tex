\documentclass[b5paper, openany]{ctexbook}

\usepackage[weather]{ifsym}

%\usepackage{marvosym}
%\usepackage{anysize}
%\papersize{26cm}{18.5cm}
%\marginsize{2.25cm}{2.25cm}{2cm}{2cm}
\usepackage[margin=2.5cm]{geometry}


% 修改脚注的编号为加圈样式,并且各页单独编号

\usepackage{pifont}
\usepackage[perpage,symbol*]{footmisc}
\DefineFNsymbols{circled}{{\ding{192}}{\ding{193}}{\ding{194}}
{\ding{195}}{\ding{196}}{\ding{197}}{\ding{198}}{\ding{199}}{\ding{200}}{\ding{201}}}
\setfnsymbol{circled}



\usepackage{amsmath,amsfonts,mathrsfs,amssymb}
\usepackage{graphicx}

\usepackage[font=bf,labelfont=bf,labelsep=quad]{caption}
\usepackage{circuitikz}
\usepackage{tikz}
\usepackage{tkz-euclide}


\usepackage{physics}
%\usepackage{amsthm}
\usepackage{ntheorem}
\theoremseparator{\;}



\usepackage{blkarray}
\usepackage{bm}
\usepackage[colorlinks=true, linkcolor=black]{hyperref}

%\usepackage{enumerate}


\theoremstyle{plain}
\theoremheaderfont{\normalfont\bfseries} 
\theorembodyfont{\normalfont}


% \newtheorem{defn}{定义}[chapter]
% \newtheorem{thm}{定理}[chapter]
% \newtheorem{yl}{引理}[chapter]
% \newtheorem{tl}{推论}[chapter]

%\usepackage[framemethod=tikz]{mdframed}
%
%\newmdtheoremenv[
%  linecolor=black,
%  leftmargin=20pt,
%  innerleftmargin=5pt,
%  innerrightmargin=2pt, roundcorner=5pt
%  ]{defn}{定义}[chapter]
%  \newmdtheoremenv[
%	linecolor=black,
%	leftmargin=20pt,
%	innerleftmargin=5pt, 
%	innerrightmargin=2pt, roundcorner=5pt
%	]{thm}{定理}[chapter]
%
%	\newmdtheoremenv[
%  linecolor=black,
%  leftmargin=20pt,
%  innerleftmargin=5pt,
%  innerrightmargin=2pt, roundcorner=5pt
%  ]{yl}{引理}[chapter]
%
%  \newmdtheoremenv[
%  linecolor=black,
%  leftmargin=20pt,
%  innerleftmargin=5pt,
%  innerrightmargin=2pt, roundcorner=5pt
%  ]{tl}{推论}[chapter]
%
%
\newtheorem{example}{\bf 例题}[chapter]
\newenvironment{solution}{\noindent {\bf 解答:}}{}%{\hfill $\blacksquare$\par}


\renewcommand{\proofname}{\bf 证明:}
\newenvironment{proof}{{\noindent \bf 证明:}}{}%{\hfill $\square$\par}
\usepackage{multicol}
\newcommand{\E}{\mathbb{E}}
\renewcommand{\Pr}{\mathbb{P}}
\newcommand{\EP}{\mathbb{E}^{\mathbb{P}}}
\newcommand{\EQ}{\mathbb{E}^{\mathbb{Q}}}
\newcommand{\dif}{\,{\rm d}}
\newcommand{\Var}{{\rm Var}}
\newcommand{\Cov}{{\rm Cov}}



 \usepackage{tcolorbox}
 \tcbuselibrary{breakable}
 \tcbuselibrary{most}

% \tcolorboxenvironment{tl}{colback = cyan!5!white, colframe = cyan!75!black,
%     colbacktitle = cyan!85!black, enhanced,  breakable}

% \tcolorboxenvironment{thm}{colback = magenta!5!white, colframe = magenta!75!black, 
%     colbacktitle = magenta!85!black, enhanced,     breakable}

% \tcolorboxenvironment{yl}{colback = red!5!white, colframe = red!75!black, 
%     colbacktitle = red!85!black, enhanced,     breakable}

% \tcolorboxenvironment{defn}{colback = blue!5!white, colframe = blue!75!black,
%     colbacktitle = blue!85!black, enhanced,    breakable
% }


%\tcolorboxenvironment{tl}{colback = cyan!5!white, colframe = cyan!75!black, fonttitle = \bfseries,
%    colbacktitle = cyan!85!black, enhanced,
%    attach boxed title to top center={yshift=-2mm}, breakable}
\setcounter{tocdepth}{1}

\setcounter{secnumdepth}{3}



\ctexset {
section = {
	name = {第,节},
	number = \chinese{section}},
subsection = {
	name = {\hspace{2em},、\hspace{-1em}},
	number = \chinese{subsection}
},
subsubsection = {
	name = {\hspace{2em}(,)\hspace{-1em}},
	number = \chinese{subsubsection},
}
}


%\usepackage{fancyhdr}
%
%% \usepackage{enumitem}   [itemindent=1em]
%
%
\renewcommand{\contentsname}{目~~录}

\usepackage{paralist}
\let\itemize\compactitem
%\let\enditemize\endcompactitem
\let\enumerate\compactenum
%\let\endenumerate\endcompactenum
\let\description\compactdesc
%\let\enddescription\endcompactdesc


\usepackage{titlesec}
\titlespacing{\chapter}{0pt}{*1}{*1}
\titlespacing{\section}{0pt}{*1}{*1}
\titlespacing{\subsection}{0pt}{*1}{*1}

\titlespacing{\subsubsection}{0pt}{*1}{*1}

\renewcommand{\le}{\leqslant}
\renewcommand{\ge}{\geqslant}
\usepackage{mathtools}

\setlength{\abovecaptionskip}{0.cm}
\setlength{\belowcaptionskip}{-0.cm}

%\setmainfont{Times New Roman}
\usetikzlibrary{decorations.pathmorphing, patterns}
\usetikzlibrary{calc, patterns, decorations.markings}
\usetikzlibrary{positioning, snakes, arrows.meta}

%\cover{cover.pdf}

\usepackage{yhmath}
\newcommand{\ms}{\text{m}/\text{s}}
\newcommand{\cms}{\text{cm}/\text{s}}
\newcommand{\msq}{\text{m}/\text{s}^2}
\newcommand{\cmsq}{\text{cm}/\text{s}^2}
\newcommand{\kmh}{\text{km}/\text{h}}
\newcommand{\NC}{\text{N}/\text{C}}


\newcommand{\atom}[3]{{}^{#2}_{#3}{\rm #1}}






\newcommand{\x}{\times}


%
\begin{document}
%\fontsize{10.5}{11}\selectfont



\title{\Huge\bfseries 高中物理(甲种本)\\
第三册\vspace*{2cm} }

\author{\Large 张同恂\and \Large 方玉珍\and \Large 马淑美}
\date{\Large 1985年11月\\~\\~\\ \includegraphics[scale=1.8]{fig/front.pdf}}

\maketitle

\tableofcontents


\frontmatter

\input{preface.tex}

\mainmatter

\chapter{磁场}
\section{磁场}
我们在初中学过,把一根磁铁放在另一根磁铁的附近,两根磁铁的碰极之间会产生相互作用的磁力:同名磁极互相推斥,异名磁极互相吸引.我们知道,两个电荷之间相互作用的电力,不是在电荷之间直接发生的,而是通过电场传递的.同样,磁极之间相互作用的磁力,也不是在磁极之间直接发生的,而是通过磁场传递的.磁极在周围的空间里产生磁场,磁场对处在它里面的磁极有磁场力的作用.

磁铁并不是磁场的唯一来源,1820年丹麦物理学家奥斯特(1777—1851)做过下面的实验:把一条导线平行地放在磁针的上方,给导线通电,磁针就发生偏转(图1.1),这说明
不仅磁铁能产生磁场,电流也能产生磁场,电和磁是有密切联系的.
\begin{figure}[htp]
\centering
\begin{minipage}[t]{0.48\textwidth}
\centering
\includegraphics{fig/1-1.pdf}
\caption{奥斯特实验}
\end{minipage}
\begin{minipage}[t]{0.48\textwidth}
\centering
\includegraphics{fig/1-2.pdf}
\caption{磁场对电流发生作用}
\end{minipage}
\end{figure}

电流能产生磁场,反过来,磁场会不会对电流产生磁场力的作用呢?我们在初中做过的图1.2所示的实验回答了这个问题.把一段直导线放在磁铁的磁场里,当导线中通过电流时,可以看到导线因受力而发生运动,这个实验使我们进一步知道电和磁的联系,磁场不仅对磁极产生磁场力的作用,对电流也产生磁场力的作用,这是一个重要实验,后面我们常要提到它.

\begin{figure}[htp]\centering
\includegraphics{fig/1-3.pdf}
\caption{电流之间通过磁场发生相互作用}
\end{figure}

实验表明:电流和电流之间也会通过磁场发生相互作用.图1.3是两条平行的直导线,当通以相同方向的电流时,它们相互吸引;当通以相反方向的电流时,它们相互推斥,这时每个电流都处在另一个电流产生的磁场里,因而受到磁场力的作用,这就是说,电流和电流之间,就象磁极和磁极之间一样,也要通过磁场而发生相互作用.

磁场跟电场一样,是一种物质,磁极或电流在自己周围
的空间里会产生磁场,而磁场的基本特性就是对处在它里面的磁极或电流有磁场力的作用,这样,我们对磁极和磁极之间、磁极和电流之间、电流和电流之间的相互作用获得了统一的认识,所有这些相互作用都是通过同一种场——磁场来传递的.

\section{磁场的方向~~磁力线}
把小磁针放在磁极或电流磁场中的任一点,我们看到小磁针因受磁场力的作用,它的两极静止时不再指向南北方向,而指向一个别的方向,在磁场中的不同点,小磁针静止时指的方向一般并不相同,这个事实说明,磁场是有方向性的,我们规定,在磁场中的任一点,小磁针北极受力的方向,亦即小磁针静止时北极所指的方向,就是那一点的磁场方向.

正象在电场中可以利用电力线来形象地描写各点的电场方向一样,在磁场中可以利用磁力线来形象地描写各点的磁场方向,所谓磁力线,是在磁场中画出的一些有方向的曲线,在这些曲线上,每一点的切线方向都跟该点的磁场方向一致(图1.4).
\begin{figure}[htp]\centering
\includegraphics{fig/1-4.png}
\caption{磁力线}
\end{figure}

实验上常用铁屑在磁场中被磁化的性质,来显示磁力线的形状,在磁场中放一块玻璃板,在玻璃板,上均匀地撒一层细铁屑,细铁屑在磁场里被磁化成“小磁针”,轻敲玻璃板使
铁屑能在磁场作用下转动,铁屑静止时有规则地排列起来,就
显示出磁力线的形状.

\begin{figure}[htp]\centering
\includegraphics{fig/1-5.pdf}
\caption{磁铁磁场的磁力线分布}
\end{figure}

图1.5是条形磁铁的磁力线分布情况,磁铁外部的磁力线是从磁铁的北极出来,进入磁铁的南极.

\begin{figure}[htp]
\centering
\begin{minipage}[t]{0.48\textwidth}
\centering
\includegraphics{fig/1-6-1.pdf}
\caption*{甲:磁力线分布}
\end{minipage}
\begin{minipage}[t]{0.48\textwidth}
\centering
\includegraphics{fig/1-6-2.pdf}
\caption*{乙:安培定则}
\end{minipage}
\caption{直线电流的磁场}
\end{figure}

图1.6是直线电流的磁场.直线电流磁场的磁力线,是
一些以导线上各点为四心的同心圆,这些同心圆都在跟导线垂直的平面上,实验表明,改变电流的方向,各点的磁场方向都变成相反的方向,即磁力线的方向随着改变.直线电流的方向跟它的磁力线方向之间的关系可以用\textbf{安培定则}(也叫右
手螺旋定则)来判定:用右手握住导线,让伸直的大拇指所指的方向跟电流的方向一致,弯曲的四指所指的方向就是磁力线的环绕方向.

\begin{figure}[htp]
\centering
\begin{minipage}[t]{0.48\textwidth}
\centering
\includegraphics[scale=1.2]{fig/1-7-1.pdf}
\caption*{甲:磁力线分布}
\end{minipage}
\begin{minipage}[t]{0.48\textwidth}
\centering
\includegraphics{fig/1-7-2.pdf}
\caption*{乙:安培定则}
\end{minipage}
\caption{环形电流的磁场}
\end{figure}

图1.7是环形电流的磁场,环形电流磁场的磁力线,是
一些围绕环形导线的闭合曲线.在环形导线的中心轴线上,
磁力线和环形导线的平面垂直,环形电流的方向跟它的磁力线方向之间的关系,也可以用\textbf{安培定则}来判定:让右手弯曲的四指和环形电流的方向一致,伸直的大拇指所指的方向就是环形导线中心轴线上磁力线的方向.

\begin{figure}[htp]\centering
\includegraphics[scale=1.3]{fig/1-8.pdf}
\caption{通电螺线管的磁场}
\end{figure}

图1.8是通电螺线管的磁场.螺线管通电以后表现出来的磁性,很象是一根条形磁铁,一端相当于北极,另一端相当
于南极.改变电流的方向,它的南北极就对调,通电螺线管外部的磁力线和条形磁铁外部的磁力线相似,也是从北极出
来,进入南极的,通电螺线管内部具有磁场,内部的磁力线跟螺线管的轴线平行,方向由南极指向北极,并和外部的磁力线连接,形成一些闭合曲线,通电螺线管的电流方向跟它的磁力线方向之间的关系,也可用安培定则来判定:用右手握住螺线管,让弯曲的四指所指的方向跟电流的方向一致,大拇指所指的方向就是螺线管内部磁力线的方向,也就是说,大拇指指向通电螺线管的北极.

\section{磁现象的电本质~~磁性材料}
\subsection{磁现象的电本质}


磁极和电流同样能够产生磁场,磁场对磁极和电流同样有磁场力的作用.通电螺线管和条形磁铁又那么相似,这些现象使我们想到:磁极的磁场和电流的磁场是不是有相同的起源?这个问题现在已经有了明确的回答,这个相同的起源就是电荷的运动.

导体中的电流是由电荷的运动形成的,因而我们不难理解通电导线的磁场是由电荷的运动产生的.那么,能不能进一步用实验直接证实:原来静止的电荷,当它运动起来的时
候就会产生磁场呢?这个问题早在一百多年以前就提出来了.1876年美国的罗兰用实验证实了这一点,罗兰把大量的电荷加在一个橡胶圆盘上,然后使盘绕中心轴高速转动,在盘的附近用小磁针来检验运动电荷产生的磁场(图1.9),结果他发现:当带电盘转动时,小磁针果然发生了偏转,而且改变盘的转动方向或者改变所带电荷的正负时,小磁针的偏转方向电改变,磁针的偏转方向跟运动电荷所形成的电流方向间的关系同样符合安培定则.这个实验证明了运动电荷确实产生磁场,进一步揭示了磁现象的电本质.
\begin{figure}[htp]\centering
\includegraphics{fig/1-9.png}
\caption{罗兰实验的示意图}
\end{figure}

磁铁的磁场是否也是由电荷的运动产生的呢?法国科学家安培(1775—1836),从奥斯特实验得到启示,提出了著名的分子电流的假说.他认为:在原子、分子等物质微粒内部存
在着一种环形电流,叫做分子电流,分子电流使每一个物质微粒都成为一个微小的磁体,它的两侧相当于两个磁极(图1.10),这两个磁极跟分子电流不可分割地联系在一起.

\begin{figure}[htp]\centering
\includegraphics{fig/1-10.pdf}
\caption{}
\end{figure}

安培的假说能够解释各种磁现象,一根软铁棒,在未被磁化的时候,内部各分子电流的取向是杂乱无章的(图1.11甲),它们的磁场互相抵消,对外界不显磁性,当软铁棒受到外界磁场的作用时,各分子电流的取向变得大致相同(图1.11乙),软铁棒就被磁化了,两端对外界显示出较强的磁作用,形成磁极.

\begin{figure}[htp]\centering
\includegraphics{fig/1-11.pdf}
\caption{}
\end{figure}

磁体受到高温或猛烈的敲击会失去磁性,这是因为在激烈的热运动或机械运动的影响下,分子电流的取向又变得杂乱了.

在安培所处的时代,人们对物质内部为什么会有分子电流还不清楚,直到二十世纪初,人类了解了原子的结构,才知道分子电流是由原子内部电子的运动形成的,这样看来磁极的磁场和电流的磁场,它们的来源相同,都来源于电荷的运动,
\textit{运动的电荷(电流)产生磁场,磁场对运动的电荷(电流)有磁场力的作用,所有的磁现象都可以归结为运动电荷(电流)之间通过磁场而发生的相互作用},这就是磁现象的电本质.

\subsection{磁性材料}

实验表明,任何物质在磁场中都能够或多或少地被磁化,只是磁化的程度不同,象铁、钴、镍那样能够被强烈磁化的物质,叫做铁磁性材料,磁化后的铁磁性物质,它们的磁性并不因外磁场的消失而完全消失,仍然剩余一部分磁性,叫做剩磁.

铁磁性物质按剩磁的情形分为软磁性材料和硬磁性材料,软磁性材料的剩磁弱,而且容易退磁,软磁性材料适用于需要反复磁化的场合,可以用来制造变压器、交流发电机、电磁铁和各种高频电磁元件的铁心,软铁、硅钢、坡莫合金(镍铁合金)等是软磁性材料,硬磁性材料的剩磁强,而且不易退磁,适合于制成永久破铁,应用在磁电式仪表、扬声器、话筒、永磁电机等电器设备中,常见的金属硬磁性材料有碳钢、钨钢、铝镍钴的合金等.

还有一种磁性材料,叫做铁氧体,它是由氧化铁和二价金属(如Ni, Co, Mn, Mg等)的氧化物组成的,在电性能上与半导体相似,在磁性上与铁磁性材料相似,铁氧体在电子技术中已经成为不可缺少的磁性材料.在电子计算机中利用铁氧体作记忆元件,在电子线路中广泛利用铁氧体作电感线圈的磁心.

\subsection*{练习一}
\begin{enumerate}
    \item 磁体的北极在磁场中所受的磁场力跟磁场方向同向,南极所受的磁场力跟磁场方向反向.图1.12是放在磁场中的小磁针,试根据小磁针所受的力说明,它将怎样转动以及静止在哪个方向.
\begin{figure}[htp]
\centering
\begin{minipage}[t]{0.48\textwidth}
\centering
\begin{tikzpicture}[>=latex, scale=.6]
\foreach \x in {-.5,.5,...,4.5}
{
   \draw [->](-2,\x)--(5,\x);
}

\draw [rotate=60, fill=red](2.5,-.4)--(2.5,.4)--(5,0);
\draw [rotate=60, fill=blue](2.5,-.4)--(2.5,.4)--(0,0);

\draw [rotate=60](0,0)--(2.5,-.4)--(5,0)--(2.5,.4)--(0,0);
\draw [rotate=60](2.5,-.4)--(2.5,.4);

\node at (0,-.3){$S$};
\node at (2.8,4){$N$};

\node at  (5.5,2.2){$B$};
\end{tikzpicture}
\caption{}
\end{minipage}
\begin{minipage}[t]{0.48\textwidth}
\centering
\includegraphics[scale=1]{fig/1-13.PDF}
\caption{}
\end{minipage}
\end{figure}
    \item 在图1.13中,当电流通过导线时,导线下面的磁针北极转向读者.试判断AB中电流的方向.
    \item 在图1.14中,当电流通过线圈时,磁针的南极指向读者.试确定线圈中电流的方向.
\begin{figure}[htp]
\centering
\begin{minipage}[t]{0.48\textwidth}
\centering
\includegraphics[scale=1]{fig/1-14.pdf}
\caption{}
\end{minipage}
\begin{minipage}[t]{0.48\textwidth}
\centering
\includegraphics[scale=.6]{fig/1-15.png}
\caption{}
\end{minipage}
\end{figure}

    \item 试确定图1.15中电源的正极和负板.
    \item 离开你向前运动的质子流产生的磁场是怎样的?向着你运动的电子流产生的磁场又是怎样的?
\end{enumerate}

\section{磁感应强度}
磁场不仅有方向性,而且有强弱的不同,巨大的电磁铁能够吸起成吨的钢铁,小的磁铁只能吸起小铁钉,我们怎样来表示磁场的强弱呢?

电场的基本特性是对其中的电荷有电场力的作用,研究电场强弱的时候,我们从分析电荷在电场中的受力情况着手,得出电场强度这个物理量来表示电场的强弱,磁场的基本特性是对其中的电流有磁场力的作用,研究磁场的强弱,我们可以从分析电流在磁场中的受力情况着手,找出表示磁场的强弱的物理量,为此,我们需要把一小段通电导线放在磁场中的某处,来研究它的受力情况.

把一小段通电导线放在磁场中的某处,我们发现,当导线方向跟该处的磁场方向一致时,通电导线所受的力最小,等于
零.当导线方向跟该处的磁场方向垂直时,所受的力最大,当导线方向跟磁场方向斜交时,所受的力介于零和最大值之间.下面,为了确定起见,我们总是把一小段通电导线垂直放入磁场,也就是放在与该处磁场方向垂直的方向上.

垂直放入磁场的通电导线所受的磁场力不仅跟其中的电流强度有关,而且跟导线的长短有关.如图1.2那样,把一段通电导线垂直地放入磁场中,实验指出:导线长度一定时,电流强度$I$越大,导线受到的磁场力$F$也越大;电流强度一定时,导线$\ell$越长,导线受到的磁场力$F$也越大.精确的实验表明:通电导线受到的磁场力$F$跟通过的电流强度$I$和导线的长度$\ell$成正比,或者说,$F$跟乘积$I\ell$成正比,这就是说,把通电导线垂直放入磁场中的某处,无论怎样改变电流强度$I$和
导线长度$\ell$,乘积$I\ell$增大多少倍,$F$也增大多少倍,比值$P/I\ell$跟乘积$I\ell$无关,是一个恒量,在磁场中不同的地方,这个比值可以是不同的值,这个比值越大的地方,表示一定长度的通电导线受到的磁场力越大,即那里的磁场越强,因此我们可以用这个比值来表示磁场的强弱.

\textit{在磁场中垂直于磁场方向的通电导线,所受的磁场力$F$跟电流强度$I$和导线长度$\ell$的乘积$I\ell$的比值叫做通电导线所在处的磁感应强度}\footnote{这个物理量所以叫做磁感应强度,而没有叫做磁场强度,是由于历史上磁场强度一词已用来表示另外一个物理量.}.如果用$B$表示磁感应强度,那么,
\[B=\frac{F}{I\ell}\]

磁感应强度是一个矢量,它的大小如上式所示,它的方向
就是该点的磁场方向.磁感应强度$B$的单位由$F$、$I$和$\ell$的单位决定,在国际单位制中,磁感应强度的单位是特斯拉,简称特,国际符号是T.1米长的导线,通过1安的电流,受到的磁场力为1牛时,磁感应强度就是1特.
\[1{\rm T}=1\frac{\rm N}{\rm A\cdot m}\]

一般永磁铁的磁极附近的磁感应强度大约是0.4—0.7特,在电机和变压器的铁心中,磁感应强度可达0.8—1.4特,通过超导材料的强电流的磁感应强度可高达1000特,而地面附近地磁场的磁感应强度大约只有$0.5\x 10^{-4}$特.

正象用电力线的疏密程度可以形象地表示电场强度的大小一样,用磁力线的硫密程度也可以形象地表示磁感应强度的大小,在磁感应强度大的地方磁力线密一些,在磁感应强度小的地方磁力线稀一些.

如果在磁场的某一区域里,磁感应强度的大小和方向都相同,这个区域就叫做\textbf{匀强磁场},匀强磁场的磁力线,方向相同,疏密程度也一样,是一些分布均匀的平行直线.

匀强磁场是最简单但又是很重要的磁场,在电磁仪器和科学实验中常常要用到它,通电长螺线管内部的磁场,距离相当近的两个平行的异名磁极间的磁场,都是匀强磁场(图1.16).

\begin{figure}[htp]\centering
\includegraphics{fig/1-16.pdf}
\caption{永磁铁间的匀强磁场}
\end{figure}

\section{磁通量}
在电学和电工学里常常要讨论穿过某一个面的磁场.为此需要引入一个新的物理量——磁通量,在下一章里就要用到它,设在匀强磁场中有一个与磁场方向垂直的平面(图1.17),磁场的磁感应强度为$B$,平面的面积为$S$.我们定义磁感应强度$B$与面积$S$的乘积,叫做穿过这个面的磁通量(简称磁通),如果用中表示磁通量,那么
\[\phi=BS \]

磁通量的意义也可以用磁力线形象地加以说明,我们知道,磁力线越密的地方,也就是穿过单位面积的磁力线条数越多的地方,磁感应强度$B$越大.因此,$B$越大,$S$越大,穿过这个面的磁力线条数就越多,磁通量所表示的就是穿过磁场中某个面的磁力线条数.

\begin{figure}[htp]
\centering
\begin{minipage}[t]{0.48\textwidth}
\centering
\includegraphics{fig/1-17.pdf}
\caption{}
\end{minipage}
\begin{minipage}[t]{0.48\textwidth}
\centering
\includegraphics{fig/1-18.pdf}
\caption{}
\end{minipage}
\end{figure}


当平面$S$不跟磁场方向垂直时(图1.18),穿过这个面的磁力线条数比垂直时少,因此磁通量也小,设平面$S$在垂直于磁力线方向上的投影为$S_n$,从图中可以看出,穿过平面$S$
的磁力线条数等于穿过投影平面$S_n$的磁力线条数,所以穿过平面$S$的磁通量
\[\phi=BS_n=BS\cos\theta\]
如果平面跟磁场方向平行,则没有磁力线穿过这个面,这时$\theta=90^\circ$,$\cos\theta=0$,穿过这个面的磁通量为零.

在国际单位制中,磁通量的单位是\textbf{韦伯},简称韦,国际符号是Wb.在磁感应强度是1特的匀强磁场里,穿过跟磁场方向垂直的面积是1${\rm m^2}$的平面的磁通量,就是1韦.
\[1{\rm Wb}=1{\rm T\cdot m^2}\]
引入了磁通量这个概念,反过来我们也可以把磁感应强
度看作是通过单位面积的磁通量,因此磁感应强度也常叫做\textbf{磁通密度},并且用${\rm Wb}/{\rm m^2}$作单位.

\subsection*{练习ニ}
\begin{enumerate}
    \item 有人根据$B=F/I\ell$提出:磁场中某处的磁感应强度$B$跟磁场力$F$成正比,跟电流强度$I$和导线长度$\ell$的乘积$I\ell$成反比.这种提法有什么问题?错在哪里?
    \item 能不能用一小段通电导线在磁场中所受磁场力的方向来定义磁感应强度的方向?讨论一下这个问题.
    \item 长10厘米的导线,放入匀强磁场中,它的方向和磁场的方向垂直,导线中的电流强度是3.0安,受到的磁场力是$1.5\x10^{-3}$牛.求该处的磁感应强度.
    \item 把矩形线圈$abcd$放在匀强磁场中(图1.19),线圈面积为$5\x10^{-2}{\rm m^2}$,磁感应强度为$2\x10^{-3}{\rm T}$.
\begin{figure}
	\centering
	\begin{tikzpicture}[>=latex]
	\foreach \x in{1,2,...,6}
		\foreach \y in {1,2,3}
		{
		   \node at (\x,\y) {$\times$};
		}
	\draw [thick](2.5,1.5) rectangle (4.5,2.5);
\draw [dashed](3.5,0.2)node[below] {$O'$}--(3.5,3.8)node[above] {$O$};
	\node at (2.3, 2.7) {$a$};	\node at (2.3, 1.3) {$c$};
	\node at (4.7, 2.7) {$b$};	\node at (4.7, 1.3) {$d$};

\draw[->] (3.5+.25, 3.5) arc [start angle=10, end angle=360, x radius=.25,y radius=.125];

	\end{tikzpicture}
	\caption{}
\end{figure}
     \begin{enumerate}
        \item 当线圈平面与磁场方向垂直时,穿过线圈的磁通量是多少?
        \item 线圈平面从图所示的位置绕$OO'$轴转过$60^\circ$时,穿过线圈的磁通量是多少?
    \end{enumerate}
    \item 通电螺线管内部的磁感应强度大,还是管口外部的磁感应强度大?你是根据什么判断的?
\end{enumerate}



\section{直线电流的磁场}
前面我们已经介绍了直线电流磁场的磁力线分布(图1.6),从磁力线的分布就可以知道磁场中各点的磁感应强度的方向.那么,直线电流磁场中各点的磁感应强度的大小又是怎样的呢?实验表明,磁感应强度的大小跟导线中的电流强度$I$有关系,也跟离开导线的距离$r$有关系.
\begin{figure}[htp]\centering
\begin{tikzpicture}[>=latex, xscale=.6, yscale=.4]
\draw (-2,2)--(2,2)--(2.2, -2)--(-2.2,-2)--(-2,2);
\fill[white](-.1,0) rectangle (.1,5);
\draw (-0.1,0)--(-0.1,5); \draw (0.1,0)--(0.1,5);
\draw (-0.1,-2)--(-0.1,-3);
\draw (0.1,-2)--(0.1,-3);
\draw[->] (-.5, 3)--(-.5,4)node[above]{$I$};
\end{tikzpicture}
	\caption{ }
\end{figure}

如图1.20所示,在垂直于
长直导线的平面上,放一个可以自由转动的小磁针,先让小磁针的位置不变,改变电路里的电流强度,来研究磁感应强度跟电流强度的关系.实验指出,电流强度越大,小磁针的偏转角度越大,这说明电流强度越大,磁感应强度就越大.再让导线中的电流强度不变,改变小磁针的位置,来研究磁感应强度跟离开导线的距离的关系.实验指出,小磁针离导线越近,偏转角度越大.这说明离导线越近,磁感应强度就越大.

可以证明:直线电流磁场的磁感应强度$B$的大小跟电流强度$I$成正比,跟离开导线的距离$r$成反比,写成公式就是
\[B=k\frac{I}{r}\]
其中$k$是比例恒量.

严格说来,上式只适用于无限长的直线电流,但是实际上,当直导线的长度远大于离开导线的距离$r$时,除了导线两端附近,上式都是适用的.

环形电流磁场和通电螺线管磁场的磁感应强度的大小也可以计算出来,它们的计算公式虽然跟直线电流不同,但有一点是相同的,即磁感应强度的大小都跟电流强度成正比,电流强度越大,磁感应强度也越大.

\section{磁场对电流的作用力}
前而我们研究磁感应强度的时候,对电流所受的磁场力
已经作过一些讨论,电流所受的磁场力通常叫做\textbf{安培力}.这一节讨论安培力的大小和方向.

\subsection{安培力的大小}

把一小段通电导线垂直放入磁场中,我们根据通电导线受的力$F$、导线中的电流强度$I$和导线长度$\ell$定义了磁感应强度$B=F/I\ell$把这个公式变形,就得到电流所受的安培力的公式:
\[F=I\ell B\]

严格说来,这个公式只适用于一小段通电导线的情形;导
线较长时,导线所在处各点的磁感应强度矢量一般并不相同,不能应用这个公式.如果磁场是匀强磁场,这个公式就适用于长的通电导线了.
\begin{figure}[htp]\centering
\begin{tikzpicture}[>=latex]

\draw (0,6)--(0,0);
\draw (.2,6)--(.2,0);
\draw[->] (.5,5.5)--(.5,5)node[right]{$I$};
\foreach \x in {0,1,...,4}
{
    \draw[->](-4,6-\x*.7)--(1,3-\x*.7);
}
\node at (1.5, 2){$B$};
\draw (0,2.9-.5) arc (-90:-30:.5)node[below]{$\theta$};
\draw[->, very thick] (-3,5.3-.6)--(-2,5.3-.6)node[right]{$B_2$};
\draw[->, very thick] (-3,5.3-.6)--(-3,5.3-.6-.6)node[below]{$B_1$};
\draw[->, very thick] (-3,5.3-.6)--(-2,5.3-.6-.6)node[right]{$B$};
\draw[dashed] (-3,5.3-.6-.6)--(-2,5.3-.6-.6);
\draw[dashed] (-2,5.3-.6)--(-2,5.3-.6-.6);
\draw (-3,5.3-.6-.2) arc (-90:-30:.2)node[below]{$\theta$};

\end{tikzpicture}
	\caption{ }
\end{figure}

如果电流方向不跟磁场方向垂直,电流受到的安培力又怎样呢?我们知道,电流方向跟磁场方向垂直时,电流受的力最大,其值可由公式$F=I\ell B$计算出;电流方向跟磁场方向平行时,电流根本不受力,即所受的力为零,知道了电流在这两种特殊情况下所受的力,我们不难求出电流在磁场中任意方向上所受的力.如图1.21所示,当电流方向与磁场方向间有一个夹角$\theta$时,我们可以把磁
感应强度矢量$B$分解为两个分量:一个是跟电流方向平行的分量$B_1=B\cos\theta$,另一个是跟电流方向垂直的分量$B_2=B\sin\theta$.前者对电流没有作用力,电流受到的力完全是由后者决定的,即$F=I\ell B_2$.代入$B2=B\sin\theta$,我们得到
\[F=I\ell B\sin\theta\]
这就是电流方向与磁场方向成某一角度时安培力的公式.这就是说,\textit{安培力的大小等于电流强度$I$、导线长度$\ell$、磁感应强度$B$以及$I$和$B$间的夹角$\theta$的正弦$\sin\theta$的乘积}.从上式可以看出,当$\theta =90^{\circ}$时,安培力最大,等于$I\ell B$;电流方向越偏离与磁场相垂直的方向,即$\theta$越小,安培力也越小;当$\theta =0$时,安培力最小,等于零.

在国际单位制中,上式中的各个物理量分别用牛顿、安培、米、特斯拉作单位.

\subsection{安培力的方向}


上述公式给出了安培力的大小,安培力的方向是怎样的呢?怎样判定安培力的方向,同学们在初中已经学过,这里再讨论一下,在图1.2所示的实验中,如果把磁铁的两极调换位置来改变磁场方向,或者不改变磁场方向而改变电流方向,导线就向着相反的方向运动,可见通电导线在磁场中的受力方向跟磁场方向,导线中的电流方向都有关系.实验表明:电流所受安培力的方向既跟磁场方向垂直,又跟电流方向垂直;也就是说,安培力的方向总是垂直于磁力线和通电导线所在的平面.

在图1.2的实验中,磁场方向跟电流方向也是垂直的,这时可以用左手定则来判定安培力的方向(图1.22):伸开左手,使大拇指跟其余四个手指垂直,并且都跟手掌在一个平面内,让磁力线垂直进入手心,
并使四指指向电流方向,这时手掌所在的平面跟磁力线和导线所在的平面垂直,拇指所指的方向就是通电导线在磁场中的受力方向.
\begin{figure}[htp]\centering
\includegraphics[scale=1.2]{fig/1-22.pdf}
\caption{}
\end{figure}

我们再来看一下磁场方向不跟电流方向垂直的情形.把图1.2装置中的导线在竖直平面上转过一个角度,使磁场方向不再跟电流方向垂直,再进行观察.结果发现,导线受力的方向并没有改变,只是所受的力减小了.因此,我们仍旧可以用左手定则来判定力的方向,只是这时磁力线是倾斜进入手心的.

\section*{阅读材料:电流强度的单位——安培}
在国际单位制中,基本单位有七个,我们已经知道其中的六个,这就是:米(长度的单位),千克(质量的单位),秒(时间的单位),安培(电流强度的单位),开尔文(热力学温度的单位),摩尔(物质的量的单位).还有一个基本单位叫做坎德拉,是光学中发光强度这个物理量的单位.这个单位在中学课程中涉及不到.这里我们介绍一下安培这个单位是怎样确定的.

以前我们是用电量的单位库仑来确定安培这个单位的,似乎库仑是一个基本单位.其实在国际单位制中库仑不是基本单位,而是导出单位,安培才是基本单位.以前所以那样讲,是因为我们的知识不够,不能先讲安培这个单位.安培是利用电流之间的相互作用力来确定的,学过直线电流的磁场
和磁场对电流的作用力,现在我们有条件来介绍这个问题了.
\begin{figure}[htp]
\centering
\begin{minipage}[t]{0.48\textwidth}
\centering
\begin{tikzpicture}[>=latex]
\foreach \x in {0,.2,1,1.2}
{
   \draw (\x, 0)--(\x, 4);
}

\draw[dashed] (1.1,2) ellipse [x radius = 1, y radius =.5];
\node at (0,-.5){$b$};\node at (1,-.5){$a$};
\draw [->](-.25, 3)--(-.25, 3.5)node[above]{$I$};
\draw[->] (.75, 3)--(.75, 3.5)node[above]{$I$};
\draw[->, thick] (.1,2)--(.8,2)node[above]{$F_1$};
\end{tikzpicture}
\caption*{甲}
\end{minipage}
\begin{minipage}[t]{0.48\textwidth}
\centering
\begin{tikzpicture}[>=latex]
\foreach \x in {0,.2,1,1.2}
{
   \draw (\x, 0)--(\x, 4);
}

\draw[dashed] (.1,2) ellipse [x radius = 1, y radius =.5];
\node at (0,-.5){$b$};\node at (1,-.5){$a$};
\draw [->](-.25, 3)--(-.25, 3.5)node[above]{$I$};
\draw[->] (.75, 3)--(.75, 3.5)node[above]{$I$};
\draw[->,thick] (1.1,2)--(.4,2)node[above]{$F_2$};
\end{tikzpicture}
\caption*{乙}
\end{minipage}
\caption{}
\end{figure}

图1.23表示两根平行的直导线,其中通以大小和方向都相同的电流$I$,导线$b$处于导线$a$的磁场中,因而受到这个
磁场的安培力$F_1$,从左手定则可以知道,$F_1$在纸面内自左向右,指向导线$a$,如图甲所示.设两导线的长度为$\ell$,两导线间的距离为$r$,导线$a$在导线$b$处产生的磁感应强度$B=kI/r$,导线$b$所受的安培力$F_1=I\ell B=kI^2\ell/r$.同样,导线$a$处于导线$b$的磁场中,受到的安培力$F_2$在纸面内自右向左,指向导线$b$,如图乙所示.同样可以算出$F_2$的大小:$F_2=kI^2\ell/r$.可见两导线受到大小相等方向相反的引力,这个引力的大小$F$为
\[F=k\frac{I^2\ell}{r}\]
其中$k$是比例恒量,它的数值跟选用的单位制以及导线周围的介质有关.

在国际单位制中,安培这个单位就是利用上式来定义的.放在真空中的两根平行直导线,通以相同的稳恒电流,当两导线相距1米,每根导线在每米长度上所受的力等于$2\x10^{-7}$牛顿时,这个稳恒电流就是1安培.

根据这个定义,由上式得到
\[ 2\x10^{-7} {\rm N}=k\frac{1{\rm A^2}\x {\rm m}}{\rm m}\]
所以
\[k=2\x10^{-7}{{\rm N}/{\rm A^2}}\]

有了安培的定义,根据公式$I=q/t$就可以定义电量的单位,如果导线中的电流为1安培,那么每秒内通过导线横截面的电量就是1库仑.所以$1{\rm C}=1{\rm A}\cdot {\rm s}$.

\subsection*{练习三}
\begin{enumerate}
    \item 图1.24表示一根放在磁场里的通电直导线,图中已分别标明电流、磁感应强度和安培力这三个量中两个的方向,试画出第三个量的方向.
    \begin{figure}[htp]\centering
\begin{tikzpicture}[>=latex, scale=.8]
		\foreach \x in {1,2,3,4}
		{
			\draw[->] (0,\x)--(4,\x);
		}
		\draw[->](2,2.5)--(2,3.5)node[right]{$F$};
		\node at (4.5,2.5){$B$};
		\draw [fill=white](2,2.5) circle (5pt);
		
		\draw[->](6.5,2.5)node[left]{$I$}--(7.5,2.5)node[above]{$F$};
		\draw [fill=white](6.5,2.5) circle (5pt);
		\node at (6.5,2.5){$\times$};
		
	\foreach \y in {1,2,3,4}
	{
		\draw[->] (9.5,\y)--(13.5,\y);
	}	
\draw [fill=white](11.5,2.5) circle (5pt);
\node at (11.5,2.5){$\times$};
		\node at (14,2.5){$B$};\node at (11.1,2.5){$I$};
	\end{tikzpicture}
    	\caption{ }
    \end{figure}
    \item 试解释为什么两根平行直导线中通以相反方向电流时,它们互相推斥.
    \item 如图1.25所示,把一根通电的直导线放在蹄形磁铁的两个磁极上方.导钱可以自由移动和转动,如果电流的方向如图所示,导线将产生怎样的运动?
        \begin{figure}[htp]\centering
    	\includegraphics[scale=.75]{fig/1-25.png}
    	\caption{ }
    \end{figure}
    \item 把30厘米的通电直导线放入匀强磁场中,导线中的电流强度是2.0安,磁场的磁感应强度是1.2特.求电流方向跟磁场方向垂直时导线所受的安培力.

\begin{figure}[htp]
\centering
\begin{minipage}[t]{0.48\textwidth}
\centering
\begin{tikzpicture}[>=latex, scale=.7]
\foreach \x in {-.5,.5,...,4.5}
{
   \draw [->](-2,\x)--(5,\x);
}

\draw [rotate=60, thick](0,0)node [left]{$a$} rectangle (4.5,.2)node [above]{$b$};
\node at  (5.5,2.2){$B$};

\draw (.8,.5) arc (0:60:.5) node[right]{$60^{\circ}$};
\end{tikzpicture}
\caption{}
\end{minipage}
\begin{minipage}[t]{0.48\textwidth}
\centering
\begin{tikzpicture}[>=latex]
\draw (0,.5)--(0,4.5);
\draw[->] (-.5,2)--(-.5,3)node [above]{$I$};
\draw[thick] (1-.3,1) rectangle (2+.3,4);
\draw[->] (1-.3,1.5)--(1-.3,2.5)node [left]{$a$};
\draw [->](2+.3,3)--(2+.3,2.5)node [right]{$c$};
\draw[->] (1-.3,4)--(1.5,4)node [above]{$b$};
\draw [->](2+.3,1)--(1.5,1)node [below]{$d$};
\end{tikzpicture}
\caption{}
\end{minipage}
\end{figure}



    \item 在磁感应强度是$4.0\x10^{-2}$特的匀强磁场里,有一条和磁场方向相交成$60^\circ$角、长8厘米的通电直导线$ab$(图1.26).通电导线$ab$所受的安培力是$1.0\x10^{-2}$牛,方向和纸面垂直指向读者,求导线里电流的大小和方向.
    \item 一根长2米的直导线,通有1安的电流,把它放在$B=0.2$特的匀强磁场中,当导线与磁力线的夹角为$0^\circ$, $30^\circ$和$90^\circ$时,导线所受的安培力分别有多大?

    \item 如图1.27所示,在通电长导线的旁边放一个可以自由移动和转动的矩形通电线圈,线圈和导线在同一平面上,它的$a$, $c$两边和导线平行,试讨论一下线圈各边的受力情况,线圈在磁场中将怎样运动?
\end{enumerate}

\section{电流天平}
应用通电导线在磁场中受力的原理,可以制成灵敏的电
流天平.用电流天平可以测出通电导线在匀强磁场中受力的大小,从而求出磁感应强度.
    \begin{figure}[htp]\centering
	\includegraphics[scale=.75]{fig/1-28.png}
	\caption{电流天平}
\end{figure}

整个装置如图1.28甲所示,它的横臂(图1.28乙)能绕
通过点$O$和$O'$的轴自由转动,轴的左右两侧,臂长相等,在轴的一侧,沿着横臂的边沿固定一条U形绝缘导线.这样,
在天平的一端就有了一段短直导线$CD$,它的长度是$\ell$\footnote{为了可以改变$\ell$的长度,仪器的U形导线中间又加了一条引线,因此,还可以用这个仪器来研究磁场对电流的作用力跟导线的长度的关系.}.天平的另一端可以悬挂砝码或细金属丝等轻小物体.调整天平,使它平衡.把有U形导线的一端放入待测的磁场中(图1.29),然后给U形导线通电,如果磁场方向和U形导线中的电流方向如图所示,$CD$段导线就受到一个向下的安培力,天平因而倾斜,在天平的另一端加上适当的砝码,使天平恢复平衡.设待测的磁感应强度是$B$,U形导线中通过的电流强度是$I$,砝码的质量是$m$,我们就有$I\ell B=mg$,由此可求出待测的磁感应强度
\[B=\frac{mg}{I\ell}\]
    \begin{figure}[htp]\centering
	\includegraphics[scale=.75]{fig/1-29.png}
	\caption{测定螺线管的磁感应强度}
\end{figure}

\subsection*{练习四}
\begin{enumerate}
    \item 设图1.29所示的长螺线管内部$B=1.0\x10^{-2}$特.
    \begin{enumerate}
        \item 与螺线管轴线平行的一条通电导线所受的力有多大?
        \item 如果$CD$的长度是2厘米,U形导线中通过的电流强度是1.0安,导线所受的力有多大?
        \item 为了使电流天平保持平衡,在另一端要加多重的砝码?
    \end{enumerate}
    \item 你自己设计一种测量磁感应强度的仪器,并说明它的原理和测定方法.
\end{enumerate}


\section{电流表的工作原理}
这一节我们讨论常用的磁电式仪表的工作原理.磁电式仪表是利用通电线圈在磁场中发生偏转的现象制成的,下面我们先讲磁场对通电线圈的作用.

\subsection{磁场对通电线圈的作用}
    \begin{figure}[htp]\centering
	\includegraphics[scale=.75]{fig/1-30.png}
	\caption{磁场对通电线圈的作用}
\end{figure}

图1.30甲表示放在匀强磁场中的通电矩形线圈,线圈平面跟磁力线成$\theta$角.线圈顶边$da$和底边$bc$所受的安培力$F_{da}$和$F_{bc}$大小相等,方向相反,彼此
平衡,$ab$和$cd$两个侧边与磁力线垂直,它们受到的安培力$F_{ab}$和$F_{cd}$虽然大小相等,方向相反,但是它们形成力偶,使线圈绕竖直轴$OO'$转动.

现在我们来求这个力偶矩.设磁感应强度为$B$,力$F_{ab}=
F_{cd}=BI·ab$,从图1.30乙(图甲的俯视图)可以看出,力偶臂$d=ad\cdot \cos\theta$,所以力偶矩$M=BI\cdot ab\cdot ad\cdot \cos\theta$,而$ab\cdot ad$等于矩形线圈的面积$S$,所以$$M=BIS\cos\theta$$

从上式可以看出,当线圈平面跟磁力线平行时,$\theta=0$,$\cos\theta=1$,所受力偶矩最大.当线圈平面跟磁力线垂直时,$\theta=90^{\circ}$,$\cos\theta=0$,力偶矩为零,这时$F_{ab}$和$F_{cd}$彼此平衡,所以线圈会停在这个位置上.

\subsection{电流表的工作原理}
    \begin{figure}[htp]\centering
	\includegraphics[scale=.75]{fig/1-31.png}
	\caption{电流表的构造}
\end{figure}
常用的电流表的构造如图1.31所示.在很强的蹄形磁铁的两极间有一个固定的圆柱形铁心,铁心外面套一个可以绕轴转动的铝框,铝框上绕有线圈,铝框的转轴上装有两个螺旋弹簧和一个指针.线圈的两端分别接
在这两个螺旋弹簧上,被测电流就是经过这两个弹簧通入线圈的.
\begin{figure}[htp]\centering
\includegraphics[scale=1]{fig/1-32.pdf}
\caption{}
\end{figure}

蹄形磁铁和铁心间的磁场是均匀地辐向分布的(图1.32),不管通电线圈转到什么角度,它的平面都跟磁力线平行,
因此磁场使线圈偏转的力偶矩$M_1$不随偏角而改变.另一方面,线圈的偏转使弹簧扭紧或扭松,于是弹簧产生一个阻碍线圈偏转的力矩$M_2$.线圈偏转的角度越大,弹簧的力矩$M_2$也越大.到$M_1$跟$M_2$平衡时,线圈就停在某一偏角上,固定在转轴上的指针也转过同样的偏角,指到刻度盘的某一刻度.

设电流表通电线圈的匝数为$N$,则线圈受到的力偶矩$M_1=NBIS$.由于$NBS$为定值,所以$M_1$跟电流强度$I$成正比.设$k_1=NBS$,则$M_1=k_1I$.另一方面,弹簧产生的力矩$M_2$跟偏角$\theta$成正比;即$M_2=k_2\theta$,其中$k_2$是一个比例恒量.$M_1$和$M_2$平衡时,$k_1I=k_2\theta$,即$\theta =kI$,其中$k=k_1/k_2$也是一个恒量.可见,测量时指针偏转的角度跟电流强度成正比,这就是说,这种电流计的刻度是均匀的.

这种利用永久磁铁来使通电线圈偏转的仪表叫数磁电式仪表,这种仪表的优点是刻度均匀,准确度高,灵敏度高,可以测出很弱的电流;缺点是价格较贵,对过载很敏感,如果通入的电流超过允许值,就很容易把它烧掉,使用时要特别注意.

\subsection*{练习五}
\begin{enumerate}
    \item 如图1.33所示,把通电线圈放入永久磁铁的匀强磁场中.
    \begin{figure}[htp]\centering
    	\includegraphics[scale=.75]{fig/1-33.png}
    	\caption{ }
    \end{figure}
    \begin{enumerate}
        \item 图甲中,线圈怎样转动?
        \item 图乙中,由上往下看线圈是顺时针转动的,磁铁哪一边是$N$极?哪一边是$S$极?
        \item 图丙中,由上往下看线圈是反时针转动的,画出线圈中电流的方向.
    \end{enumerate}
    \item 有一个匝数为10匝的矩形线圈,长为25厘米,宽为10厘米,放在$B=1.5\x10^{-3}$特的匀强磁场中,通以1.5安的电流,求它所受的最大的力偶矩.
    \item 图1.30所示的放在磁场中的线圈,当转到线圈平面跟磁力线垂直的位置时,会不会立即停在这个位置上?为什么?定性地分析一下线圈在停下来之前的运动情况,有什么办法可以使通电线圈不停下来而继续转动?
    \item 电流表中通以相同的电流时,指针的偏转角度越大,表示电流表的灵敏度越高,定性地分析一下,有哪些因素会影响磁电式电流表的灵敏度.
\end{enumerate}


\section{磁场对运动电荷的作用力}
我们知道,磁场对电流有作用力,既然电流是电荷的运
动产生的,我们自然会想到,磁场力可能是直接作用在运动电
上的.作用在整个导线上的安培力,不过是作用在运动电荷上的力的宏观表现.
\begin{figure}[htp]\centering
	\includegraphics[scale=.75]{fig/1-34.png}
	\caption{电子束在磁场中的偏转}
\end{figure}

现在用实验来检验这个想法.图1.34是一个电子射线
管,从阴极发射出来的电子束,在阴极和阳极间的高电压作用下,轰击到荧光屏上激发出荧光,我们就可以看到电子束运动的径迹.实验表明,在没有外磁场时电子束是沿直线前进的(图甲),如果把射线管放在蹄形磁铁的两极间,从荧光屏上可以看到电子束运动的径迹发生了弯曲(图乙). 这表明运动电荷确实受到了磁场的作用力,磁场对运动电荷的作用力通常叫做\textbf{洛仑兹力}.

洛仑兹力的方向也可以用左手定则来判定:伸开左手,让磁力线进入手心,四指指向正电荷运动的方向,那么拇指所指的方向就是正电荷所受的洛仑兹力的方向,运动的负电荷在磁场中所受的洛仑兹力,方向跟正电荷相反.

洛仑兹力的大小可以从磁场对电流的作用力计算出来.设导线中单位体积内含有的运动电荷数是$n$,每个电荷的电量是$q$,电荷的平均定向移动速率是$v$,导线的横截面积是$S$,
那么,通过导线的电流强度就是
\[I=nqvS\]

磁场对电流的作用力是$F=I\ell B\sin\theta$.这个力可以看作
是作用在每个运动电荷上的洛仑兹力的合力.设洛仑兹力为$f$,这段导线内运动电荷的总数为$N$,则$Nf=F$,即$Nf=I\ell B\sin\theta$,代入$I=nqvS$,得到
\[Nf=nqvS\ell B\sin\theta \]
导线中运动电荷总数$N$,等于单位体积内的运动电荷数$n$跟体积$S\ell$的乘积,即$N=nS\ell$,因此上式简化为
\[f=qvB\sin\theta \]
这就是说,\textit{洛仑兹力的大小等于电荷的电量$q$、电荷的速率$v$,磁感应强度$B$以及$v$和$B$间的夹角$\theta$的正弦$\sin\theta$的乘积}.在国际单位制中,上式中的各个物理量分别用牛顿、库仑、米/秒、特斯拉作单位.

当$\theta=90^{\circ}$时,即电荷的运动方向跟磁场方向垂直时,电荷所受的洛仑兹力最大,等于$qvB$.当$\theta=0$时,即电荷的运动方向跟磁场方向一致时,电荷所受的洛仑兹力最小,等于零.当$\theta$角为其他数值时,洛仑兹力的大小在最大值和最小值之间.

\subsection*{练习六}

\begin{enumerate}
    \item 如图1.35所示,带电粒子以速率$v$射入匀强磁场.分别标出带电粒子所受洛仑兹力的方向.
    \begin{figure}[htp]\centering
        \begin{tikzpicture}[>=stealth, scale=.5]
    \foreach \x in {1,...,4}
        \foreach \y in {1,...,4}
        {
            \node at (\x,\y){$\times$};
        }
    \draw [->] (-.5,2.5)node [left]{$+q$}--node [below]{$v$}(0.5,2.5);
    \draw (-.5,2.5)[fill=white] circle (3pt);
            \node at (2.5,-1){甲};
        \end{tikzpicture}   \quad 
        \begin{tikzpicture}[>=stealth, scale=.5]
        \foreach \x in {1,...,4}
        \foreach \y in {1,...,4}
        {
            \fill (\x,\y) circle (3pt);
        }
        \draw [->] (-.5,2.5)node [left]{$+q$}--node [below]{$v$}(0.5,2.5);
        \draw (-.5,2.5)[fill=white] circle (3pt);
        \node at (2.5,-1){乙};
        \end{tikzpicture} \quad   \begin{tikzpicture}[>=stealth, scale=.5]
        \foreach \x in {1,...,4}
        {
            \draw [->](\x,0.5)--(\x,4);
        }
        \draw [->] (-.5,2.5)node [left]{$+q$}--node [below]{$v$}(0.5,2.5);
        \draw (-.5,2.5)[fill=white] circle (3pt);
        \node at (2.5,-1){丙};
        \end{tikzpicture} \quad   \begin{tikzpicture}[>=stealth, scale=.5]
        \foreach \x in {1,...,4}
        {
                \draw [<-](\x,0.5)--(\x,4);
        }
        \draw [->] (-.5,2.5)node [left]{$+q$}--node [below]{$v$}(0.5,2.5);
        \draw (-.5,2.5)[fill=white] circle (3pt);
        \node at (2.5,-1){丁};
        \end{tikzpicture}
        \caption{}
    \end{figure}	
    \item 一个带电粒子在空间中运动时没有发生偏转,能不能说明这个空间中没有磁场?为什么?
    \item 一个电子以速率$v$射入磁感应强度为$B$的匀强磁场中,电子沿什么方向射入,受到的洛仑兹力最大?最大值是多大?沿什么方向射入,不受洛仑兹力作用?
    \item 电子的速率$v=3.0\x10^8\ms$,垂直射入$B=0.1$特的磁场中,它受到的洛仑兹力是多大?
    \item 一个电子以$1.2\x10^7\ms$的速率射入磁感应强度为0.02特的匀强磁场中.当速率$v$与磁感应强度$B$的夹角$\theta$为$30^{\circ}$和$60^{\circ}$时,电子所受洛仑兹力分别是多大?
    \item 一电荷$q$在某一匀强磁场中运动,判断下面几种说法是否正确,并说明理由.
\begin{enumerate}
    \item 只要速度的大小相同,所受的洛仑兹力就相同.
    \item 如果速度不变,把电荷$q$改为$-q$,洛仑兹力的方向将反向,但大小不变.
    \item 如果速度不变,把$B$改为反向,洛仑兹力的方向将反向,但大小不变.
\end{enumerate}

\end{enumerate}


\section{带电粒子在磁场中的运动}
带电粒子在磁场中运动时受到洛仑兹力的作用,已知洛仑兹力,利用力学中学过的运动学和动力学的知识,就可以确定带电粒子在磁场中的运动情况.

现在我们研究一种简单的情形.一个带电粒子在匀强磁场中运动,它的初速度方向跟磁场方向垂直,粒子的运动轨迹将是怎样的呢?

由于初速度的方向和洛仑兹力的方向都在跟磁场方向垂直的平面内,没有任何作用使粒子离开这个平面,所以粒子只能在这个平面内运动,洛仑兹力总是跟粒子的运动方向垂直,只改变粒子速度的方向,不改变粒子速度的大小,所以粒子的速率$v$是恒定的.这时洛仑兹力$f=qvB$的大小也是恒定的,它对运动粒子起着向心力的作用.因此粒子的运动一定是匀速圆周运动(图1.36).
\begin{figure}[htp]\centering
	\begin{tikzpicture}[>=stealth, scale=1]
	\foreach \x in {1,...,6}
	\foreach \y in {1,...,5}
	{
		\node at (\x,\y){$\times$};
	}
\draw (3.5,3) circle (1.8);

\draw [->, very thick](3.5,3+1.8)node[above]{$+q$}--(2.5,3+1.8)node [left]{$v$};
\draw [->, very thick](3.5,3+1.8) -- (3.5,3+.8)node[below]{$f$};
\draw [->](3.5,3)--+(140:1.8);
    \end{tikzpicture}
\caption{带电粒子的圆周运动}
\end{figure}

上述推论可以用实验来验证,实验所用的仪器是一种特制的电子射线管,由电子枪发出的电子射线可以
使管内的低压水银蒸气(或氢气)发出辉光,显示出电子的径迹,在暗室中可以清楚地看到:没有磁场作用时,电子的径迹是直线;在管子外面加上一个匀强磁场(这个磁场是由两个平行的通电环形线固产生的),电子的径迹就弯曲成圆弧.

在高中二年级我们学过带电粒子在电场中的运动,在现代科学技术中,常常要使带电粒子在电场的作用下运动,或者在磁场的作用下运动,或者在电场和磁场的共同作用下运动.例如电视机中的显象管、电子显微镜和我们将要学到的回旋加速器等,都是利用电场和磁场来控制电荷的运动的.

\begin{example}
    一个初速度为零的质子,经过电压是$1.3\x10^3$伏的电场加速后,垂直进入磁感应强度$B$是0.20特的匀强磁场中,试求:
    \begin{enumerate}
        \item 质子进入磁场时的速率;
        \item 质子在磁场中运动的轨道半径;
        \item 质子做匀速圆周运动的周期.
    \end{enumerate}
质子的质量$m=1.67\x10^{-27}$千克,电量$q=1.6\x10^{-19}$库.
\end{example}

\begin{solution}
    \begin{enumerate}
        \item 质子在电场中得到的动能等于电场所做的功,即$\dfrac{1}{2}mv^2=qU$.
        所以质子进入磁场时的速率
        \[\begin{split}
            v&=\sqrt{\frac{2qU}{m}}\\
            &=\frac{2\x 1.6\x 10^{-19}\x1.3\x10^3}{1.67\x 10^{-27}}\ms\\
            &=5.0\x 10^5\ms
        \end{split}\]
        \item 质子在磁场中做匀速圆周运动的向心力是由洛仑兹力$f=qvB$提供的,所以$qvB=mv^2/r$,解出$r$,得到
        \begin{equation}
            r=\frac{mv}{qB}
        \end{equation}
        代入数值得到
       \[\begin{split}
           r&=\frac{mv}{qB}=\frac{1.67\x 10^{-27}\x 5.0\x 10^5}{1.6\x 10^{-19}\x 0.2}{\rm m}\\
           &=2.6\x 10^{-2}{\rm m}
       \end{split}\]
        从(1.1)式知道,带电粒子在匀强磁场中做匀速圆周运动的
        轨道半径跟粒子的速率成正比,速率越大,轨道半径也越大.
        
        \item 匀速圆周运动的周期$T=2\pi r/v$,把(1.1)式代入,得到
\begin{equation}
    T=\frac{2\pi m}{qB}
\end{equation}
        代入数值得到
        \[\begin{split}
            T&=\frac{2\pi m}{qB}=\frac{2\x 3.14\x 1.67\x 10^{-27}}{1.6\x 10^{-19}\x 0.2}\\
            &=3.3\x 10^{-7}{\rm s}
        \end{split}\]
    \end{enumerate}
\end{solution}

从(1.2)式知道,带电粒子在匀强磁场中做匀速圆周运动的周期跟轨道半径和运动速率无关.这是一个很重要的结论,我们在后面要讲到的回旋加速器就是依据这个道理制成的.
        
\subsection*{练习七}
\begin{enumerate}
    \item 电子以$1.6\x10^6\ms$的速率垂直射入$B=10^{-4}$特的匀强磁场中,求电子做圆周运动的轨道半径和周期.
    \item 电子垂直射入$B=7.0\x10^{-4}$特的匀强磁场中,做圆周运动的轨道半径为$3.0\x10^{-2}$米,求电子运动的速率.
    \item 能量是5.3兆电子伏的$\alpha$粒子垂直进入磁感应强度是1.0特的匀强磁场中,试确定作用在$\alpha$粒子上的洛仑兹力和$\alpha$粒子的轨道半径.$\alpha$粒子即氦核,质量为$6.6\x10^{-27}$千克,带有两个基本电荷的正电.
    \item 具有相同动能的质子、氘核和$\alpha$粒子垂直进入同一匀强磁场,试比较这些粒子轨道半径的大小,氘核的质量为质子的两倍,带有一个基本电荷的正电.
    \item 两个电子分别以速率$v$和$2v$垂直射入匀强磁场中,经磁场偏转后,哪个电子先回到原来的出发点?
    \item 在匀强磁场中,带电粒子的运动方向不和磁感应强度的方向垂直,它的运动径迹将是什么样的曲线?(只要求定性说明)
\end{enumerate}


\section{荷质比的测定~~质谱仪}
\subsection{荷质比的测定}

带电粒子的电荷与质量之比,叫做\textbf{荷质比}.每种带电微观粒子都带有一定的电荷,具有一定的质量,所以荷质比是带电微观粒子的基本参量.

带电粒子在电场和磁场中运动时,所受的电场力和磁场力跟粒子所带的电量成正比,得到的加速度跟粒子的质量成反比,因而粒子的运动情况依赖于粒子的荷质比.这样,我们研究带电粒子在电场和磁场中的运动情况,反过来就可以确定粒子的荷质比.

图1.37是测定荷质比的一种装置.让中性的气体分子进入电离室$A$,在那里被电离成离子.这些离子从电离室的
小孔飘出,从缝$S_1$进入加速电场中被加速.然后让粒子垂直进入匀强磁场中做匀速圆周运动,最后打在照相底片$D$上.
\begin{figure}[htp]\centering
\includegraphics[scale=1.2]{fig/1-37.pdf}
\caption{}
\end{figure}

设粒子所带的电量是$q$,加速电场两极间的电势差是$U$,粒子进入缝$S_1$时速度很小,接近于零,粒子离开加速电场时所获得的动能就是
\begin{equation}
\frac{1}{2}mv^2=qU    
\end{equation}

设匀强磁场的磁感应强度是$B$,粒子做匀速圆周运动的轨道半径是$r$,由于向心力是洛仑兹力提供的,所以
\begin{equation}
    \frac{mv^2}{r}=qvB
\end{equation}
由(1.3)和(1.4)两式中消去$v$,我们得到
\[\frac{q}{m}=\frac{2U}{B^2r^2}\]

上式右方的各物理量都可以由实验测出来,这样就可以得到粒子的荷质比.这里我们看到,微观量的大小是通过宏观量的测定而得到的.

测定荷质比,对人类认识微观粒子有重要作用.人类认
识的第一种亚原子粒子——电子,最初就是由测定它的荷质比而被发现的.十九世纪末,英国科学家汤姆生在研究阴极射线时测定了阴极射线粒子的荷质比,他所用的方法虽然跟上述方法不同,但根据的原理也是带电粒子在电场和磁场中的偏转,汤姆生的测定导致他发现阴极射线粒子就是电子,这个问题我们将在第八章讲述.

汤姆生测得的阴极射线粒子的荷质比约为$2\x10^{11}{\rm C}/{\rm kg}$,现在测得的电子荷质比的精确值是
\[e=1.7588047\x10^{11}{\rm C}/{\rm kg}\]
通常可取作$e/m=1.76\x10^{11}{\rm C}/{\rm kg}$.

\subsection{质谱仪}

在图1.37所示的装置中,如果带电粒子的电量相同,而质量$m$有微小差别,它们进入磁场后将沿着不同的半径做圆周运动,打到照相底片的不同地方,在底片上形成若干谱线状的细条,叫做质谱线.每一条谱线对应于一定的质量;从谱线的位置可以知道圆周的半径$r$,已知带电粒子的电量$q$,就可以算出它的质量$m$.这种仪器叫做质谱仪,图1.37就是质谱仪的原理图.利用质谱仪对某种元素进行测量,可以准确地测出各种同位素的原子量.图中所示的是锗的质谱线,在谱线上标出的数字是锗同位素的质量数.

质谱仪最初也是由汤姆生设计的,他用质谱仪首先得到了氖20和氖22的质谱线,证实了同位素的存在.后来经过多次改进,质谱仪已成了一种十分精密的仪器,是测定带电粒子质量和分析同位素的重要工具.


\subsection*{练习八}
\begin{enumerate}
    \item 在图1.37中,设离子室$A$中产生的是钠离子,加速电压$U=705$伏,磁感应强度$B=3. 85\x10^{-1}$特,$r=5$厘米,求钠离子的荷质比.
    \item 在图1.37所示的装置中,离子从小孔飘出时速度并不相同,因此经加速电场加速后,从缝$S_2$射出时速度也不相同,这对实验有一定影响.于是人们提出在缝$S_2$和$S_3$之间加一个\textbf{速度选择器}(图1.38),其中$D_1$和$D_2$是两个平行金属板,分别连在电源的两极上,其间有一定的电场强度$E$;同时在这空间加有垂直于电场方向的磁场,磁感应强度为$B$.这时具有一定速度的带电粒子,从缝$S_2$垂直进入后,可以不发生偏转,由缝$S_2$射出;而具
有其他速度的带电粒子都发生偏转,不能由缝$S_2$射出,为什么?这个一定的速度$v$有多大?
\begin{figure}[htp]\centering
	\begin{tikzpicture}[>=stealth, scale=1]
	\foreach \x in {1,...,6}
	\foreach \y in {1,...,4}
	{
		\node at (\x,\y){$\times$};
	}

\draw [ultra thick] (2.5,0.5)--(2.5,4.5)	;
\draw [ultra thick] (4.5,0.5)--(4.5,4.5)	;
\draw  [ultra thick](2.5,2.5)--(0.5,2.5)node [left]{$-$};
\draw  [ultra thick](4.5,2.5)--(6.5,2.5)node [right]{$+$};

\draw [ultra thick] (2,4.75)--(3.4,4.75);  \draw  [ultra thick](3.6,4.75)--(5,4.75);
\draw  [ultra thick](2,.25)--(3.4,.25);  \draw  [ultra thick](3.6,.25)--(5,.25);
\draw [->](3.5, 5.5)--(3.5, 4.5) node [below]{$v$};
\node at (3.5,5){$S_2$};\node at (3.5,0){$S_3$};
\node at (2,3.5){$D_1$};\node at (5,3.5){$D_2$};

	\end{tikzpicture}
	\caption{速度选择器的原理示意图}
\end{figure}
\item 带电粒子带正电或者带负电,会不会影响速度选择器对它们的速度的选择?如果把图1.38中的电场方向改变为相反的方向,或者把磁场方向改变为相反的方向,速度选择器还能不能使用?如果把电场和磁场同时改变为相反的方向,还能不能使用?
\end{enumerate}

\section{回旋加速器}


在现代物理学中,为了研究物质的微观结构,往往要用能量很高的带电粒子去轰击各种原子核,观察它们的变化情况.例如,要从原子核中把中子或质子打出来,就得用8兆电子伏的质子.为了探索质子的内部结构,使用了200亿电子伏的电子去轰击质子.怎样才能在实验室大量产生这样高能量的带电粒子呢?这就要用一种新的实验设备——加速器.

我们已经学过,利用电场可以使带电粒子加速.早期制成的加速器,就是用高压电源的电势差来加速带电粒子的,这种类型的加速器受到实际所能达到的电势差的限制,粒子获得的能量并不太高,只能达到几十万到几兆电子伏,1932年美国物理学家劳仑斯发明了回旋加速器,很巧妙地克服了这个困难,这种加速器不是利用高电压使粒子一次得到巨大的速度,而是用电压较低的高频电源,使粒子每隔一定的时间受到一次加速,经过多次加速后达到巨大的速度.

现在来看一看回旋加速器的工作原理.
\begin{figure}
    \centering
    \begin{tikzpicture}[>=stealth, scale=1.3]
    \foreach \x in {1,...,7}
    \foreach \y in {1,...,5}
    {
        \node at (\x,\y){$\times$};
    }
    
    \draw (0,3.2)--(8,3.2);
    \draw (0,3.4)--(8,3.4);
    
    \draw[very thick] (3.5,3.2) arc (180:360:.5);  \draw [->, thick] (3.5,3.2)--(3.5,3.2-.5)node[below]{$v_0$};
    \draw[very thick] (3.5+1,3.2)--(3.5+1,3.2+.2);
    \draw[very thick] (3.5+1,3.2+.2) arc (0:180:1);  \draw [->, thick] (3.5+1,3.2+.2)--(3.5+1,3.2+.2+.8)node[above]{$v_1$};
    \draw[very thick] (3.5-1,3.2+.2)--(3.5-1,3.2);
    \draw[very thick] (3.5-1,3.2) arc  (180:360:1.5);  \draw [->, thick] (3.5-1,3.2)--(3.5-1,3.2-1)node[below]{$v_2$};
    \draw[very thick] (3.5+2,3.2)--(3.5+2,3.2+.2);
    \draw[very thick] (3.5+2,3.2+.2) arc  (0:180:2);  \draw [->, thick] (3.5+2,3.2+.2)--(3.5+2,3.2+.2+1.2)node[above]{$v_3$};
    \draw[very thick] (3.5-2,3.2+.2)--(3.5-2,3.2);
    \draw[very thick] (3.5-2,3.2) arc  (180:360:2.5);  \draw [->, thick] (3.5-2,3.2)--(3.5-2,3.2-1.5)node[below]{$v_4$};
    \draw[very thick] (3.5-2+5,3.2)--(3.5-2+5,3.2+.2);\draw [->, thick] (3.5-2+5,3.2+.2)--(3.5-2+5,3.2+.2+1.8)node[above]{$v_5$};
    
    \node at (3.5-.2,3){$A_0$};
    \node at (4.5+.2,3){$A_1$};
    \node at (2.5-.2,3){$A_2$};
    \node at (5.5+.2,3){$A_3$};
    \node at (1.5-.2,3){$A_4$};
    \node at (6.5+.2,3){$A_5$};
    
    \node at (4.5+.2,3.6){$A'_1$};
    \node at (2.5-.2,3.6){$A'_2$};
    \node at (5.5+.2,3.6){$A'_3$};
    \node at (1.5-.2,3.6){$A'_4$};
    \node at (6.5+.2,3.6){$A'_5$};
    
    \end{tikzpicture}
    \caption{}
    \end{figure}

图1.39表示从放在$A_0$处的粒子源发出一个带正电的粒子,它以某一速率$v_0$垂直进入匀强磁场中,在磁场中做匀速圆周运动,经过半个周期,当它沿着半圆弧$\widering{A_0A_1}$到达$A_1$时,我们在$A_1A'_1$处造成一个向上的电场,使这个带电粒子在$A_1A_1'$处受到一次电场的加速,速率由$v_0$增加到$v_1$.然后粒子以速率$v_1$在磁场中做匀速圆周运动,我们知道,粒子的轨道半径跟它的速率成正比,因而粒子将沿着半径增大了的圆周运动.
又经过半个周期,当它沿着半圆弧$\widering{A'_1A'_2}$到达$A'_2$时,我们在$A_2A'_2$处造成一个向下的电场,使粒子又一次受到电场的加速,速率增加到$v_2$.如此继续下去,每当粒子运动到$A_1A'_1$、$A_3A'_3$等处时都使它受到一个向上电场的加速,每当粒子运动到$A_2A'_2$、$A_4A'_4$等处时都使它受到一个向下电场的加速,这样,粒子将沿着图示的螺线$A_0A_1A'_1A'_2A_2$……回旋下去,速率将一步一步地增大.
\begin{figure}[htp]
	\centering
	\begin{tikzpicture}[>=latex, scale=.5]
	
\draw (0,0)node [left]{$A$}--(9,0)node [right]{$A$};
\draw (0,2)node [left]{$A'$}--(9,2)node [right]{$A'$};

\foreach \x in {1,2,...,8}
{
    \draw [->](\x,0)--(\x,2);
   \node at (\x, 2.5){$-$};  \node at (\x, -.5){$+$};
}
\draw [dashed] (3.5,0) arc (180:360:1);


\draw (5.5,.3) circle (.25) node {\tiny +};

\draw [->](5.5,.55)-- (5.5,2.75)node[above]{$v_1$};
\node at (3.5-.2,-0.5) {$A_0$};
\node at (5.5+.2,-0.5) {$A_1$};
\node at (4.5, -2){甲};
	\end{tikzpicture}
	\begin{tikzpicture}[>=latex, scale=.5]
	
\draw (0,0)node [left]{$A$}--(9,0)node [right]{$A$};
\draw (0,2)node [left]{$A'$}--(9,2)node [right]{$A'$};

\foreach \x in {1,2,...,8}
{
    \draw [<-](\x,0)--(\x,2);
   \node at (\x, 2.5){$+$};  \node at (\x, -.5){$-$};
}
\draw [dashed] (7,2) arc (0:180:2.25);	
\draw (2.5,2-.3) circle (.25) node {\tiny +};

\node at (2.5-.2, 2.5){$A'_2$}; 
\node at (7+.2, 2.5){$A'_1$};
\draw [->](2.5,2-.55)-- (2.5,-1.3)node[below]{$v_2$};	
\node at (4.5, -2){乙};
	\end{tikzpicture}
	\caption{甲:粒子运动到$A_1$时被电场加速.乙:经过半周期,粒子运动到$A'_2$,电场改变了方向,粒子又被电场加速.}
\end{figure}

我们讲过,带电粒子在匀强磁场中做匀速圆周运动的周期$T=2\pi m/qB$跟运动速率和轨道半径无关,对一定的带电粒子和一定的磁感应强度来说,这个周期是恒定的.因此,尽管粒子的速率和半径一次比一次增大,运动周期却始终不变,即粒子由$A$沿半圆弧$\widering{A_0A_1}$运动到$A_1$,由$A'_1$沿半圆弧$\widering{A'_1A'_2}$
运动到$A'_2$……经过的时间都等于半个周期,即$T/2$.
如果象图1-40所示那样,在直径$AA$、$A'A'$处造成一个交变电场,使它也
以相同的周期$T$往复变化,那就可以保证粒子每经过直径$AA$
和$A'A'$时都正好赶上适合的电场方向而被加速.这一点仔细研究一下图1.40就会明白了.

\begin{figure}[htp]
\centering
\includegraphics[scale=1.2]{fig/1-41.pdf}
\caption{回旋加速器的$D$形盒}
\end{figure}

回旋加速器是怎样实现上述加速粒子过程的呢?回转加速器的核心部分是两个$D$形的金属扁盒(图1.41).这两个$D$形盒就象是沿着直径把一个圆形的金属扁盒切成的两半.两个$D$形盒之间留一个窄缝,在中心附近放有粒子源.$D$形盒装在真空容器中,整个装置放在巨大电磁铁的两极之间,磁场方向垂直于$D$形盒的底面.

如果把两个$D$形盒分别接在频率$f=1/T=gB/(2\pi m)$的高频电源的两极上(频率的数量级为10赫),就可以在$D$形盒的窄缝中造成图1.40所示的交变电场.这样,从粒子源发出的带电粒子就可以象图1.39所示那样不断被加速.为什么
要使带电粒子在$D$形盒中运动呢?这是因为考虑到静电屏蔽作用,金属盒可以屏蔽外界电场;盒内的电场很弱,这样才能保证粒子在盒内只受磁力的作用而做匀速圆周运动.

带电粒子在$D$形盒内沿螺线轨道逐渐趋于盒的边缘,达到预期的速率后,用特殊装置把它们引出.

回旋加速器的出现,使人类在获得具有较高能量的粒子方面前进了一步,但是,在三十年代末期已经发现,用回旋加速器加速质子,在能量达到25—30兆电子伏之后,就很难进一步提高了.这是因为,在粒子的能量很高的时候,它的运动
速度接近于光速,按照狭义相对论,这时粒子的质量将随着速率的增加而增大,因此,粒子在磁场中回旋一周所需的时间要发生变化,交变电场的频率不再跟粒子运动的频率一致,这就破坏了加速器的工作条件,进一步提高粒子的速率就不可能了.

为了把带电粒子加速到更高的能量,以适应高能物理实验的需要,人们设计制造了各种类型的新型加速器,如同步加速器、电子感应加速器、直线加速器等等,这些加速器都考虑了相对论效应,可以把带电粒子加速到几千兆电子伏以上,目前同步加速器能够把质子加速到$10^6$兆电子伏.

\section*{复习题}
\begin{enumerate}
    \item 奥斯特用什么实验发现电流周围存在着磁场的?人们对磁极和磁极之间、磁极和电流之问、电流和电流之间的相互作用是怎样获得统一的认识的?
    \item 磁场中某一点的磁场方向是怎祥规定的?怎样用安培定则来判断直线电流,环形电流、通电螺线管的磁场方向?
    \item 安培的分子电流假说的内容是什么?什么是磁现象的电本质?
    \item 什么叫铁磁性材料?软磁性材料和硬磁性材料的性质有什么区别?各有什么用途?
    \item 什么叫磁场的磁感应强度?写出磁感应强度的定义式.磁感应强度的方向是怎样的?为什么说用磁力线可以形象地表示磁场?
    \item 什么叫磁通量?写出它的定义式.怎样用磁力线形象地说明磁通量?
    
    \item 直线电流磁场的磁感应强度跟电流强度、离开导线的距离各是什么关系?写出计算直线电流磁场的磁感应强度的公式.
    \item 什么叫安培力?写出计算安培力大小的公式.安培力的方向怎样来确定?
    \item 简述电流天平的原理.\item 通电线圈在磁场中为什么会发生偏转?分析一下这个问题.简述磁电式电流表的原理.
    \item 什么叫洛仑兹力?写出计算洛仑兹力大小的公式.洛仑兹力的方向怎样来确定?
    \item 在匀强磁场中运动的带电粒子,如果初速度方向跟磁场方向垂直,粒子将做匀速圆周运动,分析一下,粒子为什么做这种运动,怎样求出带电粒子在匀强磁场中做匀速圆周运动的轨道半径和周期?写出它们的计算公式.
    \item 什么叫荷质比?简述测定荷质比的原理和质谱仪的
    原理.
    \item 简述回旋加速器的原理.\item 你自己总结一下:带电粒子在电场和磁场中的运动有哪些实际应用,除了课本中所讲的,你自己能不能设想出一种新的应用?如果一时想不出,希望你多看一些课外书籍和杂志,以扩展眼界,开阔思路,力求设想出一种新的应用.
\end{enumerate}

\section*{习题}
\begin{enumerate}
    \item 照图1.42那样,把一根柔软的弹簧竖直地悬挂起来,使它的下端刚刚跟导电液体接触,给弹簧通入电流时,会发生什么现象?做一下这个实验,并解释所发生的现象.

\begin{figure}[htp]
\centering
\begin{minipage}[t]{0.48\textwidth}
\centering
    	\includegraphics[scale=.75]{fig/1-42.png}
\caption{}
\end{minipage}
\begin{minipage}[t]{0.48\textwidth}
\centering
	\includegraphics[scale=.75]{fig/1-43.png}
\caption{}
\end{minipage}
\end{figure}

    \item 把一个可以绕水平轴转动的铝盘放在蹄形磁铁之间,盘的下边缘浸在导电液体中(图1.43).把转轴和导电液体分别接到直流电源的两极上,铝盘就会转动起来.为什么?用什么方法可以改变铝盘的转动方向?
\item 在玻璃皿中放入电解质溶液,在玻璃皿的中心放一个圆柱形电极,边缘放一个围环形电极,分别接在电池的两极
上(图1.44),如果把玻璃里放在磁场中,液体就旋转起来.为什么?液体旋转的方向跟什么有关系?做一下这个实验,看你的判断是否正确?
\begin{figure}[htp]\centering
	\includegraphics[scale=.75]{fig/1-44.png}
	\caption{ }
\end{figure}
\item 下列说法中,哪些说法是正确的?
\begin{enumerate}
    \item 一小段通电导线放在磁感应强度为零的位置,所受的安培力一定等于零.
    \item 一小段通电导线在磁场中某点不受磁场力的作用,该点的磁感应强度一定为零.
    \item 一小段通电导线在磁场中所受安培力的方向、该点的磁感应强度的方向、电流的方向三者一定互
相垂直.
\end{enumerate}

\item 磁电式电流表中的磁场是均匀地辐向分布的,线圈两侧边所在位置的磁感应强度为0.02特.线圈是边长1厘米的正方形,共100匝,线圈每转$1^{\circ}$,螺旋弹簧产生阻碍线圈偏转的力矩是$2.5\x10^{-8}{\rm N}\cdot {\rm m}$.线圈中电流强度为5毫安时,指针将偏转多少度?
\item 洛仑兹力对带电粒子是否做功?为什么?    
\item 利用学过的知识,想办法把下面的带电粒子分开:
\begin{enumerate}
    \item 速度分别为$v$和$3v$的电子;
    \item 具有相同动能的质子和$\alpha$粒子;
    \item 荷质比不同的带正电的粒子.
\end{enumerate}

\begin{figure}[htp]
\centering
\begin{minipage}[t]{0.48\textwidth}
\centering
\begin{tikzpicture}[>=latex, scale=.8]
\draw [thick](0,0) arc (0:-120:1)node[left]{$a$};
\draw [thick](0,0) arc (0:-45:5)node[left]{$b$};
\draw [thick](0,0) arc (0:-20:15)node[left]{$c$};
\draw[thick] (0,0) arc (180:200:13)node[right]{$d$};
\draw [thick](0,0) arc (180:230:4)node[right]{$e$};
\draw (1,-.5) [fill=white] circle (5pt);
\node at (1,-.5){$\times$};
\node at (1.35,-.5){$B$};
\node at (0,.5){$O$};
\draw (0,0) [fill=black] circle (1.5pt);

\end{tikzpicture}
\caption{}
\end{minipage}
\begin{minipage}[t]{0.48\textwidth}
\centering
\begin{tikzpicture}[>=latex, scale=.8]
\draw[->] (-1,0)--(6,0);
\draw[->] (0,0)--(0,3)node [above]{$v_0$};
\draw[very thick] (0,0) arc (180:0:2.5); 
\draw[very thick, ->] (0,0) arc (180:90:2.5); 
\node at (0,-.5){$A$};\node at (5,-.5){$B$};

\draw (0,0) [fill=white]circle (4.5pt);
\node at  (0,0) {\small $-$};


\end{tikzpicture}
\caption{}
\end{minipage}
\end{figure}


\item 图1.45表示由$O$点发出的电子和正电子(质量和电量跟电子相同,但带的是正电荷)在匀强磁场中运动的径迹.
哪些径迹是电子的,哪些径迹是正电子的?$a$, $b$, $c$三条径迹中,哪个粒子的能量最大,哪个最小?
\item 质子、氘核和$\alpha$粒子由静止开始通过相同的电势差后垂直进入同一匀强磁场.
\begin{enumerate}
    \item 比较这些粒子的动能.
    \item 如果质子在磁场中的轨道半径为10厘米,氘核和$\alpha$粒子的轨道半径各有多大?
\end{enumerate}
\item 如图1.46所示,$A$和$B$之间的距离为0.1米,位于$A$点的电子的速度$v_0=1.0\x10^7\ms$.
\begin{enumerate}
    \item 要使电子沿半圆周由$A$运动到$B$,求磁感应强度的大小和方向.
    \item 电子从$A$
运动到$B$需要多长时间?
\end{enumerate}
\item 把图1.38所示的速度选择器加在图1.37所示装置的$S_2$和$S_3$之间,已知速度选择器的电场强度为$E$,磁感应强度为$B_1$.粒子从缝$S_3$进入磁感应强度为$B_2$的匀强磁场中,做圆周运动的半径为$r$.求粒子的荷质比.
\item  有一回旋加速器,它的交变电压的频率为$12\x10^8$赫,半圆形电极的半径为0.53米.加速氘核所需的磁感应强
度要多大?氘核的最大动能是多大?已知氘核的质量为$3.3\x10^{-27}$千克,电量为$1.6\x10^{-19}$库.
\item 目前世界上正在研究一种新型发电机,叫做磁流体发电机,它可以把气体的内能直接转化为电能.图1.47表示出了它的发电原理:将一束等离子体(即高温下电离的气体,含有大量带正电和带负电的微粒,但从总体来说呈中性),喷射入磁场,磁场中有两块金属板1和2,这时金属板上就会集聚电荷,产生电压,说明金属放上为什么会聚集电荷.在磁极配置如图中所示的情况下,电路中的电流方向如何?

磁流体发电是一项新兴技术,报纸、杂志上常有文章介
绍,希望有兴趣的同学找来看看,以扩展自己的知识面.
\item  图1.48表示两个平行金属板,它们之间的距离为$d$,分别接在电源的两极上.在两平行金属板当中的空间存在着彼此垂直的电场和磁场,电场强度为$E$,磁感应强度为$B$. 从负极板的小孔射入一个电子,经过有电场和磁场同时存在的空间,并打在正极板上,射入电子的初速度为$v0$,方向跟竖直方向成$\theta$角.求电子打在正极板上的速度的大小.

\begin{figure}[htp]\centering
	\includegraphics[scale=.75]{fig/1-47.png}
	\caption{ }
\end{figure}\begin{figure}[htp]\centering
\includegraphics[scale=.75]{fig/1-48.png}
\caption{ }
\end{figure}
\end{enumerate}

\chapter{电磁感应}
\section{电磁感应现象}

在奥斯特发现电流的磁效应以后,人们自然想到:既然电流能够产生磁场,反过来磁场是不是也能产生电流呢?最容易产生的设想是把导线绕在磁铁上,导线两端接上电流表,构成一个闭合电路,看看能不能产生电流,法拉第就是这样开始
来研究的,结果发现电流表的指针并不偏转,换用强的电磁铁,或者换用更灵敏的电流表,结果还是一样,没有电流,怎样才能获得电流呢?下面用实验来研究这个问题.


\subsection*{实验一}
如图2.1所示,如果让导体$AB$在磁场中向左
或向右运动,电流表的指针就发生偏转,表明电路中有了电流.导体$AB$停下来,电流也就消失了.这就是说,闭合电路
的一部分导体相对于磁场运动时,电路中就有电流产生.但是,如果让导体$AB$在磁场中向上或向下运动,电路中却不产生电流.怎样分析上述现象呢?我们知道,磁场可以用磁力线形象地表示出来,有了磁力线的概念,就容易分析上述现象了.导体$AB$向左或向右运动时切割磁力线,向上或向下运动时不切割磁力线.可见,闭合电路的一部分导体做切割磁力线的运动时,电路中才有电流产生.
\begin{figure}[htp]\centering
\includegraphics[scale=.6]{fig/2-1.png}
\caption{}
\end{figure}

在这个实验中,导体AB运动,如果导体不动,让磁场运动,会不会在电路中产生电流呢?让我们做下面的实验.

\subsection*{实验二}

如图2.2所示,把一个磁铁插入螺线管,或者从螺线管里拿出来,可以看到,磁铁相对于螺线管运动的时候,电流表的指针发生偏转,表明螺线管电路中有了电流.如果保持磁铁在螺线管中不动,或者让二者以同一速度运动,即保持相对静止,螺线管中就没有电流了.在这个实验中,磁铁相对于螺线管运动时,螺线管的导线切割磁力线.可见,不论是导体运动,还是磁场运动,只要闭合电路的一部分导体切割磁力线,电路中就有电流产生.
\begin{figure}[htp]\centering
\includegraphics[scale=.6]{fig/2-2.png}
\caption{}
\end{figure}

闭合电路的一部分导体切割磁力线时,穿过闭合电路的磁力线条数发生变化.由此提示我们:如果导体和磁场不发生相对运动,而让穿过闭合电路的磁场发生变化,会不会在电路中产生电流呢?为了研究这个问题,我们做下面的实验.

\subsection*{实验三}

如图2.3所示,把螺线管$B$套在螺线管$A$的外面,合上电键给螺线管$A$通电时,电流表的指针发生偏转,螺
线管$B$中有了电流.当螺线管$A$中的电流达到稳定时,螺线管$B$中的电流消失,打开电键使螺线管$A$断电时,螺线管$B$中也有电流产生,如果用变阻器来改变电路中的电阻,使螺线管$A$中的电流强度发生变化,螺线管$B$中也有电流产生.在这个实验中,螺线管$B$处在螺线管$A$的磁场中,当$A$通电和断电时,或者使$A$中的电流强度发生变化时,$A$的磁场随着发生变化.因此,这个实验表明:在导体和磁场不发生相对运动的情况下,只要闭合电路中的磁场发生变化,因而穿过闭合电路的磁力线条数发生变化,闭合电路中就有电流产生.
\begin{figure}[htp]\centering
\includegraphics[scale=.6]{fig/2-3.png}
\caption{}
\end{figure}

\subsection*{结论}


不论是闭合电路的一部分导体做切割磁力线的运动,还是闭合电路中的磁场发生变化,穿过闭合电路的磁力线条数都发生变化,这时闭合电路中就有电流产生,我们在前一章学过磁通量的概念,磁通量表示的就是穿过某个面的磁力线条数,这样,利用磁通量的概念,我们就可以总结出如下的结论:

\textbf{不论用什么方法,只要穿过闭合电路的磁通量发生变化,闭合电路中就有电流产生}.这种利用磁场产生电流的现象叫做\textbf{电磁感应},产生的电流叫做\textbf{感生电流}.

电磁感应现象是法拉第经过十多年的实验研究,在1831年发现的.这一重大发现进一步揭示了电和磁的密切联系,为后来麦克斯韦建立完整的电磁理论莫定了基础,根据这一发现后来发明了发电机、变压器等电器设备,开辟了电能在生产和生活中广泛应用的途径.

\section*{阅读材料:法拉第电磁感应的发现}

1820年奥斯特发现了电流的磁效应后,法拉第仔细地分析了电流的磁效应,他认为,既然磁铁可以使靠近它的铁块具有磁性,静电荷可以使靠近它的导体带电,那么电流也应当使靠近它的线圈感生出电流.1822年法拉第在日记中记载着“把磁转变成电”的光辉思想.后来法拉第对这一课题进行了系统的实验研究.

开始,法拉第认为:用强磁铁靠近导线,导线中就会产生稳恒电流.法拉第企图用实验证实这个设想,结果未获成功.但法拉第不怕困难,顽强奋战了十年,终于取得了重大突破,1831年发现了电磁感应现象.
\begin{figure}[htp]\centering
\includegraphics[scale=.6]{fig/2-4.png}
\caption{}
\end{figure}

1831年8月法拉第把两个线圈绕在一个铁环上(图2.4),线圈$A$接直流电源,线圈$B$接电流表,他发现,当线圈$A$中的电流接通或断开时,线圈$B$中就产生瞬时电流.法拉第
发现,铁环并不是必须的,拿走铁环,再做这个实验,电磁感应现象仍然发生,只是线圈中的电流弱些.

法拉第怎样分析他的实验呢?他的思路大致如下:

第一,线圈$B$除了处在通电线圈$A$的磁场中,同$A$没有别的联系,所以$B$的感生电流只能由$A$的磁场引起,这正是他探寻了十年的磁生电的现象.

第二,$A$中电流稳定因而周围磁场稳定时,$B$中没有感生电流,表明稳定的磁场不引起感生电流;只有当$A$通电或断电,它的电流变化因而周围磁场变化时,$B$中才有感生电流,表明变化的磁场才能引起感生电流.

第三,磁场可以由磁力线形象地描绘,$B$所在处的磁场变
化,也就是穿过线圈$B$的磁力线条数变化,即磁通量发生变化.

所以,感生电流的产生条件,可以归结为穿过线圈的磁通量变化.

为了透彻研究电磁感应现象,法拉第还做了许多其他实验,1831年11月法拉第在所写的论文中把产生感生电流概括为五种情况:变化着的电流;变化着的磁场;运动的稳恒电流;运动的磁铁;在磁场中运动的导体,法拉第按照他的力线理论把产生感生电流的条件归结为:变化的磁力线能使导体中产生感生电流.

\subsection*{练习一}
\begin{enumerate}
    \item 如图2.5所示,在磁场中有一个闭合的弹簧线圈,当图甲中人的双手离开后,线圈收缩(图乙),线圈收缩时,其中是否有感生电流?为什么?
\begin{figure}[htp]\centering
\includegraphics{fig/2-5.pdf}
\caption{}
\end{figure}

    \item 在图2.6所示的匀强磁场中有一个线圈板,线圈平面垂直于磁力线,当线圈框在磁场中上下通动对,是否会在线圈框中引起感生电流?当线圈框在磁场中左右运动时,是否会在线圈框中引起感生电流?为什么?
\begin{figure}[htp]
\centering
\begin{minipage}[t]{0.48\textwidth}
\centering
\begin{tikzpicture}[>=latex, scale=.8]
\foreach \x in {1,2,...,5}
{
    \draw[->](\x, 5)--(\x, 0);
}
\draw [fill=white](1.8,3)--(4.8,3)--(4.2,2)--(1.2,2)--(1.8,3);
\foreach \x in {2,3,4}
{
    \draw(\x, 5)--(\x, 2.5);
}

\end{tikzpicture}
\caption{}
\end{minipage}
\begin{minipage}[t]{0.48\textwidth}
\centering
\begin{tikzpicture}[>=latex, scale=.8]
\foreach \x in {-3,-2,...,2}
\foreach \y in {-2,...,2}
{
    \node at (\x,\y){$\times$};
}
\draw [dashed](-.5,-2.5)node[below]{$O'$}--(-.5,3)node[above]{$O$};
\draw[very thick] (-1.5,-1.5) rectangle (.5,1.5);
\draw[->] (-.5+.25, 2.5) arc [start angle=10, end angle=360, x radius=.25,y radius=.125];
\end{tikzpicture}
\caption{}
\end{minipage}
\end{figure}

    \item 如图2.7所示,线圈在匀强磁场中绕$OO'$轴转动时,线圈里是否有感生电流?为什么?
    \item 如图2.8所示,让闭合线圈由位置1通过一个匀强磁场运动到位置2.线圈在运动过程中,什么时候有感生电流,什么时候没有感生电流?为什么?
\begin{figure}[htp]
\centering
\begin{minipage}[t]{0.48\textwidth}
\centering
\includegraphics{fig/2-8.pdf}
\caption{}
\end{minipage}
\begin{minipage}[t]{0.48\textwidth}
\centering
\includegraphics{fig/2-9.pdf}
\caption{}
\end{minipage}
\end{figure}


    \item 矩形线圈$ABCD$位于通电直长导线附近(图2.9),线圈跟导线同在一个平面内,且线圈的两个边与导线平行.在这个平面内,线圈远离导线平动时,线圈中有没有感生电流?线圈和导线都不动,当导线中的电流$I$逐渐增大或减小时,线圈中有没有感生电流?为什么?
    \item 把一个铜环放在匀强磁场中,使环的平面与磁场的方向垂直(图2.10甲).如果使环沿着磁场的方向移动,铜环
中是否产生感生电流?为什么?如果磁场是不均匀的(图2.10乙),是否产生感生电流?为什么?
\begin{figure}[htp]\centering
\includegraphics{fig/2-10.pdf}
\caption{}
\end{figure}
\end{enumerate}


\section{感生电流的方向~~楞次定律}
在前一节的实验中,电流表的指针有时向右偏转,有时向左偏转,表示在不同情况下感生电流的方向是不同的.怎样来确定感生电流的方向呢?

现在我们利用图2.2的实验来研究这个问题.

前一节我们利用磁通量的概念概括出了产生感生电流的条件,由此自然地想到,也要利用磁通量的概念来表达确定感生电流方向的规律,当把磁铁的$N$极移近或插入螺线管时(图2.11甲),穿过螺线管的磁通量增加,从实验知道,这时感生电流的磁场方向跟磁铁的磁场方向相反,阻碍原来磁通量的增加.当磁铁的$N$极离开螺线管或者从中拔出时(图2.11乙),穿过螺线管的磁通量减少,从实验知道,这时感生电流的方向跟图甲中的方向相反,它的磁场方向跟磁铁的磁场方向相同,阻碍原来磁通量的减少,在其他电磁感应现象中,也有相同的规律,凡是由磁通量的增加引起的感生电流,它所
激发的磁场就阻碍原来磁通量的增加;凡是由磁通量的减少引起的感生电流,它所激发的磁场就阻碍原来磁通量的减少.
\begin{figure}[htp]\centering
\includegraphics[scale=.6]{fig/2-11.png}
\caption{}
\end{figure}

德国物理学家楞次(1804—1865)概括了各种实验结果,在1834年得到如下结论:

\textbf{感生电流具有这样的方向,就是感生电流的磁场总要阻碍引起感生电流的磁通量的变化},这就是\textbf{楞次定律}.

我们知道,通电螺线管相当于条形磁铁,也有两个磁极.如图2.11甲所示,当磁铁的$N$极移近螺线管时,利用安培定则可以知道,这时螺线管的上端是$N$极,因而磁铁受到推斥,阻碍磁铁相对于螺线管的运动,如图2.11乙所示,当磁铁的$N$极离开螺线管时,利用安培定则可以知道,这时螺线管的上端是$S$极,因而磁铁受到吸引,也要阻碍磁铁相对于螺线管的.运动,总之,楞次定律的内容是:从磁通量变化的角度来看,感生电流总要阻碍磁通量的变化;从导体和磁场的相对运动的角度来看,感生电流总要阻碍相对运动.

利用楞次定律可以判断各种情况下感生电流的方向.

\section{楞次定律的应用}
应用楞次定律来判断感生电流的方向,首先要明确原来磁场的方向,以及穿过闭合电路的磁通量是增加还是减少,然后根据楞次定律确定感生电流的磁场方向,最后利用安培定则来确定感生电流的方向,下面用几个例子来说明怎样应用楞次定律.

\subsection*{应用之一}


现在来确定磁铁的$S$极移近或离开螺线管时感生电流的方向,如图2.12甲所示,把磁铁的$S$极移近螺线管
时,原来的磁场方向向上,穿过螺线管的磁通量增加,从楞次定律知道,感生电流要阻碍磁通量的增加,因此感生电流的磁场方向跟原来的磁场方向相反,即感生电流的磁场方向是向下的,如图中虚线所示.知道了感生电流的磁场方向,利用安培定则就可以确定感生电流的方向.如图2.12乙所示,当磁铁的$S$极离开螺线管时,原来的磁场方向向上,穿过螺线管的磁通量减少,从楞次定律知道,感生电流要阻碍磁通量的减
少,因此感生电流的磁场方向跟原来的磁场方向相同,方向也是向上的,如图中虚线所示,知道了感生电流的磁场方向,利用安培定则就可以确定感生电流的方向.
\begin{figure}[htp]\centering
\includegraphics[scale=.6]{fig/2-12.png}
\caption{}
\end{figure}

在图甲中,螺线管的上端是$S$极,磁铁移近时受到推斥.在图乙中,螺线管的上端是$N$极,磁铁离开时受到吸引,感生电流总要阻碍磁铁和螺线管的相对运动.

\subsection*{应用之二}


现在来确定图2.3中感生电流的方向,合上电键给螺线管$A$通电时,或者减小变阻器的电阻,使螺线管A中的电流强度增大时,穿过螺线管$B$的磁通量增加(图2.13甲).设螺线管$A$中的电流沿着顺时针方向流动,因而原来的
磁场方向是向下的,如图中所示.从楞次定律知道,感生电流要阻碍磁通量的增加,因此螺线管$B$中感生电流的磁场方向跟$A$的磁场方向相反,即磁力线的方向是向上的,由此可以知道,感生电流在$B$中是沿着反时针方向流动的,打开电键使$A$断电时,或者增大变阻器的电阻时,$B$中感生电流是沿着顺时针方向流动的,如图2.13乙所示,这种情形请同学们自己利用
楞次定律来判断.
\begin{figure}[htp]\centering
\begin{minipage}[t]{0.48\textwidth}
\centering
\begin{tikzpicture}[>=latex,scale=.8]
\draw (0,0) circle (1);
\draw (0,0) circle (1.1);
\draw (0,0) circle (2);
\draw (0,0) circle (2.1);
\foreach \x in {-.5, 0, .5}
\foreach \y in {-.35,.35}
{
   \node at (\x,\y){$\times$};
}

\draw[->] (30:1.3) node [above]{$A$} arc (30:-30:1.3);
\draw[->] (-30:2.3) arc (-30:30:2.3);
\draw[->] (210:1.3) arc (210:150:1.3);
\draw[->] (150:2.3) arc (150:210:2.3) node [below]{$B$};

\end{tikzpicture}
\caption*{甲}
\end{minipage}
\begin{minipage}[t]{0.48\textwidth}
\centering
\begin{tikzpicture}[>=latex,scale=.8]
\draw (0,0) circle (1);
\draw (0,0) circle (1.1);
\draw (0,0) circle (2);
\draw (0,0) circle (2.1);
\foreach \x in {-.5, 0, .5}
\foreach \y in {-.35,.35}
{
   \node at (\x,\y){$\times$};
}

\draw[->] (30:1.3) node [above]{$A$} arc (30:-30:1.3);
\draw[->] (30:2.3) arc (30:-30:2.3);
\draw[->] (210:1.3) arc (210:150:1.3);
\draw[->] (210:2.3) node [below]{$B$} arc (210:150:2.3) ;

\end{tikzpicture}
\caption*{乙}
\end{minipage}
\caption{这是图2.3所示实验的俯视图,图中只画出了$A$中电流的磁场方向}
\end{figure}

\subsection*{应用之三} 
\begin{figure}[htp]\centering
\includegraphics[scale=1.2]{fig/2-14.pdf}
\caption{右手定则}
\end{figure}

现在来确定图2.14中感生电流的方向,这种情形可以用初中学过的\textbf{右手定则}来判断(图2.14):伸开右手,让拇指跟其余四指垂直,并且都跟手掌在一个平面内,让磁力线垂直从手心进入,拇指指向导体运动的方向,共余四指指的就是感生电流的方向,在图2.1的实验中,当导体$AB$向右运动时,用右手定则判断的结果是:感生电流是由$A$流向$B$.现在用楞次定律来判断,导体$AB$向右运动时,穿过闭合电路的磁通量减少,从楞次定律知道,感生电流要阻碍磁通量的减少,因此感生电流的磁场方向跟磁铁的磁场方向相同,即磁力线的方向也是向下的,利用安培定则可以知道,感生电流的方向是由$A$流向$B$的.可见,用楞次定律判定的感生电流的方向跟用右手定则判定的结果是一致的.右手定则可以看作是楞次定律的特殊情况,对于闭合电路中一部分导体切割磁力线而产生感生电流的情形,用右手定则来判断感生电流的方向往往比用楞次定律简便.

\subsection*{练习二}

\begin{enumerate}
    \item 在图2.9中,当线圈远离通电导线而去时,线圈中感生电流的方向如何?
    \item 如图2.15所示,导线$AB$和$CD$互相平行.试确定
    在闭合和断开开关时导线$CD$中感生电流的方向.
  \item 在图2.16中$CDEF$是金属框,当导体$AB$向右移动时,试确定$ABCD$和$ABFE$ 两个电路中感生电流的方向.应用楞次定律,我们能不能用这两个电路中的任意一个来判定导体$AB$中感生电流的方向?
  \begin{figure}[htp]
\centering
\begin{minipage}[t]{0.48\textwidth}
\centering
\begin{circuitikz}[>=latex, yscale=.5]
\draw (0,0) to [battery2] (2,0) to [cute open switch](4,0)--(5,0)--(5,2)--(0,2)--(0,0);
\draw (0,3)--(5,3)--(5,5) to [rmeter, t=G](0,5)--(0,3);
\node at (-.25, 2){$A$};
\node at (-.25, 3){$C$};
\node at (5.25, 2){$B$};
\node at (5.25, 3){$D$};
\node at (3, -.5){$K$};
\end{circuitikz}
\caption{}
\end{minipage}
\begin{minipage}[t]{0.48\textwidth}
\centering
\begin{circuitikz}[>=latex, yscale=.8]
\draw (0,0)--(5,0) to [rmeter, t=G] (5,4)--(0,4) to [rmeter, t=G] (0,0);

\foreach \x in {.5,1.5,...,4.5}
\foreach \y in {.5,1.5,2.5,3.5}
{
   \node at  (\x,\y) {$\times$};
}

\node at (0,-.5){$E$};
\node at (2.7,-.5){$A$};
\node at (5,-.5){$D$};
\node at (0,4.5){$F$};
\node at (2.7,4.5){$B$};
\node at (5,4.5){$C$};
\draw [fill=white](2.7,4.25) rectangle (2.9,-.25);
\draw [->](2.9, 2)--(3.8,2)node[right]{$v$};

\end{circuitikz}
\caption{}
\end{minipage}
\end{figure}
  \item 在图2.17所示的电路中把滑动变阻器$R$的滑动片向左移动使电流减弱.试确定这时线圈$A$和$B$中感生电流的方向.
  \item 如图2.18所示,把一个备形磁铁从闭合螺线管的右端插入,由左端抽出,在整个过程中,螺线管里产生的感生电流的方向是否发生改变?
\begin{figure}[htp]\centering
\includegraphics[scale=.6]{fig/2-17.png}
\caption{}
\end{figure}

\begin{figure}[htp]\centering
\begin{circuitikz}[>=latex]
\ctikzset{inductors/scale=2.3,  inductors/coils=6}
\draw (0,0)--(0,1) to [L] (2,1)--(2,0)--(0,0);
\draw [fill=red!70!white](2.25, .8) rectangle (3.5, 1.4);
\draw [fill=blue!70!white](3.5, .8) rectangle (4.75, 1.4);
\node at (2.75, 1.1){$N$};
\node at (4.25, 1.1){$S$};
\end{circuitikz}
\caption{}
\end{figure}


  设想存在着一种粒子,它只有一个磁极,比如$N$极(磁单
  极子),它的磁力线分布情况是什么样?那么,当磁单极子穿过螺线管时,感生电流的方向是否发生改变?
  \item 图2.19中的$A$和$B$都是很轻的铝环,环$A$是闭合的,环$B$是断开的.用磁铁的任一极来接近$A$环,会产生什么现象?把磁铁从$A$环移开,会产生什么现象?磁极移近或远离
  $B$环时,又会产生什么现象?用所学的知识解释这些现象.
\begin{figure}[htp]\centering
\includegraphics[scale=.6]{fig/2-19.png}
\caption{}
\end{figure}
\end{enumerate}

\section{法拉第电磁感应定律}

我们知道,要使闭合电路中有电流通过,这个电路中必须有电源,电流就是由电源的电动势产生的.在电磁感应现象中,既然闭合电路中有感生电流,这个电路中就一定有电动势,在电磁感应现象中产生的电动势叫做\textbf{感生电动势}.产生感生电动势的那部分导体就相当于电源.

在电磁感应现象中,不管电路是否闭合,只要穿过这个电路所围面积的磁通量发生变化,电路中就有感生电动势.如果电路是闭合的,电路里就有感生电流,感生电流的强弱决定于感生电动势的大小和电路的电阻,如果电路是断开的,电路中就没有感生电流,但感生电动势仍然存在.那么,感生电动势的大小跟什么有关呢?

在图2.1所示的实验中,导体$AB$切割磁力线的速度越大,穿过闭合电路所围面积的磁通量的变化就越快,感生电流和感生电动势就越大,在图2.2的实验中,磁铁运动得越快,穿过螺线管的磁通量的变化就越快,感生电流和感生电动势就越大.在图2.3的实验中,通电和断电时,比起逐渐改变电阻器的电阻时,$A$中电流变化得快,因而穿过$B$的磁通量变化得也快,$B$中的感生电流和感生电动势就比较大.因此实验表明:感生电动势的大小与磁通量变化的快慢有关.磁通量变化的快慢可以用单位时间内磁通量的变化来表示,单位时间内磁通量的变化量通常叫做磁通量的变化率.这就是说,感生电动势的大小跟磁通量的变化率有关.

精确的实验表明:\textbf{电路中感生电动势的大小,跟穿过这一电路的磁通量的变化率成正比},这就是\textbf{法拉第电磁感应定律}.

感生电动势是有方向的,电路中感生电流的方向不同,就是由于感生电动势的方向不同而引起的.感生电动势的方向跟感生电流的方向是一致的,也可以由楞次定律来判定.

设时刻$t_1$时穿过闭合电路的磁通量为$\phi_1$,时刻$t_2$时穿过闭合电路的磁通量为$\phi_2$,那么,在时间$\Delta t=t_2-t_1$内磁通量的变化量为$\Delta \phi=\phi_2-\phi_1$,磁通量的变化率为$\Delta \phi/\Delta t$.根据法拉第电磁感应定律,闭合电路中的感生电动势为
\[\mathcal{E}=k\frac{\Delta \phi}{\Delta t}\]
其中$k$为比例常数.在国际单位制中,上式中各量的单位都已确定:$\mathcal{E}$的单位是伏特,$\phi$的单位是韦伯,$t$的单位是秒.同学
们可以自己证明$1{\rm V}=1{\rm Wb}/{\rm s}$,所以上式中的$k=1$.这样,上式可写成
\begin{equation}
    \mathcal{E}=\frac{\Delta \phi}{\Delta t}
\end{equation}

如果闭合电路是一个$n$匝线圈,那么,由于穿过每匝线圈的磁通量变化率都相同,而$n$匝线圈可以看作是由$n$个单匝线圈串联而成的,因此整个线圈中的感生电动势就是单匝线圈的$n$倍,即
\begin{equation}
    \mathcal{E}=n\frac{\Delta \phi}{\Delta t}
\end{equation}
在实际工作中,为了获得较大的感生电动势,常常采用多匝线圈.

现在我们根据法拉第电磁感应定律来研究导体做切割磁力线运动时,感生电动势的大小.如图2.20所示,我们把矩形线框$abcd$放在匀强磁场里,线框平面跟磁力线垂直.让线框的可动部分$ab$以速度$v$向右运动,设在$\Delta t$时间内由原来的位置$ab$移到$a_1b_1$. 设$ab$的长度是$\ell$,这时线框的面积变化量$\Delta S=\ell v\Delta t$,穿过闭合电路的磁通量变化量
\[\Delta \phi=B\Delta S=B\ell v\Delta t\]
代入公式$\mathcal{E}=\Delta \phi/\Delta t$中,得到
\begin{equation}
    \mathcal{E}=B\ell v
\end{equation}
\begin{figure}[htp]
\centering
\begin{minipage}[t]{0.48\textwidth}
\centering
\begin{circuitikz}[>=latex, scale=.7]
\draw (0,0)--(6.5,0);
\draw (0,0) to [rmeter, t=G] (0,5) --(6.5,5);

\foreach \x in {1,2,...,6}
\foreach \y in {1,...,4}
{
   \node at (\x,\y){$\times$};
}

\draw [fill=white](3.5,-.15) rectangle (3.5+.2,5.15);
\draw [fill=white, dashed](5.5,-.15) rectangle (5.5+.2,5.15);

\node at (0,5.5){$d$};
\node at (0,-.5){$c$};
\node at (3.5,5.5){$a$};
\node at (3.5,-.5){$b$};
\node at (5.5,5.5){$a_1$};
\node at (5.5,-.5){$b_1$};

\end{circuitikz}
\caption{}
\end{minipage}
\begin{minipage}[t]{0.48\textwidth}
\centering
\begin{circuitikz}[>=latex]
\foreach \x in {.5,2,3.5}
{ 
   \draw[->] (\x, 4)--(\x, 0);
}
\node at (2.4, 0){$B$};
\draw[->] (1.5, 3)--(2.5,3) node[right]{$v_1$};
\draw[->] (1.5,3)--(1.5, 1)node[below]{$v_2$};
\draw[->] (1.5,3)--(2.5,1)node[right]{$v$};
\draw [fill=white] (1.5, 3) circle (1.5pt);
\node at  (1.6,2.5){$\theta$};

\end{circuitikz}
\caption{}
\end{minipage}
\end{figure}

如果导线的运动方向跟导线本身垂直,但跟磁力线方向有一个夹角$\theta$(图2.21),我们可以把速度$v$分解为两个分量:垂直于磁力线的分量$v_1=v\sin\theta$,平行于磁力线的分量$v_2=v\cos\theta$.后者不切割磁力线,不产生感生电动势.前者切割磁力线,产生的感生电动势为
$\mathcal{E}=B\ell v_1$
而$v_1=v\sin\theta$,所以
\begin{equation}
    \mathcal{E}=B\ell v\sin\theta
\end{equation}

可见,\textit{导线切割磁力线时产生的感生电动势的大小,跟磁感应强度$B$、导线长度$\ell$、运动速度$v$以及运动方向和磁力线方向的夹角$\theta$的正弦$\sin\theta$成正比}.

在国际单位制中,(2.3)和(2.4)两式中的$\mathcal{E}$、$B$、$\ell$、$v$的单位分别用V、T、m、$\ms$.同学们可以自己证明,公式等号两边的单位是一致的,即$1{\rm V}=1{\rm T}\x1{\rm m}\x1\ms$.

\begin{example}
    在图2.20中,设匀强磁场的磁感应强度$B=0.1$T,导体$ab$的长度$\ell=40$cm,向右匀速运动的速度$v=5.0\ms$,框架的电阻不计,导体$ab$的电阻$R=0.5\Omega$.试求:
    \begin{enumerate}
        \item 感生电动势和感生电流的大小;
        \item 感生电流和感生电动势的方向.
    \end{enumerate}
\end{example}

\begin{solution}
\begin{enumerate}
    \item 线框中的感生电动势
\[\mathcal{E}=Blv=0.1\x0.4\x5.0=0.2{\rm V}\]    
线框中的感生电流
\[I=\frac{\mathcal{E}}{R}=\frac{0.2}{0.5}=0.4{\rm A}\]

\item 利用楞次定律或右手定则,都可以确定出线框中的感
生电流方向是沿反时针方向流动的,在导体$ab$中是由$b$指向$a$. 导体$ab$中的感生电动势的方向和感生电流的方向一致,也是由$b$指向$a$.
\end{enumerate}
\end{solution}
   
\section*{阅读材料:寻找磁单极子}
人们早就发现电和磁有很多相似之处.例如,带电体的周围有电场,磁体的周围有磁场.同种电荷互相推斥,异种电荷互相吸引;同名磁极互相推斥,异名磁极互相吸引,然而尽管电与磁有这样多的相似之处,它们却不是完全相似的.在电现象里,有电荷,正、负电荷可以单独存在,在磁现象里却没有发现磁荷,南北极也不能单独存在.一块磁体,无论把它
分得多么小,总是有南极和北极.

但是,1931年,著名的英国物理学家狄拉克从理论上预言了存在着只有一个磁极的粒子——磁单极子.根据磁单极子的理论,电和磁之间的相似将更加完美.理论的动人前景,吸引了一批物理学家,用各种方法,在岩石中,在宇宙射线(即从字宙空间飞来的粒子)中,在加速器实验中,去寻找磁单极子.但是,半个世纪的时间过去了,并没有找到磁单极子.因此,人们推测,磁单极子可能是在宇宙形成初期产生的,残存下来的为数较少,而且分散在广漠的宇宙之中,要找到它不是很容易的.
\begin{figure}[htp]\centering
\begin{tikzpicture}[>=latex]
    \draw [dashed](-2,2) --(2,-2) ;
    \node at (-2.15,2.15){\Sun};    \node at (2.15,-2.15){\Sun};
    \draw [fill=white] (0,0) ellipse [x radius=2.1, y radius=2.1*.4];
\draw [fill=white] (0,0) ellipse [x radius=2, y radius=.8];
\draw [dashed] (-2,2) --(0,0) ;
\end{tikzpicture}
\caption{}
\end{figure}

1982年,美国的一位物理学家卡布莱拉宣布,在他的实
验仪器中通过了一个磁单极子.他的实验所根据的原理就是电磁感应现象,仪器的主要部分是一个由超导体做成的线圈.
超导体的电阻为零,一个很微小的电动势就可以在超导线圈中引起感生电流,而且这个电流将长期维持下去,并不减弱.设想有一个磁单极子穿过超导线圈(图2.22),穿过超导线圈的磁通量将发生改变,而且引起的感生电动势的方向不变,于
是在超导线圈中将引起稳定的电流.1982年2月14日,这位物理学家发现在超导线圈中出现了稳定的电流,经过周密分析,实验所得的数据跟磁单极子理论符合得很好,因而认定这是磁单极子穿过了超导线圈,不过由于以后没有重复观察到那次实验中观察到的现象,所以这一事例还不能确证磁单极子的存在.

目前,寻找磁单极子的实验还在进行中,有关磁单极子的理论,探讨得更深入了,如果磁单极子确实存在,现在的电磁理论就要做重大的修改,对整个物理学基础理论的发展,也将产生重大的影响.

\subsection*{练习三}

\begin{enumerate}
    \item 下列说法哪个正确?
    \begin{enumerate}
        \item 电路中感生电动势的大小,跟穿过这一电路的磁通量成正比;
        \item 电路中感生电动势的大小,跟穿过这一电路的磁通量的变化量成正比;
        \item 电路中感生电动势的大小,跟穿过这一电路的磁通量的变化率成正比;
        \item 电路中感生电动势的大小,跟单位时间内穿过这一电路的磁通量的变化量成正比.
    \end{enumerate}
\item 试证明:$1{\rm V}=1{\rm Wb}/{\rm s}$;$1{\rm V}=1{\rm T}\cdot 1{\rm m}\cdot 1\ms$.
\item  长5cm的导线在0.02T的匀强磁场中运动,运动的方向跟磁力线垂直,运动的速率$v=0.1\ms$,求感生电动势.
\item 在0.4T的匀强磁场中,长度为25cm的导线以6$\ms$的速率做切割磁力线的运动,运动方向跟磁力线成$30^{\circ}$角,并跟导线本身垂直.求感生电动势.
\item 50匝的线圈,穿过它的磁通量的变化率为0.5${\rm Wb}/{\rm s}$,求感生电动势.
\item 有一个1000匝的线圈,在0.4秒内穿过它的磁通量从0.02韦增加到0.09韦,求线圈中的感生电动势,如果线圈的电阻是10欧,把它跟一个电阻为990欧的电热器串联组成闭合电路时,通过电热器的电流是多大?
\item 图2.23是法拉第做成的世界上第一个发电机模型
的原理图.把一个铜盘放在磁场里,使磁力线垂直穿过铜盘;转动铜盘,就可以获得持续的电流.试解释其作用原理.
\end{enumerate}

\begin{figure}[htp]\centering
\includegraphics[scale=.6]{fig/2-23.png}
\caption{}
\end{figure}

\section{电磁感应现象中能量的转化}

在电磁感应现象中,产生感生电动势的那部分导体相当于电源.图2.1中的导体$AB$,图2.2中的螺线管,图2.3中的螺线管$B$,对外电路来说,都相当于电源.如果使外电路闭合,在感生电动势的作用下就有了感生电流.这时电流做功消耗了电能,我们知道,能量不能无中生有,只能从一种形式转化成另一种形式,各种电源都是把其他形式的能转化成电能的装置,电磁感应现象中的电能是怎样转化而来的呢?

在图2.2所示的实验中,按照楞次定律,感生电流总要阻碍磁铁相对于螺线管的运动.把磁铁移近螺线管时,磁铁受到斥力,必须有外力克服这种斥力做功,才能把磁铁移近.让磁铁离开螺线管时,磁铁受到引力,也必须有外力克服这种引
力做功,才能使磁铁离开.外力克服这种电磁的引力或斥力做功的过程中,外部的机械能就转化为感生电流的电能.在图2.1中,导线$ab$切割磁力线运动时,按照楞次定律,感生电流也要阻碍导线相对于磁铁的运动.例如当导线$ab$向右运动时,感生电流在磁场中所受的安培力,方向是向左的,外力克服安培力做功的过程中,外部的机械能就转化为感生电流的电能,从这里我们也看到,楞次定律跟能的转化和守恒定律是相符的,我们在初中学过,发电机是利用电磁感应现象制成的.发电机就是把机械能转化成电能的装置.

现在让我们进一步定量地从能量守恒的观点来研究图2.20的实验,并由此推出上一节给出的公式(2.3).当导线$ab$向右运动时,感生电流$I$在磁场中所受的安培力,方向是向左的,大小为$F=I\ell B$.为了保持导线$ab$匀速向右运动,加在导线$ab$上向右的外力,大小应该跟安培力$F$相等.在$\Delta t$时间内外力克服安培力所做的功为
\[W=Fv\Delta t=I\ell Bv\Delta t\]
设感生电动势为$\mathcal{E}$,在时间$\Delta t$内感生电流所做的功为
\[W'=\mathcal{E}I\Delta t\]
根据能量守恒定律,这两个功一定相等,因此
\[\mathcal{E}I\Delta t=I\ell B v \Delta t\]
由此就得到上节给出的公式(2.3):
\[\mathcal{E}=B\ell v\]

可见,法拉第电磁感应定律跟能的转化和守恒定律也是相符的.

\begin{example}
继续做上一节的例题,求
\begin{enumerate}
    \item 使导体$ab$向右匀速运动所需的外力;
    \item 外力做功的功率;
    \item 感生电流的功率.
\end{enumerate}
\end{example}

\begin{solution}
    \begin{enumerate}
        \item 外力跟感生电流所受的安培力平衡,因此外力的大小为
        \[F=I\ell B=0.4\x0.4\x0.1=0.016{\rm N}\]
        \item 外力做功的功率为
        \[P=Fv=0.016\x5.0=0.08{\rm W}\]
        \item 感生电流的功率为
        \[P'=\mathcal{E}I=0.2\x0.4瓦=0.08{\rm W}\]
    \end{enumerate}
\end{solution}

我们看到,$P=P'$,这正是能量守恒定律所要求的.线框是纯电阻电路,电流的功全部用来生热,所以热功率$I^2R$也应该等于$P$或$P'$.简单的计算指出,情况正是这样的.

\subsection*{练习四}
\begin{enumerate}
    \item 在图2.20中,以一定速度向右拉动导体$ab$,如果导体$ab$的电阻增大,在相同的时间里,外力做的功是增大还是减小?为什么?
    \item 在图2.20中,是快拉还是慢拉所需的功率多?为什么?
    \item 在图2.20中,导体$ab$刚开始受到恒力拉动时,它做什么运动?随后它的运动情况将如何变化?感生电动势和感生电流将如何变化?外力做的功产生了哪些效果?如果在导体$ab$还没拉出磁场时停止外力,情况又将怎样?(不考虑摩擦和空气阻力)
\end{enumerate}

\section{直流电动机的反电动势}
\subsection{直流电动机}

我们在初中已经学过直流电动机的构造原理,这里简单复习一下.电动机是把电能转化为机械能的装置,用直流电源供电的电动机叫做直流电动机.直流电动机是利用通电线圈在磁场里转动的原理制成的.我们知道,通电线圈在磁场中要受到力矩的作用而发生转动,但线圈中通入一定方向的电流时,线圈转到它的平面跟磁力线垂直的平衡位置时,就会停下来.因此,直流电动机中装有换向器(图2.24),它由两个铜制的半环组成.有了换向器,线圈转到平衡位置时可以自动改变线圈里的电流方向,线圈就可以不停地转动下去了.
\begin{figure}[htp]\centering
\includegraphics[scale=.6]{fig/2-24.png}
\caption{直流电动机的模型}
\end{figure}

\subsection{直流电动机的反电动势}

电动机的线圈在磁场里转动时,线圈导线切割磁力线,所以在线图中必然要产生感生电动势,由楞次定律知道,这个感生电动势的方向跟使线圈转动的电流方向相反,因此也跟外加电压的方向相反,通常把这个
电动势叫做\textbf{反电动势}.电动机的线图转动得越快,线圈导线切割磁力线越快,反电动势就越大,显然,只有线圈转动时,才有反电动势.

如果用$\mathcal{E}$表示反电动势,$U$表示外加电压,$R$表示线圈的电阻,那么,电动机工作时的电流强度$I$就是
\begin{equation}
    I=\frac{U-\mathcal{E}}{R}
\end{equation}
(2.5)式可以改写为
\begin{equation}
    U=\mathcal{E}+IR
\end{equation}
上式表明:电路中有反电动势存在时,加在电路两端的电压$U$等于反电动势$\mathcal{E}$跟线圈电阻上损失的电压$IR$之和.

如果用$I$乘(2.6)式的各项就得到
\begin{equation}
    UI=\mathcal{E}I+I^2R
\end{equation}
上式中的$UI$是电路供给电动机的功率(输入功率),$\mathcal{E}I$是转化为机械能的功率(输出功率),$I^2R$是在电动机线圈上损失的热功率.上式表示,电路供给电动机的功率等于转化为机械能的功率与线圈上损失的热功率之和,可见能的转化和守恒定律对直流电动机是完全适用的.从这里我们也看到电功和电热的区别,在直流电动机中,由于存在着反电动势,有一部分电能转化为机械能,电功并不等于电热,而是大于电热.

在外加电压$U$一定的情况下,电动机的输入功率并不是在任何情况下都相同,电动机带动负载工作时,线圈受到两个力矩的作用.一个是通电线圈在磁场中受到的力矩,它使线圈转动,叫做动力矩;一个是负载和机械摩擦所造成的力矩,它阻碍线圈转动,叫做阻力矩.当动力矩和阻力矩平衡
时,电动机匀速转动.当负载增加时阻力矩增大,线圈的转速减小,反电动势随着减小.从(2.5)式知道,这时电流强度增大;从(2.7)式知道,这时输入功率也增大.可见,电动机的输入功率是随着负载的增加而增大的.线圈中的电流强度增大时,电动机的动力矩也增大,当动力矩增大到和阻力矩平衡时,电动机就在较小的转速下重新做匀速转动.

\section{自感}
在电磁感应现象中,有一种叫做自感现象的特殊情形,现在来研究这种现象.

\subsection{自感现象}

在图2.25所示的实验中,先合上开关$K$,调节变阻器$R$的电阻,使同样规格的两个灯泡$A_1$和$A_2$的明亮程度相同,再调节变阻器$R_1$使两个灯泡都正常发光,然后断开开关$K$.
\begin{figure}[htp]\centering
\begin{circuitikz}[>=latex,european]
\draw (2,0) to [battery] (0,0)--(0,3)--(.5,3);
\draw (2,0) to [cute open switch](3,0)--(4,0) to [R] (5.1,0);

\node at (2.5,-.5){$K$};
\node at (4.5,-.5){$R_1$};
\draw [american] (.5,3)--(.5,3.7) to [cute choke,l=$L$] (2.5,3.7) to [lamp] (4.5,3.7);
\draw [->](.5,3)--(.5,3-.7)--(1.5,2.3)--(1.5,1.8);

\draw  (1,1.6) to [R] (2.1,1.6)  to [lamp] (4.5,1.6)--(4.5,3.7);
\draw [->](4.5,5.3/2)--(5.5,5.3/2)--(5.5, 1)--(4.5,1)--(4.5,.2);

\node at (3.3,2.3){$A_2$};
\node at (3.5,4.3){$A_1$};  \node at (1.5,1.2){$R$};
\draw [fill=black](4.5,5.3/2) circle(1.5pt);
\draw [fill=black](.5,3)circle(1.5pt);
\end{circuitikz}
\caption{}
\end{figure}

再接通电路时可以看到,跟变阻器$R$串联的灯$A_2$立刻正常发光,而跟有铁芯的线圈$L$串联的灯$A_1$却是逐渐亮起来的,为什么会出现这样的现象呢?原来,在接通电路的瞬间,电路中的电流增大,穿过线圈$L$的磁通量也随着增加,根据电磁感应定律,线圈中必然会产生感生电动势,这个感生电动势阻碍线圈中电流的增大.所以通过$A_1$的电流只能逐渐增大,灯$A_1$只能逐渐亮起来.
\begin{figure}[htp]\centering
\begin{circuitikz}[>=latex, yscale=.7]
\draw (4,3) -- (5,3) -- (5,0) to [cute open switch] (2.5,0) to [battery] (0,0)--(0,3)--(1,3);
\node at (7.5/2,.5){$K$};
\draw (1,3)--(1,3.75) to [lamp] (4,3.75)--(4,3);
\draw (1,3)--(1,2.25) to [L] (4,2.25)--(4,3);
\node at (2.5,1.6){$L$};
\node at (2.5,4.8){$A$};
\draw [<-](2,3.75)--(1.5,3.75)node [above]{$I$};
\draw [->](2,2.25)--(1.5,2.25);
\end{circuitikz}
\caption{}
\end{figure}

现在再来做图2.26的实验,把灯泡$A$和带铁芯的电阻
较小的线圈$L$并联接在直流电路里,接通电路,灯$A$正常发光后,再断开电路,这时可以看到,灯$A$要过一会儿才熄灭,为什么会出现这种现象呢?这是由于电路断开的瞬间,通过线圈的电流突然减弱,穿过线圈的磁通量也就很快地减少,因而在线圈中产生感生电动势.虽然这时电源已经断开,但线圈$L$和灯泡$A$组成了闭合电路,在这个电路中有感生电流$I$通过,所以灯泡不会立即熄灭.

从上述两个实验可以看出,当导体中的电流发生变化时,导体本身就产生感生电动势,这个电动势总是阻碍导体中原来电流的变化的.这种由于导体本身的电流发生变化而产生的电磁感应现象,叫做\textbf{自感现象}.在自感现象中产生的感生电动势,叫做\textbf{自感电动势}.

\subsection{自感系数}
自感电动势跟其他感生电动势一样,是跟穿
过线圈的磁通量的变化率$\Delta\phi/\Delta t$成正比的,我们知道,磁通量$\phi$跟磁感应强度$B$成正比,$B$又跟产生这个磁场的电流$I$成正比.所以$\phi$跟$I$成正比,$\Delta \phi$跟$\Delta I$也成正比,由此可知自感
电动势$\mathcal{E}=\Delta\phi/\Delta t$跟$\Delta I/\Delta t$成正比,即
\[\mathcal{E}=L\frac{\Delta I}{\Delta t}\]
式中的比例恒量$L$叫做线圈的\textbf{自感系数},简称自感或电感,它
是由线圈本身的特性决定的.线圈越长,单位长度上的匝数越多,截面积越大,它的自感系数就越大,另外,有铁芯的线图的自感系数,比没有铁芯时要大得多.对于一个现成的线圈来说,自感系数是一定的.

自感系数的单位是\textbf{亨利},简称亨,国际符号是H.如果通过线圈的电流强度在1秒钟内改变1安时产生的自感电动势是1伏,这个线圈的自感系数是1亨,所以
\[1{\rm H}=1{\rm V}\cdot {\rm s}/{\rm A}\]

常用的较小单位有毫亨(mH)和微亨($\mu$H).
\[1{\rm mH}=10^{-3}{\rm H},\qquad 1\mu{\rm H}=10^{-6}{\rm H}  \]

\section{自感现象的应用}
自感现象在各种电器设备和无线电技术中有广泛的应用.日光灯的镇流器就是利用线圈自感现象的一个例子.

图2.27是日光灯的电路图,它主要是由灯管、镇流器和起动器组成的,镇流器是一个带铁芯的线圈,起动器的构造
如图2.28所示,它是一个充有氖气的小玻璃泡,里面装有两个电极,一个固定不动的静触片和一个用双金属片制成的U形触片,灯管内充有稀薄的水银蒸气,当水银蒸气导电时,就发出紫外线,使涂在管壁上的荧光粉发出柔和的白光.由于激发水银蒸气导电所需的电压比220伏的电源电压高得多,因此,日光灯在开始点燃时需要一个高出电源电压很多的瞬时电压,在日光灯点燃后正常发光时,灯管的电阻变得很小,只允许通过不大的电流,电流过强就会烧毁灯管,这时又要使加在灯管上的电压大大低于电源电压.这两方面的要求都是利用跟灯管串联的镇流器来达到的.
\begin{figure}[htp]\centering
	\begin{minipage}[t]{0.48\textwidth}
		\centering
		\includegraphics[scale=.7]{fig/2-27.png}
		\caption{}
	\end{minipage}
	\begin{minipage}[t]{0.48\textwidth}
		\centering
		\includegraphics[scale=.7]{fig/2-28.png}
		\caption{}
	\end{minipage}
\end{figure}

当开关闭合后,电源把电压加在起动器的两极之间,使氖气放电而发出辉光,辉光产生的热量使U形触片膨胀伸长,跟静触片接触而把电路接通.于是镇流器的线圈和灯管的灯丝中就有电流通过,电路接通后,起动器中的氖气停止放电,U形触片冷却收缩,两个触片分离,电路自动断开.在电路突然中断的瞬间,在镇流器两端产生一个瞬时高电压,这个电压加上电源电压加在灯管两端,使灯管中的水银蒸气开始放电,于是日光灯管成为电流的通路开始发光,在日光灯正常发光时,由于交流电不断通过镇流器的线圈,线圈中就有自感电动势,它总是阻碍电流变化的,这时镇流器起着降压限流作用,保证日光灯的正常工作.

自感现象也有不利的一面.在自感系数很大面电流又很强的电路(如大型电动机的定子绕组)中,在切断电路的瞬间,由于电流强度在很短的时间内发生很大的变化,会产生很高的自感电动势,使开关的闸刀和固定夹片之间的空气电离而
变成导体,形成电弧,这会烧坏开关,甚至危及工作人员的安全.因此,切断这类电路时必须采用特制的安全开关.常见的安全开关是将开关放在绝缘性能良好的油中,防止电弧的产生,保证安全.

\subsection*{练习五}
\begin{enumerate}
    \item 制造电阻箱时,要用双线绕法,如图2.29所示,这样就可以使自感现象的影响减弱到可以略去的程度,为什么?
    \begin{figure}[htp]\centering
\includegraphics[scale=.6]{fig/2-29.png}
\caption{}
\end{figure}

    \item 有一个线圈,它的自感系数是1.2亨,当通过它的电流在$1/200$秒内由零增加到5.0安时,线圈中产生的自感电动势多大?
    \item 一个线圈的电流强度在0.001秒内有0.02安的变化时,产生50伏的自感电动势,求线圈的自感系数,如果这个电路中电流强度的变化率变为40${\rm A}/{\rm s}$,自感电动势是多大?
    \item 试证明$1{\rm H}=1\Omega\cdot {\rm s}$.
    \item 根据感生电动势$\mathcal{E}=\dfrac{\Delta \phi}{\Delta t}$和 $\mathcal{E}=L\dfrac{\Delta I}{\Delta t}$,证明$L=\dfrac{\Delta \phi}{\Delta I}$,
    并说明这个式子的物理意义.
\end{enumerate}

\section{涡流}

仔细观察发电机、电动机和变压器,可以看到它们的铁芯都不是整块金属,而是用许多薄的硅钢片叠合而成的,为什么要这样呢?

原来,把块状金属放在变化的磁场中,或者让它在磁场中
运动时,金属块内将产生感生电流.这种电流在金属块内白成闭合回路,很象水的旋涡,因此叫做涡电流,简称\textbf{涡流}.整块金属的电阻很小,所以涡流常常很强.

如图2.30所示,在块状铁芯上绕绝缘导线,当交流电通过导线时,穿过铁芯的磁通量不断变化,铁芯中就会产生如图中虚线所示的涡流.块状铁芯中的涡流很强,这将使铁芯大量发热,浪费大量的电能.因此用整块铁芯的电机和变压器,涡流损失很大,效率很低.

\begin{figure}[htp]\centering
	\begin{minipage}[t]{0.48\textwidth}
		\centering
    \includegraphics{fig/2-30.pdf}
		\caption{}
	\end{minipage}
	\begin{minipage}[t]{0.48\textwidth}
		\centering
    \includegraphics{fig/2-31.jpg}
\caption{用硅钢片做变压器的铁芯}
	\end{minipage}
\end{figure}


为了减少涡流损失,电机和变压器通常用涂有绝缘漆的薄硅钢片叠压制成的铁芯,如图2.31所示.这样涡流被限制在狭窄的薄片之内,回路的电阻很大,涡流大为减弱,从而使涡流损失大大降低.铁芯采用硅钢片,是因为这种钢比普通钢的电阻率大,可以进一步减少涡流损失.硅钢的涡流损失只有普通钢的$1/5$—$1/4$.

在各种电机、变压器中,涡流是有害的,我们要采取各种办法来减弱它,但是涡流也是可以利用的,下面举两个例子.

\begin{figure}[htp]
    \centering
    \includegraphics{fig/2-32.pdf}
    \caption{}
\end{figure}

图2.32是冶炼金属的高频感应炉的示意图,这种电炉就
是利用涡流来熔化金属的,冶炼锅内装入被冶炼的金属,让高频交流电通过线圈,被冶炼的金属中就产生很强的涡流,从而产生大量的热使金属熔化.这种冶炼方法速度快,温度容易控制,并能避免有害杂质混入被冶炼的金属中,因此适于冶炼特种合金和特种钢.

电学测量仪表要求指针的摆动很快停下来,以便可以迅速读出读数,我们在第一章讲的电流表的线圈要绕在铝框上,铝框就是起这个作用的,原来,当被测电流通过线圈时,线圈带动指针和铝框一起转动,铝框在磁场中转动时产生涡流,磁场对这个涡流的作用力阻碍它们的摆动,于是使指针很快地稳定指到读数位置上.

\subsection*{练习六}
\begin{enumerate}
    \item 如图2.33所示,$A$是可以在电磁铁两磁极间摆动的铝片(或铜片).电磁铁线围没有通电时,铝片可以摆动较长时间才停下来.电磁铁线圈通电时,铝片很快就会停下来.解释这个现象.
    
这种现象叫做电磁阻尼,在实际中有很多应用.课文中
讲的,使电学测量仪表的指针很快地停下来,就是电磁阻尼的应用.电磁阻尼还常用于电气机车的电磁制动器中.
\begin{figure}[htp]
\centering
\begin{minipage}[t]{0.48\textwidth}
\centering
\includegraphics[scale=.6]{fig/2-33.png}
\caption{电磁阻尼}
\end{minipage}
\begin{minipage}[t]{0.48\textwidth}
\centering
\includegraphics[scale=.6]{fig/2-34.png}
\caption{电磁驱动}
\end{minipage}
\end{figure}
\item 如图2.34所示,把蹄形磁铁的两极靠近一个金属圆盘,但不接触.当磁铁绕轴转动时,圆盘也绕轴转动起来,解释这个现象.

这种现象叫做电磁驱动,在实际中也有很多应用,下一章要讲的感应电动机就是利用这个道理驱动的.家庭中用的电度表,汽车上用的电磁式速度表,也利用这种电磁驱动.
\end{enumerate}

\section*{复习题}
\begin{enumerate}
    \item 什么是电磁感应现象?什么是感生电流?产生感生电流的条件是什么?
    \item 楞次定律的内容是什么?应用楞次定律判定感生电流方向的步骤是怎样的?怎样用右手定则来判断感生电流的方向?
    \item 什么是感生电动势?法拉第电磁感应定律的内容是什么?写出计算单匝线圈和$n$匝线圈的感生电动势的公式,写出导线切割磁力线运动时计算感生电动势的公式.
    \item 在电磁感应现象中能量是怎样转化的?为什么说楞次定律和法拉第电磁感应定律都跟能的转化和守恒定律是相符的?
    \item 什么是直流电动机的反电动势?在直流电动机中能的转化情况是怎样的?
    \item 什么是自感现象?什么是自感系数?写出计算自感电动势的公式.
    \item 什么是涡流?为什么发电机、电动机和变压器的铁芯要用许多相互绝缘的簿硅钢片叠合起来制成?
\end{enumerate}

\section*{习题}
\begin{enumerate}
    \item 在图2.35中,条形磁铁以速度$v$向螺线管靠近,下面哪种说法是正确的:

\begin{figure}[htp]
\centering
\begin{minipage}[t]{0.48\textwidth}
\centering
\includegraphics[scale=1.2]{fig/2-35.pdf}
\caption{}
\end{minipage}
\begin{minipage}[t]{0.48\textwidth}
\centering
\includegraphics[scale=.6]{fig/2-36.png}
\caption{}
\end{minipage}
\end{figure}

    \begin{enumerate}
\item 螺线管中不产生感生电流;
\item 螺线管中产生感生电流,方向如图2.35所示;
\item 螺线管中感生电流的方向与图中所示的方向相反.
    \end{enumerate}
    
    \item 在图2.36中,线圈$M$和线圈$P$绕在同一铁芯上.
    \begin{enumerate}
        \item 当合上电键$K$的一瞬时,线圈$P$里有没有感生电流?
        \item 当线圈$M$里有稳恒电流通过时,线圈$P$里有没有感生电流?
        \item 当断开电键$K$的一瞬时,线圈$P$里有没有感生电流?
    \end{enumerate}
    在上面三种情况里,如果线圈$P$里有感生电流,指出线圈$P$的哪一端是$N$极.
    \item 宇航员飞到某一个不熟悉的行星上,他们想用一只灵敏电流表和一个线圈来探测一下行星周围是否有磁场,应当怎样办?
   
\begin{figure}[htp]\centering
\includegraphics{fig/2-37.pdf}
\caption{}
\end{figure}
 \item 如图2.37所示,在匀强磁场中有一个线圈.
    \begin{enumerate}
        \item 当线圈分别以$P_1$和$P_2$为轴按逆时针方向转动时(如图中箭头所示),感生电流的方向各是什么?
        \item 当转速恒定时,线圈以$P_1$和$P_2$为轴转动时,两种情况下感生电流的大小有何关系?
        \item 当转速恒定时,感生电动势的大小跟线圈面积有何关系?
        \item 设磁感应强度B为1.5特,$AB$为10厘米,$BC$为4厘米,转速为每秒$120/2\pi$转,分别求出以$P_1$和$P_2$为转轴时感生电动势的最大值.
    \end{enumerate}
    \item 如图2.38所示,在磁感应强度为0.5特的匀强磁场中,让长为0.2米的导体$AB$在无摩擦的框架上以5$\ms$的速度向右滑动,如果$R_1=R_2=2$欧,其他导线的电阻不计,外力做功的功率有多大?感生电流的功率有多大?在电阻$R_1$和$R_2$上消耗的功率有多大?验证一下:能的转化是否符合守恒定律?
\item 如图2.39所示,让铜线圈$A$自由落下,并通过一段有匀强磁场的空间,试定性说明线圈的运动情况.(不考虑空气阻力)
\begin{figure}[htp]
\centering
\begin{minipage}[t]{0.48\textwidth}
\centering
\begin{circuitikz}[>=latex, european, scale=.7]
\foreach \x in {-3,-2,...,3}
   \foreach \y in {-1,0,1}
{
   \node at (\x,\y){$\times$};
}

\draw (-4,-1.5)--(4,-1.5) to [R=$R_2$] (4,1.5)--(-4,1.5) to [R=$R_1$] (-4,-1.5);
\draw [ultra thick] (.5,-1.8)node[right]{$B$}--(.5,1.8)node[right]{$A$};

\end{circuitikz}
\caption{}
\end{minipage}
\begin{minipage}[t]{0.48\textwidth}
\centering
\begin{circuitikz}[>=latex, scale=.7]
\foreach \x in {-3,-2,...,2}
   \foreach \y in {-1,0,1}
{
   \node at (\x,\y){$\times$};
}

\draw (-1,1.5) rectangle (0,2.2);
\node at (0.25,2){$A$};
\end{circuitikz}
\caption{}
\end{minipage}
\end{figure}
\item 弹簧上端固定,下端悬挂一根磁铁.将磁铁抬到某一高度后放开,磁铁能上下振动较长时间才停下来.如果在磁铁下端放一个固定的闭合线圈使磁铁上下振动时穿过它
(图2.40),磁铁就会很快地停下来,试用能量观点来解释这个现象.


\begin{figure}[htp]
\centering
\begin{minipage}[t]{0.48\textwidth}
\centering
\includegraphics{fig/2-40.pdf}
\caption{}
\end{minipage}
\begin{minipage}[t]{0.48\textwidth}
\centering
\includegraphics{fig/2-41.pdf}
\caption{}
\end{minipage}
\end{figure}

\item 图2.41是生产中常用的一种延时继电器的示意图.铁芯上有两个线圈$A$和$B$.线圈$A$跟电源连接,线圈$B$的两端接在一起,构成一个闭合电路,在拉开电键$K$的时候,弹簧$S$并不能立即将衔铁$D$拉起,从而使触头$C$(连接工作电路)
立即离开,过一段短时间后触头$C$才能离开;延时继电器就是这样得名的,试说明这种继电器的原理.
\item 普通磁带录音机是用同一个磁头来录音和放音,磁头结构如图2.42所示,在一个环形铁芯上绕着一个线圈,录音时,磁头线圈跟微音器相连,放音时,磁头线圈改为跟扬声器相连,微音器的作用是把声音的变化转变为线圈中电流的变化,扬声器的作用是把线圈中电流的变化转化为声音的变化,铁芯有个缝隙,工作时磁带就贴着这个缝隙移动.磁带上涂有一层磁粉,磁粉能被磁化,并且留下剩磁.

根据学过的知识先自己想一想录、放音的原理;再找有关的书或文章看看自己想得对不对,然后把录、放音的基本原理
简明扼要地写下来.
\begin{figure}[htp]
\centering
\includegraphics[scale=.6]{fig/2-42.png}
\caption{}
\end{figure}

\begin{figure}[htp]
\centering
\begin{circuitikz}[>=latex, scale=.8]
\draw (0,0) to [cute open switch] (3,0)--(6,0);
\draw (6,4)--(0,4) to [battery2] (0,0);
\foreach \x in {.5,1.25,...,6}
\foreach \y in {.5,1.25,...,3.5}
{
   \node at (\x, \y) {$\times$};
}

\node at (1.5,-.2){$K$};

\draw[very thick] (4,0)node[below]{$B$}--(4,4)node[above]{$A$};
\draw[very thick] (4,0) arc (0:-180:0.1);
\draw[very thick] (4,4) arc (0:180:0.1);

\draw[->] (4,2.3)--(4.8,2.3)node[above]{$F$};

\end{circuitikz}
\caption{}
\end{figure}

\item 如图2.43所示,电源的电动势$\mathcal{E}=1.5$伏,内电阻$r=0.5$欧,$AB=0.5$米,$AB$的电阻$R=0.1$欧,框架的电阻不计,磁感应强度为0.5特,金属框对$AB$的滑动摩擦力为$0.25$牛.
\begin{enumerate}
    \item 分析一下,当电键$K$闭合后,会发生哪些电磁现象?
    \item 当$AB$的速度达到稳定时(即速度为最大时),电路中的电流强度是多大?
    \item $AB$的最大速度是多少?
    \item 这时电源消耗的电能转化为什么形式的能?通过计算验证一下:能的转化是否符合守恒定律?
\end{enumerate}


\end{enumerate}





















\chapter{交流电}
我们已经学过了直流电,除了直流电,还有大小和方向
都随时间作周期性变化的电流,叫做交流电,交流电比起
直流电来有许多优点,交流电可以利用变压器升高或者降低
电压,交流电可以驱动结构简单、运行可靠的感应电动机,因
此在工农业生产和日常生活中普遍使用交流电。这一章我们
就来学习交流电。

交流电和直流电都是电荷定向移动形成的,它们存在着
共同点。学习交流电,应该注意这种共同点。但是交流电不
同于直流电,有它自己的特点。正是这些特点使交流电具有
上述优点。学习交流电,更要注意它的特点。

\section{交流电的产生}


法拉第发现的电磁感应现象的一个重大应用是制成发电
机。交流电就是从交流发电机发出的。

我们在初中学过,照图3.1那样,使矩形线圈$ABCD$在匀
强磁场中匀速转动,电流表的指针就随着线圈的转动而左右
摆动,表明电路里产生了交流电。

\begin{figure}[htp]\centering
    \includegraphics[scale=1.1]{fig/3-1.pdf}
    \caption{交流电的产生}
    \end{figure}

线圈在磁场中转动时,它的$AB$边和$CD$边切割磁力线,在
线圈中产生感生电动势,在电路中就产生感生电流。当线圈
平面转到图3.1乙所示的位置时,$AB$边向下运动,$CD$边向上
运动,感生电动势和感生电流的方向如图中所示。线圈转过
半周,当线圈平面转到图3.1丁所示的位置时,$AB$边变成向
上运动,$CD$边变成向下运动,感生电动势和感生电流的方向
都发生了改变。当线圈平面转到跟磁力线垂直的平面时(图3.1甲和丙),在
这一时刻,$AB$边和$CD$边的线速度方向跟磁力线平行,都不切
割磁力线,因而没有感生电动势和感生电流。跟磁力线垂直
的这个平面叫做\textbf{中性面},线圈平面每经过中性面一次,感生电
动势和感生电流的方向就改变一次,因此,线圈转动一周,感
生电动势和感生电流的方向改变两次。

图3.1其实就是一个交流发电机的模型。实际的发电机,
结构比较复杂,但发电机的基本组成部分仍是线圈(通常叫电
枢)和磁极。电枢转动、磁极不动的发电机,叫做旋转电枢式
发电机,磁极转动,而电枢不动,线圈依然切割磁力线,电枢
同样会产生感生电动势,这种发电机叫旋转磁极式发电机。不
论哪种发电机,转动的部分都叫转子,不动的部分都叫定子。

旋转电枢式发电机,转子产生的电流必须经过裸露着的
滑环和电刷引到外电路,如果电压很高,就容易发生火花放
电,有可能烧毁电机。同时,电枢可能占有的空间也受到限
制,线圈匝数不能很多,产生的感生电动势不能很高。这种发
电机提供的电压一般不超过500伏。旋转磁极式发电机克服
了上述缺点,能够提供几千到几万伏的电压,输出功率可达几
十万千瓦,所以大型发电机都是旋转磁极式的。

发电机的转子是由蒸汽轮机、水轮机或其他动力机带动
的,动力机将机械能传递给发电机,发电机把机械能转化为
电能输送给外电路。

\section{交流电的变化规律}

\begin{figure}[htp]\centering
    \includegraphics[scale=1]{fig/3-2.pdf}
    \caption{}
    \end{figure}

现在我们来研究交流电的变化规律。图3.2中标$A$的小
圆圈表示线圈$AB$边的横截面,标$D$的小圆圈表示线圈$CD$边
的横截面。假定线圈平面从中性面开始转动,角速度是$\omega$(弧
度/秒),经过时间$t$,线圈转过的角度是$\omega t$,$AB$边的线速度$v$
的方向跟磁力线方向间的夹角也等于$\omega t$。设$AB$边的长度是$\ell$,
磁感应强度是$B$,$AB$边中的感
生电动势就是
\[e_{AB}=B\ell v\sin \omega t\]
$CD$边中的感生电动势跟$AB$边
中的大小相同,而且两边又是
串联的,所以这一瞬间整个线
圈中的感生电动势$e$可用下式
表示:
\[e=2B\ell v\sin \omega t\]

当线圈平面转到跟磁力线平行的位置时,$AB$边和$CD$边的
线速度方向都跟磁力线垂直,两边都垂直切割磁力线。这时,
\[\omega t=\frac{\pi}{2},\qquad \sin \omega t=1 \]
感生电动势最大,用$\mathcal{E}_m$来表示,即
$\mathcal{E}_m=2B\ell v$,把它代入上式就得到
\begin{equation}
    e=\mathcal{E}_m \sin\omega t
\end{equation}
式中的$e$随着时间而变化,不同时刻有不同的数值,叫做电动
势的\textbf{瞬时值},$\mathcal{E}_m$叫做电动势的\textbf{最大值}。上式表明,电动势是
按照正弦规律变化的。

这时电路中电流强度也是按照正弦规律变化的,设整个
闭合电路的电阻为$R$,电流的瞬时值
\[i=\frac{e}{R}=\frac{\mathcal{E}_m}{R} \sin\omega t \]
其中$\mathcal{E}_m/R$是电流的最大值,用$I_m$表示,所以
\begin{equation}
    i=I_m \sin\omega t
\end{equation}

外电路中一段导线上的电压同样是按照正弦规律变化
的,设这段导线的电阻为$R'$,电压的瞬时值
\[u=iR'=I_mR'\sin\omega t \]
其中$I_mR'$是电压的最大值,用$U_m$表示,所以
\begin{equation}
    u=U_m \sin\omega t
\end{equation}

\begin{figure}[htp]\centering
    \begin{tikzpicture}[>=latex, scale=1.5]
    
    \foreach \x in {.25,.75,-.25,-.75}
    {
        \draw[->] (-2,\x)--(2,\x);
    }
    \draw [dashed](0,-1.5)--(0,2.5)node [right]{中性面};
    \draw[rotate=-60, dashed] (-1.5,-.07) rectangle (1.5,.07);
    \draw[rotate=-30] (-1.5,-.07)node[below]{$D$} rectangle (1.5,.07)node[above]{$A$};
    \draw[rotate=-30, fill=white] (-1.5,0) circle (2.5pt) node{\tiny$\cdot$};
    \draw[rotate=-30, fill=white] (1.5,0) circle (2.5pt) node{\tiny$\times$};
    \draw[rotate=-60, fill=white] (-1.5,0) circle (2.5pt) node{\tiny$\cdot$};
    \draw[rotate=-60, fill=white] (1.5,0) circle (2.5pt) node{\tiny$\times$};
\node at (2.5,0){$B$};

\draw [->] (0,1) arc (90:120:1) node [right]{$\phi_0$};
\draw [->] (0,2) arc (90:150:2) node [left]{$\omega t+ \phi_0$};

    \end{tikzpicture}
    \caption{}
    \end{figure}

上述各式都是从线圈平面
跟中性面重合的时刻开始计时
的,如果不是这样,而是如图
3.3所示从线圈平面跟中性面
有一夹角$\phi_0$时开始计时,那么,
经过时间$t$,线圈平面跟中性
面间的角度是$\omega t+\phi_0$,感生电动势的公式就变成
\begin{equation}
    e=\mathcal{E}_m \sin (\omega t+\phi_0)
\end{equation}
电流和电压的公式分别变成
\begin{align}
    i&=I_m\sin (\omega t+\phi_0) \\
    u&=U_m\sin (\omega t+\phi_0)
\end{align}

公式(3.1)、(3.2)、(3.3)或者(3.4)、(3.5)、(3.6),表示交流电是按照正弦规律变化的,这种按照正弦规律变化的交流电叫做\textbf{正弦交流电}。正弦交流电是一种最简单而又最基本的交流电,正如同简谐振动是一种最简单而又最基本的振动一样。
	
	
交流电的变化规律也可以用图象直观地表示出来,图3.4是交变电动势$e=\mathcal{E}_m \sin \omega t$的图象,图象上方画出了对应于交变电动势等于零或正、负最大值时的线圈位置,图3.5是交变电流$i=I_m\sin(\omega t+\phi_0)$或交变电压 $u=U_m\sin(\omega t+\phi_0)$的图象,其中$\phi_0=\pi/6$。

\subsection*{练习一}
\begin{enumerate}
    \item 有人说线圈平面转到中性面的瞬间,穿过线图的磁通量最大,因而线圈中的感生电动势最大;线圈平面跟中性面垂直的瞬间,穿过线圈的磁通量为零,因而线圈中的感生电动势为零,这种说法对不对?为什么?
    \item 在图3.1所示的实验中,在不改变$B$、$\ell$和$v$的数值的情况下,你有什么办法来提高电动势的最大值?并说明所依据的原理。
    \item 在图3.1所示的实验中,设$AB$边的长度为20厘米,线圈的宽$AD$为10厘米,磁感应强度$B$为0.01特,线圈的转数为50转/秒,求电动势的最大值,如果从线圈平面转过中性面30$^\circ$角的瞬时开始计时,经过0.01秒时电动势的瞬时值是多大?
    \item 画出$u=30\sin(\omega t+\pi/2)$伏的电压图象和$i=2\sin(\omega t-\pi/4)$安的电流图象,以$\omega t$为横轴。
    \item 图3.1是一个发电机模型,只能供课堂演示之用,如果你有兴趣,可约请几位同学,共同研究一下怎样在此模型的	
	基础上加以改进,设计一个小发电机,如果感到知识不足,可自学或查阅有关电工学的书籍。
\end{enumerate}


\section{表征交流电的物理量}

稳恒电流不随时间而变化,要描述电路中的电流或电压,只要指出电流强度或电压的数值就够了,交流电则不然,交流电的电流或电压,大小和方向都随时间作周期性的变化,要描述它们,需要的物理量就要多些,下面讨论表征正弦交流电的物理量。

\subsection{最大值和有效值}

交流电的最大值$(I_m,U_m)$是交流电在一周期内所能达到的最大数值,可以用来表示交流电的电流强弱或电压高低,在实际中有重要意义,例如把电容器接在交流电路中,就需要知道交变电压的最大值,电容器所能承受的电压要高于交变电压的最大值,否则电容器可能被击穿。但是,在研究交流电的功率时,最大值用起来却不够方便,它不适于用来表示交流电产生的效果,在电工技术中通常用有效值来表示交流电的大小。

交流电的有效值是根据电流热效应来规定的,让交流电和直流电通过相同阻值的电阻,如果它们在相同时间内产生的热量相等,就把这一直流电的数值叫数这一交流电的\textbf{有效值}。例如某一交变电流通过一段电阻丝,在一段时间内产生的热量为$Q$,如果改用3安的直流电通过这段电阻丝,在相同的时间内产生的热量也为$Q$,那么,这一交变电流的有效值就
是3安,交流电压的有效值可以用同样的方法来确定,通常用$I$和$U$分别表示交变电流和电压的有效值。这样,知道了交流电的有效值,很容易求出交流电通过电阻时产生的热量。设电流的有效值为$I$,电阻为$R$,在时间$t$内产生的热量$Q=I^2Rt$。这跟直流电路中焦耳定律的形式完全相同,所不同的是在交流电中电流要用有效值。

计算表明,正弦交流电的有效值与最大值之间有如下的关系:
\[\begin{split}
   I&=\frac{I_m}{\sqrt{2}}=0.707I_m\\
   U&=\frac{U_m}{\sqrt{2}}=0.707U_m 
\end{split}\]

我们通常说照明电路的电压是220伏,便是指有效值,各种使用交流电的电气设备上所标的额定电压和额定电流的数值,一般交流电流表和交流电压表测量的数值,也都是有效值,以后提到交流电的数值,凡没有特别说明的,都是指有效值。

\subsection{周期和频率}

跟别的周期性过程一样,交流电也要用周期或频率来表示变化的快慢,在图3.1所示的实验里,线圈匀速转动一周,电动势、电流都按正弦规律变化一周,我们把交流电完成一次周期性变化所需的时间,叫做交流电的\textbf{周期},通常用$T$表示,单位是秒,交流电在1秒钟内完成周期性变化的次数,叫做交流电的\textbf{频率},通常用$f$表示,单位是赫兹。

根据定义,周期和频率的关系是
\[T=\frac{1}{f},\qquad f=\frac{1}{T}\]
	
我国工农业生产和生活用的交流电,周期是0.02秒,频
率是50赫,电流方向每秒改变100次。

\subsection{相和相差}

从交流电瞬时值的表达式可以看出,交流电瞬时值何时为零,何时最大,不是简单地由时间$t$来确定,而是由$\omega t+\phi_0$来确定的。这个相当于角度的量$\omega t+\phi_0$对于确定交流电的大小和方向起着重要作用,叫做交流电的\textbf{相}。相,又叫相位、位相或周相,$\phi_0$是$t=0$时的相,叫做\textbf{初相}。在交流电中,相这个物理量可以用来比较交流电的变化步调。

两个交流电的相位之差叫做它们的\textbf{相差},用$\Delta \phi$来表示。如果交流电的频率相同,相差就等于初相之差,即
\[\Delta \phi=(\omega t+\phi_1)-(\omega t+\phi_2)=\phi_1-\phi_2 \]
这时相差是恒定的,不随时间而改变。

两个频率相同的交流电,如果它们的相位相同,即相差为零,这两个交流电叫做\textbf{同相}的,这两个交流电的变化步调一致,同时到达零和正负最大值(图3.6)。	
\begin{figure}[htp]\centering
    \begin{tikzpicture}[>=latex, xscale=.6]
        \draw [->](-2,0)--(6,0)node[right]{$\omega t$};
        \draw [->](0,-1.5)--(0,1.5)node[right]{$i$};
        \draw [ultra thick] plot[domain=0:5.3, samples=1000] function{sin(x+1)} ;
        \draw [ultra thick, dashed] plot[domain=-1:0, samples=100] function{sin(x+1)};
        \node at (2,1) [below]{$i_1$};
        \draw [very thick] plot[domain=0:5.3, samples=1000] function{.7*sin(x+1)} ;
        \draw [very thick, dashed] plot[domain=-1:0, samples=100] function{.7*sin(x+1)};
        \node at (1,.5)[below]{$i_2$};
        \draw [dashed] (-1,0)--(-1,-1.5);
        \draw [<->] (0,-1)--node[above]{$\phi_0$}(-1, -1);

\end{tikzpicture}
\caption{图中的$i_1=I_m\sin(\omega t+\phi_0)$,$i_2=I'_m\sin(\omega t+\phi_0)$}
\end{figure}

两个频率相同的交流电,如果相差为180$^\circ$,即$\Delta\phi=\pi$,这两个交流电叫做\textbf{反相}的。这两个交流电的变化步调恰好相反(图3.7):一个到达正的最大值,另一个恰好到达负的最大值;一个减小到零,另一个恰好增大到零。
\begin{figure}[htp]\centering
    \begin{tikzpicture}[>=latex, xscale=.8]
        \draw [->](-2,0)--(6,0)node[right]{$\omega t$};
        \draw [->](0,-1.5)--(0,1.5)node[right]{$i$};
        \draw [ultra thick] plot[domain=0:5.3, samples=1000] function{sin(x+1)} ;
        \draw [ultra thick, dashed] plot[domain=-1:0, samples=100] function{sin(x+1)};
        \node at (2,1) [below]{$i_1$};
        \draw [very thick] plot[domain=0:5.3, samples=1000] function{-.7*sin(x+1)} ;
        \draw [very thick, dashed] plot[domain=-1:0, samples=100] function{-.7*sin(x+1)};
        \node at (1,-.55)[below]{$i_2$};
        \draw [dashed] (-1,0)--(-1,-1.5);
        \draw [<->] (0,-1)--node[above]{$\phi_0$}(-1, -1);

\end{tikzpicture}
\caption{图中的$i_1=I_m\sin(\omega t+\phi_0)$,$i_2=I'_m\sin(\omega t+\phi_0+\pi)$}
\end{figure}

图3.8表示两个频率相同的交流电,但初相不同,且$\phi_1>\phi_2$。从图中可以看出,这两个交流电的变化步调不一致,$i_1$比$i_2$先到达正的最大值,零或负的最大值。这时我们说$i_1$比$i_2$超前$\Delta\phi$角,或者$i_2$比$i_1$落后$\Delta\phi$角。
\begin{figure}[htp]\centering
    \begin{tikzpicture}[>=latex, xscale=1]
        \draw [->](-2,0)--(7,0)node[right]{$\omega t$};
        \draw [->](0,-1.5)--(0,1.5)node[right]{$i$};
        \draw [ultra thick] plot[domain=0:5.3, samples=1000] function{sin(x+1)} ;
        \draw [ultra thick, dashed] plot[domain=-1:0, samples=100] function{sin(x+1)};
        \node at (1,1) [above]{$i_1$};
        \draw [very thick] plot[domain=0:5.3+.5, samples=1000] function{.7*sin(x+.5)}node [above]{$i_2$} ;
        \draw [very thick, dashed] plot[domain=-.5:0, samples=100] function{.7*sin(x+.5)};
        \draw [dashed] (-1,0)--(-1,-1.5);\draw [dashed] (-.5,0)--(-.5,-1);
        \draw [<->] (0,-1.2)--node[below]{$\phi_1$}(-1, -1.2);
        \draw [<->] (0,-.65)--node[above]{$\phi_2$}(-.5,-.65);
        \draw [<->] (-1,-.75)--node[above]{$\Delta \phi$}(-.5,-.75);


\end{tikzpicture}
\caption{图中的$i_1=I_m\sin(\omega t+\phi_1)$,$i_2=I'_m\sin(\omega t+\phi_2)$}
\end{figure}

最大值(或有效值)、频率(或周期)、初相是表征正弦交流电的三个重要物理量,知道了这三个量,就可以写出交流电瞬时值的表达式,从而知道正弦交流电的变化规律。

\subsection*{练习二}
\begin{enumerate}
    \item 有一个电容器,能耐压250伏,是否能接在220伏的交流电路上?为什么?
    \item 线圈转动的角速度$\omega$也叫角频率(或圆频率),试就图3.1导出角频率$\omega$跟周期$T$或频率$f$的关系式。
    \item 已知:$u_1=220\sqrt{2}\sin(100\pi t+\pi /6)$伏,$u_2=380\sqrt{2}\sin(100\pi t+\pi /3)$伏,求这两个交流电压的最大值、有效值、周期、频率、角频率和初相。这两个电压哪个超前?相差是多大?
    \item 有一正弦交流电,频率是50赫,有效值是5安,初相
    是$-\pi/2$。写出瞬时值的表达式,并画出图象。
    \item 图3.9是某一正弦交流电的图象。根据图象求出最大值、有效值、周期、角频率和初相,并写出瞬时值的表达式。
    \begin{figure}[htp]\centering
        \begin{tikzpicture}[>=latex, xscale=.6]
    \draw [->](-.5,0)--(8.5,0)node[right]{$t$(s)};
    \draw [->](0,-1.5)--(0,2.5)node[right]{$i$(A)};
    \draw [ultra thick] plot[domain=0:3.14*2.5, samples=1000] function{cos(x)};
    \draw [dashed](0,1)--(2*3.1416,1);
    \draw [dashed](0,-1)--(3.1416,-1);
    \node at (0,1) [left]{$+10$};
    \node at (0,-1) [left]{$-10$};
    \foreach \x in {0.05,0.1,0.15,0.2,0.25}
    {
        \node at (\x*31.416, 0)[below]{$\x$};
        \draw(\x*31.416, 0)--(\x*31.416, .1);
    }
    \node at (-.25,-.25){$0$};
    
    \end{tikzpicture}
    \caption{}
    \end{figure}
    \item 图3.10是两个正弦交流电的图象,哪个超前,哪个落后?超前或落后的角度是多大?其中$\phi_1$和$\phi_2$的绝对值都是60$^\circ$。
    \begin{figure}[htp]\centering
        \begin{tikzpicture}[>=latex, xscale=.6]
    \draw [->](-2,0)--(10,0)node[right]{$\omega t$};
    \draw [->](0,-1.5)--(0,1.5)node[right]{$i$};
    \draw [ultra thick] plot[domain=0:9, samples=1000] function{-cos(x+.7)*.7} node [above]{$i_2$};
    \draw [ultra thick] plot[domain=0:8.5, samples=1000] function{sin(x+1)} node [below]{$i_1$};
    \draw [ultra thick, dashed] plot[domain=-1:0, samples=100] function{-cos(x+.7)*.7};
    \draw [ultra thick, dashed] plot[domain=-1:0, samples=100] function{sin(x+1)};
    \node at (-.25,-.25){$O$};
    \draw [dashed] (-1,0)--(-1,-1.5);
    \draw [dashed] (3.1416/2-.7,0)--(3.1416/2-.7,-1.5);
    \draw [<->] (0,-1)--node[below]{$\phi_1$}(-1, -1);
    \draw [<->] (0,-1)--node[below]{$\phi_2$}(3.1416/2-.7,-1);
    
    \end{tikzpicture}
    \caption{}
    \end{figure}
    
\end{enumerate}

\section{纯电阻电路}
交流电路中如果只有电阻,这种电路就叫做\textbf{纯电阻电路}。白炽电灯、电炉、电烙铁等的电路,就是纯电阻电路。

纯电阻电路中电流跟电压的关系,前面已经用过,这里我们再归结一下,把问题明确起来。在纯电阻电路中,在任一时刻电流跟电压的关系都服从欧姆定律,设电阻为$R$,加在它上面的交变电压是$u=U_m\sin \omega t$,通过这个电阻的电流的瞬时值为
\[i=\frac{u}{R}=\frac{U_m}{R}\sin\omega t=I_m\sin\omega t\]
式中$I_m=U_m/R$,如果都换用有效值,就得到
\begin{equation}
    I=\frac{U}{R}
\end{equation}
这就是纯电阻电路中欧姆定律的表达式。这个表达式跟直流电路中欧姆定律的形式完全相同,所不同的是在交流电路中
电压和电流要用有效值。在图3.11所示的电路中通以交流电,用伏特表和安培表量出电压和电流,可以证实上述表达式是正确的。
\begin{figure}[htp]\centering
    \begin{circuitikz}[european]
        \draw (0,0)--(3,0);
        \draw (3,0) to [R=$R$] (3,3);
        \draw (0,3)--(1,3) to [rmeter, t=A] (3,3);
        \draw (1,0) to [rmeter, t=V, *-*] (1,3);
        \draw (0,0)--(0,1.3); \draw (0,1.7)--(0,3);
        \draw (0,1.3) [fill=white]circle (1.5pt);
        \draw (0,1.7) [fill=white]circle (1.5pt);
        \node at (0,1.5){$\sim$};
    \end{circuitikz}\qquad\qquad 
    \begin{tikzpicture}[>=latex, xscale=.7]
        \draw [->](-.5,0)--(8,0)node[right]{$\omega t$};
        \draw [->](0,-1.5)--(0,1.5)node[right]{$i, u$};
        \draw [ultra thick] plot[domain=0:3.1416*2, samples=1000] function{sin(x)};
        \draw [very thick] plot[domain=0:3.1416*2, samples=1000] function{.6*sin(x)};
\node at (1.7,1.2){$u$};
\node at (4.9,-.4){$i$};
\end{tikzpicture}
\caption{}
\end{figure}

在纯电阻电路中,电流和电压是同相的,即电阻对电流和电压的相位关系没有影响。在图3.12所示的实验中,如果用手摇发电机或低频交流电源给电路通以低频交流电,可以看到电流表和电压表的指针的摆动步调一致,表示电流和电压是同相的。

纯电阻电路中交流电的功率也容易计算出来,在纯电阻
电路中,电能全部转化成内能,电功等于电热,因此,求出单位时间内的电热,就得到交流电的功率。现在用$P$表示交流电的功率,则$P=I^2R$。把$U=IR$代入上式得到
\begin{equation}
    P=UI
\end{equation}

我们看到,在纯电阻电路中,功率的表达式跟直流电路中的形式完全相同,所不同的是在交流电路中电流和电压要用有效值,上述表达式也可以用实验证实。在图3.13所示的电路中,标有$W$的仪表叫瓦特表,是用来测量功率的,接通电路后,把瓦特表的读数跟安培表和伏特表的读数加以比较,就可以知道$P=UI$。
\begin{figure}[htp]\centering
    \begin{circuitikz}[european]
        \draw (0,0)--(5,0);
        \draw (5,0) to [R=$R$] (5,3);
        \draw (5,3)--(4,3) to [rmeter, t=A] (1,3);
        \draw (3.5,0) to [rmeter, t=V, *-*] (3.5,3);
        \draw (0,0)--(0,1.3); \draw (0,1.7)--(0,3);
        \draw (0,1.3) [fill=white]circle (1.5pt);
        \draw (0,1.7) [fill=white]circle (1.5pt);
        \draw (.2,3) [fill=black]circle (1.5pt);
        \node at (0,1.5){$\sim$};
        \draw(1,3) to  (1,0); \draw(1,3)--(1,3.7)--(.2,3.7)--(.2,3);
        \draw (2,3) to [rmeter, t=W, fill=white] (0,3);

    \end{circuitikz}
\caption{}
\end{figure}

在纯电阻电路中,既然欧姆定律的公式和功率的公式跟直流电路中相同,纯电阻电路中的计算就跟直流电路中的计算相同。以前在直流电那一章里所作的计算,有不少本来就是纯电阻电路中的问题,例如关于白炽电灯的计算就是。

需要注意的是:公式(3.6)和(3.7)只适用于纯电阻电路,不适
用于其他交流电路,在其他交流电路中有类似的公式,下面就会学到。	
	
\subsection*{练习三}
\begin{enumerate}
    \item 把“220V,40W”的灯泡接到照明电路中,通过灯泡的电流的最大值是多大?
    \item 在电阻$R$的两端加交变电压$u=220\sqrt{2}\sin(\omega t+\phi)$伏,$R=110$欧,写出电流瞬时值的表达式。
    \item 在电阻为500欧的电阻丝中通以交流电,每秒钟产生5焦的热。求电流和电压的有效值和最大值,交流电的功率是多大?
    \item 把一个电热器接到10伏的直流电路中,每秒钟产生的热量为$Q$。现在把它改接到交流电路中,每秒钟产生的热量为$Q/2$,求交流电压的最大值。
\end{enumerate}

\section{纯电感电路}
在直流电路中,影响电流跟电压关系的只有电阻。在交流电路中,情况要复杂一些,影响电流跟电压关系的,除了电阻,还有电感和电容。	
	
\subsection{电感对交流电的阻碍作用} 

如图3.14所示,把电感线圈$L$和白炽灯泡串接在电路里。利用双刀双掷开关K可以分别把这个电路接到直流电源或交流电源上,实验中取直流电压和交变电压的有效值相等。实验表明,接通直流电源时,灯泡亮些;接通交流电源时,灯泡变暗,这表明电感对交流电有阻碍作用。

为什么电感对交流电有阻碍作用呢?交流电通过电感线圈时,电流时刻在改变,电感线圈中必然产生自感电动势,阻
碍电流的变化,这样就形成了对电流的阻碍作用。

\subsection{感抗}

在电工技术中,变压器、电磁铁等的线圈,一般是用铜线绕的,铜的电阻率很小,在很多情况下,线圈的电阻比较小,可以略去不计,而认为线圈只有电感。有电感的电路叫\textbf{纯电感电路}。
\begin{figure}[htp]\centering
    \begin{circuitikz}[european]
        \draw (0,0)--(3,0);
        \draw (3,3) to [cute choke] (3,0);
        \draw (0,3)--(1,3) to [rmeter, t=A] (3,3);
        \node at (2.7,1.5){$L$};
        \draw (1,0) to [rmeter, t=V, *-*] (1,3);
        \draw (0,0)--(0,1.3); \draw (0,1.7)--(0,3);
        \draw (0,1.3) [fill=white]circle (1.5pt);
        \draw (0,1.7) [fill=white]circle (1.5pt);
        \node at (0,1.5){$\sim$};
    \end{circuitikz}
\caption{}
\end{figure}
	
现在我们用图3.15所示的电路来研究纯电感电路中电流跟电压的关系。电路中的$L$是电阻可以路去不计的电感线圈。改变交流电源的电压,通过$L$的电流就随着改变。记下几组相应的电流和电压的数值,就会发现,\textit{在纯电感电路中,电流强度跟电压成正比},即$I\propto U$。 用$1/X_L$作为比例恒量,写成等式,就得到
\[I=\frac{U}{X_L} \]
这就是纯电感电路中欧姆定律的表达式。把这个表达式跟
$I=U/R$
相比,可以看出$X_L$相当于电阻$R$。$X_L$表示出电感对交
流电阻碍作用的大小,叫做\textbf{感抗},它的单位也是欧姆。

感抗的大小跟哪些因素有关呢?在图3.15所示的实验中,如果把铁芯从线圈中取出,使线图的自感系数减小,灯泡就变亮;重新把铁芯插入线圈,使线圈的自感系数增大,灯泡又变暗,这表明线圈的自感系数越大,感抗就越大,在图3.15所示的实验中,如果保持电源的电压不变,而改变交流电的频率,可以看到,频率越高,灯泡越暗,这表明交流电的频率越	
高,线圈的感抗也越大。

为什么线圈的感抗跟线圈的自感系数和交流电的频率有关呢?感抗是由自感现象引起的,线圈的自感系数$L$越大,自感作用就越大,因而感抗越大;交流电的频率$f$越高,电流的变化率越大,自感作用也越大,因而感抗越大。进一步的研究指出,线圈的感抗$X_L$跟自感系数$L$和交流电的频率$f$间有如下的关系:
\[X_L=2\pi fL\]
由于$1{\rm H}=1{\rm V\cdot s/A}$,$1{\rm H/s}=1{\rm V/A}=1\Omega$,因此上式中的$X_L$、$f$、$L$的单位应分别用欧姆、赫兹、亨利。

\subsection{扼流圈}

我们知道,电阻是由导体本身的电阻率、长度和横截面积决定的,跟通过的电流无关。$X_L=2\pi fL$告诉我们,感抗却跟通过的电流的频率有关。例如,自感系数是1亨的线圈,对于直流电,$f=0$,$X_L=0$;对于50赫的交流电,$X_L=314$欧;对于500千赫的交流电,$X_L=3.14$兆欧。所以电感线圈在电路中有“通直流、阻交流”或“通低频、阻高频”的特性。

在电工和电子技术中,用来“通直流、阻交流”的电感线圈,叫\textbf{低频扼流圈}(图3.16甲)。线圈绕在闭合的铁芯上,匝
数为几千甚至超过一万,自感系数为几十亨。这种线圈对低频交流电就有很大的阻碍作用,用来“通低频、阻高频”的电感线圈,叫\textbf{高频扼流圈}(图3.16乙).线圈有的绕在圆柱形的铁氧体心上,有的是空心的,匝数为几百,自感系数为几个毫亨。这种线圈对低须交流电的阻碍作用较小,对高频交流电的阻碍作用很大。

\subsection*{练习四}
\begin{enumerate}
    \item 一个线圈的自感系数为0.6亨,电阻只有几欧姆,把这个线圈接到50赫的交流电路中,它的感抗是多大?比较感抗和电阻的大小,说明为什么可以略去电阻,而认为它只有电感。
    \item 有一个高频扼流圈,自感系数是25毫亨,对于1兆赫的交流电,它的感抗是多大?对于1千赫的交流电,它的感抗又是多大?
    \item 一线圈的自感系数为0.5亨,电阻可以忽略。把它接到频率为50赫,电压为220伏的交流电路中,求通过线圈的电流。
    \item 有一线圈,电阻可略去不计。把它接到220伏、50赫的交流电路中,测得通过线圈的电流为2安。求线圈的自感系数。
\end{enumerate}
	
\section{纯电容电路}
\subsection{交流电能够通过电容器}

照图3.17那样,把白炽灯泡和
,电容器串联在电路里,如果接通直流电源,灯泡不亮,说明直
流电不能通过电容器。如果接通交流电源,灯泡就亮了,说明交流电能够“通过”电容器。我们看到,这里交流电又表现出跟直流电不同的特性。

直流电不能通过电容器是容易理解的。交流电为什么能够“通过”容器呢?原来,电流实际上并没有通过电容器的电介质,只不过在交变电压的作用下,当电源电压升高时,电容器充电,电荷向电容器的极板上集聚,形成充电电流;当电源电压降低时,电容器放电,电荷从电容器的极板上放出,形成放电电流。电容器交替进行充电和放电,电路中就有了电流,似乎交流电通过了电容器。

\subsection{电容对交流电的阻碍作用}
在图3-17的实验中,如果把
电容器从电路中取下来,使灯泡直接与交流电源相接,可以看到,灯泡要比接有电容器时亮得多。这表明电容也对交流电有阻碍作用。

为什么电容对交流电有阻碍作用呢?原来,对导线中形成电流的自由电荷来说,当电源的电压推动它们向某一方向做定向运动的时候,电容器两极板上积累的电荷却反抗它们
向这个方向做定向运动,这就产生了电容对交流电的阻碍作
用。

\subsection{容抗}

只接有电容器的电路,连接导线的电阻一般都很小,可以略去不计,而认为电路中只有电容。只有电容的电路叫\textbf{纯电容电路}。
\begin{figure}[htp]\centering
    \begin{circuitikz}[european]
        \draw (0,0)--(3,0);
        \draw (3,3) to [C] (3,0);
        \draw (0,3)--(1,3) to [rmeter, t=A] (3,3);
        \draw (1,0) to [rmeter, t=V, *-*] (1,3);
        \draw (0,0)--(0,1.3); \draw (0,1.7)--(0,3);
        \draw (0,1.3) [fill=white]circle (1.5pt);
        \draw (0,1.7) [fill=white]circle (1.5pt);
        \node at (0,1.5){$\sim$};
    \end{circuitikz}
\caption{}
\end{figure}

我们用图3.18所示的电路来研究纯电容电路中电流跟电压的关系。改变电路两端的电压,电路中的电流就随着改变。记下几组相应的电流和电压的数值,就会发现,\textit{在纯电容电路
中,电流强度跟电压成正比},即$I\propto U$,用$1/X_C$作为比例恒量,
写成等式,就得到
\[I=\frac{U}{X_C}\]
这就是纯电容电路中欧姆定律的表达式。把这个表达式跟$I=U/R$
相比,可以看出$X_C$相当于电阻$R$。$X_C$表示出电容对交
流电阻碍作用的大小,叫做\textbf{容抗},它的单位也是欧姆。

容抗的大小跟哪些因素有关呢?在图3.17所示的实验中,换用电容不同的电容器来做实验,可以看到,电容越大,灯泡越亮。这表明电容器的电容越大,容抗越小,如果保持电源的电压不变,而改变交流电的频率,可以看到,频率越高,灯泡越亮。这表明交流电的频率越高,容抗越小。

为什么电容器的容抗与它的电容和交流电的频率有关呢?这是因为电容越大,在同样电压下电容器容纳的电荷越多,
充电电流和放电电流就越大,因而容抗就越小,交流电的频率越高,充电和放电就进行得越快,充电电流和放电电流就越大,因而容抗就越小,进一步的研究指出,电容器的容抗$X_C$跟电容$C$和交流电的频率$f$间有如下的关系:
\[X_C=\frac{1}{2\pi fC} \]
由于$1{\rm F}=1{\rm C\cdot V}$,$1{\rm s/F}=1{\rm s}\cdot {\rm V/C}=1{\rm V/A}=1\Omega$,因此上式中的$X_C$、$f$、$C$的单位应分别用欧姆、赫兹、法拉。

\subsection{隔直电容器和旁路电容器}

跟感抗类似,容抗也跟通过的电流的频率有关,容抗与频率成反比,频串越高,容抗越小.例如,10微法的电容器,对于直流电,$f=0$,$X_C$为$\infty$;对于50赫的交流电,$X_C=318$欧;对于500千赫的交流电,$X_C=0.0318$欧。所以电容器在电路中有“通交流、隔直流”或“通高频、阻低频”的特性。这种特性,使电容器成为电子技术中的一种重要元件。

在电子技术中,从某一装置输出的电流常常既有交流成分,又有直流成分。如果只需要把交流成分输送到下一级装置,只要在两级电路之间串联一个电容器(图3.19甲),就可
以使交流成分通过,而阻止直流成分通过,作这种用途的电容器叫做隔直电容器。隔直电容器的电容一般较大。

在电子技术中,从某一装置输出的交流电常常既有高频成分,又有低频成分,如果只需要把低频成分输送到下级装置,只要在下一级电路的输入端并联一个电容器(图3.19乙),就可以达到目的,电容器对高频成分的容抗小,对低频成分的容抗大,高频成分就通过电容器,而使低频成分输入到下一级,作这种用途的电容器叫做高频旁路电容器。高频旁路电容器的电容一般较小。

\subsection*{练习五}
\begin{enumerate}
    \item 电容是100皮法的电容器,对频率是$10^8$赫的高频电流和频率是$10^3$赫的音频电流,容抗各是多大?
    \item 把电容为5微法的电容器接到220伏、50赫的交流电路中,通过电容器的电流是多少?把电容器换为0.05微法的,通过的电流是多少?
    \item 在有220伏、50赫交流电源的地方,使用一个交流电流表可以测定电容器(耐压在311伏以上)的电容,说明测定的方法和原理。
    \item 线圈的自感系数为$L$,电容器的电容为$C$。要使感抗和容抗相等,交流电的频率应该满足什么条件?
\end{enumerate}

\section{电感和电容对交流电相位的影响}
我们知道,电阻对交流电的相位没有影响,电感和电容则不然,它们不但对交流电有阻碍作用,影响交流电的电流和电压的数值关系,而且要影响交流电的电流和电压的相位关系。

我们先来看电感对交流电相位的影响,实验装置如图3.20所示,把一个电感线圈$L$连接到电路中,用手摇发电机或低频交流电源给电路通以低频交流电,可以看到电流表和电压表的指针的摆动步调不同,这表明电感线圈中的电流跟线圈两端的电压不是同相的。
\begin{figure}[htp]\centering
    \includegraphics[scale=.6]{fig/3-20.png}
    \caption{}
    \end{figure}

进一步研究这个问题可以使用示波器,把电感线圈两端的电压和线圈中的电流的变化输送给示波器,在荧光屏上就可以看到电压和电流的图象。从图象看出,电感使交流电的电流落后于电压,可以证明,\textit{在纯电感电路中,电流比电压
落后$\pi/2$}(图3.21)。
\begin{figure}[htp]\centering
    \begin{tikzpicture}[>=latex, xscale=.6]
\draw [->](-2,0)--(11,0)node[right]{$\omega t$};
\draw [->](0,-1.5)--(0,1.5)node[right]{$i,u$};
\draw [ultra thick] plot[domain=0:3*3.1416, samples=1000] function{sin(x)} node [below]{$u$};
\draw [very thick] plot[domain=0:10, samples=1000] function{.8*sin(x-3.1416/2)} node [above]{$i$};
%\draw [ultra thick, dashed] plot[domain=-1:0, samples=100] function{-cos(x+.7)*.7};
%\draw [ultra thick, dashed] plot[domain=-1:0, samples=100] function{sin(x+1)};
\node at (-.25,-.25){$O$};

\foreach \x in {1,5}
{
    \draw[dashed] (\x*3.1416/2,0)--(\x*3.1416/2,1);
    \draw[dashed] (\x*3.1416/2+3.1416/2,0)--(\x*3.1416/2+3.1416/2,.8);
}

\draw[dashed] (3*3.1416/2,0)--(3*3.1416/2,-1);
    \draw[dashed] (4*3.1416/2,0)--(4*3.1416/2,-.8);

\end{tikzpicture}
\caption{电感对交流电相位的影响}
\end{figure}

电容对交流电相位的影响也可以用实验来演示,演示装置仍如图3.20所示,只是用电容器来代替电感线图.给电路通以低频交流电,可以看到两个电表指针的摆动步调也不同,表明电流跟电压也不是同相的,用示波器进一步研究,从荧光屏上的图象可以看出,电容使交流电的电流超前于电压。可以证明,\textit{在纯电容电路中,电流比电压超前$\pi/2$}(图3.22)。

\begin{figure}[htp]\centering
    \begin{tikzpicture}[>=latex, xscale=.6]
        \draw [->](-2,0)--(11,0)node[right]{$\omega t$};
        \draw [->](0,-1.5)--(0,1.5)node[right]{$i,u$};
        \draw [ultra thick] plot[domain=0:3*3.1416, samples=1000] function{sin(x)} node [above]{$u$};
        \draw [very thick] plot[domain=0:3*3.1416, samples=1000] function{.8*sin(x+3.1416/2)} node [below]{$i$};
        \draw [very thick, dashed] plot[domain=-3.1416/2:0, samples=100] function{.8*sin(x+3.1416/2)};
        %\draw [ultra thick, dashed] plot[domain=-1:0, samples=100] function{sin(x+1)};
        \node at (-.25,-.25){$O$};
        
        \foreach \x in {1,5}
        {
            \draw[dashed] (\x*3.1416/2,0)--(\x*3.1416/2,1);
            \draw[dashed] (\x*3.1416/2+3.1416/2,0)--(\x*3.1416/2+3.1416/2,-.8);
        }
        
        \draw[dashed] (4*3.1416/2,0)--(4*3.1416/2,.8);
            \draw[dashed] (5*3.1416/2,0)--(5*3.1416/2,1);
            \draw[dashed] (3*3.1416/2,0)--(3*3.1416/2,-1);
        \end{tikzpicture}
\caption{电容对交流电相位的影响}
\end{figure}

实际的交流电路中常常是既有电阻元件,又有电感元件
和电容元件;这些元件可以串联,也可以并联,由于电感和电容要影响交流电的相位,因此关于交流电的串联和并联的计算要比直流电复杂。除非电路中只有电阻元件,直流电路中计算总电流、总电压、总电阻的公式不再适用,这方面的知识,本书不再讲述,有兴趣的同学可以找电工学的书籍来学习。

\section{交流电的功率}
\begin{figure}[htp]\centering
    \begin{circuitikz}[european]
        \draw (0,0)--(5,0);
        \draw (5,3) to [american inductors, L=$L$] (5,1.5) to [R=$R$] (5,0);

        \draw (5,3)--(4,3) to [rmeter, t=A] (1,3);
        \draw (3.5,0) to [rmeter, t=V, *-*] (3.5,3);
        \draw (0,0)--(0,1.3); \draw (0,1.7)--(0,3);
        \draw (0,1.3) [fill=white]circle (1.5pt);
        \draw (0,1.7) [fill=white]circle (1.5pt);
        \draw (.2,3) [fill=black]circle (1.5pt);
        \node at (0,1.5){$\sim$};
        \draw(1,3) to  (1,0); \draw(1,3)--(1,3.7)--(.2,3.7)--(.2,3);
        \draw (2,3) to [rmeter, t=W, fill=white] (0,3);

    \end{circuitikz}
\caption{}
\end{figure}

由于电感和电容对交流电的相位有影响,交流电在功率
向题上也表现出跟直流电不同的特点,在一般的交流电路中,
功率的表达式$P=UI$不再成立。这一点从图3.23所示的实
验中就可以看出来,把电阻和
电感串联在电路中,接通电路
后,用仪表分别测出电流、电压
和功率,从仪表的读数可以知
道,$P<UI$。计算表明,交流电
的功率表达式是
\[P=UI\cos\phi\]
其中,$\phi$是电流和电压之间的相差,$\cos\phi$叫做\textbf{功率因数}。$P$是
交流电路中实际消耗的功率,叫做\textbf{有功功率},
单位用瓦或千
瓦,乘积$UI$叫做\textbf{视在功率},单位用伏安或千伏安。

功率因数的大小,决定于电路的性质。在白炽电灯、电炉
等纯电阻电路中,电流和电压同相,$\phi=0$,$\cos\phi=1$,$P=UI$。
在电动机、日光灯的电路中,不仅有电阻,还有电感,电流和电
压不是同相,$\phi\ne 0$,$\cos\phi<1$,$P<UI$。

发电机或变压器都有一定的额定电流和额定电压,通常
所说的发电机或变压器的容量,就是指它的额定电流和额定
电压的乘积,也就是它的额定视在功率,在正常情况下,电流
和电压都不允许超过额定值,否则发电和变压器就会损坏
或缩短寿命,由公式$P=UI\cos\phi$知道,如果$\cos\phi$很小,即使
电流和电压都已达到额定值,有功功率$P$还是很小,供电设备
的利用率就很低。为了充分利用供电设备的容量,必须提高
电路的功率因数。

\begin{figure}[htp]\centering
    \begin{circuitikz}[european]
        \draw (0,0)--(6,0);
        \draw (6,3) to [american inductors, L=$L$] (6,1.5) to [R=$R$] (6,0);
\draw (5,0) to [C=$C$] (5,3);
        \draw (6,3)--(4,3) to [rmeter, t=A] (1,3);
        \draw (3.5,0) to [rmeter, t=V, *-*] (3.5,3);
        \draw (0,0)--(0,1.3); \draw (0,1.7)--(0,3);
        \draw (0,1.3) [fill=white]circle (1.5pt);
        \draw (0,1.7) [fill=white]circle (1.5pt);
        \draw (.2,3) [fill=black]circle (1.5pt);
        \node at (0,1.5){$\sim$};
        \draw(1,3) to  (1,0); \draw(1,3)--(1,3.7)--(.2,3.7)--(.2,3);
        \draw (2,3) to [rmeter, t=W, fill=white] (0,3);

    \end{circuitikz}
\caption{}
\end{figure}

怎样才能提高功率因数呢?让我们来做一个实验,在图
3.23的电路中,并联一个适当的电容器,如图3.24所示,接
通电路后,从仪表的读数可以
知道,$P$跟$UI$的比值,比没接
电容器以前大了,即功率因数提高了,在有电感的电路中,电流落后于电压,并联上电容器
以后,由于电容有使电流超前于电压的作用,这个作用跟电感使电流落后的作用抵消或
部分抵消,所以电流和电压之间的相差减小,功率因数就提
高了。

我国规定,工厂用电的功率因数不得低于0.85。在变电
所或工厂的配电室里,常装有电容器,并联在电路中,就是为
了提高电路的功率因数。

\section{变压器}
在实际应用中,
常常需要改变交流电的电压。大型发电
机发出的交流电,电压有几万伏,而远距离输电却需要高达儿
十万伏的电压,各种用电设备所需的电压也各不相同,电
灯、电炉等家用电器需要220伏的电压,机床上的照明灯需要
36伏的安全电压,一般电子管的灯丝只需6.3伏的低电压,电
视机显象管却需要一万多伏的高电压,交流电便于改变电
压,以适应各种不同的需要,变压器就是改变交流电电压的
设备。

\subsection{变压器原理}
\begin{figure}[htp]\centering
    \includegraphics[scale=1]{fig/3-25.pdf}
    \caption{}
    \end{figure}

图3.25是变压器的示意图.变压器是由
闭合铁芯和绕在铁芯上的两个线圈组成的。一个线圈跟电源
连接,叫原线圈(也叫初级线圈);另一个线圈跟负载连接,叫
副线圈(也叫次级线圈),两个线圈都是用绝缘导线绕制成的,
铁芯由涂有绝缘漆的硅钢片叠合而成。


在原线圈上加交变电压$U_1$,原线圈中就有交变电流,它在
铁芯中产生交变的磁通量,这个交变磁通量既穿过原线圈,
也穿过副线圈,在原、副线圈中都要引起感生电动势,如果副
线圈电路是闭合的,在副线圈中就产生交变电流,它也在铁
心中产生交变磁通量,这个交变磁通量既穿过副线圈,也穿
过原线圈,在原、副线圈中同样要引起感生电动势,在原、副
线圈中由于有交变电流而发生的互相感应现象,叫做\textbf{互感现
象},互感现象是变压器工作的基础。

原线圈和副线圈中的电流共同产生的磁通量,绝大部分
通过铁芯,只有一小部分漏到铁芯之外。在粗略的计算中可
以略去漏掉的磁通量,认为穿过这两个线圈的交变磁通量相
同,因而这两个线圈的每匝所产生的感生电动势相等,设原
线圈的匝数是$n_1$,副线圈的匝数是$n_2$,穿过铁芯的磁通量是
$\phi$,那么原、副线圈中产生的感生电动势分别是
\[\mathcal{E}_1=n_1\frac{\Delta \phi}{\Delta t},\qquad  \mathcal{E}_2=n_2\frac{\Delta \phi}{\Delta t}\]
由此可得
\begin{equation}
    \frac{\mathcal{E}_1}{\mathcal{E}_2}=\frac{n_1}{n_2} 
\end{equation}

在原线圈中,感生电动势$\mathcal{E}_1$起着阻碍电流变化的作用,跟
加在原线圈两端的电压$U_1$的作用相反,是反电动势。原线圈
的电阻很小,如果略去不计,则有$U_1=\mathcal{E}_1$。副线圈相当于一
个电源,感生电动势$\mathcal{E}_2$相当于电源的电动势。副线圈的电阻
也很小,如果忽略不计,副线圈就相当于无内阻的电源,因而
副线圈的端电压$U_2$等于感生电动势$\mathcal{E}_2$,即$U_2=\mathcal{E}_2$。因此得到
\begin{equation}
    \frac{U_1}{U_2}=\frac{n_1}{n_2} 
\end{equation}

可见,\textit{变压器原、副线圈的端电压之比等于这两个线圈
的匝数比}。如果$n_2>n_1$,$U_2$就大于$U_1$,变压器就使电压升高,
这种变压器叫做\textbf{升压变压器},如果$n_1>n_2$,$U_1$就大于$U_2$,变压
器就使电压降低,这种变压器叫做\textbf{降压变压器}。

变压器原、副线圈的电流之间又有什么关系呢?

变压器工作时,输入的功率主要由副线圈输出,小部分在
变压器内部损耗了。变压器的线圈有电阻,电流通过时要生
热,损耗一部分能量,这种损耗叫做铜损。铁芯在交变磁场中
反复磁化,也要损耗一部分能量使铁芯发热,这种损耗叫做铁
损。变压器的能量损耗很小,效率很高,特别是大型变压器,
效率可达97—99.5\%。所以,在实际计算中常常把损耗的能
量略去不计,认为变压器的输出功率和输入功率相等\footnote{根据交流电功率的公式$P=UI\cos\phi$,(3.11)式应该写成$U_1I_1\cos\phi_1
=U_2I_2\cos\phi_2$。式中 $\cos\phi_1$是原线圈电路的功率因数,$\cos\phi_2$是副线圈
电路的功率因数。$\phi_1$和$\phi_2$通常相差很小,在实际计算中可以认为它们
相等,因而得到(3.11)式。}。即
\begin{equation}
    U_1I_1=U_2I_2
\end{equation}

由(3.10)式和(3.11)式可得
\begin{equation}
    \frac{I_1}{I_2}=\frac{n_2}{n_1}
\end{equation}

这就是变压器工作时原、副线圈中电流之间的关系。可
见,\textit{变压器工作时原线圈和副线圈中的电流强度跟线圈的匝
数成反比}。
变压器的高压线圈匝数多而通过的电流小,可用
较细的导线绕制;低压线圈匝数少而通过的电流大,应当用较
粗的导线绕制。

\subsection{几种常用的变压器}

变压器的种类很多,除常见的电力
变压器外,我们再介绍几种常用的变压器。

图3.26是\textbf{自耦变压器}的示意图。这种变压器的特点是
铁芯上只绕一个线,如果把整个线圈作原线圈,只取线圈
的一部分作副线圈,就可以降低电压(图3.26甲);如果把线
圈的一部分作原线圈,整个线圈作副线圈,就可以升高电压
(图 3.26乙).

\textbf{调压变压器}就是一种自耦
变压器,它的构造如图3.27所
示。线圈$AB$绕在圆环形的铁
心上,$AB$之间加上输入电压
$U_1$。移动滑动触头$P$的位置,
就可以调节输出电压$U_2$。

\textbf{互感器}也是一种变压器,交流伏特表和安培表都有一定
的量度范围,不能直接测量高电压和大电流,高电压对人有
危险,为了保证工作人员的安全也不能把电表直接接入高压
电路里。用变压器来把高电压变成低电压,或者把大电流变
成小电流,这个问题就可以解决了,这种变压器叫做互感
器,互感器分电压互感器和电流互感器两种。

\textbf{电压互感器}(图3.28)用来把高电压变成低电压,它的原
线并联在高压电路中,副线圈上接入交流伏特表,根据伏
特表测得的电压$U_2$和铭牌上注明的变压比($U_1/U_2$),可以算
出高压电路中的电压。为了工作安全,电压互感器的铁壳和
副线圈应该接地。

\textbf{电流互感器}(图3.29)用来把大电流变成小电流。它的
原线圈串联在被测电路中,副线圈上接入交流安培表,根据
安培表测得的电流$I_2$和铭牌上注明的变流比($I_1/I_2$),可以算
出被测电路中的电流。如果被测电路是高压电路,为了工作
安全,同样要把电流互感器的外壳和副线圈接地。

\subsection*{练习六}
\begin{enumerate}
    \item 变压器能不能改变直流电的电压?说明理由。
    \item 收音机中的变压器,原线圈有1210匝,接在220伏
的交流电源上,要得到5伏、6.3伏和350伏三种输出电压,
三个副线圈的匝数各是多少?
\item 为了安全,机床上照明电灯用的电压是36伏,这个
电压是把220伏的电压降压后得到的,变压器的原线圈是
1140匝,副线圈是多少匝?用这台变压器给40瓦的电灯供
电,原副线圈中的包流强度各是多大?
\item 利用变压器的原理可以测量线圈的匝数;用被测线
圈作原线圈,用一个匝数已知的线圈作副线圈,通入交流电,
测出两线圈的电压,就可以求出被测线圈的匝数,已知副线
圈有400匝,把原线圈接到220伏的线路中,测得副线圈的电
压是55伏,求原线圈的匝数。
\item 在图3.28所示的电压互感器的电路中,为什么副线
圈的匝数比原线圈的少?在图3.29所示的电流互感器的电路
中,为什么副线圈的匝数比原线圈的多?
\end{enumerate}

\section{电能的输送}
发电是为用电服务的。从发电站到用电设备或长或短总
有一段距离。水电站只能建筑在河流上,火电站有时要建筑
在煤矿附近,与城市以及分布很广的农村都有距离。即使建
筑在城市里的发电站,与城市里的用电设备也有距离,这样
就产生了电能输送的问题。

电能便于输送,用导线把电源和用电设备连接起来,就
可以实现电能的输送,但是导线有电阻,电流通过时要发热,
这个热量毫无用处,徒然是一种浪费。在电能的输送中要研
究如何减少这种能量损耗,以便有效地利用电能。

设导线的电阻为$R$,通过的电流为$I$,那么在导线上损耗
的电功率是
\[P=I^2R \]

又设导线的总长度为$\ell$,导线的横截面积为$S$,从电阻定
律知道
\[ R=\rho\frac{\ell}{S}\]

要减少能量损耗,一种办法是减小导线的电阻。导线材
料一经选定(现在多用铝导线),电阻率$\rho$就确定了。导线的
总长度由输电距离来决定,不能改变。所以要减小电阻,只有
增大导线的横截面积。

增大导线的横截面积,就要多耗费金属材料,而且导线太
重,给架设也带来困难。因此,单纯用增大导线的横截面积来
减少能量损耗,不是一种理想的办法。实际上,有时即使把横
截面积增大到十分惊人甚至实际做不到的程度,也不能把能
量损耗降低到所能允许的数值范围之内,例如要把功率为200
千瓦的电能输送到10千米外的地方,如果用110伏的电压输
电,并且使能量损耗为输送电能的10\%,就要用横截面积约为
96000${\rm mm}^2$的铝线。这个结果你自己不难算出来。这是个多
么粗的导线,你可以估量一下。显然,导线是太粗了!

减少能量损耗的另一种办法是减小电流。在输送功率不
变的情况下,要减小电流,就必须提高输电电压。在上面的
例子中,如果用11千伏的电压来输电,也就是说,把输电电压
提高到$10^2$倍,电流就减小为原来的$10^{-2}$倍。能量损耗与电
流的平方成正比,所以能量损耗就减小为原来的$10^{-4}$倍。要
使能量损耗仍为输送电能的10\%,导线的横截面积就可以减
小为原来的$10^{-4}$倍,也就是用9.6${\rm mm}^2$的导线就行了。

可见,用提高输电电压的办法来减少能量损耗是行之有
效的。

但输电电压也不是无条件地越高越好。电压提高了,花
在绝缘上的费用要相应地增加,能量损耗在所能允许的数值
范围之内,提高输电电压固然可以减小导线的横截面积,但从
机械强度方面来考虑,导线又不能太细,导线太细容易被拉
断,高压输电还必须考虑在导线上因产生电晕放电而损失能
量。电压越高,导线越细,产生电晕放电的可能性越大,为了
避免电晕放电,有时还需要把导线加粗。可见,输电电压并不
是毫无限制可以任意提高的。

实际输送电能时,要综合考虑各种因素,依照不同情况选
择适合的输电电压,如果输送功率比较大,输电距离比较远,
就要采用较高的电压输电。电压低了,势必要加大导线的横
截面积,如果输送功率不太大,距离也不太长,就不必用太高
的电压输电,这时能量损耗不会太大,电压高了反而增加花
在绝缘上的费用,而且导线因机械强度的限制又不能太细。
例如,输送功率为100千瓦以下,距离为几百米以内,一般采用
220伏的电压送电。这就是通常用的低压线路。输送功率为
几千千瓦到几万千瓦,距离为几十千米到上百千米,一般采用
35千伏或110千伏的电压送电。这就是所谓高压输电。如果
输送功率为10万千瓦以上,距离为几百千米,就必须采用220
千伏甚至更高的电压送电。这就是所谓超高压输电。

我国远距离输电采用的电压有110千伏、220千伏和330
千伏,在少数地区已开始采用500千伏的电压送电,目前世
界上正在试验的最高输电电压是1000千伏。

大型发电机发出的电压,等级有10.5千伏、13.8千伏、
15.75千伏、180千伏,不能直接用这个电压进行远距离输电。
在发电站内,要由升压变压器升压后,再向远处输电。远距离
输电的电压很高(例如220千伏或330千伏),用电设备不能直
接应用,因此送到用电区,要在一次高压变电所把电压降到
110千伏,再在二次高压变电所降到10千伏,其中一部分送往
需要高压电的工厂,另一部分送到低压变电所降到220/380
伏,送给一般用户(图3.30).
\begin{figure}[htp]\centering
    \includegraphics[scale=.75]{fig/3-30.pdf}
    \caption{从发电站到用户的输送线路}
    \end{figure}

交流电可以很方便地由变压器升压或降压,从而实现远
距离输电,并且给额定电压不同的用电设备供电,这是交流电
的一个优点。

\section*{阅读材料:直流输电}
人类历史上最早的输电线路就是直流的,开始时输电电
压只有100伏,后来逐步提高,到1885年提高到6千伏。但
是,此后在输电技术上发生了巨大变化。

一方面,制造高电压大功率的直流发电机很困难,又不能
直接给直流电升压,而远距离输电需要高电压。另一方面,十
九世纪八十年代末发明了三相交流发电机、变压器,特别是发
明了结构简单,运行可靠、价格便宜的感应电动机,于是,适
应这种情况,输电技术转为发展交流输电。

但是,现在直流输电作为解决输电技术困难的方向之一,
又重新受到重视。近二十年来,高压直流输电有了迅速发展。
那么交流输电遇到了什么困难,直流输电又有什么优点呢?随
着电力系统的扩大,输电功率的增加,输电距离的增长,交流
输电遇到了一些技术困难,例如:

导线不但有电阻,还有电感,较细的导线,电阻的作用超
过电感,在输电功率大,输电导线横截面超过95毫米的情
况下,对50赫的交流电来说,感抗超过了电阻,但对稳定的直
流电则只有电阻,没有感抗。

输电线一般是架空线,但需要跨过海峡给海岛输电时要
用水下电缆,需要穿过人口密集的城市输电时要用地下电缆,
电缆在金属心线的外面包着一层绝缘皮,水和大地都是导体,
被绝缘皮隔开的金属心线和水(或大地)构成了电容器;一条
200千伏的水下电缆,每千米长的电容约0.2微法,在交流输
电的情况下,这个电容对受电端起旁路电容的作用,并且随电
缆增长而增大,电缆长度超过50千米,旁路电容会增大到交
流电几乎送不出去的程度,这时交流输电已无实际意义,而只
能用直流输电,因为电容对稳定的直流不起作用。

交流电每个周期有正、负最大值各一次。设想有甲、乙两台
交流发电机给同一条电路供电,假如甲的是正的最大值时,乙
恰好是负的最大值,它们发的电在电路里恰好互相抵消,电路
无法工作,所以要电路正常工作,给同一条电路供电的所有
发电机都必须同步运行,即同时达到正的最大值,同时达到负
的最大值,同时为零,现代的供电系统是把方圆数百千米内
的电站连成一个电力网,而且用高压输电线路将相隔数百甚
至上千千米的电力网连接起来,使它们可以互相支援,调剂余
缺,而要使这么大范围的这么多发电机同步运行,技术上是很
困难的,但直流输电就不存在同步问题。

现代的直流输电,只是输电这个环节是直流,发电仍是交
流。在输电线路的起端,有专用的换流设备将交流变换为直
流,在输电线路的末端(受电端),也有专用的换流设备将直流
换为交流,目前换流设备存在着制造难,价格高等困难,有待
研究解决。

高压直流输电主要用于远距离大功率输电、海底电缆输
电、非同步运行的交流系统之间的连络等方面。我国在1977
年建成了31千伏的直流输电工业性试验线路。随着大型水
电站的开发和坑口电站的建设,以及大电网的互相联接,远距
离大功率的直流输电必将在我国得到发展。

\subsection*{练习七}
\begin{enumerate}
    \item 在课文所给的例子中,用110伏和11千伏的电压输
电,分别要用96000${\rm mm}^2$和9.6${\rm mm}^2$的铝导线。如果不用铝
导线,而用铜导线,导线的横截面积分别要多大?
\item 从发电站输出的功率为200千瓦,输电线的总电阻
为0.05欧,用110伏和11千伏的电压输电,在这两种情况
下,在输电线上损失的电压各是多少伏?输送到用户的电压各
是多少伏?在输电线上损失的功率各是多少千瓦?
\item 用220伏和11千伏两种电压来输电,输送的功率相
同,在输电线上损失的功率相同,导线的长度和电阻率也相
同,求导线的横截面积之比。
\end{enumerate}

\section{交流电的整流}
由供电线路供给的通常都是交流电,但在生产和生活中
也要用直流电,如电解、电镀、蓄电池充电、直流电动机等都要
用直流电,这就需要把交流电变成直流电。把交流电变成直
流电的过程叫做\textbf{整流}。晶体二极管(简称二极管)具有单向导
电性,只许电流单方向通过,可以用来整流。常用的整流电路
有半波整流、全波整流和桥式整流。

\subsection{半波整流}

图3.31是半波整流电路。$B$是电源变压器,
$D$是二极管,$R$是用电器的电阻,也叫负载电阻或负载。
\begin{figure}\centering
    \begin{circuitikz}[>=latex,european]


\draw (0,1.3)--(0,0)--(1,0) to [cute inductor] (1,3)--(0,3)--(0,1.7) ;
\draw (1.5,0)--(5,0) to [R=$R$] (5,3) --(1.5,3) node[above]{$a$} to  [cute inductor] (1.50,0)node[below]{$b$};
\draw (1.5,3) to[empty diode, fill=black](5,3);
 \draw (0,1.3) [fill=white]circle (1.5pt);
        \draw (0,1.7) [fill=white]circle (1.5pt);
        \node at (0,1.5){$\sim$};
\draw [ultra thick](1.25,1)--(1.25,2);   \node at (3.125,3.5){$D$};
\node at (1.25,2.5){$B$};
\draw [<->](2.25,0)--node [fill=white]{$u$} (2.25,3);
\draw [<->](4,0)--node [fill=white]{$u_R$} (4,3);
%\draw [ultra thick] (3.75,3.25)--(3.75,2.75);
\node at (5.25, .8){$-$}; \node at (5.25, 2.2){$+$};
\end{circuitikz}
\caption{}
\end{figure}

当变压器的原线圈有交流电输入时,变压器的副线圈就
有交变电压输出,设输出的交变电压为$u=U_m\sin\omega t$,它的
波形如图3.32甲所示。
\begin{figure}\centering
    \begin{tikzpicture}[>=latex, xscale=.5]
\foreach \x in {0,-3,-6}
{
    \draw [<->](0, \x+2)--(0,\x)--(14,\x)node[right]{$t$};
    \node at (-.25,\x-.25){$O$};
}

\foreach \y in {1,2,3,4}
{
    \draw [dashed] (3.1416*\y, 0)--(3.1416*\y, -6);
}

\draw [ultra thick] plot[domain=0:4*3.1416, samples=1000] function{1.2*sin(x)} ;
\draw [ultra thick] plot[domain=0:3.1416, samples=1000] function{1.2*sin(x)-3} ;
\draw [ultra thick] plot[domain=2*3.1416:3*3.1416, samples=1000] function{1.2*sin(x)-3} ;
\draw [ultra thick] plot[domain=0:3.1416, samples=1000] function{0.8*sin(x)-6} ;
\draw [ultra thick] plot[domain=2*3.1416:3*3.1416, samples=1000] function{0.8*sin(x)-6} ;

\node at (-.25, 2){$u$};
\node at (-.45, 2-3){$u_R$};
\node at (-.45, 2-6){$i_R$};

\node at (3.1416-.25, -.5){$\dfrac{T}{2}$};
\node at (2*3.1416+.25, -.25){$T$};
\node at (3*3.1416-.25, -.5){$\dfrac{3T}{2}$};
\node at (4*3.1416+.25, -.25){$2T$};

\node at (-1, 1.5){甲};
\node at (-1, 1.5-3){乙};
\node at (-1, 1.5-6){丙};


\end{tikzpicture}
\caption{}
\end{figure}

当变压器输出的交变电压处于正半周时,$a$正$b$负,二极
管因加正向电压而导通,电流方向由$a$经$D$、$R$到$b$、二极管
导通时正向电阻很小,与负载电阻相可以忽略,因而电压全
部加在负载电阻上,这时负载电阻上的电压波形与变压器输
出的电压波形相同。

当变压器输出的交变电压处于负半周时,$a$负$b$正,二极
管因加反向电压而截止,二极管截止时,它的反向电阻可以
看作无限大,电路中的电流近似为零。这时电压全部加在二
极管上,负载电阻上的电压为零。

图3.32乙是负载电阻的电压波形,图3.32丙是负载电
阻的电流波形,可见,整流后负载电阻获得的是强度随时间
变化的直流电,也叫脉动直流电。

半波整流电路简单,使用元件少,但只利用交流电的半个
周期,负载获得的直流电脉动性大,变压器的利用率低,只适
宜在小功率和要求不高的设备中使用。

\subsection{全波整流}

图3.33是全波整流电路,两只二极管的负极
都通过负载电阻$R$与变压器副
线圈中心抽头$O$连接,当变压
器输出的变电压处于正半周
时,$a$正$b$负,$O$点电势介于$a$、$b$
之间,二极管$D_1$因加正向电压
而导通,$D_2$因加反向电压而截
止,这时电流方向是由$a$经$D_1$、
$R$到$O$,如图中实线箭头所示。
当变压器输出的交变电压处于负半周时,$a$负$b$正,二极管
$D_1$截止,$D_2$导通,这时电流方向是由$b$经$D_2$、$R$到$O$,如图
中虚线箭头所示,可见,无论正半周或负半周,通过负载电阻
R的电流方向总是相同的。图3.34是全波整流的波形,其中
甲是变压器副线圈$a$、$O$间或$O$、$b$间的交变电压波形,乙是负
载电阻的电压波形,丙是负载电阻的电流波形。

全波整流使交流电的两个半周期都得到利用,负载获得
的直流电脉动性比较小,但变压器副线圈要有中心抽头,副线
圈的圈数是半波整流的二倍,变压器利用率也不高,二极管
在截止状态时承受的反向电压是副线圈$a$、$b$间的全部电
压,因此要选用耐压性能较高
的二极管。

\subsection{桥式整流}

图3.35是桥式整流电路,当变压器输出的
交变电压处于正半周时,$a$正$b$
负,二极管$D_1$、$D_3$,因加正向电
压而导通,$D_2$、$D_4$因加反向电压而截止,这时电流方向是由$a$
经$D_1$、$R$、$D_3$、到$b$。当变压器输出的交变电压处于负半周时,$a$
负$b$正,二极管$D_1$、$D_3$因加反向电压而截止,$D_2$、$D_4$因加正向
电压而导通。这时电流方向是由$b$经$D_2$、$R$、$D_4$到$a$。可见,无
论正半周或负半周,通过负载电阻$R$的电流方向也总是相同
的。桥式整流的波形跟全波整流的波形相似。

桥式整流克服了半波整流和全波整流的缺点,在无线电
技术中应用较广,它的缺点是需要的整流元件较多。


\section{滤波}
交流电经整流后变成脉动直流电,脉动直流电可以看作是由强度和方向都不随时间变化的直流成分与强度和方向都
随时间变化的交流成分组成的。例如,图3.36甲所示的脉动直流电就可以看作是由直流成分(图乙)和交流成分(图丙)叠加形成的。其他的脉动直流电,也可以看作是由强度不变的直流成分和一个或几个交流成分叠加形成的。脉动直流电含有交流成分,往往不能适应实际需要,例如用来作收音机的电源,收音机就会发出嗡嗡的交流声,为了使脉动直流电变得比较平稳,需要把其中的交流成分滤掉,这叫做\textbf{滤波}。常用的滤波电路有电容滤波、电感滤波和$\pi$型滤波。
\begin{figure}[htp]\centering
    \begin{tikzpicture}[>=latex, xscale=.6]
        \foreach \x in {0,-3,-6}
        {
            \draw [<->](0, \x+2)node[left]{$i$}--(0,\x)--(14,\x)node[right]{$t$};
            \node at (-.25,\x-.25){$O$};
        }
      
        \draw [ultra thick] plot[domain=0:4*3.1416, samples=1000] function{sin(x)+1} ;
        \draw [ultra thick] plot[domain=0:4*3.1416, samples=1000] function{sin(x)-6} ;
        \draw [ultra thick] plot[domain=0:4*3.1416, samples=1000] function{-2} ;
        
        \node at (-1, 1){甲};
        \node at (-1, 1-3){乙};
        \node at (-1, 1-6){丙};
        
        
        \end{tikzpicture}
        \caption{}
\end{figure}

电容器具有通交流隔直流的作用,可以作滤波元件。图3.37甲是带有电容滤波的半波整流电路,电解电容器$C$和负载电阻$R$并联。并联电路有分流作用,当电容器的电容比较大时,容抗比较小,脉动电流的交流成分大部分通过电容器而滤掉,直流成分和一小部分交流成分通过负载电阻。因此,负载电阻的电压和电流就平稳多了。电容器的电容越大,负载电阻的电压和电流就越平稳,图3.37乙的粗实线是经过电容滤波的电压波形。这个电压波形可以用示波器观察到。

电感线圈具有通直流阻交流的作用,也可作滤波元件。图3.38甲是带有电感滤波的全波整流电路,低频扼流圈$L$和负
载电阻R串联。串联电路有分压作用,低频扼流圈的电感比较大,感抗也比较大,电压的交流成分大部分降在低频扼流圈上,低频扼流圈的电阻一般比负载电阻小,电压的直流成分大部分降在负载电阻上,因此,负载电阻的电压和电流也平稳多了。电感线圈的电感越大,负载电阻的电压和电流就越平稳,图3.38乙的粗实线是经过电感滤波的电压波形.

把电容滤波和电感滤波组合起来,可以组成滤波效果更好的$\pi$型滤波,图3.39是带有$\pi$型滤波的桥式整流电路。如果负载需要的电流不大,电感线圈也可以用电阻来代替,这样做虽然滤波效果差一些,但成本低、体积小,所以常常采用。

\subsection*{练习八}
\begin{enumerate}
	\item 图3.33中变压器副线圈$aO$间和$bO$间的交变电压都是6伏特。当$D_1$导通而$D_2$截止时,加在$D_2$上的反向电压的最大值是多大?
	\item 图3.40是哪一种整流电路?试用带箭头的线画出当$a$正$b$负时电流的通路。
	\item 图3.41中的信号是正弦交流电,最大值为1伏,直流电源的电动势为4.5伏。画出负载$R$两端的电压波形,如果二极管$D$反接,负载$R$两端的电压又怎样?直流电源的内电阻略去不计。
	\item 在图3.35所示的桥式整流电路中,如果二极管$D_1$的极性接反,会发生什么现象?如果二极管$D_1$已被击穿,又会发生什么现象?说明理由,并画出负载电阻上的电压波形。
\end{enumerate}

	
\section{三相交流电}
象图3.1那样,使一个线圈在磁场里转动,电路里只产生
一个交变电动势,这时发出的交流电叫做\textbf{单相交流电}。如果在磁场里有三个互成角度的线圈同时转动,电路里就产生三个交变电动势,这时发出的交流电叫做\textbf{三相交流电}。

图3.42是三相发电机的示意图。在铁心上固定着三个相同的线圈$AX$、$BY$、$CZ$,始端是$A$、$B$、$C$,末端是$X$、$Y$、$Z$。三个线圈的平面互成120$^\circ$角,匀速地转动铁心,三个线圈就在磁场里匀速转动,三个线圈是相同的,它们发出的三个电动
势,最大值和频率都相同,如果象图3.43那样把三个线圈分别跟负载连接起来,三相发电机就相当于三个独立的电源同时供电。

这三个电动势的最大值和频率虽然相同,但是它们的相位并不相同,由于三个线圈平面互成120$^\circ$角,所以三个电动势的相位也互差120$^\circ$。取图3.42所示的瞬间作为时间起点,这三个电动势可以分别表示为
\[\begin{split}
e_A&=\mathcal{E}_m\sin \omega t\\
e_B&=\mathcal{E}_m\sin (\omega t-120^\circ)\\
e_C&=\mathcal{E}_m\sin (\omega t-240^\circ)\\
\end{split}\]
它们的图象如图3.44所示。
\begin{figure}[htp]\centering
    \begin{tikzpicture}[>=latex, yscale=2]
\draw[<->] (0,1.5)node [right]{$e$}--(0,0)--(7,0)node[right]{$t$};
\draw [dashed](0,-1)--(2*3.1416,-1);
\draw [dashed](0,1)--(2*3.1416,1);
\foreach \x in {1,2,...,6}
{
    \draw [dashed](\x*3.1416/3, -1)--(\x*3.1416/3, 1);
}
\draw [ultra thick] plot[domain=0:2*3.1416, samples=1000] function{sin(x)} ;
\draw (0,0)--(0,-1.5);
\draw [ultra thick] plot[domain=0:2*3.1416, samples=1000] function{sin(x-3.1416*2/3)} ;
\draw [ultra thick] plot[domain=0:2*3.1416, samples=1000] function{sin(x+3.1416*2/3)} ;
\draw(3.1416*2/3,-1)--(3.1416*2/3,-1.5);
\draw [<->] (0,-1.3)--node[fill=white]{$\dfrac{T}{3}$}(3.1416*2/3,-1.3);
\node at (3.1416/2,1.2){$e_A$};
\node at (3.1416/2+3.1416*2/3,1.2){$e_B$};
\node at (3.1416/2+3.1416*4/3,1.2){$e_C$};




\end{tikzpicture}
\caption{}
\end{figure}

在实际应用中,三相发电机和负载都不是象图3.43那样连接,因为这种连接比起单相供电来没有什么优越之处,我们可以把图3.43中线圈末端和负载之间的三条导线合在一起,象图3.45那样用一条导线来连接,这样,每相负载上的电压不会改变,却可以节省两条导线。从每相线圈始端引出的导线叫做\textbf{端线},在照明电路里俗称火线,从公共点引出的导
线叫做\textbf{中性线},在照明电路里中性线是接地的,叫零线。这种有中性线的连接叫做\textbf{三相四线制}。照明电路通常采用三相四线制,灯泡连在端线和中性线之间(图3.46)。

\section{三相电路的连接}
现在我们进一步讨论三相电路的连接。先讨论电源的连接,然后讨论负载的连接。

\subsection{三相电源的连接}

在实际应用中,三相发电机的线圈通常都是采用图3.47所示的连接方法,这种连接方法叫做星形连接(符号是Y)。这时电源提供两种电压,一种是每相线圈两端的电压,叫做\textbf{相电压};另一种是每两根端线之间的电压,叫做\textbf{线电压}(图3.47)。在实际的发电机中,各相线圈的情况相同,所以三个相电压相等,通常用$U_{\text{相}}$来表示。这时三个线电压也相等,通常用$U_{\text{线}}$来表示。

在三相电源的星形连接中,相电压和线电压并不相等,这一点用伏特表测量一下就可以看出来,测量和计算表明,\textit{在
三相电源的星形连接中,线电压是相电压的$\sqrt{3}$倍},即
\[U_{\text{线}}=\sqrt{3}U_{\text{相}}\]

\subsection{负载的星形连接}

负载也可以采用星形连接,图3.45和图3.46中的负载就是星形连接的,在这种连接中,加在每相负载上的电压就是相电压(图3.48),通过每相负载的电流,可以利用交流电路中的欧姆定律计算出来,各相的负载不同,通过它们的电流也不相同,通过每相负载的电流叫做\textbf{相电流};通过端线的电流叫做\textbf{线电流}。在这种连接中,一根端线连接一相负载,各个线电流分别等于各个相电流。
\begin{figure}\centering
    \begin{circuitikz}[european,scale=.6,>=latex]
    
\draw (0,0) to [R=$2$, *-*] (2,-3) ;
\draw (0,0) to [R=$3$, *-*] (-2, -3);
\draw (0,0) to [R=$1$] (0,4);
\draw (0,4)--(-4-2,4);
\draw (-2,-3)--(-4-2,-3);
\draw (-4-2,-4)--(2,-4)--(2,-3);
\draw (0,0)--(-4-2,0);
\draw (0,4)--(-4-2,4);
\foreach \x in {0,-4,-3,4}
{
    \draw [fill=white] (-4-2,\x) circle (3pt);
}

\node at (-4.8-2,0){$O$};
\node at (-4.8-2,-4){$B$};
\node at (-4.8-2,-3){$C$};
\node at (-4.8-2,4){$A$};
\node at (.5,0){$O'$};
\draw [<->] (-3-2,0)--node[fill=white]{$U_{\text{相}}$}(-3-2,4);
\draw [->](-5,4)--(-4,4) node [above]{$I_{\text{线}}$};
\draw [->](0,4)--(0,3) node [right]{$I_{\text{相}}$};

    \end{circuitikz}
    \caption{负载的星形连接}
\end{figure}

中性线是三相负载的公共回线,通过的电流是三个相电流的总合,现在看一下三相负载相同时,通过中性线的电流。
例如三相负载是三个等值的电阻,或是三个相同的线圈,就属于负载相同的情况,图3.49表示在这种情况下三个相电流的曲线。三个电流的最大值和频率相同,位相互差120$^\circ$。从图中可以知道,三个相电流的总合在任何时刻都等于零,这一点,利用数学中学过的三角知识也可以得到证明,这就是说,在三相负载相同的情况下,在中性线里没有电流通过,既然
如此,中性线就可以省去不用了,这就成为\textbf{三相三线制}(图3.50)。


三相三线制适用于负载相同的情况,下节要讲的感应电动机,它的三个定子线圈是相同的负载,这种电动机就可以接在三相三线制电路中(图3.51)。在照明电路中,电灯的使用经常有变化,很难保证三相负载相同,必须用三相四线制。三相四线制比三相三线制多一根中性线,但可以提供两种电(220/380伏),照明和动力可以共用,相电压 220伏 供给照明,线电压380伏供给动力。

\subsection{负载的三角形连接}

负载也可以如图3.52所示那样连接起来,这种连接方法叫做三角形连接(符号是$\triangle$),在这种连接中,加在每相负载上的电压是线电压,通过每相负载的相
电流也可以利用交流电路中的欧姆定律计算出来,各相负载相同时,三个相电流相等,通过端线的三个线电流也相等,显然,这时线电流不等于相电流,测量和计算表明,在负载的三角形连接中,三相负载相同时,线电流是相电流的$\sqrt{3}$倍,即
\[I_{\text{线}}=\sqrt{3}I_{\text{相}}\]
\begin{figure}\centering
    \begin{circuitikz}[european,scale=.8,>=latex]
    
\draw (0,0) to [R=$2$, *-*] (5,0) to [R=$1$, *-*] (2.5, 4) to [R=$3$] (0,0);
\draw (5,0)--(5,-2)--(-4,-2);
\draw (0,0)--(-4,0);
\draw (2.5,4)--(-4,4);
\foreach \x in {0,-2,4}
{
    \draw [fill=white] (-4,\x) circle (3pt);
}

\node at (-4.3,0){$C$};

\node at (-4.3,-2){$B$};
\node at (-4.3,4){$A$};
\draw [<->] (-3.5,0)--node[fill=white]{$U_{\text{线}}$}(-3.5,4);
\draw [->](-2,4)--(-1,4) node [above]{$I_{\text{线}}$};
\draw [->](2.5, 4)--(2,4-.8)node [left]{$I_{\text{相}}$};
\draw [->](2.5, 4)--(3,4-.8)node [right]{$I_{\text{相}}$};

    \end{circuitikz}
    \caption{负载的三角形连接}
\end{figure}

三相负载应该怎样连接,要看电源电压和负载额定电压的情况而定,例如对于线电压是380伏的三相电路,如果每相负载的额定电压是380伏,就用三角形连接;如果每相负载的额定电压是220伏,就用星形连接。

\subsection*{练习九}
\begin{enumerate}
    \item 根据图3.49写出三相负载相同时的相电流的瞬时值的表达式,并利用数学中学过的三角知识证明:三个相电流之和在任何时刻都等于零。
    \item 在图3.46所示的三相四线制电路中,相电压是220伏。现有“220V,100W”的灯泡90盏,第一相安装40盏,第二相安装30盏,第三相安装20盏。求灯泡全接通时,各个相电流和线电流。
    \item 如果电源采用图3.53所示的三角形连接,线电压和相电压的关系是怎样的?设各相线圈的情况相同。
\end{enumerate}

\section{感应电动机}
\subsection{旋转磁场}

感应电动机是利用三相交流电可以产生旋转磁场以及电磁感应现象制成的。照图3.54那样,在磁铁中间放一个铝框,如果转动磁铁,造成一个旋转磁场,铝框就随着转动,磁铁转动时,铝框切割磁力线,其中产生感生电流,从楞次定律知道,感生电流和磁铁磁场的相互作用,要阻碍磁铁
和铝框的相对运动,现在磁铁是在外力作用下转动的,结果不是铝框阻止磁铁的转动,而是铝框被磁铁带着转动起来。

不仅转动磁铁可以产生旋转磁场,三相交流电也可以产生旋转磁场。取三个相同的线圈,照图3.55那样使它们的平面交成120$^\circ$角,线圈中通以三相交流电,中间的铝框就象被转动的磁铁带着一样转动起来,说明三相交流电也产生旋转磁场。

现在用图3.56来说明三相交流电为什么会产生旋转磁场。在开始的时刻($t=0$),即在图3.56甲所示的瞬间,线圈$AX$中的电流$i_{AX}$是零;线圈$BY$中的电流$i_{BY}$是负值,电流从末端$Y$流向始端$B$;线圈$CZ$中的电流$i_{CZ}$是正值,电流从始端$C$流向末端$Z$。电流$i_{BY}$、$i_{CZ}$形成的合成磁场如图中的虚线所示,方向指向$A$。

经过$T/3$,在图3.56乙所示的瞬间,线圈$AX$中的电流
$i_{AX}$是正值,电流从始端$A$流向末端$X$;线圈$BY$中的电流$i_{BY}$是零;线圈$CZ$中的电流$i_{CZ}$是负值,电流从末端$Z$流向
始端$C$。电流$i_{AX}$、$i_{CZ}$形成的合成磁场如图中的虚线所示,方向指向$B$。可以看出,磁场方向沿逆时针转过120$^\circ$角。

图3.56丙和丁分别表示经过$2T/3$和$T$时,各个线圈中电流的方向和合成磁场的方向。可以看出,每经过$T/3$,磁场方向就沿逆时针转过120$^\circ$角。在交流电的一周期内磁场旋转一周。

\subsection{感应电动机}

感应电动机是利用图3.55中的现象制成
的,它有一个定子(图3.57甲)和一个转子(图3.57乙),在定子内侧的凹槽里联有互成120$^\circ$角的三组线圈(定子绕组),把这三组线圈用星形接法或三角形接法连入三相电路中,就产生旋转磁场。

感应电动机的转子是由铁心和嵌在铁心上的闭合导体构成的,闭合导体是由嵌在铁心凹槽中的铜条(或铝条)和两个铜环(或铝环)连在一起制成的,形状象个鼠笼(图3.57丙),
所以这种电动机也叫鼠笼式感应电动机,这个闭合导体相当于图3.56中的铝框,有了旋转磁场,它就转动起来。

鼠笼式感应电动机的构造简单,要改变转动方向,只要把定子上的任意两组线圈的电流互换一下就行了。这种电动机制造、使用、保养都比较简单,广泛应用在工农业生产中。能够使用鼠笼式感应电动机,是三相交流电的另一个优点。

\section*{阅读材料:直线电机和磁悬浮列车}

一般电动机工作时都是转动的,但是用旋转电机来驱动
的交通工具要做直线运动,用旋转电机来驱动的机器的一些部件也要做直线运动。这就需要有把旋转运动变为直线运动的一套装置。能不能直接利用直线运动来驱动,而省去这套装置呢?几十年前人们就提出了这个问题,现在已制成了利用直线运动的电动机,即直线电机。

直线电机的原理并不复杂。设想把一台旋转感应电动机沿着一条半径的方向剖开,并且展平,就成了一台直线感应电动机(图3.58),在直线电机中,相当于旋转电机定子的,叫初级;相当于旋转电机转子的,叫次级。初级中通以交流电,
次级在电磁力的作用下就沿着初级做直线运动,这时初级要做得很长,延伸到运动所需要达到的位置,而次级不需要那么长,实际上,直线电机既可以把初级做得很长,也可以把次级做得很长;既可以初级固定、次级移动,也可以次级固定,初级移动。

直线电机是一种新型电机,近年来应用日益广泛,磁悬浮列车就是用直线电机来驱动的。磁悬浮列车是一种新型列车,是铁路技术中一项新兴的技术成果,一般的列车由于车轮和铁轨之间有摩擦,所能达到的最高运行速度为300$\kmh$左右,要超过这个速度,用旋转电机来驱动并不理想,用直线电机来驱动却很含适,直线电机的一个级固定于地面,跟导轨一起延伸到远处;另一个级安装在列车上,初级通以交流电,列车就沿导轨前送。同时利用磁悬浮使列车跟导轨脱离接触,减小摩擦。列车上装有磁体(有的就是兼用直线电机的线園),磁体随列车运动时,使设在地面上的金属板或线圆中出现感生电流,利用磁体和感生电流之间的电磁力把列车悬浮起来,有的国家已建立了几千米的磁悬浮列车试验线,运行速度高达500$\kmh$以上。人们预计,本世纪末磁悬浮列车可达到实用化的阶段。

直线电机除了用于磁悬浮列车,还可广泛地用于其他方
面,如用于传送系统、电气锤、电磁搅拌器等。在我国,直线电视也逐步得到推广和应用。直线电机的原理虽然不复杂,但在设计、制造方面有它自己的特点,产品还不如旋转电机那样成熟,有待进一步研究和改进。

\subsection*{练习十}
\begin{enumerate}
    \item 有一台感应电动机,铭牌上标有“220/380”“$\triangle$/Y”的字祥。这表示这台电动机每相定于线圈的额定电压是 220伏,如果线电压是220伏,定子线圈要连成三角形;如果线电压是380伏,定子线圈要连成星形,为什么?
    \item 实验表明:只要把定子上任意两组线圈的电流互换一下,旋转磁场就向相反方向旋转。画出类似于图3.56的
    图,并加以分析说明。
    \item 在图3.54所示的实验中,铝框总要比磁铁转得慢,即比旋转磁场转得慢,而不能跟磁场转得一样快,即不能同步旋转,因此感应电动机也叫异步电动机,试说明铝框为什么不能跟磁铁同步旋转。
\end{enumerate}

\section*{复习题}
\begin{enumerate}
\item 就图3.1所示的装置简述正弦交流电的产生和正弦交流电的变化规律,
\item 表征正弦交流电的三个物理量是什么?什么叫交流电的最大值和有效值?对正弦交流电来说,二者之间有什么关系?什么叫做交流电的相和初相?什么叫做相差?
\item 纯电阻电路中欧姆定律的表达式是怎样的?功率的表达式是怎样的?电阻对交流电的电流和电压的相位关系有没有影响?
\item 纯电感电路中欧姆定律的表达式是怎样的?感抗跟
自感系数和交流电频率间有什么关系?电感线圈在电路中表现出什么特性?
\item 纯电容电路中欧姆定律的表达式是怎样的?容抗跟电容和交流电频率间有什么关系?电容器在电路中表现出什么特性?
\item 纯电感电路中,电流和电压的相位关系是怎样的?在纯电容电路中,电流和电压的相位关系是怎样的?
\item 交流电的功率表达式是怎样的?什么叫有功功率、视在功率和功率因数?
\item 简述变压器的工作原理。变压器原副线圈上的端电压之比等于什么?原副线圈中的电流之比等于什么?
\item 简述电能输送的原理。
\item 简述交流电整流和滤波的原理和方法。
\item 简述三相交流电的产生,在电源的星形连接中,线电压和相电压的关系是怎样的?在负载的星形连接中,线电流和相电流的关系是怎样的?在负载的三角形连接中,线电流和相电流的关系是怎样的?
\item 简述感应电动机的原理。
\item 你自己总结一下,交流电比直流电有哪些优点。
\item 你自己总结一下,交流电和直流电存在什么共同点,交流电又有什么特点,哪些规律形式相同,哪些规律有所不同。
\end{enumerate}

\section*{习题}
\begin{enumerate}
    \item 把电阻和电客器串联在交流电路中,测得电阻上的电压为30伏,电容器上的电压为40伏。已知电阻的阻值为200欧,交流电的频率为50赫,求电容器的电容,这道题提供了一种用伏特表测定电容的方法。
    \item 把电阻和电容器并联在交流电路中,已知电阻的阻值是500欧,电容器的电容是30微法。当交流电的频率为50赫时,通过电阻和电容器的电流之比是多大?当交流电的频率是500赫时,电流之比又是多大?
    \item 变压器的原线圈为1100匝,副线圈为180匝,原线圈接到220伏的交流电路中,副线圈上并联了3个阻值都是90欧的用电器。如果原线圈允许通过的最大电流为0.9安,副线圈上最多还可以并联多少个阻值为60欧的用电器?
    \item 有一个教学用的可拆变压器,它的原副线圈外部还可绕线,现在要测定原副线圈的匝数。现有一根足够长的绝缘导线,还需要什么器材?简要说明实验原理。
    \item 图3.59是电工常用的钳形电流表,可以用来测定交变电流,把钳口打开,把被测的通电导线放在钳口中间(右图),交流电表就可以测出导线中的电流强度。试说明钳形电流表的工作原理。
    \item 在图3.35所示的桥式整流电路中,变压器原副线圈的匝数比等于8,原线圈接在220伏的交流电路中,能不能选用最高反向工作电压为50伏的晶体二极管进行整流?为什么?
    \item 图3.60是电子技术中用到的限幅电路,电池组的电动势都为$\mathcal{E}$,左端输入的是正弦交流电,电压$u_1$的最大值为$2\mathcal{E}$。试画出右端输出的电压$u_2$的图象,并分析说明理由。电池组的内电阻略去不计。
    \item 在图3.46所示的三相四线制照明电路中,设$A$相接通了8盏“220V,100W”的灯泡,$B$相接通了2盏“220V,100W”的灯泡,$C$相中没有接通灯泡,这时接通的灯泡都正常发光,因某种原因中性线断开了(在该图中$O$处断开),将会发生什么现象?说明理由。
    
    在三相四线制电路中,中性线在任何时候都不能断开。
    为了避免中性线断开,在中性线上也不能安装开关和保险丝。做完这道题,你将会对此有所了解。
    \item 如图3.61所示,电源采用星形连接,负载采用三角形连接。电源的相电压是220伏,各相负载相同,阻值都是110欧,求通过各相负载的相电流和线电流。
    \item 每一台电动机都有一定的额定功率。在实际中要根据负载的功率来选择电动机的功率,使电动机的额定功率等于或稍大于负载的功率,有一台水浆,抽水量$Q=0.03{\rm m^3}/{\rm s}$,抽水高度$h=20{\rm m}$,效率$\eta_1=0.55$,用一台感应电动机道过皮带传动来带动,皮带传动的效率$\eta_2=0.8$。现有三台感应电动机,额定功率分别是14千瓦、20千瓦、28千瓦,应当选择哪一台?计算时取$g=10\msq$。
\end{enumerate}





































\chapter{电磁振荡和电磁波}
无线电广播和电视广播都是利用电磁波传播的.导弹和人造卫星的控制,宇宙飞船跟地面的通讯联系,也要利用电磁波.电磁波是什么呢?怎样利用它来传递各种信号呢?这一章就要学习这方面的知识,正如机械振动能够产生机械波一样,电磁振荡能够产生电磁波,我们就从电磁振荡开始学习.

\section{电磁振荡}
\subsection{电磁振荡的产生}


\begin{figure}[htp]\centering
\begin{circuitikz}[>=latex]
\draw (0,0)--(3,0);
\draw (4,0)--(5.5,0);
\draw (0,3) to [rmeter, t=G] (4,3)--(5.5,3) to [battery] (5.5,0);
\draw (0,0) to [american inductors, L] (0,3);
\draw (3.5, 3) to [C=$C$] (3.5,1)--(3.5,.5);
\draw [fill=white] (3,0) circle (1.5pt);
\draw [very thick] (3.5,.5)--(3.5,-.2)node [below]{$K$}; \draw [fill=white] (3.5,.5) circle (1.5pt);
\draw [fill=white] (4,0) circle (1.5pt);
\node at (.5, 1.5){$L$};\node at (6.2, 1.5){$E$};
\draw [very thick]  (-.4,1)--(-.4,2);
\draw [fill=black] (3.5,3) circle (1.5pt);
\end{circuitikz}
\caption{振荡电路}
\end{figure}

在图4.1所示的电路中,先把开关扳到电池组一边,给电容器充电.稍后再把开关扳到线圈一边,让电容器通过线圈放电,我们会看到电流表的指针左右摆动,表明电路里产生了大小和方向作周期性变化的交变电流.通常把这样产生的交变电流叫做
\textbf{振荡电流}.能够产生振荡电流的电路叫做\textbf{振荡电路},图4.1中由电感线圈和电容器组成的电路就是一种简单的振荡电路,简称$LC$回路.

用示波器来观察振荡电流,可以看到,在$LC$回路里产生的振荡电流也是按正弦规律变化的.

下面分析$LC$回路里产生振荡电流的过程.

在开关刚扳到线圈一边的瞬间,已被充电的电容器尚未放电,电路里没有电流,电路里的能量全部是电容器里储存的电场能(图4.2甲).
\begin{figure}[htp]\centering
\includegraphics[scale=.75]{fig/4-2.png}
\caption{电磁振荡的过程}
\end{figure}

电容器开始放电后,由于线圈的自感作用,电路里的电流不能立刻达到最大值,而是由零逐渐增大,放电过程中,线圈周围产生磁场,并且随着电流的增大而增强;电容器极板上的电荷逐渐减少,电容器里的电场逐渐减弱.这样,电路里的电场能逐渐转化为磁场能,到放电完了时,电流达到最大值,电容器极板上已经没有电荷,电场能全部转化为磁场能(图4.2乙).


电容器放电完了后,由于线圈的自感作用,电路里的电流并不立即减小为零,而是保持原来的方向继续流动,使电容器在反方向上重新充电.在反方向充电过程中,随着电流的减小,线圈周围的磁场逐渐减弱;电容器两极板带上相反的电荷,电容器里的电场随着极板上电荷的增多而增强.这样,电
路里的磁场能又逐渐转化为电场能,充电完了时,电流减小到零,电容器极板上的电荷达到最大值,磁场能全部转化为电场能(图4.2丙).

此后电容器再放电,再充电,这样不断地充电和放电,电路中就有了振荡电流,同时电场能和磁场能发生周期性的转化.这种现象叫做\textbf{电磁振荡}.

图4.2中的电磁振荡跟机械振动中的自由振动类似,叫做\textbf{自由振荡}.最初给电容器充电,相当于使单摆的摆锤偏离平衡位置,给摆锤一定的重力势能,电路中电场能和磁场能的相互转化,相当于单摆中重力势能和动能的相互转化.

\subsection{无阻尼振荡和阻尼振荡}

在自由振荡中,如果没有能量损失,振荡应该持续下去,振荡电流的振幅应该保持不变.这种振荡叫做\textbf{无阻尼振荡}(图4.3甲).可是,实际上在电磁振荡中总要有能量损失,一部分能量由于电路中有电阻而转化为热,还有一部分能量辐射到周围空间中去.这样,振荡电路
的能量逐渐损耗,振荡电流的振幅逐渐减小,直到最后停止下来.这种振荡叫做\textbf{阻尼振荡}(图4.3乙).
\begin{figure}[htp]\centering
\begin{tikzpicture}[>=latex, xscale=.6]
\draw[->](0,0)node[left]{$O$}--(13,0);
\draw[->](0,-1)--(0,1)node[right]{$i$};
\draw [very thick] plot[domain=0:3.1416*4, samples=1000] function{0.5*cos(4.5*x)} ;
\node at (6.5, -1.5){甲:无阻尼振荡};
\end{tikzpicture}

\begin{tikzpicture}[>=latex, xscale=.6]
    \draw[->](0,0)node[left]{$O$}--(13,0);
    \draw[->](0,-1)--(0,1)node[right]{$i$};
    \draw [very thick] plot[domain=0:3.1416*4, samples=1000] function{.5*exp(-0.2*x)*cos(4.5*x)} ;
    \node at (6.5, -1.5){乙:阻尼振荡};


\end{tikzpicture}
\caption{}
\end{figure}

实际工作中常常需要保持振幅不变的等幅振荡,这种等幅振荡要用\textbf{振荡器}来产生.振荡器靠晶体管(或电子管)周期地把电源的能量补充到振荡电路中,用来补偿振荡过程中的能量损耗,以维持等幅振荡.

\subsection*{练习一}
\begin{enumerate}
	\item 画一条按正弦规律变化的振荡电流的曲线,并在这条曲线上标出对应于图4.2甲、乙、丙、丁、戊各个时刻电流值的点.
	\item 在图4.2所示的电磁振荡中,何时电容器里的电场最强?何时线圈里的磁场最强?电场能和磁场能是怎样相互转化的?
	\item 把$LC$回路中产生的自由振荡跟单摆的简谐振动相对比,说明它们类似的地方.
\end{enumerate}


\section{电磁振荡的周期和频率}
电磁振荡完成一次周期性变化需要的时间叫做\textbf{周期},一秒钟内完成的周期性变化的次数叫做\textbf{频率}.

振荡电路里发生无阻尼自由振荡的周期和频率,叫做振荡电路的\textbf{固有周期}和\textbf{固有频率},简称振荡电路的周期和频率.

$LC$回路的周期和频率跟哪些因素有关呢?让我们改变
电容和电感的大小,重做图4.1的实验,可以看到:电容或电感增加时,周期变长,频率变低;电容或电感减小时,周期变短,频率变高.

上述现象可以这样来说明,加在电容器上的电压一定时,电容器的电容越大,它容纳的电荷就越多,放电和充电需要的时间就越长,因而周期就越长,频率就越低,线圈的电感越大,阻碍电流变化的作用就越强,放电和充电需要的时间就越长,因而周期就越长,频率就越低.

进一步的研究可以证明,周期$T$和频率$f$跟自感系数$L$和电容$C$的关系是:
\[T=2\pi\sqrt{LC},\qquad f=\frac{1}{2\pi\sqrt{LC}} \]
式中$T$、$f$、$L$、$C$的单位分别是秒、赫兹、亨利、法拉.

根据上述公式,选用适当的电容器和线圈,就可以使振荡电路的周期和频率符合我们的需要,要改变振荡电路的周期和频率,可以通过改变电容或电感的办法来实现.

\subsection*{练习二}
\begin{enumerate}
	\item 一个$LC$回路能够产生535千赫到1605千赫的电磁振荡,已知线圈的自感系数是300微亨,可变电容器的最大电容和最小电容各是多少?
	\item $LC$回路中的可变电容器的电容可从30皮法变到15皮法.要使这个回路的最低固有频率为1000千赫,线圈的自感系数应为多大?用这个线圈,回路的最高固有频率是多大?
	\item 如果把$LC$回路中电容为$C$的电容器用两个电容也为$C$的电容器串联起来代替,回路的固有周期怎样变化?如果不是串联而是并联,固有周期又怎样变化?
\end{enumerate}


\section{电磁场}

电磁振荡能够产生电磁波,为了说明这个问题,我们需要知道一些有关电磁场的知识,十九世纪六十年代,英国物理学家麦克斯韦(1831—1879)在总结前人研究电磁现象的基础上,建立了完整的电磁场理论,这个理论不仅全面地说明了当时已知的电磁现象,而且成功地预言了电磁波的存在.下面我们简要地介绍一下这个理论的要点.

麦克斯韦用场的观点分析了电磁感应现象,并作出结论:\textit{变化的磁场能够在周围的空间产生电场}.这是电磁场理论的第一个要点,如图4.4所示,在变化的磁场中放一个闭合
电路,电路中就产生感生电流,这是我们学过的电磁感应现象.我们知道,要产生电流,必须有使电荷做定向移动的电场
存在,现在闭合电路里有了电流,说明有电场存在;这个闭合电路中没有其他电源,这个电场只能认为是由于磁场的变化而产生的.麦克斯韦进一步认为,这个电场的产生,跟是否存在着闭合电路没有关系,只要磁场发生变化,在周围空间里就要产生电场.
\begin{figure}[htp]\centering
\includegraphics[scale=1.2]{fig/4-4.pdf}
\caption{变化的磁场产生电场(磁场增强时)}
\end{figure}

麦克斯韦还指出:变化的磁场所产生的电场,是由磁场的变化情况决定的,如果磁场的变化是均匀的,即在相等的时间内磁感应强度的变化相等,产生的电场就是稳定的,即电场强度不随时间而变化.如果磁场的变化是不均匀的,产生的电场就是变化的.

既然变化的磁场能够产生电场,相反地,变化的电场是否也能够产生磁场呢?麦克斯韦用场的观点分析了电磁现象,认为\textit{变化的电场能够在周围的空间产生磁场},这是电磁场理论的第二个要点,一个静止的电荷,它产生的是静电场,即空间各点的电场强度不随时间而变化,这个电荷一且运动起来,电场就发生变化,即空间各点的电场强度将随着时间而变化.另一方面,运动电荷要产生磁场.用场的观点来分析这个问题,就可以说:这个磁场是由变化的电场产生的,在电容器充放电时,两极板间的电场发生变化,这个变化的电场产生磁场(图4.5),而且这个磁场跟假定两极板间存在电流时所产生的磁场是一样的.
\begin{figure}[htp]\centering
\includegraphics[scale=1.2]{fig/4-5.pdf}
\caption{变化的电场产生磁场}
\end{figure}


变化的电场所产生的磁场,是由电场的变化情况决定的.
如果电场的变化是均匀的,产生的磁场就是稳定的.如果电场的变化是不均匀的,产生的磁场就是变化的.

变化的磁场产生电场,变化的电场产生磁场,这是麦克斯韦理论的两大支柱.按照这个理论,变化的电场和磁场总是相互联系的,形成一个不可分离的统一的场,这就是电磁场.电场和磁场只是这个统一的电磁场的两种具体表现.

\section{电磁波}
\subsection{电磁波的产生}

从麦克斯韦的电磁场理论可以知道:如果在空间某处发生了不均匀变化的电场,就会在邻近的空间引起变化的磁场,这个变化的磁场又会在较远的空间引起新的变化的电场,接着又在更远的空间引起新的变化的磁场,这样,变化的电场和变化的磁场并不局限于空间某个区域,而要由近及远向周围空间传播开去,电磁场这样由近及远地传播就形成\textbf{电磁波}.
\begin{figure}[htp]\centering
\includegraphics[scale=.75]{fig/4-6.png}
\caption{电磁波传播过程的示意图}
\end{figure}

在图4.1所示的振荡电路中有振荡电流时,会产生周期性变化的电场和磁场,这种电场和磁场的变化是不均匀的,因而会激起电磁波向外传播(图4.6).电磁波的周期和频率等于激起电磁波的振荡电流的周期和频率,振荡电路中有振荡
电流时,电荷做快速振动,可见做快速振动的电荷会激起电磁波.

\subsection{电磁波的特点}
\begin{figure}[htp]\centering
\includegraphics[scale=1.2]{fig/4-7.pdf}
\caption{沿$z$轴传播的电磁波在某一时刻的波的图象}
\end{figure}

图4.7表示作正弦变化的电场或磁场所
引起的电磁波在某一时刻的波的图象.波峰表示在该点的电场强度$E$或磁感应强度$B$在正方向具有最大值,波谷表示在该点的$E$或$B$在反方向具有最大值.两个相邻的波峰(或波谷)之间的距离等于电磁波的波长.在任一时刻,$E$和$B$沿电磁波的传播方向是作正弦变化的,在传播方向上的任一点,$E$和$B$都是随时间作正弦变化的,即$E$和$B$在振动.$E$的振动方向平行于$x$轴,$B$的振动方向平行于$y$轴,它们彼此垂直,而且都跟波的传播方向垂直,因此电磁波是横波.
\begin{figure}[htp]\centering
\begin{tikzpicture}[>=latex, xscale=.5]
\foreach \x in {0,2,4}
{
   \draw[->] (-.5, \x)--(20,\x)node [right]{$z$};
\draw [->](0,-.8+\x)--(0,1+\x)node [right]{$E$};
\node at (-.25,\x-.25){$O$};
}

\draw [very thick] plot[domain=0:4*3.1416, samples=1000] function{.7*sin(x)+4} ;
\draw [very thick] plot[domain=0:5*3.1416, samples=1000] function{-.7*sin(x)+2} ;
\draw [very thick] plot[domain=0:6*3.1416, samples=1000] function{.7*sin(x)} ;

\foreach \x in {1,2,3}
{
\draw[dashed] (2*3.1416*\x, -1)--(2*3.1416*\x, 5);
}

\draw[<->] (4*3.1416, 4.5) to node [fill=white]{$\lambda$} (6*3.1416, 4.5);

\node at (-4, 0){丙};\node at (-4, 4){甲};\node at (-4, 2){乙};
\node at (-2, 0){$t=T$};\node at (-2, 4){$t=0$};\node at (-2, 2){$t=\dfrac{1}{2}T$};

\draw[->] (3.1416/2, 0)--(3.1416/2, 0.7);
\draw[->] (3.1416/2, 2)--(3.1416/2, 2-0.7);
\draw[->] (3.1416/2, 4)--(3.1416/2, 4+0.7);

\draw[->] (3.1416*1.5, 0)--(3.1416*1.5, -0.7);
\draw[->] (3.1416*1.5, 2)--(3.1416*1.5, 2+0.7);
\draw[->] (3.1416*1.5, 4)--(3.1416*1.5, 4-0.7);

\draw[->] (3.1416/4, 0)--(3.1416/4, 0.7/1.414);
\draw[->] (3.1416/4, 2)--(3.1416/4, 2-0.7/1.414);
\draw[->] (3.1416/4, 4)--(3.1416/4, 4+0.7/1.414);

\draw[->] (3.1416*.75, 0)--(3.1416*.75, 0.7/1.414);
\draw[->] (3.1416*.75, 2)--(3.1416*.75, 2-0.7/1.414);
\draw[->] (3.1416*.75, 4)--(3.1416*.75, 4+0.7/1.414);

\draw[->] (3.1416*1.25, 0)--(3.1416*1.25, -0.7/1.414);
\draw[->] (3.1416*1.25, 2)--(3.1416*1.25, 2+0.7/1.414);
\draw[->] (3.1416*1.25, 4)--(3.1416*1.25, 4-0.7/1.414);

\draw[->] (3.1416*1.75, 0)--(3.1416*1.75, -0.7/1.414);
\draw[->] (3.1416*1.75, 2)--(3.1416*1.75, 2+0.7/1.414);
\draw[->] (3.1416*1.75, 4)--(3.1416*1.75, 4-0.7/1.414);

\end{tikzpicture}
\caption{电磁波以一定速度在空间传播(为简单起见只画出了$E$矢量)}
\end{figure}

电磁波在空间以一定的速度传播,图4.8表示经过一个周期$T$电磁波向前传播的情形.可以看出,经过一个周期$T$,电磁波传播的距离等于波长$\lambda$.因此,我们在高中一年级学过的波长、频率(或周期)和波速之间的关系,即$v=\lambda/T=\lambda f$,对电磁波也完全适用.

麦克斯韦从理论上预见:电磁波在真空中的传播速度等于光在真空中的传播速度,即电磁波在真空中的传播速度$c=3.00\x10^8\ms$,这个论断后来得到了实验证实.从理论上发现电磁波以光速传播,这是物理学史上最伟大的成就之一.后面我们学到光学时还要说明这一点.

从场的观点来看,必须把场看作能量的贮存场所,电场
贮存电能,磁场贮存磁能,电磁场贮存电磁能.因此,电磁波的发射过程,也就是辐射能量的过程.电磁波在空间传播,电磁能就随同着一起传播.

电磁波虽然可以由做快速振动的电荷所激起,但它可以脱离电荷而独立存在,设想一个振动的电荷突然停止运动,它的场就变成了静电场.原先它所激起的电磁波并不随同消失,而是在空间继续传播着.这时电磁波独立地存在着,我们可以象研究任何其他事物的变化过程一样来研究电磁波的变化过程.

电磁波有一点跟机械波大不相同,机械波要靠媒质来传播,电磁波可以在真空中传播,它的传播并不需要靠别的物质来做媒质;由于变化的电场和变化的磁场具有密不可分的联系,电磁波本身就能够传播.

电磁波可以脱离电荷而独立存在,并且不需要别的物质来做媒质就能够在空间传播,电磁波也具有能量.这样看来,电磁场跟由原子、分子组成的物质(通常叫做实物)一样,都是不依赖于我们感觉的客观存在,电磁场是物质的一种特殊形态.

现在让我们引用大科学家爱因斯坦的话作为本节的结束.他说:“在麦克斯韦的理论中,电场和磁场,或简单些说电磁场,是一种实在的东西”,“在一个现代物理学家看来,电磁场正和他所坐的椅子一样地实在”,“对目前来说,我们仍然需要认定场与实物都一起存在”.(引自爱因斯坦,英费尔德著:《物理学的进化》)

\section{赫兹实验}
麦克斯韦从理论上预见电磁波的存在,在这个预见以后,过了二十多年,在1888年,德国物理学家赫兹(1857—1894)第一次用实验证实了电磁波的存在.

\begin{figure}[htp]\centering
	\includegraphics[scale=1]{fig/4-9.PDF}
	\caption{赫兹实验}
	\end{figure}

赫兹实验的原理如图4.9所示.左方是电磁波发射器,$A$和$B$是两根金属杆,它们相对的那端各带一个金属球,两球之间留有间隙,把两杆接到感应圈$C$的两极上.感应圈是一个特殊的变压装置,它可以把低电压变成高电压.当两金属
球之间的电压足够高时,空气被击穿,在两球间隙中发生火花放电,每跳一次火花,电荷在两球之间往复多次,形成高频的振荡电流.火花放电是间断性的,跳过一次火花之后,接着跳过第二次火花,这样,就间断性地发射出电磁波.

图4.9右方是电磁波接收器,它是一个金属圆环,也留有
一个间隙,在间隙处环的两端带有金属球,当电磁波传到接
收器时,电磁波的电场使环的两个金属球间产生电压,这个电压足够高时,在两球间隙中就会发生火花放电.

赫兹发现,在发射器的间隙有火花跳过时,接收器的间隙确有火花跳过,这样,赫兹用实验证实了电磁波的存在,赫兹还用实验测定了电磁波的波长和频率,得到了电磁波的传播速度,证实了这个速度等于光速.赫兹还用实验证明,电磁波跟所有波动现象一样,能产生反射、折射、干涉、衍射等现象,从而充分证实了麦克斯韦的电磁场理论.

今天,赫兹实验以及类似的实验很容易做成,我们在教室里就可以进行演示,在赫兹实验中电磁波传播的距离不过几米,而现在发射的电磁波可以传到几千千米的远处,麦克斯韦的理论成了无线电技术的基础.

\subsection*{练习三}
\begin{enumerate}
	\item 从地球向月球发射电磁波,经过多长时间才能在地球上接收到反射回来的电磁波?地球到月球的距离为$3.84\x10^5$千米.
	\item 我国第一颗人造地球卫星采用20.009兆赫和19.995兆赫的频率发送无线电信号,这两种频率的电磁波的波长各是多少?
	\item 有一台收音机,它接收的波长范围由560.7米到186.9米,它接收的频率范围是多少?
	\item 一个振荡电路辐射出的电磁波的波长是300米,这个振荡电路的周期是多大?
\end{enumerate}

\section{电磁波的发送(一)~~开放电路}
从赫兹成功地证实电磁波的存在,到现在不过一百年,电磁波在科学技术上已经得到广泛的应用.广播电台和电视台每天都在发射强大的电磁波,它带着能量和信号传到各地,我们打开收音机和电视机,就听到声音,看到图像(图4.10).这是怎样实现的呢?从这一节开始我们将学习有关这方面的一些基本知识.
\begin{figure}[htp]\centering
\includegraphics[scale=.75]{fig/4-10.png}
\caption{电磁波的发送和接收}
\end{figure}

\begin{figure}[htp]\centering
	\includegraphics[scale=1.2]{fig/4-11.pdf}
	\caption{由闭合电路变成开放电路}
	\end{figure}

要利用电磁波,必须有效地向外发射电磁波.但是,由普通的电容器和线圈组成的振荡电路(图4.11甲),向外辐射能量的本领是很差的.这种振荡电路通常叫做闭合电路,电场和磁场几乎完全集中在电容器和线圈内,电场能和磁场能主要是在电路内互相转化,辐射出去的能量极少,实际上不能用来发射电磁波,为了有效地发射电磁波,必须尽可能使电场和磁场分散开.如果把电路改成图4.11乙那样,辐射能量的本领会好些;如果改成图4.11丙那样,辐射能量的本领就更好了,图4.11丙所示的电路叫做开放电路.

实际应用的开放电路,线圈下部用导线接地,这条导线叫做地线;线圈上部接到比较高的导线上,这条导线叫做天线.天线和地线形成了一个敞开的电容器.

研究表明,振荡电路辐射能量的本领跟振荡频率有关.电磁振荡的频率越高,向外辐射能量的本领就越大.因此,在无线电技术中使用的振荡电流频率都很高,在几百千赫以上.由
频率公式$f=\dfrac{1}{2\pi\sqrt{LC}}$知道,要提高振荡频率,必须降低振荡
电路的电感和电容.上面讲的具有天线和地线的开放电路,不但形式开敞,同时它的电感和电容都很小,有很高的固有频率,这对于发射电磁波是很有利的.
\begin{figure}[htp]\centering
\begin{circuitikz}[>=latex]
\draw (0,3)-- (2,3) to [american inductor] (2,0)-- (0,0)  ;
\draw (2.5, -.5) node [ground]{地线} to(2.5,0) to [american inductor] (2.5,3)to (2.5,3.5)node[bareantenna]{天线} ;
\draw (-.2, .5) rectangle (.2,2.5);
\draw (0,0)--(0,.5);
\draw (0,2.5)--(0,3);
\node [rotate=90]at (-.5,1.5){振荡器};
\node at (1.75,1.5){$L_2$};
\node at (2.75,1.5){$L_1$};
\end{circuitikz}
\caption{感应耦合}
\end{figure}

使开放电路里产生振荡电流,通常用如图4.12所示的感
应耦合的办法,使开放电路的线圈$L_1$跟振荡器电路中的线圈$L_2$接近.这样,振荡器电路里有振荡电流时,由于$L_2$和$L_1$的互感作用,在开放电路里就产生了同样频率的振荡电流,从而发射出电磁波.

无线电技术中使用的电磁波通常叫做\textbf{无线电波},无线电波的波长从几毫米到几十千米,例如中央人民广播电台发射一种频率为15.55兆赫的无线电波,波长为19.3米.通常根
据波长或频率把无线电波分成几个波段,如下表所示.

\begin{center}
\begin{tabular}{cp{.18\textwidth}p{.18\textwidth}cp{.25\textwidth}}
	\hline
	波段  &  波长  & 频率 & 传播方式 & 主要用途
\\
\hline
长波    &  30000—3000m  & 10—100KHz   &地波    & 超远程无线电通讯和导航\\
中波   & 3000—200m   & 100—1500KHz   & 地波和天波   & 无线电广播和电报通讯\\
中短波   & 200—50m   & 1500—6000KHz  & 地波和天波   & 无线电广播和电报通讯\\
短波 & 50—10m   & 6—30MHz & 天波   & 无线电广播和电报通讯\\
米波 & 10—1m   & 30—300MHz   & 近似直线传播   &  无线电广播、电视、导航\\
分米波 & 10—1dm   & 300—3000MHz   & 直线传播   &  电视、雷达、导航\\  
厘米波& 10—1cm   &  3000—30000MHz   & 直线传播   &  电视、雷达、导航\\ 
毫米波 & 10—1mm 
  & 30000—300000MHz   & 直线传播   &  电视、雷达、导航\\
\hline
\end{tabular}

\end{center}


\section{电磁波的发送(二)~~调制}
发射电磁波是为了用它来传递信号,无线电报传递的是
电码符号,无线电广播传递的是声音,无线电视传递的不仅有声音还有图像,怎样利用电磁波把电码、声音、图像等信号发射出去呢?

无线电技术中发射信号,先把要传递的信号转变为电信号,这种电信号频率比较低,不能直接用来发射电磁波,如果把这种电信号“加”到高频等幅振荡电流上,那么,载有信号的高频振荡电流产生的电磁波就载着要传递的信号一起发射出去,把要传递的电信号“加”到高频等幅振荡电流上叫做\textbf{调制}.进行调制的装置叫做\textbf{调制器}.要传递的电信号叫做调制信号.

常用的调制方法有\textbf{调幅}和\textbf{调频},调幅是使高频振荡电流的振幅随着调制信号而改变,调频是使高频振荡电流的频率随着调制信号而改变.

\begin{figure}[htp]
\centering
\begin{minipage}[t]{0.48\textwidth}
\centering
\begin{circuitikz}[>=latex]
\draw (0,3)to  (2,3) to [american inductor] (2,0) to [mic] (0,0)  ;
\node [waves,rotate=90] at(1,-1.75) {};
\draw (2.5, -.5) node [ground]{} to(2.5,0) to [american inductor] (2.5,3)to (2.5,3.5)node[bareantenna]{} ;
\draw (-.2, .5) rectangle (.2,2.5);
\draw (0,0)--(0,.5);
\draw (0,2.5)--(0,3);
\node [rotate=90]at (-.5,1.5){振荡器};
\node at (1.75,1.5){$L_2$};
\node at (2.75,1.5){$L_1$};

\end{circuitikz}
\caption{调制}
\end{minipage}
\begin{minipage}[t]{0.48\textwidth}
\centering
	\includegraphics[scale=1.2]{fig/4-14.pdf}
	
	甲:高频等幅振荡电流的波形;乙:声音信号的波形;丙:经过调幅的高频振荡电流的波形
	\caption{调幅的作用}
\end{minipage}
\end{figure}


图4.13是调幅装置的示意图,接在振荡器和线圈$L_2$之间的话筒就是一个最简单的调制器.由声源发出的声音振动使话筒里的碳粒发生时松时紧的变化,它的电阻也发生时大时小的变化,所以,虽然振荡器产生的是高频等幅振荡电流
(图4.14乙),但在线圈$L_2$中通过的却是振幅随声音而改变的高频调幅电流(图4.14丙).由于$L_2$和$L_1$的互感作用,通过开放电路中的也是这种高频调幅电流,于是从开放电路就发射出振幅随声音面改变的电磁波.这种电磁波叫做\textbf{调幅波}.

根据这节和前节所讲的,我们看到,在电磁波的发送中必须有振荡器、调制器、天线和地线.振荡器产生高频等幅振荡电流,经过调制器后,成为被调制的高频振荡电流,于是开放电路就发射出带有调制信号的电磁波.为了使电磁波能传得
很远,还需要用放大器把被调制的高频振荡电流加以放大,以增强开放电路发射电磁波的功率,整个发送装置的方框图如图4.15所示.
\begin{figure}[htp]\centering
\includegraphics[scale=.75]{fig/4-15.png}
\caption{电磁波发送的方框图}
\end{figure}

\section{电磁波的接收(一)~~电谐振}
电磁波在空间传播时,如果遇到导体,就把自己的一部分能量传给导体,使导体中产生感生电流.感生电流的频率跟激起它的电磁波的频率相同.因此,利用放在电磁波传播空间中的导体,就可以接收到电磁波.无线电技术中,用天线和地线组成的接收电路来接收电磁波.

世界上有许许多多的无线电台,它们发出的电磁波的频
率各不相同,我们首先要从这些电台发出的电磁波中把我们要接收的选出来,通常这叫做选台.不进行选台,例如在收音机中,如果同时把各个广播电台发出的电磁波都接收下来,
并且把它们都转变成声音,那只能是一片嘈杂声,什么也听不清楚.怎样才能从许多电台中选出我们需要的那个电台呢?这就要设法使它发出的电磁波在接收电路里激起的感生电流最强,而其他电台发出的电磁波激起的感生电流都非常弱.

在无线电技术里是利用电谐振现象来实现这一点的,电谐振相当于机械振动中的共振,下面说明这种现象.

用莱顿瓶\footnote{莱顿瓶是由玻璃圆筒和贴在玻璃筒内外面上的锡箔构成的.它是一个电容器,两层锡箔是电容器的两极,它们之间的玻璃是电介质.内层的锡箔跟伸到瓶口外的带有金属球的金属棒相连接.}和带有间隙$AB$的矩形线圈组成第一个电路
(图4.16甲),用同样的莱顿瓶和带有氖管的一边可移动的矩形线圈组成第二个电路(图4.16乙).莱顿瓶具有电容,矩形线圈具有电感,所以这两个电路都是振荡电路.
\begin{figure}[htp]\centering
	\includegraphics[scale=.6]{fig/4-16.png}
	\caption{电谐振}
	\end{figure}

给第一个电路中的莱顿瓶带电,当$AB$间的电压达到一定程度时,$AB$间就发生火花放电,这时移动第二个电路中矩形线圈的可移动的一边,可以看到:两个矩形线圈的大小相差不
多时,氖管开始发光;两个线圈的大小完全相同时,氖管最亮.

第一个电路实际上是电磁波发射器,跟赫兹实验中的发射器一样,发生火花放电时向外发射电磁波,电磁波的频率等于第一个电路的固有频率.第二个电路实际上是电磁波接收器,当它接收到第一个电路发射的电磁波时,就在电路中激起振荡电流,上面的实验说明激起的振荡电流的强弱跟接收电路的固有频率有关系.当两个线圈的大小完全相同时,它们的自感是相同的;而两个莱顿瓶是一样的,它们的电容也相同.因此两个电路的固有频率相同,这就是说,当接收电路的固有频率跟接收到的电磁波的频率相同时,激起的振荡电流最强,这就是电谐振现象.

使接收电路产生电谐振的过程叫做\textbf{调谐},能够调谐的接收电路叫做调谐电路.
\begin{figure}[htp]\centering
\begin{circuitikz}[>=latex]
    \draw (0,0) to [vC] (0,3) --(2,3)to  [american inductor](2,0)--(0,0);
    \draw (1, 3) to(1,4) node [bareantenna]{} ;
    \draw (1, 0) to(1,-.5) node [ground]{} ;
    \draw (1, 0) [fill=black] circle (1.5pt);
    \draw (1, 3) [fill=black] circle (1.5pt);
\end{circuitikz}
	\caption{调谐电路}
	\end{figure}

图4.17是收音机的调谐电路.调节可变电容器的电容来改变调谐电路的频率,使它跟我们要接收的电台发出的电磁波的频率相同,由于电谐振现象,只有这个频率的电磁波才在调谐电路里激起较强的感生电流,这样,我们就可以选出需要的电台.

\section{电磁波的接收(二)~~检波}
由无线电台发射出的是经过调制的载有信号的电磁波,
在调谐电路中因电谐振而产生的也是经过调制的高频振荡电流.这种电流通过收音机的耳机或扬声器,并不能使它们振动而发声.为什么呢?假定某一个半周期电流的作用是使振动片向某个方向运动,下一个半周期电流就以几乎同样大的作用使振动片向反方向运动,高频电流的周期非常短,半周期更短,而振动片的惯性相当大,所以在振动片还没有来得及在电流的作用下向某个方向运动的时候,就立刻有一个几乎同样大的作用要使它向反方向运动,结果振动片实际上不发生振动,因此,在电磁波的接收中还必须设法从经过调制的高频振荡电流中取出发射时加上去的调制信号.

从经过调制的高频振荡电流中取出调制信号的过程,叫做\textbf{检波}.检波是调制的逆过程,也叫做解调.由于调制的方法不同,检波的方法也不同,下面介绍收音机中对调幅波的检波.
\begin{figure}[htp]\centering
\begin{circuitikz}[>=latex]
\draw (0,0) to [vC] (0,3)--(1.8,3) to [american inductor] (1.8,0)--(0,0);
\draw (.9, 0) to(.9,-.5) node [ground]{} ;
\node at (-.75,1.5){$C_1$};\node at (1.6,1.5){$L_1$}; \node at (2.8,1.5){$L_2$};
\ctikzset{diodes/scale=0.6}  \draw (2.6,0) to  [american inductor] (2.6,3) to [full diode=$D$](5,3)--(6,3)to [mic=耳机](6,0) -- (0,0) ;
\draw (5,0)to [C=$C_2$, *-*] (5,3);
\draw (.9, 3) to(.9,3.9) node [bareantenna]{} ;

\draw[dashed](2.2, 0)--(2.2,3.5);\draw[dashed](3.2, -.5)--(3.2,3.5);
\draw (2.6,0) [fill=black] circle (1.5pt);
\draw (1.8,0) [fill=black] circle (1.5pt);
\draw (.9, 0) [fill=black] circle (1.5pt);
\draw (.9, 3) [fill=black] circle (1.5pt);
\end{circuitikz}
\caption{}
\end{figure}

图4.18虚线的右边是晶体二极管的检波电路,是利用晶体二极管的单向导电性来进行检波的.$L_1C_1$调谐电路中产生的是经过调幅的高频振荡电流(图4.19甲).$L_1$和$L_2$绕在同一磁棒上,由于互感作用,在$L_2$上产生的是高频交变电压.由于晶体二极管$D$有单向导电性,通过它的是单向脉动电流
(图4.19乙).这个单向脉动电流既含有高频成分,又含有音频成分.由于电容器有通高频阻低频的作用,高频成分基本上从电容器$C_2$通过,剩下的音频电流(图4.19丙)通过耳机,
使耳机的振动片随着信号而振动发声.
\begin{figure}[htp]\centering
\includegraphics[scale=.75]{fig/4-19.png}
\caption{检波的作用}
甲:高频调幅电流的波形,乙:检波后的脉动电流的波形,丙:通过耳机的音频电流的波形.
\end{figure}

图4.18实际上就是一个晶体二极管收音机的电路图.这种收音机声音很小,只能用耳机收听本地电台.为了提高收音机的接收性能,需要用放大器把微弱的信号放大,图4.20是加有放大器的收音机方框图.由天线和调谐电路接收到的高频调幅电流,先通过放大器进行高频放大,然后进行检波和
低频放大,放大后的音频电流通过耳机或扬声器,使它们发出声音,在学生实验中,我们将熟悉这种收音机的电路图,并且要根电路图进行收音机的安装和调试.
\begin{figure}[htp]\centering
\includegraphics[scale=.75]{fig/4-20.png}
\caption{收音机的方框图}
\end{figure}


\subsection*{练习四}
\begin{enumerate}
	\item 有一台收音机,把它的调谐电路中的可变电容器的动片从完全旋入到完全旋出,仍然接收不到某一较高频率电
	台的信号,要想接收到该电台的信号,应该增加谐振线圈的
	数,还是减小谐振线圈的匝数?为什么?
	\item 收音机由收听某一较高频率的电台改为收听某一较低频率的电台,调谐电路中可变电容器的动片应该旋进一些还是旋出一些?为什么?
	\item 某收音机调谐电路的可变电容器的动片完全旋入时,电容是390皮法,这时能接收到520千赫的无线电波.动片完全旋出,电容变为39皮法,这时能接收到的无线电波的频率是多大?在这两种情形下,接收到的电磁波的波长分别是多长?
\end{enumerate}

\section{传真~~电视~~雷达}
这一节我们介绍无线电波的现代应用——传真、电视和雷达的原理.

\subsection{传真}

无线电传真是利用无线电波传送图表、书信、照片等图像的一种方法,报纸印的许多照片都是传真照片.

跟无线电广播不同,在无线电传真中要把图像反射出的光转换为电信号.光电管的制成解决了这个问题.利用光电管可以使电流随着光的强弱而改变,光电管是传真装置中的主要元件,光电管的原理将在本书第七章介绍.
\begin{figure}[htp]\centering
	\includegraphics[scale=.6]{fig/4-21.png}
	\caption{无线电传真示意图}
	\end{figure}

在发射端,把图片贴在图4.21左方的转动筒上,转动筒在光电管近旁,一边转动,一边沿轴向移动.这样,图片各部分反射出的强弱不同的光,按照一定顺序照射到光电管上,在光电管电路中就出现了强弱变化的信号电流.用光电管电路作调制器,把信号电流调制到高频等幅振荡电流上,经过发射机
就发射出带有传真信号的无线电波.

在接收端,接收机收到带有传真信号的无线电波后,经过放大、检波,把传真信号取出并送给辉光管,辉光管是一种充气管,它发出的光的强弱随着通过的电流的强弱而变化.这样,辉光管所发的光的强弱变化,跟发射端图片各部分的反射光的强弱变化相同.辉光管发的光会聚到卷在转动筒上的感
光纸上,接收端用的转动筒跟发射端的相同,移动和转动的情况也完全一致,感光纸依次曝光,再经过显影、定影,就得到跟原来一样的图片.

\subsection{电视}

传真传递的是静止的图像,电视传递的是活动的景像,电视的活动景像是怎样形成的呢?
\begin{figure}[htp]\centering
	\includegraphics[scale=.6]{fig/4-22.png}
	\caption{摄像管}
	\end{figure}

在电视的发射端,摄取景物并将景物反射的光转换为电信号,是由摄像管(图4.22)来完成的,摄像时,摄像镜头将
被摄物体的像成在摄像管中的屏上.电子枪发出的电子束按一定规律偏转,对屏上的图像进行扫描,扫描的路线如图4.23所示,从$a$开始,逐行进行,直到$b$.这样,把一幅图像各个部分的明暗情况逐点变成强弱不同的信号电流,经过发射机就发射出带有图像信号的无线电波.
\begin{figure}[htp]
	\centering
	\begin{minipage}[t]{0.48\textwidth}
	\centering
	\includegraphics[scale=1.2]{fig/4-23.pdf}
	\caption{扫描}
	\end{minipage}
	\begin{minipage}[t]{0.48\textwidth}
	\centering
	\includegraphics[scale=1.2]{fig/4-24.pdf}
	\caption{显像管}
	\end{minipage}
	\end{figure}
	
在电视接收端,把电信号转换成景物的像,是由显像管(图4.24)来完成的.电视机接收到带有图像信号的无线电波后,经过放
大、检波,把图像信号送到显像管,显像管里也有一个电子枪,它发射电子束的强弱受图像信号的控制,并且按照跟摄像管的电子枪同样的步调和方式扫描,这样,当电子束射到显像管底部的荧光屏上时,在屏上就出现跟摄像管屏上相同的图像.

摄像机在1秒钟内要传送25张画面,电视机也以相同的速率在荧光屏上显现这些画面.由于画面更换迅速和视觉暂留,我们便看到了活动的影像.

电视机接收到的无线电波,除了带有图像信号,还带有伴音信号,伴音信号经检波取出后,送入扬声器,扬声器就伴随着活动景像发出声音来.

从二十世纪二十年代开始试验电视广播以来,电视技术有了很大的发展,已由黑白电视发展到彩色电视.电视技术目前仍在迅速发展中.

电视的应用日益扩大,除了电视广播,有线电视(又叫闭路电视)也得到了广泛应用.例如,在自动化企业的控制中心,可以利用电视来监视各条生产线的工作情况.一些不便直接观察的地方,如有毒气或强烈放射性的地方,可以用电视作间接的观察.现在,电视技术已经应用到工业、交通、文化教育、国防和科学研究等各个方面.

\subsection{雷达}

雷达是利用无线电波来测定物体位置的无线电设备.

电磁波遇到障碍物要发生反射,雷达就是利用电磁波的反射来工作的.波长越短的电磁波,传播的直线性越好,反射性能越强,因此雷达用的是微波波段的无线电波.

雷达有一个特制的可以转动的天线(图4.25),它能向一定的方向发射不连续的无线电波,每次发射的时间约为百万分之一秒,两次发射的时间间隔大约是发射时间的一百倍.发射出去的无线电波遇到障碍物时,可以在这个时间间隔内反射回来为天线接收.
\begin{figure}[htp]\centering
	\includegraphics[scale=1.2]{fig/4-25.jpg}
	\caption{雷达天线}
	\end{figure}

已知无线电波的传播速度$c$,测出从发射无线电波到接收到反射回来的无线电波的时间$t$,就可以根据公式$2s=ct$来确定障碍物的距离,再根据发射无线电波的方向和仰角,便可以确定障碍物的位置.

实际上,障碍物的距离等情况是由雷达的指示器直接显示出来的,当雷达向目标发射无线电波时,在指示器的荧光屏上呈现出一个尖形波;在收到反射回来的无线电波时,在荧光屏上呈现出第二个尖形波(图4.26).根据两个波的距离,可直接从荧光屏上的刻度读出障碍物的距离.
\begin{figure}[htp]\centering
	\includegraphics[scale=.6]{fig/4-26.png}
	\caption{}
	\end{figure}

利用雷达可以探测飞机、舰艇、导弹以及其他军事目标.除了军事用途外,雷达在交通运输上可以用来为飞机、船只导航,在天文学上可以用来研究星体,在气象上可以用来探测
台风、雷雨、乌云.

\section{电磁波的传播}
不同波长的电磁波有着不同的传播特性,这里只介绍无线电波的传播.无线电波的传播方式大致有三种:地波、天波和直线传播(图4.27).
\begin{figure}[htp]\centering
	\includegraphics[scale=.6]{fig/4-27.png}
	\caption{}
	\end{figure}

\subsection{地波}

沿地球表面空间传播的无线电波叫做地波.

地面上有高低不平的山坡和房屋等障碍物,无线电波要绕过这些障碍物;才能被各处的接收机收到.波的重要特性之一是衍射,当波长大于或相当于障碍物的尺寸时,波可以绕到障碍物的后面,地面上的障碍物一般不太大,长波能很好地绕过它们.中波和中短波也能较好地绕过它们,短波和微波的波长过短,绕过障碍物的本领很差.

地球是导体,地波沿地面传播时地球表面会因电磁感应
而激起感生电流,这就要损失能量,这种能量损失随频率的增高而增大.因此,从能量损失的角度来看,这种传播方式对长波、中波和中短波比较适宜,对短波和微波则不适宜.

地波在传播过程中要不断损失能量,因此中波和中短波的传播距离不太大,一般在儿百千米范围内.收音机在这一波段一般只能收听到本地或附近省市的电台,就是这个缘故.虽然长波的传播距离可以远得多,但发射长波的设备庞大,造
价高,因此无线电广播一般不用长波,由于地波传播稳定可靠,近年来在超远程无线电通讯和导航技术等方面,发射长波的技术已有很大发展.

\subsection{天波}

依靠电离层的反射来传播的无线电波叫做天波.什么是电离层呢?在地球表面的大气层中,在大约50千米到几百千米的范围内,一部分中性气体分子由于受到太阳光的照射而发生电离,分解为带正电的离子和自由电子,这层大气就叫做电离层.

电离层对于不同波长的电磁波表现出不同的特性,对于波长短于10米的微波,电离层能让它穿过,飞向宇宙.对于波长超过3000米的长波,电离层基本上把它吸收掉.对于中波、中短波和短波,波长越短,电离层对它吸收得越少而反射得越多.因此,短波最适宜以天波的形式传播,可以传到几千千米以外的远处.

电离层是不稳定的,白天电离程度高,夜晚电离程度低.夜晚电离程度低,电离层对中波和中短波的吸收减弱,这时中波和中短波也可以用天波的形式传播.收音机在夜晚能够收听到许多远地的中波或中短波电台,就是这个缘故.由于电离层不稳定,电离程度和高度经常变化,无线电波到达接收机时强弱也随着时刻变化,因此在用一般收音机收听短波广播时,声音常是忽大忽小,高级收音机里,增设特殊的线路自动控制音量,来解决这个问题.

\subsection{直线传播}

微波又叫超短波,它既不能以地波的形式传播,又不能依靠电离层的反射以天波的形式传播.微波的传播形式跟光相似,是沿直线传播的.这种沿直线传播的无线电波叫做空间波或视波.

地球表面是圆球形的,沿直线传播的微波能够传播的距离不大,一般为几十千米.这种传播方式受大气的干扰小,能量耗损小,接收到的信号较强而且比较稳定,电视、雷达采用的都是微波.

远距离传送微波,需要设立中继站,由某地发射出的微波被中继站接收,放大后再传向下一站,这样一站传一站,把微波传向远方,同步通讯卫星可以用来作微波中继站.同步通讯卫星相对于地面静止在赤道上空36000千米的高处,只要有三颗这样的卫星,就可以把微波信号传遍全世界.

\section{电子技术一瞥}
前面讲过的无线电广播、电视、传真等通常叫做无线电技术,在无线电技术中,要利用各种电子器件如电子管、晶体管、摄像管、显像管、光电管等,也要利用无线电波在空间的传
播.电子技术是在无线电技术的基础上发展起来的,现在它已经大大超出了无线电技术的范围,成为一门领域十分广泛
的现代科学技术.可以说,凡是要用到各种电子器件的技术,不论是否利用电磁波在空间的传播,都属于电子技术的范围.示波器,收音机和电视机是电子技术的成果,电子显微镜和电子计算机也是电子技术的成果.

电子技术广泛地应用于工业、农业、国防等部门中,是现代化生产和科学研究的重要手段.电子技术在通讯方面的应用很广,除了广播和电视而外,还有无线电报、无线电话、无线电导航等,空间科学离不开电子技术.没有精确的电子测量和电子控制,卫星就不能上天.在工业上利用电子技术进行自动控制、自动调节、自动监视和保护等.各种程序控制机床是利用电子技术来自动控制的.自动调节可以把温度、压力、速度或液面高度准确地调节在一定数值.利用电子技术可以
自动监视机器的运转,有危险时能发出警报或采取保护措施.利用电子技术还能够实行遥控(远距离操纵),例如在一个地点可以同时控制远处的几座水电站的运行,利用电子技术可以作各种精密测量,例如用电子测微计能测出$10^{-8}$厘米的距离,利用电子技术还可以研究天文学、气象学、医学等.

电子计算机是二十世纪的重大发明,是电子技术的一项卓越成就.从电子计算机出现以来,已经经历了“四代”的变化.随着电子器件的微型化,即由电子管、晶体管到集成电路(在一小片半导体上制造出晶体管、电容、电阻等元件,并连接起来形成电路)、大规模集成电路(一般有几千个到几万个晶体管),并向着超大规模集成电路发展,七十年代以后出现了第四代电子计算机,不但有大中型电子计算机,应用于科学技术领域,而且出现了微型电子计算机(微电脑),微电脑的体积
小,耗能少,便宜可靠,便于推广使用,它在气象、水电、医疗、商业、军工、教育、财经、铁路、交通、邮电等领域取得广泛的应
用.微电脑的广泛应用和推广,将使整个生产结构和社会生活发生巨大变化.

电子技术的发展是世界范围的新技术革命的重要组成部分,以电子计算机为代表的电子技术水准已成为衡量一个国家现代化水平的重要标志.

我国的电子技术现在已经有了相当的发展.
在通讯方面,我国的广播和电视事业发展很快.广播已基本普及.到1983年底,全国拥有广播电台122座,发射的总功率是解放初期的250倍.全国拥有电视台52座,省以下的电视专业微波线路14156千米,电视覆盖地区的人口约占总人口的60\%.我国已经成功地发射了通信广播卫星,成为世界上少数几个能独立发射这种卫星的国家之一,这标志着我国卫星通信技术已接近世界先进水平.

我国于1958年就研制成功了第一台电子管元件计算机.此后陆续又研制出一批大、中、小型集成电路计算机,近年来,我国在集成电路的研制和生产、计算机的推广和应用等方面都取得了很大成绩.1983年我国成功地研制成每秒运算一亿次的“银河”巨型机.1984年我国在研制超大规模集成电路上取得了新的突破,我国的微电脑已批量定型生产.

我国的电子技术,目前虽然同国际先进水平还有差距,但
不久的将来一定会进入国际先进水平的行列,同学们在发展我国电子技术方面将会作出贡献.

\section*{复习题}
\begin{enumerate}
	\item 简述$LC$回路产生电磁报荡的过程.
	\item $LC$回路的周期和频率跟哪些因素有关?写出周期和频率的公式.
	\item 麦克斯韦电磁场理论的两个要点是什么?
	\item 电磁波是怎样产生的?为什么说电磁场是物质的一种特殊形态?
	\item 简述证实电磁波存在的赫兹实验.
	\item 发射电磁波为什么要用开放电路?又为什么要用高频振荡电流?
	\item 什么叫调制?什么叫调幅?设用音频电信号来调幅,画出调幅前后高频振荡电流的波形.
	\item 简述电谐振现象,什么叫调谐?在收音机电路里是怎样实现调谐的?
	\item 什么叫检波?晶体二极管的检波电路是怎样的?画出检波前后高频振荡电流的波形.
	\item 无线电波主要有几种传播方式?不同波段的无线电波各以什么方式传播?
\end{enumerate}

\section*{习题}
\begin{enumerate}
	\item 要使$LC$回路的固有频率增大,应该采用下述哪种方法?
	\begin{enumerate}
		\item 增大电容器两极板间的距离;
		\item 增大电容器两极板的正对面积;
		\item 在线圈中插入铁心;
		\item 减小线圈的匝数.
	\end{enumerate}
	\item 赫兹实验中的发射器为什么用图4.9那种形式,面不用该实验中接收器那种形式?赫兹实验中需要把接收器的两金属球调节到合适的距离,才有明显的接收效果,为什么?
	\item 回旋加速器中的磁感应强度为$B$,被加速的粒子的电量为$q$,质量为$m$.用$LC$振荡器作为高频电源,电感$L$和电容$C$的数值应该满足什么条件?
	\item 附近有电焊机进行电焊,或者有汽车通过,收音机和电视机会受到干扰,解释这个现象.
\end{enumerate}















\chapter{光的反射和折射}

光学是物理学中发展得最早的分支之一,远在二千四百
多年前我国的墨翟(公元前468—376)及其弟子们所著的《墨
经》一书,就记载了光的直线传播、光的反射等现象,称得上是
世界上最早的光学著作.宋代沈括(1031—1095)在《梦溪笔
谈》中记载了极为丰富的光学知识.后来,各国科学家们发明
了眼镜、望远镜、显微镜、照相机等光学仪器,并对光的本性进
行了研究,把光学应用于生产技术和生活中.本世纪六十年代
激光出现以后,光学的研究和应用有了新的发展,光学与其他
科学技术紧密结合,在各个领域中得到了广泛的应用,成为现
代物理学和现代科学技术的重要前沿之一.

在中学我们主要学习光的传播和光的本性的知识.

\section{光的直线传播}
\subsection{光源}

在漆黑的屋子里,我们什么也看不见.如果点上
一盏灯,就可以看见灯,也可以看见屋子里的桌、椅、墙壁等.
我们能够看见灯的发光部分,是因为它发出的光进入了我们
的眼睛,引起了视觉,我们能够看见桌、椅、墙壁等,是因为它
们反射光.

电灯、蜡烛、太阳、萤火虫等,能够自行发光,我们把这样
的物体叫做\textbf{光源}.

光具有能量,它可以使物体变热,使照相底片感光,使光
电池供电,光源发光要消耗其他形式的能,由其他形式的能
转化成光能.电灯发光消耗了电能,蜡烛发光消耗了化学能,
太阳发光消耗了太阳内部的原子能.

\subsection{光的直线传播}

我们知道,光是沿直线传播的.把一块
有窄缝的硬纸板挡在光源前面,可以看到,从窄缝射出的光束
沿直线前进(图5.1).
\begin{figure}[htp]\centering
    \includegraphics[scale=.6]{fig/5-1.png}
    \caption{光沿直线传播}
    \end{figure}

在研究光的传播规律时,为了方便,常用一条直线来表示
光的传播方向,这样的直线,叫做\textbf{光线}.光线是个很有用的概
念,利用光线我们就可以用几何学的方法来研究光的传播问
题,应该注意的是,光线并不是实际存在的东西,实际中只
能得到很窄的光束,不能得到象几何线那样的光线,正如质
点是物体的抽象一样,光线是光束的抽象.

自然界中的许多光现象,例如影、日食、月食等,都是光沿
直线传播而产生的.日常生活中,我们也经常运用光沿直线
传播的性质,例如,知道了一个发光点发出的两条光线的方
向,根据光沿直线传播的性质,就可以确定这个发光点的位
置;照图5.2那样,把两条光线向反方向延长,它们的交点的
位置就是发光点的位置.人的眼睛在观察物体时,就是利用
这种方法来确定物体的位置的.根据从铅笔尖射到眼睛的两
条光线,人们判断铅笔尖的位置在这两条光线反方向延长线
的交点(图5.3).
\begin{figure}[htp]
    \centering
    \begin{minipage}[t]{0.48\textwidth}
    \centering
\begin{tikzpicture}[>=latex]
\node at (0,0){\Sun};
\node at (-.5,0){$S$};
\draw[dashed] (0,0)--(15:2)node[above]{$A$};
\draw[dashed] (0,0)--(-15:2)node[below]{$C$};
\draw[->](15:2)--(15:3.5)node[above]{$B$};
\draw[->](-15:2)--(-15:3.5)node[below]{$D$};


\end{tikzpicture}
    \caption{利用光线
    确定物体的位置}
    \end{minipage}
    \begin{minipage}[t]{0.48\textwidth}
    \centering
    \includegraphics[scale=.6]{fig/5-3.png}
    \caption{眼睛根据光的直线传播确定物体的位置}
    \end{minipage}
    \end{figure}

\section{光的速度}
光从光源发出,以有限的速度向外传播.光传播得很快,
在日常接触到的距离内,光从光源到达我们的眼睛所用的时
间很短,凭感觉根本无法察觉出来,所以在历史上很长一段时
间里,人们一直认为光的传播是不需要时间的,直到十七世
纪才发现光是以有限的速度传播的.

1607年伽利略最早做了测定光速的尝试.让两个实验
者在夜间每人各带一盏遮蔽着的灯,站在相距约1.6千米的
两个山顶上.第一个实验者先打开灯,同时记下开灯的时间.
第二个实验者看到传来的灯光后,立刻打开自己的灯.第一
个实验者看到第二个实验者的灯光后,再立刻记下时间,然
后根据记下的时间间隔和两山顶间的距离计算光的传播速
度.这种测量光速的方法原理虽然正确,却没有得出什么实
验结果.这是因为光的速度很大,在相距约1.6千米的两山
顶间来回一次,所用的时间大约只有十万分之一秒,用这种简
单粗糙的方法根本不能测出这样短的时间来.伽利略测定光
速的实验虽然失败了,但是它表明了光的速度是很大的,在不
太长的距离上不能用粗糙的计时方法测量出来.
要测定光速,必须利用很大的距离,或者用精巧的办法准
确地测量出很短的时间间隔.伽利略以后的学者们正是沿着
这两个方向探求测定光速的方法的.

\begin{figure}[htp]\centering
    \includegraphics[scale=.6]{fig/5-4.png}
    \caption{罗默用天文观测的方法测定光速的示意图.
    设卫星第一次食毕,地球在$C$点,卫星绕木
    星公转一周,第二次食毕,地球在$D$点,因此
    光要多传播一段距离$\ell$才能到达地球.}
    \end{figure}

1676年丹麦天文学家罗默(1644—1710)用天文观测的
方法,发现了光是以有限的速度传播的.罗默是在观测木星
的卫星食的过程中得到这一发现的.如图5.4所示,木星的
卫星绕木星每转一周都要消失在木星的影内一次,即发生一
次卫星食,相继的两次卫星食间隔的时间,就是这个卫星绕
木星公转一周的周期,这个周期不大,约为1.75天.罗默在
整年中连续观测了卫星的公转周期,发现这个周期不是恒定
的.当地球在绕日运行的轨道上远离木星而去时,即地球在
图中所示由$A$运行到$B$的半年里,卫星的周期长;当地球在轨
道上向着木星运行时,即地球在由$B$运行到$A$的半年里,周
期短.罗默认为,这是由于光以有限的速度传播而造成的.
当地球远离木屋运动时,卫星第二次食毕,从木星的影中出
来,卫星发出的光到达地球要多传播一段距离,所以观测到的
周期长;当地球向着木星运动时,光到达地球要少传播一段距
离,所以观测到的周期短.周期长或短的数值跟光速的大小
有关系.利用罗默观测到的数据和地球绕日运行的轨道直径,
可以计算出光速的大小.

为了在地面上不太长的距离内测定光速,科学家们设计
了各种巧妙的实验方法,以便准确地测出很短的时间间隔,下
面我们简略地介绍一下美国物理学家迈克耳孙(1852—1931)
的旋转棱镜法.
\begin{figure}[htp]\centering
    \includegraphics[scale=.6]{fig/5-5.png}
    \caption{迈克耳孙测定光速实验的示意图}
    \end{figure}

迈克耳孙选择了两个山峰,测出两山峰间的距离,在第一
个山峰上安装一个强光源$S$和一个正八面棱镜$A$(图5.5),光
源$S$发出的光,经过狭缝射到八面镜$A$的面1上,反射后射到
放置在另一个山峰上的凹镜$B$上,经平面镜$M$反射后,再由凹
镜$B$反射回第一个山峰.如果八面镜静止不动,反射回来的
光就射到八面镜的另一个面3上,经面3反射后,通过望远镜
$C$进入观察者的眼中,看到光源$S$的像.

如果使八面镜转动,那么光经凹镜$B$反射回来时,八面镜
的面3已经偏离了原来的方向,经面3反射后的光将不再进
入望远镜中,观察者就观察不到光源$S$的像了.适当调节八
面镜的转速,使反射回来的光到达八面镜时,八面镜恰好转过
$1/8$转,这时面2正好转到原来面3所在的位置,经面2反射后
的光就可以进入望远镜中,观察者就可以重新看到S的像.根
据八面镜转过$1/8$
转所用的时间和两山峰间的距离,就可以算
出光在空气里的速度.迈克耳孙经过校正,得出光在真空中
的速度$c=299796{\rm km}/{\rm s}$.

1970年以后,开始利用激光测量光速,这种测量方法的
原理是测出激光的波长$\lambda$和频率$\nu$,利用$c=\lambda\nu$ 计算得出光
速$c$来(关于光是一种波动的知识将在下一章中学习),激光
测速法大大提高了测量的精确度,根据1975年第15届国际
计量大会决议,真空中的光速值定为
$$c=299792458\ms$$
在通常的计算中,可取$c=3.00\x10^8\ms$.

\subsection*{练习一}
\begin{enumerate}
    \item 在图5.6中,$S$是一个点光源,$AB$是物体,$E$是屏
幕,试画出$AB$在屏幕上所成
的影的范围.
\begin{figure}[htp]
	\centering
\begin{tikzpicture}
\node at (-1.1,0) {$S$};
\node at (-.7,0){\Sun};
\draw (1, -.5)node[below]{$B$}--(1,.5)node [above]{$A$};
\draw (3, -2)--(3,2)node [above]{$E$};
\end{tikzpicture}
	\caption{}
\end{figure}


\item 为什么在发生日食的
时候,有的地方能看到日全食,
有的地方只能看到日偏食?在
什么情况下能看到日环食?画
出地球、月球和太阳位置的简图来解释发生这几种日食的原
因.
\item 现在需要知道学校里升国旗的旗杆的高度,你能否
利用光的直线传播知识想出一种测量的办法来?
\item 边长5厘米的正方形卡片,放在小灯泡正前方15厘
米的地方,在卡片后方放一个跟卡片平行的纸屏,纸屏距卡片
15厘米,卡片在屏上的影是什么形状?面积多大?
\item 在迈克耳孙测定光速的一次实验中,测得两山峰间
的距离为35373.21米,八面镜的旋转速度为528转/秒,试利
用这些数据算出光在空气中的速度.
\end{enumerate}

\section{光的反射~~平面镜}
\subsection{光的反射}
前面讲的光沿直线传播,是就光在同一种媒
质里的传播而言的,当光从一种媒质射入另一种媒质,例如
从空气射入玻璃或水里时,在两种媒质的分界面上,光将改变
传播方向,一部分光被反射回原来的媒质中,这种现象叫做
\textbf{光的反射}.

我们在初中学过,光的反射遵守下面的\textbf{反射定律}:

\begin{enumerate}
    \item 反射光线跟入射光线和法线在同一平面上,反射光
线和入射光线分居在法线的两侧;
\item 反射角等于入射角.
\end{enumerate}

根据光的反射定律,如果使光线沿着反射光线的路径射
到界面上,这时的反射光线一定会沿着原来入射光线的路径
射出,就是说,反射时光路是可逆的.

平滑的表面,例如镜面、平静的水面等,能使平行的入射
光线沿同一方向平行地反射出去(图5.8),这时只在这个方
向有反射光线,其他方向没有反射光线,这样的反射叫做镜面
反射.
\begin{figure}[htp]
    \centering
    \begin{minipage}[t]{0.48\textwidth}
    \centering
    \begin{tikzpicture}[>=Stealth, scale=1]
        \draw (-2,0)--(2,0);
    \draw(0,0)--(0,3)node [above]{法线};
    \draw[very thick] (135:3)node [above]{入射光线}--(0,0);  \draw[very thick] (45:3)node [above]{反射光线}--(0,0);	
    \draw[->] (135:3)--(135:1.5);  \draw[->] (0,0)--(45:1.5);
    \draw (0,.65) arc (90:45:.65) node [above]{$r$};	
    \draw (0,.5) arc (90:135:.5)node [above]{$i$};	
        \end{tikzpicture}
    \caption{光的反射}
    \end{minipage}
    \begin{minipage}[t]{0.48\textwidth}
    \centering
        \begin{tikzpicture}[>=stealth, xscale=.6]
    
    \foreach \x in {1,2,3,4,5}
    {
        \draw[->, very thick] (\x,0)--(\x+3,3);
        \draw[very thick] (\x,0)--(\x-3,3);
        \draw[<-, very thick] (\x-2,2)--(\x-3,3);
    }
    
    \fill [pattern=north east lines] (-1,-.25) rectangle (7,0);
        \draw (-1,0)--(7,0);
        \end{tikzpicture}
    \caption{镜面反射}
    \end{minipage}
    \end{figure}

如果表面是粗糙不平的,那么沿同一方向射到面上的光
线将向不同的方向反射(图5.9),这样的反射叫做漫反射.我
们能从不同方向看见本身不发光的物体,比如看见书上的字,
就是因为光在纸面上发生漫反射的缘故.
\begin{figure}[htp]
    \centering
    \begin{minipage}[t]{0.48\textwidth}
    \centering
    \includegraphics[scale=1]{fig/5-9.pdf}
    \caption{漫反射}
    \end{minipage}
    \begin{minipage}[t]{0.48\textwidth}
    \centering
    \includegraphics[scale=.8]{fig/5-10.PDF}
    \caption{平面镜成虚像}
    \end{minipage}
    \end{figure}

    \subsection{平面镜}

    日常生活中用的镜子都是平面镜,从镜子里可
    以看到镜前物体的像.平面镜里的像是怎样产生的呢?

    在图5.10中,$M$是平面镜,一支蜡烛位于镜前.为了研
    究蜡烛所成的像,我们先来研究烛焰上的一点$S$在镜中的像.
    从$S$射向镜面的光线中,任取两条光线$SA$和$SB$,这两条光
    线被镜面反射后,分别沿着$AC$和$BE$的方向进入眼睛.发
    光点虽然在$S$,但是我们根据光沿直线传播的经验,认为光线
    $AC$和$BE$是从它们的反向延长线的交点$S'$射来的,从$S$点
    发出的其他光线经镜面反射后,反向延长线也通过$S'$点(请
    同学们自己证明一下),$S'$就是$S$在平面镜中的像,因为光
    线实际上不是从$S'$发出的,$S'$并不是光线的实际交点,而是
    光线的反向延长线的交点,所以光学中把这样的像叫做\textbf{虚像}.

    从$S$点发出的垂直于镜面的光线$SO$,经镜面反射后,沿
    原路返回.这条反射光线的反向延长线也通过$S'$点.根据
    反射定律,可以证明直角三角形$SAO$和$S'AO$全等,所以$S'O=SO$.
    就是说,像到镜面的距离等于物到镜面的距离.像和
    物对镜面是对称的.

    整个蜡烛可以看成是由许多点组成的,每个点都在镜子
    里产生一个虚像,这许多点的像就组成了蜡烛的虚像.

    平面镜除了在生活中作镜
    子用,还广泛应用在各种仪器
    中.例如,在高一讲过的显示微
    小形变的装置中,在库仑扭秤
    装置中,都装有平面镜,用反射
    光线把微小效应放大.潜望镜
    中用平面镜来改变光线的行进
    方向(图5.11).
    \begin{figure}[htp]\centering
        \includegraphics[scale=1]{fig/5-11.pdf}
        \caption{潜望镜示意图}
    \end{figure}

\subsection*{练习二}
\begin{enumerate}
    \item 甲乙二人面对着镜子,甲能从镜中看到乙的眼睛,乙
    是否也能看到甲的眼睛?说明理由.
    \item 在图5.11的潜望镜中,两平面镜倾斜的角度是多大?
    \item 晚上在灯下看书,如果书的纸面很光滑,有时会看到
    纸面上发出刺眼的光泽,为什么会出现这种现象?怎样消除
    它?
    \item 如图5.12所示,两个平面镜互相垂直,在跟这两个
    镜面垂直的平面内,有一条入射光线$AB$,经两个镜面反射后,
    沿$CD$方向射出,试证明不论光线$AB$以多大的入射角射入,反射光线$CD$都平行于$AB$射出.
    
    根据上面的现象,在六十年代,曾制作了由三块平面镜组
    成的反射器,由登月宇航员带到了月球上.这三块平面镜象
    房子里的墙角那样,彼此相交成直角,能把任何方向射到镜面
    上的光线逆着原方向反射回去.精确测出激光从地球射到这
    个反射器再返回地球的时间,再利用光速值就可以算出月球
    到地球的距离.
    \item 图5.13中,$M$是一个平面镜,$S$是光源,通过狭缝使
    光线射到镜面上,$M$的位置,最初是与$S$射来的光线垂直,这
    时反射光线射在标尺的零刻度上;当$M$转动一角度后,反射光
    线射到标尺的$n$刻度上.如果标尺到镜$M$的距离为$\ell$,且$\ell\gg On$,
    求$M$转动的角度是多大.
    \item 有的镜子照出的像会变形,这是什么原因?
    \begin{figure}[htp]
        \centering
        \begin{minipage}[t]{0.48\textwidth}
        \centering
\begin{tikzpicture}[>=latex ]


\fill  [pattern=vertical lines, rotate=50] (-4,0)--(0,0)--(0,4)--(.2,4)--(.2,-.2)--(-4,-.2)--(-4,0);
\draw [rotate=50](-4,0)--(0,0)--(0,4);
\draw[thick] (-4,1.25)node[left]{$A$}--(-1.5,1.25)node[above]{$B$}--(-1,-1.2)node[above]{$C$}--(-4,-1.2)node[left]{$D$}  ;
\draw[thick, ->] (-4,1.25)--(-5.5/2,1.25);
\draw[thick, <-] (-2.5,-1.2)--(-4,-1.2);


\end{tikzpicture}
        \caption{}
        \end{minipage}
        \begin{minipage}[t]{0.48\textwidth}
        \centering
\begin{tikzpicture}[>=latex]
\fill [pattern=north east lines] (-3.2,-.5) rectangle (-3, .5);
\draw (-3,-.5)node[below]{$M$}--(-3,.5);
\draw (0,-2)--(0,2);
\draw[->](2,0)--(1,0);
\foreach \x in {-2,-1.8,...,2}
{
    \draw (0,\x)--(.1,\x);
}
\draw (-3,0)--(2,0)node[right]{\Sun};
\node at (2.75,0){$S$};
\node at (0,0.2) [right] {$O$};
\draw (-3,0)--(0,1);
\draw[dashed, very thick] (-3+.15,-.5)--(-3-.15,.5);
\node at (0.2, 1.75){$n$};
\draw [very thick] (1.5, .5)--(1.5, .1);
\draw [very thick] (1.5, -.5)--(1.5, -.1);
\node at (-1.5,-.2){$\ell$};
\draw [<-](-1.5,0)--(0,0);
\draw [<-](-1.5,0.5)--(-3,0);

\end{tikzpicture}
        \caption{}
        \end{minipage}
        \end{figure}
\end{enumerate}
    
\section{球面镜}
反射面是球面一部分的镜叫做\textbf{球面镜}.用球面的内表面
作反射面的叫做\textbf{凹镜},用球面的外表面作反射面的叫做\textbf{凸镜}.

\subsection{球面镜的焦点和焦距}

凹镜对光线起会聚作用.射到凹
镜上的平行光线,被反射后会聚于一点(图5.14甲),这一点叫
做凹镜的焦点,通常用$F$表示.如果把一张纸放在焦点,纸上
会出现一个很亮的光点.凹镜的焦点是反射光线的实际会聚
点,是实焦点.
\begin{figure}[htp]\centering
\begin{minipage}[t]{0.48\textwidth}
\centering
\begin{tikzpicture}[>=latex, scale=1.5]
            \fill [pattern=north east lines] (2,-1)--(2.2,-1) to [bend left=-30] (2.2,1) --(2,1) to [bend left=30](2,-1);
            \draw [thick](2,1) to [bend left=30] (2,-1);
        \foreach \x in {-.8,-.4,0,...,.8}
        {
            \draw[->] (0,\x)--(0.5,\x);
        }    
        \draw  (0,-.8)--(2.1,-.8)--(1,0)--(.95/2,.4);\draw  (0,.8)--(2.1,.8)--(1,0)--(.95/2,-.4);
        \draw  (0,-.4)--(2.2,-.4)--(1,0)--(.9/2,.2);\draw  (0,.4)--(2.2,.4)--(1,0)--(.9/2,-.2);
        \draw  (0,0)--(2.3,0);

        \draw[->](2.1,-.8)--(3.1/2,-.4);  \draw[->](2.1,.8)--(3.1/2,.4);
        \draw[->](2.2,-.4)--(3.2/2,-.2);   \draw[->](2.2,.4)--(3.2/2,.2);
        \draw[->] (2.3,0)--(3.3/2,0);
        \node at (1,0)[above]{$F$};      \node at (2.6,0){$P$};   
        \draw (1,0)[ fill=black] circle (1pt);

        \end{tikzpicture}
\caption*{甲}
\end{minipage}
\begin{minipage}[t]{0.48\textwidth}
\centering
\begin{tikzpicture}[>=latex,scale=1.5]
    \fill [pattern=north east lines] (-2,-1)--(-2.2,-1) to [bend left=30] (-2.2,1) --(-2,1) to [bend left=-30]
    (-2,-1);
\draw [thick](-2.2,1) to [bend left=-30] (-2.2,-1);
\foreach \x in {-.8,-.4,0,...,.8}
        {
            \draw[->] (-4,\x)--(-3.5,\x);
        }    

        \draw(-3.5,-.8)--(-2.3,-.8);\draw(-3.5,-.4)--(-2.45,-.4);
        \draw(-3.5,.4)--(-2.45,.4);\draw(-3.5,.8)--(-2.3,.8);
        \draw(-3.5,0)--(-2.5,0);  \draw[->](-2.5,0)--(-3.2,0);
\draw[dashed] (-2.5,0)--(-1,0)node [below]{$F$}--(0,0)node [right]{$C$};
\draw [dashed](-2.3,-.8)--(-1,0);\draw [dashed](-2.45,-.4)--(-1,0);
\draw [dashed](-2.45,.4)--(-1,0);\draw [dashed](-2.3,.8)--(-1,0);

\draw [->](-2.3,-.8)--(-2.95,-1.2);\draw [->](-2.45,-.4)--(-6.35/2,-.6);
\draw [->](-2.45,.4)--(-6.35/2,.6);\draw [->](-2.3,.8)--(-2.95,1.2);
\node at (-2.7,0)[above]{$P$};
\draw (-1,0)[ fill=black] circle (1pt);

\end{tikzpicture}
\caption*{乙}
\end{minipage}
    \caption{球面镜的焦点和焦距}
    \end{figure}

凸镜对光线起发散作用,射到凸镜上的平行光线,被反
射后变得发散,把反射光线反向延长后,它们将会聚于一点
(图5.14乙),这一点叫做凸镜的焦点,凸镜的焦点不是反射
光线的实际会聚点,是虚焦点.

镜面的中心点$P$叫做球面镜的顶点,连接球心$C$和顶点
$P$的直线叫做主光轴,简称主轴.靠近主轴射向镜面的光线
叫做近轴光线.严格说来,只有平行于主轴的近轴光线经球
面镜反射后,才能会聚于一点,我们这里研究的只限于近轴
光线.

焦点到顶点的距离叫做焦距,通常用$f$表示.对于近轴
光线,球面镜的焦距等于球半径$R$的一半,即$f=R/2$.

利用凹镜对光的会聚作用,人们制造了生活中用的太阳
灶和工业上用的太阳炉,这是利用太阳能的一种重要方法.

反射时光路是可逆的,如果把光源放在凹镜的焦点,那么
从光源射向凹镜的光线,反射后将平行射出.探照灯、汽车头
灯、手电筒等射出的光束比较集中,能够照亮远处的物体,就
是利用了凹镜的这一性质.

凸镜可以用来扩大观察范围.从图5.15可以看出,对于
口径相同的平面镜和凸镜,观察者离镜同样远时,从凸镜观察
到的范围要比平面镜大,用一个口径不太大的凸镜,就能观
察到比较大的范围内的景像,因此,汽车上的观后镜都用
凸镜.

\begin{figure}[htp]\centering
    \includegraphics[scale=1]{fig/5-15.pdf}
    \caption{}
    \end{figure}

\subsection{球面镜成像}
球面镜也可以使物体成像,物体离球面镜
的距离不同,所成的像也不同.我们可以用下面的实验来观
察球面镜的成像情况.
\begin{figure}[htp]\centering
\begin{tikzpicture}[>=latex, scale=1.4]
    \fill [pattern=north east lines] (2,-1)--(2.2,-1) to [bend left=-30] (2.2,1) --(2,1) to [bend left=30](2,-1);
    \draw [thick](2,1) to [bend left=30] (2,-1);
\node at (0,0)[above]{$F$}; \draw [fill=black] (0,0) circle(1.5pt);
\node at (-2,0)[above]{$C$}; \draw [fill=black] (-2,0) circle(1.5pt);
\node at (-5,0){甲};
\draw (-4,0) rectangle (-3.6,.4); \draw (-3.8, .4) to [bend left=-30](-3.8, .8)to [bend left=-30] (-3.8,.4) ;
\draw [fill=gray] (-1, .5)--(-1,-1.25)--(0, -1.75)--(0, 0)--(-1, .5);
\draw [fill=white](-.5-.1,0) rectangle (-.5+.1,-.2); \draw [fill=white] (-.5, -.2) to [bend left=-30](-.5, -.4)to [bend left=-30] (-.5, -.2) ;
\draw[dashdotted] (-4.5,0)--(3,0);


\end{tikzpicture}

\begin{tikzpicture}[>=latex, scale=1.4]
    \fill [pattern=north east lines] (2,-1)--(2.2,-1) to [bend left=-30] (2.2,1) --(2,1) to [bend left=30](2,-1);
    \draw [thick](2,1) to [bend left=30] (2,-1);
\node at (0,0)[above]{$F$}; \draw [fill=black] (0,0) circle(1.5pt);
\node at (-2,0)[above]{$C$}; \draw [fill=black] (-2,0) circle(1.5pt);
\node at (-5,0){乙};
\draw (-.6,0) rectangle (-.2,.4); \draw (-.4, .4) to [bend left=-30](-.4, .8)to [bend left=-30] (-.4,.4) ;
\draw [fill=gray] (-4, .5)--(-4,-1.25)--(-3, -1.75)--(-3, 0)--(-4, .5);
\draw [fill=white](-4+.2,0) rectangle (-4+.8,-.6); 
\draw [fill=white] (-3.5, -.6) to [bend left=-30](-3.5,-1.2)to [bend left=-30] (-3.5, -.6) ;



    \draw[dashdotted] (-4.5,0)--(3,0);

\end{tikzpicture}

\begin{tikzpicture}[>=latex, scale=1.4]
    \fill [pattern=north east lines] (1.5,-1)--(1.7,-1) to [bend left=-30] (1.7,1) --(1.5,1) to [bend left=30](1.5,-1);
    \draw [thick](1.5,1) to [bend left=30] (1.5,-1);
\node at (0,0)[above]{$F$}; \draw [fill=black] (0,0) circle(1.5pt);
\node at (-2,0)[above]{$C$}; \draw [fill=black] (-2,0) circle(1.5pt);
\node at (-3,0){丙};
\draw[dashdotted] (-2.5,0)--(5,0);
\draw (.8,0) rectangle (.4,.4); \draw (.6, .4) to [bend left=-30](.6, .8)to [bend left=-30] (.6,.4) ;
\draw[dashed] (4.2,0) rectangle (3.4,.8); \draw[dashed] (3.8, .8) to [bend left=-30](3.8, 1.6)to [bend left=-30] (3.8,.8) ;

\end{tikzpicture}
    \caption{凹镜成像\\
    甲:物体位于二倍焦距以外时,成倒立缩小的实像;\\
    乙:物体位于二倍焦距和焦点之间时,成倒立放大的实像;\\
    丙:物体位于焦点以内时,成正立放大的虚像.}
    \end{figure}

如图5.16所示,把一支蜡烛放在凹镜前二倍焦距以外的
地方,用一张透明纸作光屏,在蜡烛和凹镜之间移动光屏,直
到在光屏上出现清晰的蜡烛的像.可以看到,这时的像是倒
立缩小的(图5.16甲).这个像是由反射光线实际会聚而成
的,能够用光屏接收到,所以光学中把这样的像叫做\textbf{实像}.
把蜡烛向凹镜移近,同时使光屏远离凹镜,当蜡烛位于二
倍焦距以内焦点以外时,从光屏上可以看到倒立放大的实像
(图5.16乙).使蜡灿进一步靠近镜面,当蜡烛位于焦点以内
时,无论怎样移动光屏,都得不到蜡烛的像.这时向镜里看
去,可以看到一个正立放大的像(图5.16丙).这个像不是
由反射光线实际会聚而成的,用光屏接收不到,是虚像.

把蜡烛放在凸镜前的任何位置上,用光屏都接收不到蜡
烛的实像,只能从镜中看到蜡烛的正立缩小的虚像
(图5.17).
   \begin{figure}[htp]\centering
\begin{tikzpicture}[>=latex, scale=1.5]
    \fill [pattern=north east lines] (2,-1)--(2.2,-1) to [bend left=30] (2.2,1) --(2,1) to [bend left=-30]
    (2,-1);
\draw [thick](2,1) to [bend left=-30] (2,-1);
\draw [dashed] (-3,0)--(4,0);
\node at (3,0) [above]{$F$};
\draw (3,0) [fill=black] circle (1.5pt);

\draw (-2.5,0) rectangle (-2, .5);
\draw (-2.25, .5) to [bend left=30] (-2.25,1) to [bend left=30]  (-2.25, .5);

\draw [dashed](2.5,0) rectangle (2.75, .25);
\draw [dashed](2.625, .25) to [bend left=30] (2.625, .5) to [bend left=30]  (2.625, .25);


\end{tikzpicture}
        \caption{凸镜只能成正立缩小的虚像}
        \end{figure}

\subsection*{练习三}
\begin{enumerate}
    \item 有一个凹镜,要想知道它的焦距,请你想出一个粗略
        测量的方法来.
    \item 比较凹镜和凸镜的成像情况有什么不同.
    \item 有一个凹镜,把它放在你的面前,如果能从镜中看到
        你自己的正立的像,这时凹镜顶点到你的距离是20厘米,那
        么这个凹镜的焦距至少有多大?
    \item 如图5.18所示,$M_1$和$M_2$是两个焦距相等的凹镜,
    其焦距为$f$.要想使平行于主轴的光线$a$能在$M_1$和$M_2$之
    间来回反射,两凹镜的顶点$P_1$
    和$P_2$应相距多远?在图中标出
    两凹镜的焦点$F_1$和$F_2$的位置,
    并画出光线$a$被两凹镜反射的
    光路图.
\end{enumerate}

\begin{figure}[htp]\centering
\begin{tikzpicture}[>=latex]
        \fill [pattern=north east lines] (-2,-1)--(-2.2,-1) to [bend left=30] (-2.2,1) --(-2,1) to [bend left=-30]
        (-2,-1);
\draw [thick](-2,1) to [bend left=-30] (-2,-1);
 
\fill [pattern=north east lines] (2,-1)--(2.2,-1) to [bend left=-30] (2.2,1) --(2,1) to [bend left=30](2,-1);
\draw [thick](2,1) to [bend left=30] (2,-1);
\draw [dashed] (-2.3,0)--(2.3,0);
\node at (-2.1,0)[below]{$P_1$};\node at (2.1,0)[below]{$P_2$};
\node at (-2.2,1)[above]{$M_1$};\node at (2.2,1)[above]{$M_2$};
\draw (-.5,.7)--node[above]{$a$}(2.15,.7);
        \end{tikzpicture}
    \caption{}
    \end{figure}
 
\section{光的折射}
光从空气斜射到玻璃上,在界面上一部分光线发生反射,
回到空气中;另一部分光线射入玻璃中,并改变了原来的传播
方向(图5.19).光从一种媒质
射入另一种媒质时,传播方向
发生改变的现象,叫做光的折
射.
\begin{figure}[htp]
	\centering
	\begin{tikzpicture}[>=latex, scale=.8]
\fill [cyan!20] (-3,-3) rectangle (3,0);
\draw (-3,0)--(3,0);
\node at (2.5,.5){空气};
\node at (2.5,-.5){玻璃};
\draw (0,-2.5)node[right]{$N'$}--(0,2.5)node[right]{$N$};	
\draw[very thick] (0,0)--(-45:3.5)node[right]{$B$}; \draw[->, very thick] (0,0)--(-45:1.5);
\draw[>-, very thick](150:1.5)-- (0,0); \draw[very thick] (0,0)--(150:3.5)node[right]{$A$};

\draw (0,.5) arc (90:150:.5)node[above]{$i$};	
\draw (0,-.5) arc (-90:-45:.5)node[below]{$r$};		
\node at (.25,.25){$O$};
	\end{tikzpicture}
	\caption{光从空气进入玻璃时发生折射}
\end{figure}

改变入射光线的方向,折
射光线的方向也随着改变.折
射光线与法线间的夹角叫做折
射角.折射角与入射角之间有
什么关系呢?这个问题,历史上
经过一千多年才研究清楚,公元二世纪,希腊天文学家托勒密(100—170)测量了折射角与
入射角,积累了大量的数据.根据测量结果,托勒密认为折射
角与入射角成正比.这个结论在入射角较小时大体上是正确
的,入射角较大时就不成立了.下表是实验测得的光从空气
射入玻璃时一组入射角与折射角的数据.从表中可以看出,
当入射角大于20$^\circ$时,入射角$i$与折射角$r$的比值$i/r$
随着入
射角的增大有较大的变化,不是恒量.

\begin{center}
\begin{tabular}{cccc}
\hline
    入射角$i$ & 折射角$r$ & 比值$i/r$ & 比值$\sin i/\sin r$\\
\hline
0  &  0  &  不确定 &    \\
10  &  6.7  &  1.50 &  $0.174/0.117\approx 1.49$  \\
20  &  13.3  &  1.50 &   $0.342/0.230\approx 1.49$ \\
30  &  19.6  &  1.53 &  $0.500/0.336\approx 1.49$  \\
40  &  25.2  &  1.59 &  $0.643/0.426\approx 1.51$  \\
50  &  30.7  &  1.63 &  $0.766/0.511\approx 1.50$  \\
60  &  35.1  &  1.71 &  $0.866/0.575\approx 1.51$  \\
70  &  38.6  &  1.81 &  $0.940/0.624\approx 1.50$  \\
80  &  40.6  &  1.97 &  $0.985/0.651\approx 1.51$  \\
\hline
    \end{tabular}
\end{center}

为了研究折射角与入射角之间的数量关系,在很长的一
段时间里,许多科学家作了多方面的尝试,直到1621年才由
荷兰科学家斯涅耳(1580—1626)发现了这个关系:入射角的
正弦跟折射角的正弦之比是个常数,即
\[\frac{\sin i}{\sin r}=\text{常数}\]

人们在研究折射现象时早已发现,折射光线位于入射光
线和法线所在的平面上,折射光线和入射光线分居在法线的
两侧,结合斯涅耳的发现,我们可以把光的\textbf{折射定律}表述
如下:

\begin{enumerate}
    \item 折射线在入射光线和法线所在的平面上,
折射光线和入射光线分居在法线的两侧;
\item 入射角的正弦跟折射角的正弦之比为一常数,即
\[\frac{\sin i}{\sin r}=\text{常数}\]
\end{enumerate} 

在折射现象中,光路也是可逆的.在图5.19中,如果光
线沿$BO$从玻璃射入空气中,即入射角为$r$,空气中的折射光
线将沿$OA$前进,即折射角为$i$.这样,光线由其他媒质射入
空气中时,折射角大于入射角.

利用光的折射,可以解释
水的视深比实深浅的现象.图
5.20表示一个装有水的容器,$A$
是容器底上的一点,从$A$点发
出的光线由水中射入空气时,
折射角比入射角大,折射光线
远离法线.折射光线进入眼中
后,我们根据光沿直线传播的
经验,就觉得它们是从$A'$点发出的.$A'$在$A$的上方,所以看
到容器的底部上升,水变浅了.
\begin{figure}[htp]\centering
    \includegraphics[scale=1.2]{fig/5-20.pdf}
    \caption{水的视深比实深浅}
    \end{figure}

\section{折射率} 
    光从一种媒质射入另一种媒质时,入射角的正弦跟折射
    角的正弦之比为一常数,这个规律对任何媒质都是正确的,但
    是对不同的媒质来说,这个常数是不同的.例如,光从空气射
    入玻璃时,这个常数约为1.50;光从空气射入水中时,这个常
    数约为1.33.可见这个常数跟媒质有关系,是一个反映媒质
    的光学性质的物理量,我们把它叫做媒质的\textbf{折射率}.如果用$n$
    表示折射率,那么,
    \[n=\frac{\sin i}{\sin r}\]

    在折射现象中,光通过两种媒质,所以折射率与两种媒质
    有关系,设光由媒质I射入媒质II,确切地说,这个折射率叫
    做媒质II对媒质I的\textbf{相对折射率},通常用$n_{21}$来表示.例如
    玻璃对空气的相对折射率是1.50,水对空气的相对折射率是
    1.33.

    光在不同媒质中的速度不同.研究表明,媒质的折射率
    跟光在媒质中的速度有关系.设光在媒质I中的速度是$v_1$,
    在媒质II中的速度是$v_2$,那么,媒质II对媒质I的相对折射
    率$n_{21}$
    等于$v_1$与$v_2$之比,即
\[n_{21}=\frac{v_1}{v_2} \]

    光在每种媒质中的速度是一定的,所以光从一种媒质射
    入另一种媒质时折射率是一个常数.

    光从真空射入某种媒质时的折射率,叫做该种媒质的\textbf{绝
    对折射率},通常用$n$表示.以后我们提到某种媒质的折射率
    时,就是指这种媒质的绝对折射率,设光在某种媒质中的速
    度为$v$,由于真空中的光速为$c$,所以这种媒质的绝对折射率
\[n=\frac{c}{v} \]

    光在真空中的速度大于在任何其他媒质中的速度,所以
    媒质的绝对折射率都大于1.下表是一些媒质的绝对折射率.
    \begin{center}
        \begin{tabular}{cccc}
        \hline
            媒质 & 折射率 & 媒质 & 折射率\\
        \hline
  金刚石  &  2.42 &  水晶   & 1.54\\
  玻璃   &   1.5—1.9  &  酒精  &  1.36\\
  二硫化碳  & 1.63  &  乙醚  &  1.35\\
  岩盐  &  1.54  &  水   &  1.33\\ 
        \hline
            \end{tabular}
        \end{center}

空气中的光速跟真空中的光速相差很少,可以认为空气
中的光速等于真空中的光速.因此,空气的绝对折射率可以
认为是1,某种媒质对空气的相对折射率可以认为等于这种
媒质的绝对折射率.

知道了媒质的绝对折射率,可以算出相对折射率.设光
由水射入玻璃中,已知水的折射率$n_1=1.33$,玻璃的折射率
$n_2=1.51$,根据
$n_1=c/v_1$,$n_2=c/v_2$,
可以求得玻璃对水的相对折射
率
$$n_{21}=\frac{v_1}{v_2}=\frac{n_2}{n_1}=\frac{1.51}{1.33}=1.14$$
同理,设光由玻璃射入水中,也可以求得水对玻璃的相对折射率
\[n_{12}=\frac{n_1}{n_2}=\frac{1.33}{1.51}=0.88 \]
从这个例子我们还知道$n_{21}$和$n_{12}$互为倒数,即
\[n_{21}=\frac{1}{n_{12}},\qquad n_{12}=\frac{1}{n_{21}}  \]

对于两种媒质来说,光在其中传播速度较小的,绝对折射
率较大,叫做\textbf{光密媒质};光在其中传播速度较大的,绝对折射
率较小,叫做\textbf{光疏媒质}.光密媒质和光疏媒质是相对而言的.
例如水跟空气相比,水是光密媒质;水跟玻璃相比,水是光疏
媒质.

由于
\[n_{21}=\frac{n_2}{n_1}=\frac{\sin i}{\sin r} \]
所以$n_2>n_1$时,$i>r$.这就是说,
光线由光疏媒质射入光密媒质时,折射角小于入射角.相反,
光线由光密媒质射入光疏媒质时,折射角大于入射角.

\begin{example}
    光线从空气中以入射角$i$射在玻璃砖的上表面
    上,穿过玻璃砖后,又射入空气中.如果玻璃砖的上下表面是
    平行的,求光线从玻璃砖射出后的传播方向.
\end{example}

\begin{figure}[htp]
	\centering
	\begin{tikzpicture}[>=latex, scale=.8]
\draw [fill=cyan!20] (-3,-3) rectangle (3,0);
\draw [very thick](-1,0)--(1,-3);	
\node at (-.75,0)[above]{$O$};
\node at (.75,-3)[below]{$E$};
\draw[->, very thick](-1,0)--(0,-1.5)	;
\draw[very thick] (-1,0)--+(135:2)node [left]{$A$};
\draw[very thick] (1,-3)--+(-45:2)node [right]{$B$};
\draw[very thick, -<] (-1,0)--+(135:1);
\draw[very thick, ->] (1,-3)--+(-45:1);
	\draw[dashed] (-1,-1)--(-1,1);
	\draw[dashed] (1,-2)--(1,-4);
\draw (-1,0.5) arc (90:135:.5)node[above]{$i$};
\draw (1,-3.5) arc (-90:-45:.5)node[below]{$i'$};
\draw (-1,-0.5) arc (-90:-60:.5)node[below]{$r$};
\draw (1,-2.25) arc (90:125:.75)node[right]{$r'$};


	\end{tikzpicture}
	\caption{}
\end{figure}


\begin{solution}
    光线以入射角$i$射在玻璃砖的上表面上,经玻璃砖折射
后从下表射出的光路图如图5.21所示.设光从空气进入
玻璃砖上表面后的折射角为$r$,射到下表面的入射角为$r'$,从
下表面进入空气后的折射角为$i'$,玻璃的折射率为$n$.根据
光的折射定律,
在上表面处
\[\frac{\sin i}{\sin r}=n \]
即
\begin{equation}
    \sin i=n\sin r 
\end{equation}
在下表面处
\[\frac{\sin r'}{\sin i'}=\frac{1}{n} \]
即
\begin{equation}
    \sin i' =n \sin r'
\end{equation}

由于上下两表面平行,所以
\[r'=r,\qquad \sin r'=\sin r \]
由(5.1)
(5.2)两式可得
\[\begin{split}
    \sin i'&=\sin i\\
i'&=i    
\end{split}\]

可见,从玻璃砖下表面射出的光线$EB$平行于射到上表面
的入射光线$AO$.
\end{solution}


这个例题告诉我们,光线通过两面平行的玻璃板后,传播
方向不变,但是从图中可以看出,射出的光线跟入射光线相
比,侧移了一段距离.计算表明,玻璃板越薄,光线侧移的距
离越小.

\subsection*{练习四}
\begin{enumerate}
    \item 光线以45$^\circ$的入射角从空气射入折射率为1.55的玻
璃中,折射角是多大?
\item 光线从空气射入水中,要想使折射角等于30$^\circ$,入射
角应为多大?
\item 根据水和岩盐的折射率,分别算出它们中的光速,水
中的光速大约是真空中光速的几分之几?
\item 图5.22是光线由空气进入某种媒质时的折射情况.
试由图中所给的数据求出这种媒质的折射率和这种媒质中的
光速.
\begin{figure}[htp]
	\centering
	\begin{tikzpicture}[>=latex]
\fill [cyan!20] (-3,-2) rectangle (3,0);
\node at (1,1){空气};
	\draw[dashed] (0,-1)--(0,1);
\draw(-3,0)--(3,0);
\draw[->, very thick] (0,0)--(-55:1.5); \draw[very thick] (0,0)--(-55:2);
\draw[->, very thick] (150:3)--(150:1.5); \draw[very thick] (0,0)--(150:3);
\draw (.5,0) arc (0:-55:.5) node [right] {$55^\circ$};
\draw (-1,0) arc (180:150:1) node [below] {$30^\circ$};

	\end{tikzpicture}
	\caption{}
\end{figure}

    \item 根据上节表中给出的折射率填空:
    \begin{enumerate}
        \item 光线从水中垂直射入空气中时,折射角\underline{\qquad}
    于入射
    角;从玻璃斜射入水中时,折射角\underline{\qquad}于入射角;从空气斜射
    入酒精中时,折射角\underline{\qquad}
    于入射角;从岩盐斜射入水晶中时,
    折射角\underline{\qquad}
    于入射角.
    \item 二硫化碳比水的折射率大,由此可知二硫化碳中的光
    速比水中的光速\underline{\qquad};二者相比,二硫化碳是光\underline{\qquad}
    媒质,水是光\underline{\qquad}
    媒质.
    \end{enumerate}
    
    \item 你在池边沿斜线向水面下看去,看到水中有一条鱼,
    你所看到的鱼的位置比实际的深还是浅?如果鱼也看到了你,
    鱼所看到的你的头的位置比实际的高还是低?
    \item 把一块厚玻璃板压在书上,透过玻璃板看书上的字
    跟拿走玻璃板直接看,感觉有什么不同?做做看,并解释看到
    的现象.
\end{enumerate}

\section{全反射}
\subsection{全反射现象}

光从光密媒质射入光疏媒质时,折射角大
于入射角.由此可以预料,当入射角增大到某一角度时,折射
角将等于90$^\circ$,入射角再增大,就不再有折射光线了.
    \begin{figure}[htp]
        \centering
        \begin{tikzpicture}[>=latex]
    \draw [fill=cyan!20] (-2,0) arc (180:360:2);
    
    \draw[->, very thick] (0,0)--(20:1.5); \draw[very thick] (0,0)--(20:3)node[above]{折射光线};
    \draw[->, very thick]  (0,0)--(-45:1.5); \draw[very thick]  (0,0)--(-45:3)node[below]{反射光线};
        \draw[very thick] (0,0)--(225:3);	\draw [<-,very thick](225:1.5)--(225:3);	
    \node at (-1,1){空气};
        \draw (-2,0)--(2,0);
    \draw[dashed] (0,-1)--(0,1);
    \node at (225:3.25){\Sun};
    \node at (225:3.5)[below]{光源};
        \end{tikzpicture}
        \caption{观察全反射现象}
    \end{figure}

上述现象可以用图5.23所示的半圆形玻璃砖来观察.让
光线沿着半圆形玻璃砖的半径射到直边上,可以看到,一部分
光线从直边折射到空气中,一部分光线反射回玻璃.逐渐增
大光线的入射角,将会看到,折射光线离法线越来越远,而且
折射光线越来越弱,反射光线越来越强,当入射角增大到某
一角度时,折射光线消失,只剩下反射光线,光全部反射回玻
璃中,这种现象叫做\textbf{全反射}.

\subsection{临界角}

折射角等于90$^\circ$时的入射角叫做\textbf{临界角}.光线
从光密媒质射入光疏媒质,当入射角大于临界角时,就发生全
反射现象.

利用光的折射定律,可以求出各种媒质对空气(或真空)
的临界角.如果用$C$表示临界角,$n$表示媒质的折射率,那么,
由于空气对该媒质的折射率等于$1/n$,
所以
\[\frac{\sin C}{\sin 90^\circ}=\frac{1}{n} \]
由此可得
\[\sin C=\frac{1}{n} \]

因此,已知媒质的折射率,利用上式就可以求出这种媒质
对空气(或真空)的临界角.

光的全反射现象在自然界中经常可以看到.水或玻璃巾
的气泡看起来特别明亮,就是因为光从水或玻璃射向气泡时,
在界面发生全反射.露水珠或喷泉的水珠,在阳光照耀下格
外明亮,也是因为射进水珠的光在水珠内发生全反射.

\subsection{光导纤维}
\begin{figure}[htp]\centering
    \includegraphics[scale=.6]{fig/5-24.png}
    \caption{弯曲的玻璃棒能传输光}
    \end{figure}

光从玻璃射入空气时,如果入射角大于临界
角,就发生全反射,使光不能从玻璃射到空气中.这一现象使
人们受到启发,试图用玻璃棒来传输光.如图5.24所示,把
一根弯曲的玻璃棒插在暗盒的一边,打开盒里的电灯,可以看
到从玻璃棒的下端有明亮的光传出来,如果照在纸上,就出现
一个明亮的光斑,这是因为从玻璃捧上端进入棒内的光线,
在棒的内壁上发生全反射;经过多次全反射,光线最后从棒的
下端传出来.
\begin{figure}[htp]\centering
    \includegraphics[scale=1]{fig/5-25.pdf}
    \caption{光导纤维}
    \end{figure}

现代科学技术中用的光导纤维,就是利用上述现象制成
的,光导纤维简称光纤,是一种比头发还细的玻璃丝,这种
玻璃丝分为内外两层(芯线和包层),芯线的折射率比包层的
折射率大,光从芯线射向包层时能发生全反射,这样光就在芯
线内从光纤的一端传输到另一端(图5.25).
\begin{figure}[htp]\centering
    \includegraphics[scale=1]{fig/5-26.pdf}
    \caption{光导纤维传像}
    \end{figure}

如果把许多光纤并成束,并使束中各条光纤的相对位
置保持不变,就可以用来传递图像(图5.26).医学上用光纤
来观察人体内脏的内窥镜,例如胃镜,就是用这个道理制
作的.

光导纤维在现代科学技术中有重要的应用,就象无线电
技术中把信号调制到无线电波上一样,把要传送的信号调制
到光波上,让光载着信号沿光导纤维传送出去,就可以实现光
纤通讯.光纤通讯能够同时传送大量信号,对信息的传输能
力很大,这是它的突出优点.采用光纤通讯将会引起通讯技
术的重大变革,光纤通讯在一些先进国家正在进入实用阶
段.我国也在努力提高光纤技术的水平,积极进行光纤通讯
试验,以适应现代科学技术革命的需要,这方面有许多工作
等待人们去做.

\section*{阅读材料:海市蜃楼}

夏天,在平静无风的海面上,向远方望去,有时能看到山
峰、船舶、楼台、亭阁、集市、庙宇等出现在远方的空中,古人
不明白产生这种景象的原因,对它作了不科学的解释,认为是
海中蛟龙(即蜃)吐出的气结成的,因而叫做“海市蜃楼”.1981
年8月19日文汇报曾以《蓬莱阁上观胜景,庙岛海面出蜃楼》
为题,登载了这年7月10日下午在山东省蓬莱市海面上出现
海市蜃楼的消息,当时有四、五百游人观赏了这一奇异景象.

海市蜃楼是光在密度分布不均匀的空气中传播时发生全
反射而产生的,夏天,海面上的下层空气,温度比上层低,密度
比上层大,折射率也比上层大.我们可以把海面上的空气看
作是由折射率不同的许多层气体组成的,远处的山峰、船舶、
楼房、人等反射出来的光线射向空中时,由于不断被折射,越
来越偏离法线方向,进入上层空气的入射角不断增大,以致发
生全反射,光线反射回地面,人们逆着光线看去,就会看到远
方的景物悬在空中(图5.27).
\begin{figure}[htp]\centering
    \includegraphics[scale=.6]{fig/5-27.png}
    \caption{海市蜃楼}
    \end{figure}

在沙漠里也会看到海市蜃楼现象,太阳照到沙地上,接
近沙面的热空气层比上层空气的密度小,折射率也小.从远
处物体射向地面的光线,进入折射率小的热空气层时被折射,
入射角逐渐增大,也可能发生全反射,人们逆着反射光线看
去,就会看到远处物体的倒影(图5.28),仿佛是从水面反射出
来的一样,沙漠里的行人常被这种景象所迷惑,以为前方有
水源而奔向前去,但总是可望而不可及.
\begin{figure}[htp]\centering
    \includegraphics[scale=.6]{fig/5-28.png}
    \caption{沙漠里的海市蜃楼}
    \end{figure}

在炎热夏天的柏油马路上,有时也能看到上述现象.光
线被贴近路面的热空气全反射,从远处看去,路面显得格外明
亮光滑,就象用水淋过一样.

\subsection*{练习五}

\begin{enumerate}
    \item 水和金刚石的临界角各是多大?
    \item 图5.29表示一个横
截面为等腰直角三角形的玻璃
棱镜.当光从它的一个直角边垂直射入时,就从另一个直角
边垂直射出来,而没有光从斜
边射出,这种棱镜可以使光的
传播方向改变90$^\circ$.说明它的
道理.
\begin{figure}[htp]
	\centering
\begin{tikzpicture}[>=latex]
  \draw [fill=cyan!20](0,0)--(2,0)--(0,2)--(0,0);
    \draw[ thick] (-1,1)--(1,1)--(1,-1);
\draw[->,  thick](-1,1)--(.5,1); \draw[->,  thick](1,1)--(1,-.5); 
\end{tikzpicture}
	\caption{}
\end{figure}

\item 光线从空气射入水中时,光线在水中的折射角最大为多少度?
\item 潜水的人在水面下能看到水面上的全部景象,因为
水面上180$^\circ$范围内射入水中的光线全部集中在水下97.5$^\circ$的视野内(图5.30),试说明它的道理.
\begin{figure}[htp]\centering
    \includegraphics[scale=.6]{fig/5-30.png}
    \caption{}
    \end{figure}
\end{enumerate}

\section{棱镜}

光学仪器中常用的棱镜是横截面为三角形的三棱镜,通
常简称为棱镜,下面我们来研究棱镜对光的作用.

\subsection{通过棱镜的光线~~色散}

让一束单色光从空气射向玻璃
棱镜的一个侧面,可以看到,光线通过棱镜,从另一个侧面射
出来时,方向发生了明显的变化;光线向棱镜的底面偏折(图
5.31),这是因为光线在棱镜的两个侧面上发生折射时,两次
向底面偏折所造成的.偏折角度$\theta$跟棱镜材料的折射率有
关\footnote{偏折角度$\theta$还跟入射角的大小有关系,这个问题高中不讨论.},折射率越大,偏折角度越大.
\begin{figure}[htp]\centering
    \includegraphics[scale=1]{fig/5-31.PDF}
    \caption{光线通过棱镜后,向底面偏折}
    \end{figure}

让一束白光射向玻璃棱镜,可以看到,白光通过棱镜后,
发生色散,在光屏上形成一条彩色的光带(图5.32),叫做\textbf{光
谱}.红光在最上端,紫光在最下端,中间是橙、黄、绿、蓝等色.
这表明各种色光通过棱镜后的偏折角度不同.红光的偏折角
度最小,紫光的偏折角度最大.
\begin{figure}[htp]\centering
    \includegraphics[scale=1.2]{fig/5-32.pdf}
    \caption{棱镜使白光发生色散}
    \end{figure}

    不同色光通过棱镜后的偏折角度不同,表明棱镜材料对
    不同色光的折射率不同,红光的偏折角度小,表示棱镜材料对
    红光的折射率小;紫光的偏折角度大,表示棱镜材料对紫光的
    折射率大,下表是实验测得的冕牌玻璃对各种色光的折
    射率.

\begin{center}
    \begin{tabular}{ccccccc}
        \hline
        色光&        紫&        蓝&        绿&        黄&        橙&        红\\
        \hline
        折射率&        1.532&        1.528&        1.519&        1.517&        1.514&        1.513\\
        \hline
    \end{tabular}
\end{center}

我们知道,媒质的折射率等于光在真空中的速度跟在这
种媒质中的速度之比.各种色光在真空中的速度是一样的,
都等于$c$,它们在同一媒质(例如玻璃)中的折射率不同,表明
它们在同一媒质中的速度不同,红光的折射率比其他色光小,
表明红光在媒质中的速度比其他色光大.

\subsection{全反射棱镜}

从练习五的第2题知道,横截面为等腰直
角三角形的玻璃棱镜对光能起全反射作用,这种棱镜叫做\textbf{全
反射棱镜}.

通常用的反光镜背面都镀有一层银膜,靠这层银膜来反
射光.这种镀膜的反光镜不能
使光全部反射,大约有10\%的
光被吸收掉.全反射棱镜能够
把入射光全部反射出去,所以
在一些精密的光学仪器中都
用全反射棱镜代替镀膜的平面
镜,例如,潜水艇的潜望镜中就
使用全反射棱镜(图5.33).
\begin{figure}[htp]
	\centering
	\begin{tikzpicture}[>=latex, scale=1.2]
	\draw (0,0) rectangle (.5, 4);
	\draw [very thick](0,0)--(2,0);  \draw [very thick](.5,0.5)--(2,0.5);
\draw [very thick](0,0)--(0,3.5)--(-1,3.5);  \draw [very thick](.5,0.5)--(.5,4)--(-1,4);
	%\draw (0,4)--(-2,4);  \draw (0,3.5)--(-2,3.5);
\draw [fill=cyan!20](0,0.5)--(.5,.5)--(.5,0)--(0,0.5);
\draw [fill=cyan!20](0,4)--(0,3.5)--(.5,3.5)--(0,4);

\draw(-1.5,3.75)--(.25,3.75)--(0.25,.25)--(2.5,.25);
\draw[->](-1,3.75)--(-.5,3.75);
\draw[->](.25,3.75)--(.25,2);
\draw[->](0.25,.25)--(1.5,.25);
	\end{tikzpicture}
	\caption{潜望镜中的全反射棱镜}
\end{figure}


\subsection*{练习六}
\begin{enumerate}
    \item 说明图5.31中,光线通过棱镜的$AB$和$AC$两个侧
面时,为什么都向底面偏折,如果这个棱镜的里面是空气,周
围是水,当光线通过这个空气棱镜时,射出的光线是否还向底
面偏折?画出这种情况下的光路图来.
\item 冕牌玻璃对紫光的折射率为1.532,对红光的折射率
为1.513.紫光和红光在这种玻璃中的速度各是多大?
\item 什么形状的全反射棱镜可以使光线改变180$^\circ$?这时
光线应从哪边射入?画出光路图来.
\item 光线通过棱镜时,偏折角度$\theta$跟棱镜材料的折射率
有关.设图5.31中光线对棱镜的侧面$AB$的入射角以及棱
镜的顶角$A$的大小保持不变.试定性地讨论:棱镜材料的折
射率越大,偏折角度也越大.
\end{enumerate}

\section{透镜}
折射面是两个球面,或者一个是球面,另一个是平面的透
明体,叫做透镜.透镜通常是用玻璃磨成的,它是光学仪器中
经常使用的基本元件.

\subsection{凸透镜和凹透镜}

透镜可分为两类.一类是中间厚、边缘
薄的,叫做\textbf{凸透镜},凸透镜对光线起会聚作用,又叫做\textbf{会聚透
镜},另一类是中间薄、边缘厚的,叫做\textbf{凹透镜}.凹透镜对光线
起发散作用,又叫做\textbf{发散透镜}.
\begin{figure}[htp]
    \centering
    \begin{minipage}[t]{0.48\textwidth}
    \centering
    \includegraphics[scale=1.2]{fig/5-34-1.pdf}
    \end{minipage}
    \begin{minipage}[t]{0.48\textwidth}
    \centering
    \includegraphics[scale=.8]{fig/5-34-2.pdf}
    \end{minipage}
    \caption{}
    \end{figure}


凸镜为什么能使光线会聚,凹透镜为什么能使光线发
散呢?其原理可用棱镜对光线的偏折作用来说明,如图5.34所示,透镜可以看作是由许多棱镜组成的.凸透镜上部的
棱镜底面朝下,光线通过时都向下偏折;下部的棱镜底面朝
上,光线通过时都向上偏折,因此,凸透镜使光线会聚.与此
相反,凹透镜上部的棱镜底面朝上,光线通过时都向上偏折;
下部的棱镜底面朝下,光线通过时都向下偏折.因此,凹透镜
使光线发散.

\subsection{透镜的主轴、光心和焦点}

透镜的两个球面都有自己的
球心,如图5.35中的$C_1$、$C_2$所示,我们把通过两球心$C_1$、$C_2$
的直线,叫做透镜的\textbf{主光轴},简称主轴.通常把厚度比球面的
半径小得多的透镜,叫做薄透镜.我们在后面讨论的都是薄
透镜.
\begin{figure}[htp]\centering
\begin{tikzpicture}[>=latex, scale=.6]
\draw [dashed](-3.5,0)--(5,0);
\draw (-1,0) circle (1.5);
\draw (1.6,0) circle(2);
\node at (-1,0)[above]{$C_1$};
\node at (1.6,0)[above]{$C_2$};
\node at (-0.05,0)[below]{$O$};
\draw [fill=black] (-1,0) circle (1.5pt);
\draw [fill=black] (1.6,0) circle (1.5pt);
\draw [fill=black] (-0.05,0) circle (1.5pt);
\draw [pattern=north east lines](-0.05,1.15) to [bend right=32] (-0.05,-1.15) to [bend right=54](-0.05,1.15);
\end{tikzpicture}
\begin{tikzpicture}[>=latex, scale=.6]
    \draw [dashed](-3,0)--(7,0);
    \draw (-1,0) circle (2);
    \draw (3.6,0) circle(2);
    \node at (-1,0)[above]{$C_1$};
    \node at (3.6,0)[above]{$C_2$};
    \node at (1.3,0)[below]{$O$};
    \draw [fill=black] (-1,0) circle (1.5pt);
    \draw [fill=black] (3.6,0) circle (1.5pt);
    \draw [fill=black] (1.3,0) circle (1.5pt);
    \draw [pattern=north east lines](.65,1.15) to [bend left=32] (.65,-1.15) --(1.95,-1.15) to [bend left=32](1.95,1.15)--(.65,1.15);

\end{tikzpicture}
    \caption{透镜的主轴}
    \end{figure}

主轴跟透镜的两面各有一个交点,对于薄透镜来说,这两
个交点可以看作是重合在一起的,这一点叫做透镜的\textbf{光心},用
$O$表示.透镜的中央部分相当于两面平行的薄玻璃板,通过
光心的光线相当于通过这个薄玻璃板,因此,不管从任何方向
通过光心的光线,传播方向都不改变,这是光心的重要性质.

平行于主轴的光线,通过凸透镜后会聚于主轴上的一点
(图5.36甲$F'$),这个点叫做\textbf{凸透镜的焦点}.平行于主轴的
光线通过凹透镜后变得发散(图5.36乙),这些发散光线看起
来好象是从它们的反向延长线的交点$F'$发出来的,点$F'$也
在主轴上,叫做\textbf{凹透镜的焦点}.凸透镜的焦点是实焦点,凹透
镜的焦点是虚焦点.
\begin{figure}[htp]\centering
\begin{minipage}[t]{0.48\textwidth}
\centering
\begin{tikzpicture}[>=latex, scale=1]
    \draw [pattern=north east lines] (0,1.2) to [bend left=15] (0,-1.2) to [bend left=15](0,1.2);
    \draw [dashed] (-2.5,0)--(3.2,0);
    \draw (0,0)[fill=black] circle (1.5pt);
    \draw (-2,0)[fill=black] circle (1.5pt);    \draw (2,0)[fill=black] circle (1.5pt);
    \foreach \x in {-.8,-.4,.4,.8}
    {
        \draw[->] (-2.5,\x)--(-2,\x);
    }    

\draw[->] (-2,-.8)--(0,-.8)--(2,0)--(3, .4)   ;
\draw[->] (-2,-.4)--(0,-.4)--(2,0)--(3, .2)   ;
\draw[->] (-2,.4)--(0,.4)--(2,0)--(3, -.2)   ;
\draw[->] (-2,.8)--(0,.8)--(2,0)--(3, -.4)   ;

\node at (-2,0)[above]{$F$};
\node at (2,0)[above]{$F'$};
\node at (-.5,0)[above]{$O$};


\end{tikzpicture}
\caption*{甲}
\end{minipage}
\begin{minipage}[t]{0.48\textwidth}
\centering
\begin{tikzpicture}[>=latex]
    \draw [pattern=north east lines] (0,1.2) to [bend left=15] (0,-1.2)--(.5,-1.2) to [bend left=15](.5,1.2)--(0,1.2);
    \draw [dashed] (-2.5,0)--(3.2,0);
\node at (-2,0) [above]{$F'$};
\node at (2,0) [above]{$F$};
\draw (-2,0)[fill=black] circle (1.5pt);    \draw (2,0)[fill=black] circle (1.5pt);

\foreach \x in {-.8,-.4,.4,.8}
    {
        \draw[->] (-2.5,\x)--(-2,\x);
        \draw[dashed](-2,0)--(0.25,\x);
        \draw[->](0.25,\x)--(2.5,2*\x);
    }  
    \draw  (-2,-.8)--(0.25,-.8)    ;
    \draw  (-2,-.4)--(0.25,-.4)    ;
    \draw  (-2,.4)--(0.25,.4)    ;
    \draw  (-2,.8)--(0.25,.8)    ;
    \draw[->](0.25,0)--(2.5,0);
   \node at (0,0)[above] {$O$};
   \draw (0.25,0)[fill=black] circle (1.5pt); 
\end{tikzpicture}
\caption*{乙}
\end{minipage}
    \caption{}
    \end{figure}

从透镜的焦点到光心的距离,叫做透镜的\textbf{焦距},用$f$表
示.透镜的两侧各有一个焦点,只要透镜两侧的媒质相同,两
个焦点对光心是对称的,两个焦距相等,后面讨论的都是这
种情况.

\subsection*{练习七}
\begin{enumerate}
    \item 有一个不知焦距的凸透镜,用什么办法可以粗略地测出它的焦距?
    \item 利用点光源和凸透镜如何能得到平行光线?
    \item 在图5.37中,画出光线通过透镜后的传播方向.
\end{enumerate}

\begin{figure}\centering
\begin{tikzpicture}[>=latex]
    \draw [pattern=north east lines] (0,1.2) to [bend left=15] (0,-1.2) to [bend left=15](0,1.2);
    \draw [dashed] (-2,0)--(2,0);
\node at (-1.5,0)[below] {$F$};\node at (1.5,0)[below] {$F'$};
\draw (-1.5,0) [fill=black]circle(1.5pt); \draw (1.5,0) [fill=black]circle(1.5pt);
\draw [->](-2,-.7)--(-1.5, -.7);
\draw (-1.5,-.7)--(0, -.7);


\end{tikzpicture}
\begin{tikzpicture}[>=latex]
    \draw [pattern=north east lines] (0,1.2) to [bend left=15] (0,-1.2) to [bend left=15](0,1.2);
    \draw [dashed] (-2,0)--(2,0);
    \node at (-1.5,0)[below] {$F$};\node at (1.5,0)[below] {$F'$};
    \draw (-1.5,0) [fill=black]circle (1.5pt); \draw (1.5,0) [fill=black]circle(1.5pt);
    \draw (-1.5,0)--(0, .7);
    \draw [->](-2,-.7/3)--(-1.5,0);
\end{tikzpicture}


    \begin{tikzpicture}[>=latex]
        \draw [pattern=north east lines] (0,1.2) to [bend left=15] (0,-1.2)--(.5,-1.2) to [bend left=15](.5,1.2)--(0,1.2);
        \node at (-1.5,0)[below] {$F$};\node at (1.5,0)[below] {$F'$};
        \draw (-1.5,0) [fill=black]circle (1.5pt); \draw (1.5,0) [fill=black]circle(1.5pt);
        \draw [dashed] (-2,0)--(2,0);
        \draw [->](-2,.7)--(-1.5, .7);
        \draw (-1.5,.7)--(0.25, .7);
    \end{tikzpicture}
    \begin{tikzpicture}[>=latex]
        \draw [pattern=north east lines] (0,1.2) to [bend left=15] (0,-1.2)--(.5,-1.2) to [bend left=15](.5,1.2)--(0,1.2);
        \node at (-1.5,0)[below] {$F$};\node at (1.5,0)[below] {$F'$};
        \draw (-1.5,0) [fill=black]circle (1.5pt); \draw (1.5,0) [fill=black]circle(1.5pt);
        \draw [dashed] (-2,0)--(2,0);
        \draw [->](-2,.7)--(-1.75/2, .7/2);
        \draw (-1.75/2, .7/2)--(0.25, 0);

    \end{tikzpicture}
    \begin{tikzpicture}[>=latex]
        \draw [pattern=north east lines] (0,1.2) to [bend left=15] (0,-1.2)--(.5,-1.2) to [bend left=15](.5,1.2)--(0,1.2);
        \node at (-1.5,0)[below] {$F$};\node at (1.5,0)[below] {$F'$};
        \draw (-1.5,0) [fill=black]circle (1.5pt); \draw (1.5,0) [fill=black]circle(1.5pt);
\draw [dashed] (-2,0)--(2,0);
\draw[dashed](.25,.5)--(1.5,0);
\draw (-1,1)--(.25,.5);
\draw[->] (-1.625,1.25)--(-1,1);
    \end{tikzpicture}
\caption{}
\end{figure}

 \section{透镜成像}
    利用透镜可以使物体成像,这是透镜的一个重要应用,透
    镜所成的像跟物体离透镜的距离有关系,下面我们用实验来
    研究透镜成像的情况.

    象图5.38那样,把蜡烛和光屏放在光具座的两端,把焦
    距已知的凸透镜放在蜡烛和光屏之间.调整凸透镜和光屏的
    高度,使烛焰的中点、凸透镜的光心、光屏的中点一样高,以使
    烛焰的像能成在光屏上.
    \begin{figure}[htp]\centering
        \includegraphics[scale=.6]{fig/5-38.png}
        \caption{研究透镜成像}
        \end{figure}

    先使蜡烛到透镜的距离——物距$u$大于2倍焦距($u>
    2f$),移动光屏,直到在屏上出现清晰的蜡烛的像.可以看到,
    这个像是缩小、倒立的,这时光屏到凸透镜的距离——像距
    小于2倍焦距$2f$.这个像是蜡烛发出的光通过凸透镜后会
    聚而成的,是实像.

    缩短蜡烛到凸透镜的距离,移动光屏,使像仍成在光屏
    上.可以看到,像变大.当物距等于2倍焦距时,像的大小跟
    蜡烛相同,这时像距也等于2倍焦距.

    继续缩短蜡烛到凸透镜的距离,使$2f>u>f$.可以看到,
    这时所成的像大于蜡烛,像距$v>2f$.如果再缩短物距,使$u=
    f$,则无论怎样移动光屏,也得不到像了.

    进一步减小物距,使$u<f$,光屏上仍得不到蜡烛的像.但
    是,如果从光屏这边朝着透镜看去,可以看到一个正立、放大
    的像,与蜡烛位于凸透镜的同侧,这个像是由通过凸透镜的
    光线的反向延长线会聚成的,是虚像.

    总结以上凸透镜成像的情况,可以看出:当$u>2f$时,成
    倒立、缩小的实像;当$u=2f$时,所成的倒立的实像跟物体大
    小相等;当$2f>u>f$时,成倒立、放大的实像;当$u<f$时,成
    正立、放大的虚像.

    改用凹透镜来做上面的实验,可以看到,无论怎样改变
    蜡烛到凹透镜的距离$u$,在光屏上都得不到蜡烛的实像,而
    只能通过凹透镜看到一个与蜡烛位于同侧的正立、缩小的
    虚像.

    \subsection*{练习八}
    \begin{enumerate}
        \item 放大镜所成的像是正立、放大的,这是实像还是虚
    像?放大镜应该是什么透镜?如果其焦距为$f$,镜到物的距离
    最大不能超过多少?
    \item 幻灯机所成的像是什么像?它的镜头应该是什么透
    镜?如果镜头的焦距为8厘米,镜头到幻灯片的距离只能在什
    么范围内变化?
    \item 照相机所成的像是什么像?它的镜头应该是什么透
    镜?如果镜头的焦距为7.5厘米,镜头到底片的距离最短不能
    小于多少?
    \end{enumerate}

    \section{透镜成像作图法}
    物体通过透镜所成的像,可以用作图方法求出.我们知
    道,发光点S通过凸透镜所成的实像$S'$,是从$S$射向凸透镜的
    所有光线经凸透镜折射后的会聚点(图5.39).所以我们只要
    找出其中两条光线的会聚点,就可以确定像的位置.
    \begin{figure}[htp]\centering
    \begin{tikzpicture}[>=latex, scale=.8]
    \draw [pattern=north east lines] (0,2) to [bend left=20] (0,-2) to [bend left=20](0,2);
    \draw [dashdotted] (-5,0)--(5,0);
    \node at (0,0)[fill=white, above]{$O$};
    \draw[thick] (-4,1.5)node[left]{$S$}--(0,1.5)--(2,0)--(4,-1.5)node[right]{$S'$};
    \draw[thick] (-4,1.5) --(0,-1.5)--(4,-1.5);
    \draw[thick] (-4,1.5)--(0,.5)--(4,-1.5);
    \draw[thick] (-4,1.5)--(0,-.5)--(4,-1.5);
    \node at (2,0)[above]{$F'$};
    \node at (-2,0)[below]{$F$};
    \draw (-4,1.5) [fill=black] circle (1.5pt);
    \draw (4,-1.5) [fill=black] circle (1.5pt);
    \draw (0,0) [fill=black] circle (1.5pt);
    \draw (-2,0) [fill=black] circle (1.5pt);
    \draw (2,0) [fill=black] circle (1.5pt);
    
    \draw [->](-3,1.5)--(-1,1.5);
    \draw [->](1,-1.5)--(3,-1.5);
    \draw [->](-2,0)--(-1,-1.5/2);
    \draw [->](2,0)--(3,-1.5/2);

    \draw [->](0,.5)--(2,-.5);
    \draw [>-](-2,1 )--(0,.5);
    \draw [->](0,-.5)--(2,-1);
    \draw [>-](-2,.5 )--(0,-.5);    
    
    \end{tikzpicture}
        \caption{像是通过凸透镜的所有光线的会聚点}
        \end{figure}
 
    
    当发光点不在主轴上时,它射向凸透镜的光线中,下列三
    条光线通过凸透镜后的折射光线的方向很容易确定:
    \begin{enumerate}
        \item   跟主轴平行的光线,折射后通过焦点;
        \item 通过焦点的光线,折射后跟主轴平行;
        \item 通过光心的光线经过透镜后方向不变.
    \end{enumerate}

    应用这三条光线中的任意两条,就可以求出发光点$S$的
    像$S'$(图5.40).
    \begin{figure}[htp]\centering
\begin{tikzpicture}[>=latex,scale=.8]
\draw [pattern=north east lines] (0,2) to [bend left=20] (0,-2) to [bend left=20](0,2);
\node at (0,0)[fill=white, above]{$O$};
\draw [dashdotted] (-5,0)--(5,0);
\draw[thick] (-4,1.5)node[left]{$S$}--(0,1.5)--(2,0)--(4,-1.5)node[right]{$S'$};
\draw[thick] (-4,1.5) --(0,-1.5)--(4,-1.5);
\draw[thick] (-4,1.5)--(4,-1.5);
\node at (2,0)[above]{$F'$};
\node at (-2,0)[below]{$F$};
\draw (-4,1.5) [fill=black] circle (1.5pt);
\draw (4,-1.5) [fill=black] circle (1.5pt);
\draw (0,0) [fill=black] circle (1.5pt);
\draw (-2,0) [fill=black] circle (1.5pt);
\draw (2,0) [fill=black] circle (1.5pt);

\draw [->](-3,1.5)--(-1,1.5);
\draw [->](1,-1.5)--(3,-1.5);
\draw [->](-2,0)--(-1,-1.5/2);
\draw [->](2,0)--(3,-1.5/2);
\draw [->](0,0)--(2,-1.5/2);
\draw [>-](-2,1.5/2)--(0,0);


\end{tikzpicture}
        \caption{}
        \end{figure}

    一个物体可以看作是由许多点组成的,每个点发出的光
    线经过透镜后都形成一个像点,所有的像点合在一起就是整
    个物体的像.实际作图时,只要求出物体上下两个端点的像,
    就可以求出物体的像,因为物体上其他各点的像都在这两个
    像点之间.

    作图中,通常用一根两端带箭头的直线表示透镜,跟它垂
    直的直线表示主轴,交点$O$表示光心(图5.41).
    \begin{figure}[htp]\centering
\begin{circuitikz}
\draw[<->, very thick ] (0,1.2)--(0,-1.2);
\draw [dashed](-1,0)--(1,0);
\node at (-.2,-.2){$O$};
\node at (0,-1.5){甲};
\end{circuitikz}
\qquad 
\begin{circuitikz}
\draw[>-<, very thick ] (0,1.2)--(0,-1.2);
\draw [dashed](-1,0)--(1,0);
\node at (-.2,-.2){$O$};
\node at (0,-1.5){乙};
\end{circuitikz}
        \caption{透镜的符号\\
        甲:凸透镜;        乙:凹透镜}
        \end{figure}

        图5.42是物体$AB$位于凸透镜二倍焦距以外时用作图法
        求出的像.可以看出,求出的像是倒立、缩小的,像位于二倍
        焦距以内,这跟上节实验中观察到的情况是一致的.物体位于
        二倍焦距处以及位于焦点以外、二倍焦距以内时的成像情况,
        请同学们自己练习用作图法来研究.
        \begin{figure}[htp]\centering
\begin{tikzpicture}[scale=1]
    \draw[<->, very thick ] (0,2)--(0,-2);
\draw[dashdotted](-5,0)--(5,0);
\node at (-1.5,0)[below]{$F$};\node at (1.5,0)[below]{$F'$};\node at (-.25,-.25){$O$};
\draw [fill=black] (-1.5,0) circle (1.5pt);
\draw [fill=black] (1.5,0) circle (1.5pt);
\draw [fill=black] (-3,0) circle (1.5pt);
\draw [fill=black] (3,0) circle (1.5pt);
%\draw [fill=black] (0,0) circle (1.5pt);
\draw [->, ultra thick, >=latex] (-4.5,-1)node[below]{$B$}--(-4.5,1)node[above]{$A$};
\draw [thick](-4.5,-1)--(0,-1)--(1.5,0)--(3,1);  \draw[thick] (-4.5,1)--(0,1)--(1.5,0)--(3,-1);
\draw[->,thick, >=latex] (-4.5,-1)--(-2,-1);  \draw[->,thick, >=latex] (-4.5,1)--(-2,1);
\draw[->, thick, >=latex] (-4.5,1)--(0,0)--(4.5,-1);
\draw[->, thick, >=latex] (-4.5,-1)--(0,0)--(4.5,1);
\draw [->, ultra thick, >=latex] (4.5/2,1/2)node[above]{$B'$}--(4.5/2,-1/2)node[below]{$A'$};

\draw[->, thick, >=latex] (-4.5,1)--(-4.5/2,0.5);
\draw[->, thick, >=latex] (-4.5,-1)--(-4.5/2,-0.5);
\draw[->, thick, >=latex] (0,1)--(1.5/2,0.5);
\draw[->, thick, >=latex] (0,-1)--(1.5/2,-0.5);

\end{tikzpicture}
            \caption{}
\end{figure}

            物体$AB$位于凸透镜的焦点以内时,各点射向透镜的光线
            通过透镜后呈发散状,不能会聚成实像,但是把这些光线沿反
            方向延长后能够会聚成像,这样求出的是物体的虚像(图5.
            43).从图上可以看出,凸透镜所成的虚像是正立、放大的,跟
            物体位于透镜的同侧.

            \begin{figure}[htp]\centering
\begin{tikzpicture}
    \draw[<->, very thick ] (0,2)--(0,-2);
    \draw[dashdotted](-5,0)--(5,0);
    \node at (-1.5,0)[below]{$F$};\node at (1.5,0)[below]{$F'$};\node at (-.25,-.25){$O$};
    \draw [thick](-1,.5)--(0,.5)--(1.5,0)--(4.5,-1); 
    \draw [thick](-1,-.5)--(0,-.5)--(1.5,0)--(4.5,1); 
    \draw [thick](-1,.5)--(0,0)--(4,-2);
    \draw [thick](-1,-.5)--(0,0)--(4,2);
    \draw [->, ultra thick, >=latex] (-1,-.5)node[below]{$B$}--(-1,.5)node[above]{$A$};

    \draw[dashed] (0,.5)--(-4.5, 2);   \draw[dashed] (0,-.5)--(-4.5, -2);
    \draw[dashed] (-1,.5)--(-4, 2);   \draw[dashed] (-1,-.5)--(-4, -2);

    \draw [fill=black] (-1.5,0) circle (1.5pt);
    \draw [fill=black] (1.5,0) circle (1.5pt);
    \draw [fill=black] (-3,0) circle (1.5pt);
    \draw [fill=black] (3,0) circle (1.5pt);

    \draw [->, ultra thick, >=latex, dashed] (-3,-1.5)node[below]{$B'$}--(-3,1.5)node[above]{$A'$};


    \draw [->,  thick, >=latex](-1,.5)--(-.5,.5);  \draw [->,  thick, >=latex](0,.5)--(3,-.5);
    \draw [->,  thick, >=latex](-1,-.5)--(-.5,-.5); \draw [->,  thick, >=latex](0,-.5)--(3,.5);
    \draw [->,  thick, >=latex](0,0)--(2,-1);    \draw [->,  thick, >=latex](0,0)--(2,1);

    \draw [->,  thick, >=latex](-1,.5)--(-.5,.25);\draw [->,  thick, >=latex](-1,-.5)--(-.5,-.25);
\end{tikzpicture}
                \caption{}
                \end{figure}

                凹透镜所成的像同样可以用作图法求出.只要注意这时
                平行于主轴的光线,经凹透镜折射后,是折射光线的反向延长
                线通过焦点.图5.44中$A'B'$就是物体$AB$的像.可以看出,
                求出的像与物体位于凹透镜的同侧,是正立、缩小的虚像.

                \begin{figure}[htp]\centering
\begin{tikzpicture}
    \draw[>-<, very thick ] (0,2)--(0,-2);
    \draw[dashdotted](-5,0)--(5,0);
    \node at (-1.5,0)[below]{$F$};\node at (-.25,-.25){$O$};
    \draw [fill=black] (-1.5,0) circle (1.5pt);
    \draw [fill=black] (1.5,0) circle (1.5pt);

    \draw [->, ultra thick, >=latex] (-4,-1)node[below]{$B$}--(-4,1)node[above]{$A$};

    \draw [->,  thick, >=latex](-4,1)--(0,1)--(3,3); 
    \draw [->,  thick, >=latex](-4,-1)--(0,-1)--(3,-3); 
    \draw [thick, dashed](-1.5,0)--(0,-1);
    \draw [thick, dashed](-1.5,0)--(0,1);
\draw[->,  thick, >=latex] (-4,1)--(0,0)--(4,-1);
\draw[->,  thick, >=latex] (-4,-1)--(0,0)--(4,1);
\draw[->,  thick, >=latex] (-4,1)--(-2,0.5);
\draw[->,  thick, >=latex] (-4,-1)--(-2,-.5);
\draw [->,  thick, >=latex](-4,1)--(-2,1);
\draw [->,  thick, >=latex](-4,-1)--(-2,-1);
\draw [->, ultra thick, >=latex, dashed ] (-1.1,-.3)node[below]{$B'$}--(-1.1,.3)node[above]{$A'$};



\end{tikzpicture}
                    \caption{}
                    \end{figure}

                    由此可见,用作图法求像时,如果所画光线在通过透镜后
                    相交,得到的就是实像;如果所画光线在通过透镜后的反向延
                    长线相交,得到的就是虚像.


\subsection*{练习九}
\begin{enumerate}
    \item 如图5.45所示,已知物体$AB$通过凸透镜后所成的
    实像$A'B$,试画出从$A$、$B$两点射向凸透镜的任意两条光线
    $AE$、$BD$经过凸透镜折射后的传播方向.
    \begin{figure}[htp]\centering
\begin{tikzpicture}[xscale=1.5]
    \draw[<->, very thick ] (0,2)--(0,-2);
    \draw[dashdotted](-3,0)--(3,0);
\draw[->, ultra thick, >=latex] (-1,0)node[below]{$B$}--(-1,1)node[above]{$A$};

\draw[->,  thick, >=latex] (-1,0)--(0,-.5)node[right]{$D$};
\draw[->,  thick, >=latex] (-1,1)--(0,1.2)node[right]{$E$};
\node at (.25,.25){$O$};
\draw[->, ultra thick, >=latex] (2,0)node[above]{$B'$}--(2,-2)node[below]{$A'$};

\end{tikzpicture}
        \caption{}
        \end{figure}

    \item 一物体位于凸透镜前15厘米,凸透镜的焦距为10厘
米,用作图法求出像到凸透镜的距离.
\item 焦距为10厘米的凹透镜前有一物体,物到镜的距离
为20厘米,用作图法求像到镜的距离.
\item 在图5.46中,$OO'$为透镜的主轴,$A'B'$为物体$AB$
经透镜所成的像.这个透镜是什么透镜?试用作图法求出透
镜和它的焦点的位置.                    
\begin{figure}[htp]\centering
    \begin{tikzpicture}[>=latex]
\draw [dashdotted](0,0)node[left]{$O$}--(7,0)node[right]{$O'$};
\draw [->, very thick](1,0)node[below]{$B$}--(1,1)node[above]{$A$};
\draw [->, very thick](6,0)node[above]{$B'$}--(6,-1.5)node[below]{$A'$};
    \end{tikzpicture}
    \caption{}
    \end{figure}
\end{enumerate}

\section{透镜成像公式}

\subsection{透镜成像公式}

透镜成像的物距、像距和焦距之间的关
系,也可以用公式表示出来,下面我们来推导这个公式.
\begin{figure}[htp]\centering
\begin{tikzpicture}[xscale=1.3]
\draw[<->, very thick ] (0,3)--(0,-3);
\draw (-5,0)--(4,0);
\draw [->, very thick, >=latex](-4,0)--(-4,2.5)node[above]{$A$};
\node at (-4-.2,0.25){$B$};
\draw (-4,2.5)--(0,2.5)node[right]{$C$}--(3,-2.5*.75);
\draw (-4,2.5)--(0,0)--(3,-2.5*.75);
\draw [->, very thick, >=latex](3,0)node[above]{$B'$}--(3,-2.5*.75)node[right]{$A'$};
\draw [dashed](12/7,0)node[above]{$F'$}--(12/7,-1.5);
\draw[<->,>=latex] (0,-1.3)--node[fill=white]{$f$}(12/7,-1.3);
\draw[<->,>=latex] (0,-1.5)--node[fill=white]{$u$}(-4,-1.5);
\draw [dashed](-4,0)--(-4,-1.7);  \draw [dashed](3,0)--(3,-2.5);
\draw[<->,>=latex] (0,-2.3)--node[fill=white]{$v$}(3,-2.3);
\node at (0.25,.25){$O$};
\node at (-12/7,.25){$F$};
\draw  (12/7,0)[fill=black] circle (1.5pt);
\draw  (-12/7,0)[fill=black] circle (1.5pt);
\end{tikzpicture}
    \caption{}
    \end{figure}

在图5.47中,$AB$是物体,它到凸透镜的距离为$u$;$A'B'$
是$AB$的像,像到凸透镜的距离为$v$;凸透镜的焦距为$f$.从
图中可以看出,$\triangle COF'  \sim   \triangle A'B'F'$,所以
\[\frac{CO}{A'B'}=\frac{OF'}{B'F'}  \]
另外,$\triangle ABO \sim \triangle A'B'O$,所以
\[\frac{AB}{A'B'}=\frac{BO}{B'O} \]
因为$CO=AB$,所以上面两式的左边相等,因而这两个式子的
右边也相等,即
\[\frac{OF'}{B'F'}=\frac{BO}{B'O} \]
而$OF'=f$,$B'F'=v-f$,$BO=u$,$B'O=v$,把这些代入上式,可
得
\[\frac{f}{v-f}=\frac{u}{v} \]
这个式子用起来不方便,变形后得到
\[fv+fu=uv \]
再用$uvf$除等式两边,就得到凸透镜成像的公式:
\[\frac{1}{u}+\frac{1}{v}=\frac{1}{f}  \]

上面的公式也适用于凹透镜成像(同学们可在后面的练
习中自己证明).

在运用透镜成像的公式时,需要注意:凸透镜的焦距$f$取
正值,凹透镜的焦距$f$取负值;物体到透镜的距离$u$总取正
值;实像的像距$v$取正值,虚像的像距$v$取负值.

\subsection{像的放大率}

透镜所成的像跟物体相比,可以是放大或
缩小的,也可以跟物体大小相等.为了说明像的放大情况,我
们把像的长度$A'B'$跟物体的长度$AB$之比,叫做\textbf{像的放大率},
并且用$m$表示,即放大率
\[m=\frac{A'B'}{AB} \]

从图5.47可知,
\[\frac{A'B'}{AB}=\frac{v}{u} \]
所以
\[m=\frac{v}{u}\]
即像的放大率等于像距与物距的比值.计算放大率时像距$v$
取绝对值,所以放大率$m$总是正值.

\begin{example}
    下面是一种测量凸透镜焦距的方法:如图5.48
所示,使蜡烛和光屏相距一定的距离$L$,将凸透镜从蜡烛向光屏移动,移动到位置$A$时,在光屏上得到蜡烛放大的实像;移
动到位置$B$时,在光屏上得到蜡烛缩小的实像,测出$A$、$B$间
的距离$d$,利用测得的$L$和$d$即可求出凸透镜的焦距.试求
出这个焦距.
\end{example}

\begin{figure}[htp]
	\centering
\begin{tikzpicture}[>=latex]
  \draw[dash dot](-3,0)--(6,0);
\draw [->, very thick](-2.5,0)--(-2.5,2)node[above]{蜡烛};
\draw [dashed](-2.5,0)--(-2.5, -2);
\draw [dashed](0,1.5)--(0, -.75);
\draw [dashed](3.5,0)--(3.5, -1.5);
\draw [dashed](5.5,0)--(5.5, -2);
\draw [very thick](5.5,0)--(5.5, 3)node[right]{屏};
\draw [<->](-2.5,.75)--node[fill=white]{$u_1$}(0,.75);
\draw [<->](5.5,.75)--node[fill=white]{$v_1$}(0,.75);
\draw [<->](3.5,-.5)--node[fill=white]{$d$}(0,-.5);
\draw [<->](-2.5,-1)--node[fill=white]{$u_2$}(3.5,-1);
    \draw [<->](5.5,-1)--node[fill=white]{$v_2$}(3.5,-1);
\draw [<->](-2.5,-1.5)--node[fill=white]{$L$}(5.5,-1.5);
\fill (0,0) circle (1.5pt);  \fill (3.5,0) circle (1.5pt)node[above]{$B$};
\node at (.2,0)[above]{$A$};
\end{tikzpicture}
	\caption{}
\end{figure}

\begin{solution}
    设凸透镜的焦距为$f$,透镜在$A$时物距为$u_1$,像距为$v_1$,
    透镜在$B$时物距为$u_2$,像距为$v_2$.利用透镜成像公式可得:
   \begin{align}
\frac{1}{u_1}+\frac{1}{v_1}&=\frac{1}{f}\\
\frac{1}{u_2}+\frac{1}{v_2}&=\frac{1}{f}
   \end{align}
 
    由(5.3)(5.4)两式得
\[\frac{1}{u_1}+\frac{1}{v_1}=\frac{1}{u_2}+\frac{1}{v_2} \]
    
从图上可知,$v_1=L-u_1$,$u_2=u_1+d$,$v_2=L-u_1-d$,代入
    上式,以消去$v_1$、$u_2$和$v_2$,得
    \[\frac{1}{u_1}+\frac{1}{L-u_1}=\frac{1}{u_1+d}+\frac{1}{L-u_1-d} \]

    把上式整理化简,可得
\[u_1=\frac{1}{2}(L-d) \]
    把上式和$v_1=L-u_1$代入(5.3)式,化简后可得
\[f=\frac{L^2-d^2}{4L} \]
    这就是所求的计算凸透镜焦距的公式.这个公式在后面测定
    凸透镜焦距的实验中将要用到.
\end{solution}

\subsection*{练习十}

\begin{enumerate}
    \item 试证明:
$\dfrac{1}{u}+\dfrac{1}{v}=\dfrac{1}{f}$也
适用于凹透镜成像.(提示:
对于凹透镜$f$和$v$应取负值)
\item  利用公式
$\dfrac{1}{u}+\dfrac{1}{v}=\dfrac{1}{f}$,
试讨论:
\begin{enumerate}
    \item 对于凸透镜,物距$u$在什么范围内变化时,$v$为正值,
    成实像?在什么范围内变化时,$v$为负值,成虚像?
    \item 对于凹透镜,$v$能否为正值?
\end{enumerate}
\item 一支蜡烛,距凸透镜24厘米,在离凸透镜12厘米的
屏上得到清晰的像.这个凸透镜的焦距是多长?像是放大的
还是缩小的?画出成像光路图.
\item 照相机的镜头相当于一个凸透镜,如果镜头的焦距
为10厘米,底片长3.6厘米,要想给身高1.8米的人拍一张全
身像,人到镜头的距离至少为多少米?
\item 用焦距为10厘米的凸透镜作放大镜来看微小的物
体,要想使像成在离镜25厘米的地方,镜到物的距离应为多
少?这时看到的像放大了多少倍?
\item 物体位于离凹透镜15厘米处,凹透镜的焦距是7.5
厘米,像距是多大?像的放大率是多大?
\item 幻灯机的镜头也相当于一个凸透镜,如果不改变幻
灯机的镜头,当幻灯机到幕的距离增大时,应该怎样调节镜头
到幻灯片的距离,才能得到清晰的像?这时幕上的像是变大还
是变小?
\end{enumerate}

\section{眼睛}
\subsection{眼睛为什么能看见物体}

眼晴能看见物体,从物理方面
来说跟凸透镜成像的道理是一样的.
\begin{figure}[htp]\centering
    \includegraphics[scale=1]{fig/5-49.pdf}
    \caption{眼睛构造简图}
    \end{figure}

眼睛的主要构造如图5.49所示,最外层的无色透明部
分叫做角膜,中间的透明囊状物叫做晶状体,晶状体和前面的
角膜之间充满着无色透明的液体——\textbf{水样液},晶状体和后面
的视网膜之间充满着无色透明的胶状物质——\textbf{玻璃体}.角膜、
水样液、晶状体和玻璃体都对光线产生折射,它们的共同作用
相当于一个凸透镜,这个凸透镜的前焦点约在角膜前1.5厘米
处,后焦点约在角膜后2.0厘米处.用眼晴观察的物体,距离
都大于二倍焦距,所以从物体射进眼睛里的光线经过这个凸
透镜折射后,在视网膜上形成倒立、缩小的实像,刺激分布在
视网膜上的感光细胞,通过视神经传给大脑,产生视觉,于是
我们就看到了物体.

\subsection{眼睛的调节}

眼睛要看见物体,必须使物体成像在视网膜上.视网膜的位置是固定不变的,而物体到眼睛的距离却
远近不同,眼睛是怎样使远近不同的物体都在视网膜上成清
晰的像呢?原来晶状体是有弹性的,它的弯曲程度可以靠周
围的肌肉——睫状体来调节,在观看远处物体时,由于周围
肌肉的作用,晶状体的弯曲程度变小,晶状体变得扁平,眼睛
的焦距变大;在观看近处物体时,由于周围肌肉的作用,晶状
体的弯曲程度变大,晶状体变凸,眼睛的焦距变小.因此,无
论是远处的物体还是近处的物体都能在视网膜上成清晰的
像.可见,眼睛是一个精巧的变焦距系统,在物距改变时,它
能靠改变晶状体的弯曲程度来改变焦距.眼睛的这种作用叫
做\textbf{眼睛的调节}.

眼睛的调节是有限度的,晶状体变得最扁时能够看到的
最远点,叫做眼睛的远点.正常眼睛的远点在无限远处,就
是说,从无限远处的物体射入眼睛的平行光线,它们的像恰好
能成在视网膜上.晶状体变得最凸时能看清的最近点,叫做
眼睛的近点.正常眼睛的近点约在离眼睛10厘米的地方.所
以靠眼睛的调节能看清的范围是从离眼睛10厘米到无限远
处,在合适的照明情况下,正常的眼睛看距眼睛25厘米远的
物体,不容易感到疲劳,因此把距眼睛25厘米的距离叫做\textbf{明
视距离}.

\subsection{近视眼和远视眼~~眼镜}

近视眼的视网膜到晶状体的距
离过远,或者晶状体比正常眼睛凸一些,从无限远处射来的平
行光线不能会聚在视网膜上,而会聚在视网膜前(图5.50甲).
所以近视眼的远点不在无限远处,不能看清远处的物体,只能
看清一定距离内的物体.近视眼的近点也比正常眼的近.
\begin{figure}[htp]\centering
\begin{tikzpicture}[>=latex]
    % eye begin
\draw (0,.5) arc (150:-150:1);
\draw (0,.5) arc (120:240:.58);
\draw (0.25,.5)to [bend left=35] (.25,-.5)to [bend left=35](0.25,.5) ;
% eye end
\draw[->] (-2,.3) to (-1.5,.3) ;
\draw (-2,.3) --(.25,.3)--(2-.15, -.1);
\draw[->] (-2,0) to (-1.5,0) ;
\draw (-2,0) --(.25,0)--(2-.15, 0);
\draw[->] (-2,-.3) to (-1.5,-.3) ;
\draw (-2,-.3) --(.25,-.3)--(2-.15, .1);
\node at (0,-1.5) {甲};
\end{tikzpicture}
\qquad \qquad 
\begin{tikzpicture}[>=latex]
    % eye begin
    \draw (0,.5) arc (150:-150:1);
    \draw (0,.5) arc (120:240:.58);
    \draw (0.25,.5)to [bend left=35] (.25,-.5)to [bend left=35](0.25,.5) ;
    % eye end
    \draw[->] (-2,.2) to (-1.5,.2) ;
    \draw (-2,.2) --(-1,.2)--(.25,.3)--(2-.15, 0);
    \draw[->] (-2,0) to (-1.5,0) ;
    \draw (-2,0) --(.25,0)--(2-.15, 0);
    \draw[->] (-2,-.2) to (-1.5,-.2) ;
    \draw (-2,-.2) --(-1,-.2)--(.25,-.3)--(2-.15, 0);
\draw [pattern=north east lines] (-1.25, 1)--(-.75, 1)to [bend right=20]  (-.75, -1)-- (-1.25, -1)to [bend right=20]  (-1.25, 1);

    \node at (0,-1.5) {乙};

\end{tikzpicture}
    \caption{近视眼及其矫正方法}
    \end{figure}

    那样把无限远处射来
    的平行光线会聚在视网膜上,应该用凹透镜做眼镜,使入射的
    平行光线先经过凹透镜变得发散些,再进入眼睛,会聚点就后
    移到视网膜上(图5.50乙).

    远视眼的视网膜到晶状体的距离过近,或晶状体比正常
    眼扁些,平行射入眼睛的光线将会聚在视网膜的后面(图5.51
    甲).远视眼的近点比正常眼的远,所以视力范围比正常眼小.
    矫正远视眼的方法是用凸透镜做眼镜,使射入的光线先经过
    凸透镜变得会聚一些,再进入眼睛,会聚点就前移到视网膜上
    (图5.51乙).


    \begin{figure}[htp]\centering
\begin{tikzpicture}[>=latex]
    % eye begin
    \draw (0,.5) arc (150:-150:1);
    \draw (0,.5) arc (120:240:.58);
    \draw (0.25,.5)to [bend left=35] (.25,-.5)to [bend left=35](0.25,.5) ;
    % eye end
    \draw[->] (-2,.3) to (-1.5,.3) ;
    \draw (-2,.3) --(.25,.3)--(2+.25, 0);
    \draw[->] (-2,0) to (-1.5,0) ;
    \draw (-2,0) --(.25,0)--(2+.25, 0);
    \draw[->] (-2,-.3) to (-1.5,-.3) ;
    \draw (-2,-.3) --(.25,-.3)--(2+.25, 0);
    \node at (0,-1.5) {甲};

\end{tikzpicture}\qquad \qquad 
\begin{tikzpicture}[>=latex]
    % eye begin
    \draw (0,.5) arc (150:-150:1);
    \draw (0,.5) arc (120:240:.58);
    \draw (0.25,.5)to [bend left=35] (.25,-.5)to [bend left=35](0.25,.5) ;
    % eye end
    \draw[->] (-2,.4) to (-1.5,.4) ;
    \draw (-2,.4) --(-1,.4)--(.25,.3)--(2-.15, 0);
    \draw[->] (-2,0) to (-1.5,0) ;
    \draw (-2,0) --(.25,0)--(2-.15, 0);
    \draw[->] (-2,-.4) to (-1.5,-.4) ;
    \draw (-2,-.4) --(-1,-.4)--(.25,-.3)--(2-.15, 0);
\draw [pattern=north east lines] (-1, 1)to [bend right=-20]  (-1, -1)to [bend right=-20]  (-1, 1);

    \node at (0,-1.5) {乙};

\end{tikzpicture}
        \caption{远视眼及其矫正方法}
        \end{figure}

青少年中的近视眼,多数是由于不注意用眼卫生造成的.
预防近视眼要注意;读书写字姿势要正确,眼睛与书保持30厘
米左右的距离,不要在光线太强和太暗的地方看书,看书一小
时后要休息一会,防止眼睛过度疲劳.

\subsection{视角}

物体对眼的光心$O$所张的角,叫做\textbf{视角}.从图5.52中可以看出,物体在视网膜上所成的像的大小决定于视角.
视角越大,所成的像越大,视网膜上受到刺激的感光细胞就越
多,眼睛对物体看得就越清楚.同一个物体,离眼睛近时视角
大,在视网膜上所成的像也大;离眼睛远时视角小,在视网膜
上所成的像也小.这就是物体离眼睛近比离眼睛远时看得清
楚的原因.人们在观察微小的物体时,总是把它放在离眼睛
近的地方,以增大视角,使视网膜上成的像大些.

\begin{figure}[htp]\centering
\begin{tikzpicture}[>=latex, scale=1.2]
    % eye begin
    \draw (0,.5) arc (150:-150:1);
    \draw (0,.5) arc (120:240:.58);
    \draw (0.25,.5)to [bend left=35] (.25,-.5)to [bend left=35](0.25,.5) ;
    % eye end

\draw[->, ultra thick] (-6,-1)node[below]{$B$}--(-6,1)node[above]{$A$};
\draw[->, ultra thick] (-3,-1)node[below]{$B$}--(-3,1)node[above]{$A$};

\draw (-3,-1)--(0.25,0)--(1.7, .582)node[right]{$B''$};
\draw (-3, 1)--(0.25,0)--(1.7, -.582)node[right]{$A''$};
\draw (-6,-1)--(0.25,0)--(1.85, .278)node[right]{$B'$};
\draw (-6,1)--(0.25,0)--(1.85, -.278)node[right]{$A'$};

\draw [->, thick](1.7, .582)to [bend right=-26] (1.7, -.582);
\draw [->,  thick](1.85, .278)to [bend right=-11] (1.85, -.278);

\node at (0.25,0)[above]{\small $O$};

\draw[->] (-3,-1)--(-2.75/2,-.5);
\draw[->] (-3, 1)--(-2.75/2,.5);
\draw[->] (-6,-1)--(-4.4375,-.75);
\draw[->] (-6,1)--(-4.4375,.75);

\draw [<->, >=stealth](-1.2, .45)to [bend left=-35]node[left]{$\beta$} (-1.2, -.45);
\draw [<->, >=stealth](-.8, .17)to [bend left=-35]node[left]{$\alpha$} (-.8, -.17);


\end{tikzpicture}
    \caption{物体离眼睛近时视角大}
    \end{figure}

如果物体在视网膜上的像小到只落在一个感光细胞上,
那么眼睛看到的就只是一个点.要使眼睛把物体上的两个点
区分开,必须使这两个点在视网膜上的像落在不同的感光细
胞上.这样,这两个点的视角就必须大于某一数值才行.根
据实验知道,正常眼的这一数值约等于1分($1'$),大小为0.1
毫米的物体,在离眼睛25厘米的明视距离处,所成的视角大
约就是$1'$.

把物体移得离眼睛近些可以增大视角,使眼睛看清物体,
但是这种方法是有一定限度的,物体移到近点以后就不能再
移近了.有些物体(例如天体)无法移近眼睛,不能用这种办
法来增大视角.在这些情况下,为了看清物体,就需要借助于
显微镜、望远镜等光学仪器.

\section{显微镜和望远镜}
\subsection{显微镜}

观察细菌、动植物的组织、金属的结构等细微物
体,要用显微镜.显微镜能把物体放大很多倍,下面我们来说
明它的原理.

显微镜的主要部分是装在
镜筒两端的两组透镜,每组透
镜都相当于一个凸透镜.靠近
被观察物体的一组透镜叫做\textbf{物
镜},靠近眼睛的一组透镜叫做
\textbf{目镜}.物镜的焦距很短,目镜的
焦距较长.

物镜的作用是得到被观察
物体的放大的实像,目镜的作
用是把物镜所成的实像作为物
体,一步把它放大为虚像.

\begin{figure}[htp]\centering
    \includegraphics[scale=.6]{fig/5-53.png}
    \caption{显微镜的成像光路图}
    \end{figure}

图5.53是显微镜的成像光路图.物镜$L_1$到被观察物$AB$
的距离稍大于物镜的焦距$f_1$,通过物镜得到放大的实像$A'B'$.
$A'B'$对目镜$L_2$来说是物体,使$A'B'$位于目镜的焦点$F_2$以内,
这样通过目镜就得到$A'B'$的放大的虚像$A''B''$.从图上可以
看出,$A''B''$的视角比眼睛直接看$AB$时的视角大得多.所以
用显微镜可以看清非常微小的物体.

人眼只能看清大小为0.1毫米左右的细节,光学显微镜
的放大率为1000—1500倍左右,可使我们看清物体万分之一
毫米左右的细微结构,大大提高了我们的观察能力,但是要
观察物质更细微的构造,例如晶体的结构、分子、原子等,光学
显微镜就无能为力了,必须用放大率更高的电子显微镜.

\subsection{望远镜}

观察远处的物体或天体要用望远镜.望远镜的
构造有不同的型式,下面我们介绍开普勒望远镜和反射式望
远镜.

开普勒望远镜是德国天文学家开普勒在1611年发明的,
主要用来观察天体,所以叫做天文望远镜,它由两组透镜组
成,每组透镜都相当于一个凸透镜,其中对着远处物体的一组
叫做物镜,对着眼睛的一组叫做目镜.但是跟显微镜相反,望
远镜的物镜焦距较长,目镜焦距较短.
\begin{figure}[htp]\centering
    \includegraphics[scale=.6]{fig/5-54.png}
    \caption{开普勒望远镜原理图}
    \end{figure}

开普勒望远镜的原理如图5.54所示.从天体射来的平
行光线,经过物镜$L_1$后,在焦点以外距焦点很近的地方成一
倒立缩小的实像$A'B'$.目镜$L_2$的前焦点和物镜的焦点是重
合的,所以实像$A'B'$位于$L_2$和它的焦点之间距焦点很近的
地方,$L_2$以$A'B'$为物体,形成放大的虚像$A''B''$.这样,当
我们对着目镜观察的时候,进入眼睛的光线就好象是从$A''B''$
射来的.$A''B''$的视角大于直接用眼睛观察天体时的视角,因
此从望远镜中看到的物体使人觉得离自己近了,看得清楚了.

望远镜的物镜越大,进入镜中的光就越多,所成的像就越
明亮清晰,这对于观察传来的光很弱的遥远星体是很重要的.
但是由于制造和安装上的困难,透镜的直径很难大于1米,所
以天文台用的大型望远镜多为反射式的.这种望远镜是牛顿
在1668年发明的.反射式望远镜的原理如图5.55所示.它
用一个很大的凹镜代替物镜,从遥远天体射来的平行光线,经
凹镜$C$反射后,向焦点会聚,但是在光线还没有会聚到焦点以
前,就被平面镜$M$反射到目镜$O$中,形成实像.反射式望远镜
的凹镜可以做得很大,能够集中较多的光,使成像明亮清晰.
凹镜的口径越大,能够看到的宇宙范围也就越大,现在世界
上已有口径为5米的反射式望远镜.
\begin{figure}[htp]\centering
    \includegraphics[scale=.6]{fig/5-55.png}
    \caption{反射式望远镜原理图}
    \end{figure}

\section*{阅读材料:电子显微镜和射电望远镜}
\subsection*{电子显微镜}

    \begin{figure}[htp]
        \centering
        \begin{minipage}[t]{0.48\textwidth}
        \centering
        \includegraphics[scale=.6]{fig/5-56.png}
        \end{minipage}
        \begin{minipage}[t]{0.48\textwidth}
        \centering
        \includegraphics{fig/5-56.pdf}
        \end{minipage}
        \caption{用磁场聚焦的电子显微镜的示意图(右图)}
        \end{figure}

电子显微镜是在本世纪三十年代出现的.
它是类比于光学显微镜发展起来的,光学显徽镜是用可见光照射被研究的物体,利用光学透镜使光线偏折而成像的;电子
显微镜则是让电子束穿过被研究的物体,利用电磁透镜(实际
上就是按一定要求分布的空间电场和磁场)使电子束偏转而
成像的,图5.56是用磁场聚焦的电子显微镜的示意图,
为了便于类比,左边画出了光学显微镜的示意图.发射电子
的阴极$K$相当于光学显微镜的光源,从阴极发射出来的电子,
经过磁透镜$L_1$后变为平行的电子束,$L_1$起聚光镜的作用,电
子束穿过被研究的物体$O$,产生被研究物体的透射像.磁透镜$L_2$起物镜的作用,电子束通过它,放大成像$I_1$,$I_1$再经磁透镜$L_3$放大,第二次成像$I_2$.$I_2$被投射在荧光屏$S$上,可以用
照相方法记录下来.


电子显微镜的放大率比光学显微镜的放大率高一千倍左
右,电子显微镜能观察物质的精细结构,可以拍摄出物质的
原子结构图,在现代科学技术中有重要的应用.

\subsection*{射电望远镜}
太阳、恒星和宇宙空间的物质能发出无线
电波,这种无线电波叫做射电辐射,观测射电辐射的强度,是
天文学中研究天体和宇宙的一种重要方法.射电望远镜就是
用来观测宇宙中射电辐射的仪器.
\begin{figure}[htp]\centering
    \includegraphics[scale=1]{fig/5-57.pdf}
    \caption{射电望远镜}
    \end{figure}

射电望远镜有各式各样的结构,图5.57所示的是常见的
抛物面天线射电望远镜,它有一个很大的金属抛物面状天线,
从宇宙空间射来的平行于抛物面轴的无线电波,被反射后集
中到位于抛物面焦点处的小天线上,小天线接收到的无线电
波能量通过传输线输送给接收机,接收机对电波能量进行测
量,确定射电波的强度.

利用射电望远镜进行观测有许多优点.无线电波能穿过
云雾和尘埃,因此用射电望远镜能不分晴雨昼夜连续进行观
测;对于那些难以用光学望远镜观测的天体和字宙空间,利用
射电望远镜也可以进行研究.

\section*{复习题}
\begin{enumerate}
\item 什么叫做光线?举例说明哪些现象是光沿直线传播
而产生的?
\item 什么叫光的反射?光的反射定律的内容是什么?
\item 平面镜、凹镜、凸镜各对光线起什么作用?它们的成
像情况有什么不同?
\item 什么叫光的折射?光的折射定律的内容是什么?什么
叫媒质的绝对折射率?什么叫一种媒质对另一种媒质的相对
折射率?它们跟光速的关系是怎样的?
\item 什么叫全反射现象?付么叫临界角?举两个全反射现
象的例子.
\item 射向玻璃棱镜侧面的光线,通过棱镜后,为什么向底
面偏折?什么现象表明在同种媒质中各种色光的折射率不
同?紫光的折射率大,还是红光的折射率大?
\item 凸透镜为什么能使光线会聚?凹透镜为什么能使光
线发散?为什么说凸透镜的焦点是实焦点,而凹透镜的是虚焦
点?
\item 说明凸透镜和凹透镜的成像规律.用作图法求像时
为什么要利用三条特殊光线?像是不是仅由这三条特殊光线
形成的?

写出透镜成像的公式.在应用这个公式时应注意什么?
怎样计算像的放大率?
\item 眼睛为什么能使远处的物体和近处的物体都成像在
视网膜上?近视眼和远视眼跟正常眼的差别各是什么?它们各
应戴什么眼镜?
\item 什么叫视角?为什么观察物体时视角越大看得越清
楚?
\item 说明显微镜的构造和作用,画出它的成像光路图.
\item 说明开普勒望远镜和反射式望远镜的构造和原理.

\end{enumerate}

\section*{习题}
\begin{enumerate}
    \item 为了使身高1.8米的人能从平面镜中看到自己的全
身像,如果镜和人都是直立的,平面镜的长度至少应为多长?
画图加以说明.
\item 光可以从弯曲玻璃棒的一端传到另一端,这是否跟
光在同一种媒质中沿直线传播相矛盾?
\item 在水面下1.0米处有一个点光源,从水面上看,这个
点光源能照亮水面多大的面积?
\item 用下面的方法可以测量液体的折射率:取一个半径
为$r$的软木塞,在它的圆心处插上一枚大头针,让软木塞浮在
液面上(图5.58).调整大头针插进软木塞的深度,使它露在
外面的长度为$h$.这时从软木塞周围各个方向向液体中看,恰
好看不到大头针.利用测得的数据$r$和$h$即可以求出液体的
折射率.
\begin{enumerate}
    \item 写出用$r$和$h$求折射率的计算式;
    \item 用这种方法实际做一下,求出水的折射率.
\end{enumerate}
\begin{figure}[htp]\centering
    \includegraphics[scale=1]{fig/5-58.pdf}
    \caption{}
    \end{figure}
\item 为了从坦克内部观察外界目标,在坦克壁上开有一
个长方形孔.假定坦克的壁厚为20厘米,孔的宽度为12厘
米,孔内安装一块折射率$n=1.52$的玻璃,厚度跟坦克的壁厚
相同,车内人员通过这块玻璃能看到的外界范围为多大角度?
\item 图5.59中给出了光线在光学器件上射入和射出的
方向,试在方框内画出可用的光学器件及光路图.
\begin{figure}[htp]
	\centering
\begin{tikzpicture}[>=latex]
  
\draw (0,0) rectangle (1.5,1.5);
\draw (0,2.5) rectangle (0+1.5,2.5+1.5);
\draw (4,0) rectangle (4+1.5,1.5);
 \draw (4,2.5) rectangle (4+1.5,2.5+1.5);
\draw [->] (5.5, 1.3)--(6.5,1.3);
\draw [->] (5.5, .2)--(6.5,.2);
\draw [->] (3, 1)--(4,1);
\draw [->] (3, .5)--(4,.5);


\draw [->] (5.5, 3.8)--(6.5,3.8);
\draw [->] (5.5, 2.7)--(6.5,2.7);
\draw [->] (3, 3.5)--(4,3);
\draw [->] (3, 3)--(4,3.5);

\draw [->] (-1, 3.5)--(0,3.5);
\draw [->] (-1, 3)--(0,3);
\draw [->] (.5, 4)--(.5,5);
\draw [->] (1, 4)--(1,5);

\draw [->] (-1, 1.3)--(0,1.3);
\draw [<-] (-1, .2)--(0,.2);
\draw [->] (-1, 1.1)--(0,1.1);
\draw [<-] (-1, .4)--(0,.4);

\end{tikzpicture}
	\caption{}
\end{figure}
\item 如果把凸透镜的中部用一块黑纸遮住,用它还能得
到物体的像吗?为什么?
\item 蜡烛到凸透镜的距离为20厘米,到光屏的距离为40
厘米,这时在光屏上得到清晰的烛像,如果把凸透镜向光屏
移动5厘米,光屏应向后移动多远才能再得到清晰的烛像?这
时像的放大率是增大了,还是减小了?
\item 当物体到凸透镜的距离为36厘米时,光屏上所成的
像的高度为10厘米;当物体到凸透镜的距离变为24厘米时,
光屏上像的高度变为20厘米,这个凸透镜的焦距是多大?
\item 透镜焦距的倒数$1/f$
叫做透镜的焦度,焦度的单位叫
做屈光度,透镜的焦距为1米时,焦度为1屈光度.屈光度
乘以100,就是通常所说的眼镜的度数,某同学眼镜的近视度
数是250度,他的眼镜是什么透镜?焦距是多大?
\item 如图5.60所示,如果以凸透镜的两个焦点$F_1$、$F_2$
分别作为计算物距和像距的起点,并且用x表示物体AB到
第一焦点$F_1$的距离,以$x'$表示像$A'B'$到第二焦点$F_2$的距
离.试证明;凸透镜成像的公式
\[\frac{1}{u}+\frac{1}{v}=\frac{1}{f} \]
将变为$xx'=f^2$.此式称为牛顿薄透镜公式,在光学书中经常遇到.
\begin{figure}[htp]
	\centering
\begin{tikzpicture}
  \draw [dash dot](-4,0)--(5,0);
\fill (-2,0) circle (1.5pt) node[below]{$F_1$};
\fill (2.5,0) circle (1.5pt)node[above]{$F_2$};
\draw[->, very thick, -latex](-3.5,0)node[below]{$B$}--(-3.5,.8)node[above]{$A$};
\draw[->, very thick, -latex](4.5,0)node[above]{$B'$}--(4.5,-1.5)node[below]{$A'$};
\draw[<->, ultra thick ](0,2)--(0,-2);
\node at (.25,.25){$O$};
\draw[ stealth-stealth](-3.5,.4)--node [fill=white]{$x$}(-2,.4);
\draw[ stealth-stealth](4.5,-.4)--node [fill=white]{$x'$}(2.5,-.4);
\draw (-2,0)--(-2,.5); \draw (2.5,0)--(2.5,-.6); 
\end{tikzpicture}
	\caption{}
\end{figure}
(提示:$u=x+f$,$v=x'+f$)
\item 某人用焦距为2厘米和8厘米的两个凸透镜分别
作物镜和目镜,制成一架简易显微镜.如果物镜到被观察物
体的距离为2.2厘米,要想使目镜最后所成的虚像距目镜25
厘米,目镜和物镜间的距离应为多大?这架显微镜的放大率是
多大?
\end{enumerate}































































\chapter{光的波动性}
从古代起,人们就已经熟悉光现象了。但是,光究竟是什么
呢?这却是一个不容易回答的问题。人类对光的本性的认识,
经历了漫长而曲折的道路,有一个辩证发展的过程,根据事
实建立学说,发展学说,或者决定学说的取舍;发现新的事实,
再建立新的学说。人们就是这样通过光的行为,经过分析研
究,逐渐认识光的本性的。

在这一章和下一章里,我们就基本上按照这种认识发展
过程来介绍关于光的本性的知识。

\section{光的微粒说和波动说}
光的本性问题很早就引起了人们的注意。到了十七世纪,
形成了两种学说。一种是牛顿主张的\textbf{微粒说},认为光是从光
源发出的一种物质微粒,在均匀媒质中以一定的速度传播。另
一种是惠更斯(1629—1695)提出的\textbf{波动说},认为光是某种振
动,以波的形式向周围传播。

微粒说很容易解释光的直进现象,解释光的反射也很容
易,因为小球跟光滑平面发生弹性碰撞时的反射规律跟光的
反射定律相同。然而微粒说在解释一束光射到两种媒质分界
面处会同时发生反射和折射的现象时,却发生了很大的困难。
因为根据微粒说,光在镜面上发生反射,是由于光粒子受到镜
面的推斥;发生折射,是由于受到折射物质表面的吸引。在同
时发生反射和折射的情况下,又怎样用推斥和吸引来解释呢?
波动说却比较容易解释这种现象,因为人们知道这是波经常
发生的现象。用水槽和一些简单仪器做实验就可以看到水波
同时发生反射和折射的现象,并且可以查明水波的反射和折
射规律跟光非常相似,然而波动说在解释光的直进现象时却
遇到了困难,因为人们知道波能够绕障碍物,不会象光那样
在物体的后面留下清晰的影子。

光的微粒说和波动说当时各有成功的一面,但都不能完
满地解释当时知道的各种光现象。只是由于牛顿在学术界有
很高的声望,致使微粒说在一百多年的长时期里一直占着主
导地位,波动说发展得很慢。到了十九世纪初,人们成功地在
实验中观察到了光的干涉、衍射现象,这是波的特征,无法用
微粒说来解释,于是波动说得到了公认,光的波动理论也就迅
速发展起来。

下面我们就来研究表明光的波动性的各种现象。

\section{光的干涉}
我们在力学中学过了波的干涉,研究过水波和声波的干
涉现象,我们知道,干涉是波特有的现象,只有频率相同、相
差恒定的波源——相干波源才能产生稳定的干涉现象。对于
水波或声波,相干波源是容易得到的,但是,要找到符合相干
条件的两个相干光源却很困难。在室内点两支蜡烛或两盏电
灯,只看到墙壁被均匀照亮,丝毫看不到光的干涉现象,当
然,我们不应该因此就得出光不具有波动性的结论,因为这
些光源都是独立发光的,甚至同一光源的两个发光部分,我们
也无法使它们具有相同的频率和恒定的相差,所以,即使光具
有波动性,这样的两个光源也不会产生稳定的干涉现象,无法
看到干涉图样。

\subsection{光的干涉}

1801年英国物理学家托马斯·杨(1773—1829)首先巧妙而简单地解决了相干光源的问题,成功地观察到
了光的干涉现象。
\begin{figure}[htp]\centering
    \includegraphics[scale=.6]{fig/6-1.png}
    \caption{杨氏实验}
    \end{figure}

杨氏的办法是把点光源发出的一束光分成两束,以保证
它们具有相同的频率和恒定的相,实验做法如下,让太阳
光照射到一个有小孔的屏上(图6.1),这个小孔就成了一个
“点光源”,光从小孔出来后,照射到第二个屏的两个小孔上,
这两个小孔离得很近,而且与前一小孔的距离相等,因此,如
果光足某种波动,那么任何时刻从前一小孔发出的光波都会
同时传到这两个小孔,所以这两个小孔处的光振动不但频率
相同,而且总是同相的。这两个小孔就成了两个相干光源,它
们发出的光在像屏某处叠加时,如果同相,光就加强,如果反
相,光就减弱或抵消,因此应该产生明暗条纹。实验果然产生
了预期的结果,在像屏上看到了彩色的干涉条纹。

后来用狭缝代替小孔,用单色光代替太阳光来做实验,得
到更清晰的明暗条纹,这就是著名的杨氏双缝干涉实验。图
6.2是双缝干涉的装置和产生干涉图样的示意图。
\begin{figure}[htp]\centering
    \includegraphics[scale=1.2]{fig/6-2.pdf}
    \caption{双缝干涉}
    \end{figure}

    \begin{figure}[htp]\centering
\includegraphics[scale=1.1]{fig/6-3.pdf}
\caption{}
\end{figure}
微粒说不能解释光的干涉现象,波动说则可以作出完善
的解释,并能够根据双缝的距离和缝到屏的距离以及波长计
算出屏上出现明暗条纹的位置。图6.3是用波动说研究双缝
干涉的图,设两个缝$S_1$和$S_2$的距离为$d$,到屏的距离为$\ell$,且
$\ell\gg d$。$O$是$S_1S_2$的中垂线与屏的交点,$O$到$S_1$、$S_2$的距离相
等。从$S_1$、$S_2$射出的光波到达$O$点经过的路程相等,所以两
列波到达$O$点时相差为零,也就是说,是同相的,它们互相加
强,在$O$点出现亮条纹,叫做中央亮纹,现在我们来研究离$O$
点距离为$x$的$P$点的情况。$P$到$S_1$、$S_2$的距离分别为$r_1$、$r_2$。
从$S_1$、$S_2$发出的光波到达$P$点的路程差是
\[\delta =r_2-r_1 \]
从图中可以看出,
\[r^2_1=\ell^2+\qty(x-\frac{d}{2})^2,\qquad r^2_2=\ell^2+\qty(x+\frac{d}{2})^2  \]
两式相减,可得
\[r^2_2-r^2_1=(r_2-r_1)(r_2+r_1)=2dx\]
由于$\ell\gg d$,且$\ell\gg x$,因此$r_2+r_1\approx 2\ell$,所以
\[\delta=\frac{d}{\ell}x \]
如果路程差$\delta$等于波长$\lambda$的整数倍,两列波到达$P$点时同相,
因而互相加强,这里就出现亮条纹;如果路程差$\delta$等于半波长
$\lambda/2$的奇数倍,两列波到达$P$点时反相,因而互相削弱,这里
就出现暗条纹。

所以,在屏上满足
\[x=\pm k\frac{\ell}{d}\lambda, \qquad k=0,1,2,\ldots \]
的地方出现亮条纹.当$k=0$时,$x=0$为中央亮纹;当$k=1,
2,\ldots$时,分别为中央亮纹两边的第1条、第2条……亮条纹。

在满足
\[x=\pm(2k-1)\frac{\ell}{d}\cdot \frac{\lambda}{2},\qquad k=1,2,\ldots \]
的地方出现暗条纹,当$k=1,2,\ldots$时,分别为中央亮纹两边
的第1条、第2条……暗条纹。

相邻两条亮纹(或暗纹)间的距离$\Delta x$为
\[\Delta x=\frac{\ell}{d}\lambda \]
可见,相邻两条亮纹(或暗纹)间的距离是相等的,从上式看
出,在$d$和$\lambda$相同的情况下,干涉条纹间的距离$\Delta x$跟波长$\lambda$
有关系。用不同的色光做实验,可以看到$\Delta x$的宽度不同,红
光的最宽,紫光的最窄。这表明不同色光的波长不
同,红光的波长最长,紫光的波长最短。用白光作光源时,由
于各色光的波长不同,$\Delta x$的宽度也不同,因此在中央白色亮
纹两边出现彩色条纹。

上面的公式提供了一种测量光波波长的方法:测出$n$条
亮纹(或暗纹)间的距离$a$,算出相邻两条亮纹(或暗纹)间的
距离
\[\Delta x=\frac{a}{n-1} \]
再测出$d$和$\ell$的值,就可以算出波长$\lambda$。

\subsection{波长和频率}

我们知道,波长与频率的乘积等于波速。各
种色光在真空中的速度都等于$c$,如果用$\nu$表示光波的频率;
则有$c=\lambda\nu$。由于各种色光的波长不同,可见它们的频率也
不相同。红光的波长最长,频率最小;紫光的波长最短,频率
最大。下表是各种色光的频率和在真空中波长的范围:
\begin{center}
    \begin{tabular}{cccccc}
        \hline
        色光   &     波长    &    频率    &    色光    &    波长   &     频率     \\
        &(微米)&($10^{14}$赫)&    &(微米)&($10^{14}$赫)\\
        \hline
        红   &  0.77—0.62   &     3.9—4.8    &    绿&0.58—0.49    &    5.2—6.1\\
        橙   & 0.62—0.60    &    4.8—5.0     &   蓝—靛    &    0.49—0.45    &    6.1—6.7\\
        黄&0.60—0.58   &     5.0—5.2     &   紫      &  0.45—0.39      &  6.7—7.7\\
        \hline
    \end{tabular}
\end{center}

光的波长也常用埃(\AA)作单位
\[1\text{\AA}=10^{-10}{\rm m}\]
但埃不是国际制单位。

\section{薄膜干涉及其应用}
\subsection{薄膜干涉}
把金属丝圆环在肥皂液里蘸一下,环上就形
成一层肥皂液薄膜、用单色光照射薄膜,薄膜上就产生明暗相
间的干涉条纹(图6.4)。产生这种现象是由于照射到膜上的
光会从膜的前表面和后表面分别反射回来,形成两列波,这两
列波是由同一入射波产生的,因此频率相同,相差恒定,能够
产生干涉,竖立的肥皂薄膜在重力作用下成为上薄下厚的楔
形,在薄膜的某些地方,反射回来的两列波恰好波峰和波峰
(或者波谷和波谷)叠加,光振动加强,产生亮条纹;在另外一
些地方,恰好波峰和波谷叠加,光振动削弱,产生暗条纹,这
就是薄膜干涉的原因。
\begin{figure}[htp]\centering
    \includegraphics[scale=1]{fig/6-4.pdf}
    \caption{肥皂液膜上光的干涉}
    \end{figure}


肥皂泡在太阳光照耀下会出现彩色的条纹,也是由薄膜
干涉产生的。白光中每种色光的波长不同,所以在薄膜某一厚
度的地方,某一波长的反射光互相加强,就出现这种色光的亮
纹;在另一厚度的地方,另一波长的反射光互相加强,就出现
另一色光的亮纹,这样,在薄膜上就出现了不同颜色的条纹。

\subsection{检查精密零件的表面质量}
\begin{figure}[htp]\centering
    \includegraphics[scale=.8]{fig/6-5.png}
    \caption{用干涉法检查表面}
    \end{figure}

各种精密零件,例如光学元
件,对表面加工的质量要求很高,一般精度要求在几分之一光
波波长之内,这样的表面需要用干涉法来检验,如果被检查
的表面是一个平面,可以在它的上面放一个透明的标准样板,
并在一端垫一薄片,使样板的标准平面和被检查的平面间形
成一个劈形的空气薄层(图6.5甲)。用单色光从上面照射,
入射光从空气层的上下表面反射出两列光波,于是从反射光
中就会看到干涉条纹。如果被测表面是平的,产生的干涉条
纹就是一组平行的直线;如果被测表面某些地方不平,产生的
干涉条纹就要发生弯曲(图6.5乙)。从下涉条纹弯曲的方向
和程度还可以了解被测表面的不平情况。这种测量的精度可
达$10^{-6}$厘米。

\subsection{增透膜}

现代光学装置,如摄影机、电影放映机、潜水艇
的潜望镜等,都是由许多透镜和棱镜组成的,光进入这些装
置时,在每个镜面上都有一部分光被反射,使得通过装置的光
减少,结果成的像就不清晰,计算表明,如果一个装置中包含
有六透镜,那么将有50\%左右的光被反射,为了减少光在
元件表面上的反射损失,提高成像的质量,可在元件表面涂上
一层透明薄膜(一般用氟化镁)。当薄膜厚度是入射光在薄膜
介质中波长的$1/4$
时,在薄膜的两个面上反射的光,路程差恰好
等于半个波长,因而互相抵消,这就大大减少了光的反射损
失,增强了透射光的强度。这种薄膜叫做增透膜。

在通常情况下入射光为白光,增透膜的厚度只能使一定
波长的光反射时互相抵消,不可能使白光中的所有波长的光
都互相抵消。在选择增透膜的厚度时,一般是使光谱中部的
绿光在直入射时互相抵消,因为人的视觉对这种光最敏感。
这时光谱边缘部分的红光和紫光并没有完全抵消,所以涂有
增透膜的光学镜头呈淡紫色。

\section*{阅读材料:全息照相}
同学们从报刊杂志上可能看到过全息照相这个名词,知
道全息照相是一种新的照相技术,那么,什么是全息照相?它
与普通照相有什么不同?下面我们就来介绍一下这个问题。

我们知道,普通照相是把照相机的镜头对着被拍摄的物
体,让从物体上反射的光进入镜头,在感光底片上产生物体的
像。感光底片上记录的是从物体上各点反射出来的光的强
度。

但是,光是一种波,从被摄物体上各点反射出来的光不仅
强度(它正比于光波振幅的平方)不同,而且位相也不同。全
息照相就是一种既记录反射光的强度,又记录反射光的位相
的照相术,这种照相术记录的是光波的振幅和位相的全部信
息,所以称为全息照相。
\begin{figure}[htp]\centering
    \includegraphics[scale=.8]{fig/6-6.png}
    \caption{全息照相原理示意图}
    \end{figure}

全息照相是应用光的干涉来实现的。它用激光(是良好
的相干光)作光源,全息照相的原理如图6.6所示,激光束
被分成两部分:一部分射向被摄物体,另一部分射向反射镜
(这束光叫做参考光束),从物体上反射出来的光(叫做物光
束)具有不同的振幅和位相,物光束和从反射镜来的参考光束
都射到感光片上,两束光发生干涉,在感光片上产生明暗的于
涉条纹,感光片就成了全息照片。干涉条纹的明暗记录了干
涉后光的强度,干涉条纹的形状记录了两束光的位相关系。

从全息照片的干涉条纹上不能直接看到物体的像,为了
现出物体的像,必须用激光束(参考光束)去照射全息照片,当
参考光束通过全息照片时,便复现出物光束的全部信息,于是
就能看到物体的像。

全息照相较之普通照相有许多优点。第一,它再现出来
的像是跟原来物体一模一样的逼真的立体像,跟观察实物完
全一样;第二,把全息照片分成若干小块,每一小块都可以完
整地现出原来物体的像,所以全息照片即使有缺损,也不会使
像失真;第三,在同一张感光片上可以重迭记录许多像,这些
像能够互不干扰地单独显示出来。

全息照相技术有重要的实际应用,全息照相在一张感光
片上可以重叠记录许多像,这为信息的大容量高密度储存提
供了可能,例如用全息照相方法可以把一本几百页的书的内
容存储在只有指甲大小的全息照片上。全息照相在精密测量、
无损检验、显微术等方面也得到应用,随着全息照相技术的
发展,它将会得到更广泛的应用。

\subsection*{练习一}

\begin{enumerate}
\item 绿光的干涉条纹与红光的干涉条纹有什么不同?用
白光做干涉实验,为什么会得到彩色的干涉条纹?在彩色条纹
中最靠近中央亮纹的是哪种颜色的条纹?为什么?
\item 在杨氏双缝实验中,保持双缝到屏的距离不变,调节
双缝间的距离,当距离增大时,干涉条纹间的距离将变\underline{\qquad};
当距离减小时,干涉条纹间的距离将变\underline{\qquad}。
\item 色光从真空进入媒质后,频率不变,但传播速度减小
了,波长将如何变化?
\item 用单色光做双缝干涉实验,测得双缝间的距离为0.4
毫米,双缝到屏的距离为1米,干涉条纹的间距为1.5毫米,
求所用光波的波长。
\item 取两块平玻璃板,用手指把它们紧紧捏在一起,会从
玻璃板面上看到许多彩色花纹;改变手指用力的大小,花纹的
颜色和形状也随着改变,做这个实验,并解释看到的现象。
\end{enumerate}

\section{光的衍射}
\subsection{光的衍射}

我们知道,波能够绕过障碍物产生衍射,衍射
也是波特有的现象。并且知道只有障碍物或孔的尺寸跟波长
相差不多时,才能明显地观察到波的衍射现象。

光既然是一种波动,那么,光在传播中是否也能产生衍射
现象呢?从前面讲的光的干涉实验知道,光波的波长是很短
的,只有十分之几微米,通常的物体都比它大得多,因此很难
看到光的衍射现象,但是,当光射向一个针孔、一条狭缝、一
根细丝时,就会出现衍射现象。

取一个不透光的屏,在屏中间开一个较大的圆孔。用点
光源照射,在像屏上就出现一个明亮的圆形光斑(图6.7甲)。
显然,这是光沿直线传播的结果。圆孔小一些,可以看到像屏
上的光斑也随着减小(图6.7乙).但是,圆孔很小(直径小于
0.1毫米)时,像屏上的光斑不仅不减小,反而变大了,而且光
斑的亮度也变得不均匀,成为一些明暗相间的圆环(图6.7
丙)。这些圆环的面积,远远超过了光按直线传播所能照到的
范围,就是说光绕到小孔以外的区域中去了。这就是光通过
小孔产生的衍射现象。
\begin{figure}[htp]\centering
    \includegraphics[scale=.6]{fig/6-7.png}
    \caption{光通过小圆孔的衍射}
    \end{figure}

如果在不透明的屏上装一个宽度可以调节的狭缝,代替
上面实验中的小圆孔,重做实验,可以看到:当缝比较宽时,光
沿直线传播,在像屏上出现一条亮线;当缝很窄时,光通过缝
后就明显地偏离了直线传播的方向,像屏上被照亮的范围变
宽,并且出现了明暗相间的条纹。这是光通过狭缝时产生的
衍射现象。

光的衍射现象进一步证明了光具有波动性,对建立光的
波动说起了重要的作用。关于这个问题,历史上曾有过一段
趣事,十九世纪初,法国物理学家菲涅耳(1788—1827)利用波
动理论对光的衍射现象作出了数学分析,当时一位反对波动
说的数学家泊松从菲涅耳的分析得出结论:如果菲涅耳的理论是
正确的,那么把一个小的圆盘状物体放在光束中,在距这个圆
盘一定距离的像屏上,圆盘的影的中心应当出现一个亮斑:人
们从未看到过和听说过这种现象,因而认为这是荒谬的。于
是微粒说的拥护者们认为可以驳倒波动说了。菲涅耳接受了
这一挑战,精心研究,奇迹终于出现了,实验证明盘影的中心
确实有亮斑(图6.8)。微粒说无法解释这种奇特的现象,而
波动说却能作出完满的解释。菲涅耳的理论和实验使波动说
获得了巨大的成功。

\begin{figure}[htp]\centering
\includegraphics[scale=1]{fig/6-8.jpg}
\caption{不透明盘产生的衍射,影子的中心有一个亮斑(泊松亮斑)}
\end{figure}

\subsection{衍射光栅}

光通过单狭缝产生的衍射条纹的位置跟光波
的波长有关,因此,利用衍射条纹也可以测定波长。但是单缝
的衍射条纹比较宽,测量的结果很不精确。为了精确测出光
波的波长,可以增加缝数。因为缝数增加以后,从各条单缝衍
射出来的光波要互相干涉,结果使明条纹变窄了。这个问题
的具体分析比较复杂,我们在这里就不讲了。从图6.9中可
以清楚地看到明条纹随着缝数增加而变窄的情形。
\begin{figure}[htp]\centering
    \includegraphics[scale=.6]{fig/6-9.png}
    \caption{衍射条纹随着缝数增加而变窄}
    \end{figure}

光学仪器中用的衍射光栅就是根据这个原理制成的。常
用的透射光栅是在玻璃片上刻有许多等宽而又等间距的平行
刻痕,其中刻痕是不透光的部分(图6.10)。实用的衍射光栅
一般在每毫米内有几十条乃至上千条狭缝。光栅产生的衍射
条纹又窄又亮,可以精确地测
定光波的波长,不同波长的光
通过光栅后产生的衍射条纹的
位置不同,因此,利用光栅可以
把不同波长的色光分开,就是
说,光栅跟棱镜一样具有分光
作用,用它可以产生光谱,这也
是光栅的一个用途。

\begin{figure}[htp]\centering
    \includegraphics[scale=.6]{fig/6-10.jpg}
    \caption{衍射光栅}
    \end{figure}

\subsection*{练习二}
\begin{enumerate}
\item 为什么隔着墙能听到墙那边人的说话声,但看不见
人?
\item 光的衍射现象跟光的直线传播是否矛盾?在什么情
况下光沿直线传播?
\item 把两支铅笔并在一起,中间留一条狭缝,放在眼前,
进过这条狭缝去看远处的日光灯,使狭缝的方向跟灯管平行,
就看到许多条平行的彩色条纹,做这个实验,并解释看到的
现象。
\end{enumerate}

\section{光的偏振}

光的干涉和衍射现象清楚地表明光是一种波,我们知道,
波有纵波和横波,这两种波都能够产生干涉和衍射现象,那
么,光波究竟是纵波还是横波呢?
\begin{figure}[htp]\centering
    \includegraphics[scale=1]{fig/6-11.pdf}
    \caption{横波的偏振}
    \end{figure}

我们先用机械波来说明纵波和横波的主要区别,沿着绳
传播的横波,如果在它传播的方向上放上带有狭缝的木板(图
6.11),只要狭缝的方向跟绳的振动方向相同,绳上的横波就
可以毫无阻碍地传过去;如果把狭缝的方向旋转90$^\circ$,绳上的
横波就不能通过了,这种现象叫做横波的\textbf{偏振}。纵波是沿着
波的传播方向振动的,不论狭缝方向如何,纵波都可以传过
去,不会发生偏振现象(图6.12)。
\begin{figure}[htp]\centering
    \includegraphics[scale=1]{fig/6-12.pdf}
    \caption{纵波没有偏振现象}
    \end{figure}

光是否产生偏振现象呢?十九世纪法国科学家马吕(1775
—1812)发现,光也能够产生偏振现象,我们可以用下面的方
法来观察这一现象。

取一块电气石晶体薄片或人造偏振片\footnote{常用的一种人造偏振片是把聚乙烯醇薄膜在碘溶液里浸泡
后,在较高的温度下拉伸,再烘干制成的。},通过它观察太
阳光或灯光,可以看到它是透明的。以入射光线为轴旋转晶
片,这时着到的透射光的强度并不发生变化,再取一个同样
的晶片,把它放在前一晶片的后面,通过它去观察从前一晶片
透射过来的光,就会发现,从第二个晶片透射过来的光的强度
跟两晶片的相对方向有关,把前一晶片固定,以入射光线为
轴旋转后一晶片时,从后一晶片透射过来的光的强度发生周
期性的变化:后一晶片转到某一方向时,透射光最强(图6.13
甲);再旋转90$^\circ$,转到跟前一方向垂直时,透射光最弱,几乎
等于零(图6.13乙).
\begin{figure}[htp]\centering
    \includegraphics[scale=.6]{fig/6-13.png}
    \caption{光的偏振}
    \end{figure}

把上述光现象跟机械波的偏振现象相比较,表明光通过
品片时产生偏振现象。只有横波才产生偏振现象,所以光波
是横波。

上面实验中的现象可以解释如下。太阳、电灯等普通光
源发出的光,包含着在垂直于传播方向上沿一切方向振动的
光,并没有一个占优势的方向,也就是说,沿着各个方向振动
的光波强度都相同,这种光叫做\textbf{自然光}。自然光通过第一个
晶片(叫做起偏器)后,相当于被一个“狭缝”卡了一下,只有振
动方向跟“狭缝”方向一致的光波才能通过(图6.13),这种振
动方向一定的光叫做\textbf{偏振光}。每个偏振片上都有一条标线,表
示的就是偏振片允许通过的偏振光的振动方向,这个方向叫
做偏振片的偏振化方向,自然光通过第一个晶片后虽然变成
了偏振光,但由于自然光中沿各个方向振动的光波强度都相
同,所以不论晶片转到什么方向,都会有相同强度的光透射过
来,再通过第二个晶片(叫做检偏器)去观察,情形就不同了。
不论旋转哪个晶片,两晶片的偏振化方向一致时,透射光最
强,两晶片的偏振化方向互相垂直时,透射光最弱。

光的偏振现象并不是罕见的,我们通常看到的绝大部分
光,除了从光源直接射来的,基本上都是偏振光,只是我们眼
睛不能鉴别罢了,如果通过偏振片去观察从玻璃或水面反射
的光,旋转偏振片,就会发现透射光的强度也发生周期性的变
化,从而知道反射光是偏振光。

光的偏振现象在技术中有很多应用,例如,在拍摄水面
下的景物或展览橱窗中的陈列品的照片时,由于从水面或窗
玻璃会发出很强的反射光,使得水面下的景物和橱窗中的陈
列品看不清楚,摄出的照片也模糊不清。如果在照相机镜头
上加一个偏振片,使偏振片的偏振化方向与反射光的垂直,就
可以把这些反射光滤掉,而摄得清晰的照片,汽车在夜间行
车时,迎面开来的车灯的光常常使司机看不清路面,容易发生
事故,如果在每辆汽车的车灯玻璃上和司机座席前面的窗玻
璃上各安上一块偏振片,并使它们的偏振化方向都跟水平方
向成45$^\circ$角(图6.14),就可以解决这个问题。这时,从对面
车灯射来的偏振光,由于振动方向跟司机自己座前窗玻璃上
偏振片的偏振化方向垂直,所以不会射进司机眼里。而从自
已的车灯射出去的偏振光,由于振动方向跟自己的窗玻璃上
编振片的偏振化方向相同,所以司机仍能看清自己车灯照亮
的路面和物体。

\begin{figure}[htp]\centering
    \includegraphics[scale=.6]{fig/6-14.png}
    \caption{汽车车灯和窗玻璃上的偏振片}
    \end{figure}

\section*{阅读材料:偏振光与立体电影}
看立体电影,要戴上一副特殊的眼镜,这样从银幕上看到
的人物和自然景物的影像才有立体感,犹如身临其境一般。如
果取下眼镜,银幕上的图象就模糊不清了。这是一副什么眼
镜?为什么能使我们看到的影像有立体感?学习了偏振光的知
识,就可以明白它的原理了。

为了说明立体电影的原理,首先得说说用两只眼睛看物
体跟用一只眼睛看物体的区别,用两只眼睛看物体时,由于两
眼看到的同一物体略有差别,因而产生立体感。只用一只眼
睛看物体,就没有立体感。

普通电影是用一架摄影机拍摄,一架放映机放映的,银幕
上的画面是一幅平面图像,立体电影是用两架摄影机并排在
一起,同时拍下同一景物的两幅图象,由于两架摄影机对景物
的角度不同,所以拍下的两幅图像略有差别,就如同两眼看到
的同一物体略有差别一样,放映时,用两架放映机把两架摄
影机拍下的两组影片同步放映,使略有差别的两幅图像重叠
在银幕上。这时如果用眼睛直接观看,看到的画面是模糊不
清的,要看到立体电影,需要运用光的偏振知识,使两眼各看
到一幅图像。在每架放映机前装一块偏振镜(图6.15),其作
用相当于起偏器,从两架放映机发出的带有影像的两束光,通
过偏振镜后,就成了偏振光,左右两架放映机前的偏振镜的
偏振化方向互相垂直,因此产生的两束偏振光的偏振方向也
互相垂直。这两束偏振光投射到银幕上再反射到观众,偏振
方向不改变。观众戴的眼镜是一副偏光眼镜,相当于检偏器,
偏光眼镜的两只镜片的偏振化方向也是互相垂直的,而且左
眼镜片的偏振化方向跟左边放映机前偏振镜的一致,右眼镜
片的偏振化方向跟右边放映机前偏振镜的一致。这样,左眼
只能看到左机映出的画面,右眼只能看到右机映出的画面,两
眼看到的画面略有差别,因而产生立体感。
\begin{figure}[htp]\centering
    \includegraphics[scale=.6]{fig/6-15.png}
    \caption{立体电影}
    \end{figure}

\section{光的电磁说}
十九世纪初,杨氏、菲涅耳等对光的干涉和衍射的研究,
使光的波动说获得了很大的成功,逐渐为人们所接受,物理学
家们继续发展和完善光的波动说,试图对光波作出进一步的
说明。当时人们只了解在媒质中传播的机械波,以为光波也
是这种机械波。但是,一切机械波,包括声波在内,都需要有
传播的媒质,在真空中是不能传播的,光却能够在真空中传播,
从太阳和其他恒星发出的光,能够穿过辽阔的宇宙空间传到
地球上来。那么,光是通过什么媒质传过来的呢?为了说明光
的传播问题,人们曾假设在宇宙空间里到处都充满着一种特
殊的物质,叫做“以太”,认为光是通过“以太”传播的。为了解
释光波是横波、光波传播的速度很大、光波在不同媒质中的传
播速度不同等问题,对“以太”这种物质的性质作了种种假设,
例如,“以太”的密度应该非常小,但是又应该具有很大的弹
性,对“以太”性质所作的假设有些是互相矛盾的,很难使人
相信存在这样的物质。为了证明“以太”的存在,人们曾做过
各种实验,但是都失败了,这使得认为光波是通过“以太”传
播的机械波的理论陷入了困境。

1846年,法拉第发现在磁场的作用下,偏振光的振动面
会发生改变。这个发现很重要,它表明光和电磁现象间存在
着联系,启示人们把光现象和电磁现象联系起来考虑。

十九世纪六十年代,麦克斯韦在研究电磁场理论时预见
了电磁波,并且指出电磁波是横波,电磁波的传播速度等于光
速。麦克斯韦根据电磁波跟光波的这些相似性指出,光波是
一种电磁波,这就是\textbf{光的电磁说}。

二十多年以后,赫兹用实验证实了电磁波的存在,测得电
磁波的传播速度确实等于光速,而且电磁波也能产生反射、折
射、干涉、衍射、偏振等现象,其规律都跟光波的相同。这就从
实验上证实了光是一种电磁波。

前面第四章已经讲过,电磁波跟机械波不同,电磁波可以
在真空中传播,不需要依靠别的媒质。这就解决了光波在传
播媒质上所遇到的困难。

麦克斯韦提出的光的电磁说,在物理学的发展中有很重
要的意义,它把光现象和电磁现象统一起来,指出了它们的一
致性,再一次证明了自然现象之间是相互联系的,光的电磁
说使人们对光的本性的认识前进了一大步。

\section{电磁波谱}
我们已经知道,无线电波是电磁波,其波长范围从几十千
米到几毫米,现在又知道了光波也是电磁波,其波长不到1微
米。可见,电磁波是一个很大的家族,包括的波长范围很大。
光波里能够作用于我们的眼睛并引起视觉的部分,只是一个
很窄的波段,通常也叫做可见光。正象在可闻声波范围外还
存在着大量的听不见的超声波和次声波一样,在可见光波范
围外还存在着大量的看不见的红外线和紫外线。

\subsection{红外线}

红外线是英国物理学家赫谢耳在1800年发现
的,他用灵敏温度计研究光谱里各种色光的热作用时,把温度
计移到光谱的红光区域外侧,它的温度上升得更高,说明那里
有看不见的射线照射到温度计上,这种射线后来就叫做红外
线.给电炉丝通电,电炉丝的温度大约上升到500$^{\circ}{\rm C}$以上时,
才开始发出暗红色的光,随着温度的升高,它逐渐变成橙色、
黄色;但在电炉丝发光之前,我们就已感到热了,这就是它发
射了红外线的缘故。一切物体都可以发射红外线,温度较高
的物体发出的红外线也较多。

红外线在工业、农业、军事、科研以及人民生活中都有广
泛的应用,红外线技术已发展成为一门现代科学技术,红外
线最显著的作用是热作用,所以可以利用红外线来加热,红外
线炉,红外线烤箱、红外线干燥器等,都是利用红外线来加热
的。这种加热方法的优点是能使物体从内部发热,加热效率
高,效果好。利用对红外线敏感的底片可以进行远距离摄影
和高空摄影,从卫星上用红外线对地面摄影,从照片上可以
清晰地看出地面上的城市、街道、桥梁和房屋,这种摄影不受
白天和夜晚的限制。利用红外成像技术可以制成军事上用的
夜视仪,使人们在漆黑的夜间能够看见目标,一切物体,包
括大地、云雾、冰块、人体、飞机和车船,都在不停地辐射红
外线,并且不同的物体辐射的红外线的波长和强度不同,利用
灵敏的红外线探测器吸收物体发出的红外线,然后用电子仪
器对接收到的信号进行处理,就可以察知被探测物体的特征。
这种技术叫做\textbf{红外线遥感}。利用红外线遥感技术,可以在飞
机或卫星上勘测地热、寻找水源、监测森林火情、估计农作物
的长势和收成、预报台风寒潮等。红外线遥感技术的应用范
围极其广泛,还在迅速发展中。

\subsection{紫外线}

紫外线是德物理学家里特在1801年发现的。
如果在光谱的紫外区域放一张照相底片,或者放一个光敏电
阻,都能够察知紫外线的存在。紫外线的波长比紫光还短。一
切高温物体,如太阳、弧光灯发出的光都含有紫外线,利用气
体放电也可以激发紫外线。紫外线的主要作用是化学作用。
紫外线很容易使照相底片感光。用紫外线照相能辨认出细微
差别,例如可以清晰地分辨出留在纸上的指纹。紫外线有很
强的荧光效应,能使许多物质激发荧光,日光灯和农业上诱
杀害虫用的黑光灯,都是用紫外线来激发荧光物质发光的,紫
外线还有杀菌消毒作用,医院里常用紫外线来消毒病房和手
术室,紫外线还能促进生理作用和治疗皮肤病、软骨病等,经
常在矿井下劳动的工人,适当地照射紫外线,能促进身体健
康。但过强的紫外线能伤害人的眼睛和皮肤,电焊的弧光中
有强烈的紫外线,因此电焊工在工作时必须穿好工作服,并戴
上防护面罩。

\subsection{伦琴射线}
比紫外线波长还短的电磁波,有伦琴射线。德
国物理学家伦琴(1845—1923)在1895年研究阴极射线的性质
时,发现阴极射线的高速电子流射到玻璃管壁上,管壁会发出
一种看不见的射线,这种射线的穿透本领很大,能使放在厚纸
后面的荧光物质铂氰化钡发出荧光,并能使包在黑纸里的照
相底片感光,伦琴当时不知道这是什么射线,把它叫做X射
线。后来人们做了大量实验,发现高速电子流射到任何固体
上,都会产生这种射线,并且从它产生的衍射现象知道它是波
长很短的电磁波。为了纪念伦琴,就把这种射线叫做伦琴射
线,图6.16是产生伦琴射线的装置,叫做伦琴射线管.图中
的螺旋钨丝$K$是它的阴极,用钨或铂制成的电极$A$是它的阳
极,又叫对阴极。管里的真空程度很高,气压约为$10^{-3}$—$10^{-5}$
帕($10^{-5}$—$10^{-7}$毫米汞柱),用电池组或变压器给钨丝K通
电,钨丝达到赤热状态就向周围发射电子,把管的阴阳两极
接到几万伏的高压电源上,管内就产生很强的电场,炽热钨
丝发出的电子在电场力的作用下以很大的速度射到对阴极
上,就从那里激发出相当强的伦琴射线。伦琴射线穿透物质
的本领跟物质的密度有关系,在工业上可以用它来检查金属
部件有没有砂眼、裂纹等缺陷,在医学上可以用它来透视人
体,检查体内的病变和骨折的情况。
\begin{figure}[htp]\centering
    \includegraphics[scale=1.2]{fig/6-16.pdf}
    \caption{伦琴射线管}
    \end{figure}

此外,还有比伦琴射线波长更短的电磁波,那就是放射性
元素放出的$\gamma$射线,我们将在第九章学习。

\subsection{电磁波谱}

无线电波、红外线、可见光、紫外线、伦琴射
线、$\gamma$射线合起来,构成了范围非常广阔的\textbf{电磁波谱}(图6.17),其中最长的波长是最短的波长的$10^{21}$倍以上,从图中
可以看出,各种电磁波的范围已经衔接起来,并且发生了交
错,例如长波的红外线和微波已经重叠,短波的紫外线已经
进入伦琴射线的区域,总的说来,从无线电波到$\gamma$射线,都是
本质上相同的电磁波,它们的行为服从共同的规律。另一方
面,由于它们的频率或波长不同而又表现出不同的特性,例
如,波长较长的无线电波,很容易表现出干涉、衍射等现象,但
对波长越来越短的可见光、紫外线、伦琴射线、$\gamma$射线,要观察
到它们的干涉、衍射现象,就越来越困难了。

\begin{figure}[htp]\centering
    \includegraphics[scale=1]{fig/6-17.pdf}
    \caption{电磁波谱}
    \end{figure}

不同的电磁波产生的机理不同,无线电波是振荡电路中
自由电子的运动产生的;红外线、可见光、紫外线是原子的外
层电子受到激发后产生的;伦琴射线是原子的内层电子受到
激发后产生的;$\gamma$射线是原子核受到激发后产生的。

\section{光谱}
光波是由原子内部运动的电子产生的,各种物质的原子
内部电子的运动情况不同,所以它们发射的光波也不同。研
究不同物质的发光和吸收光的情况,有重要的理论和实际意
义,已成为一门专门的学科-光谱学,下面简单介绍一些
关于光谱的知识。

\subsection{分光镜}
\begin{figure}[htp]\centering
    \includegraphics[scale=.6]{fig/6-18.png}
    \caption{分光镜构造原理示意图}
    \end{figure}

观察光谱要用分光镜,这里我们先讲一下分光
镜的构造原理,图6.18是分光镜的构造原理示意图.它是
由平行光管$A$、三棱镜$P$和望远镜筒$B$组成的。平行光管$A$的
前方有一个宽度可以调节的狭缝$S$,它位于透镜$L_1$的焦平
面\footnote{通过焦点垂直于透镜主光轴的平面,叫做透镜的焦平面。}处。从狭缝射入的光线经透镜$L_1$折射后,变成平行光线
射到三棱镜$P$上。不同颜色的光经过三棱镜沿不同的折射方
向射出,并在透镜$L_2$后方的焦平面$MN$上分别会聚成不同
颜色的像(谱线)。通过望远镜筒$B$的目镜$L_3$,就看到了放大
的光谱像。如果在$MN$那里放上照相底片,就可以摄下光谱
的像,具有这种装置的光谱仪器叫做摄谱仪。

\subsection{发射光谱}

物体发光直接产生的光谱叫做\textbf{发射光谱}。发
射光谱有两种类型;连续光谱和明线光谱。

连续分布的包含有从红光到紫光各种色光的光谱叫做\textbf{连
续光谱}。炽热的固体、液体和高压气体的发射光谱
是连续光谱。例如电灯丝发出的光、炽热的钢水发出的光都
形成连续光谱。

只含有一些不连续的亮线的光谱叫做\textbf{明线光谱}。
明线光谱中的亮线叫做谱线,各条谱线对应于不同波长的光。
稀薄气体或金属的蒸气的发射光谱是明线光谱。明线光谱是
由游离状态的原子发射的,所以也叫\textbf{原子光谱}。
\begin{figure}[htp]\centering
    \includegraphics[scale=.6]{fig/6-19.png}
    \caption{光谱管}
    \end{figure}

观察气体的原子光谱,可以使用光谱管(图6.19),它是
一支中间比较细的封闭的玻璃管,里面装有低压气体,管的两
端有两个电极。把两个电极接到高压电源上,管里稀薄气体发
生辉光放电,产生一定颜色的光。

观察固态或液态物质的原子光谱,可以把它们放到煤气
灯的火焰或电弧中去烧,使它们气化后发光,就可以从分光镜
中看到它们的明线光谱。

实验证明,原子不同,发射的明线光谱也不同,每种元素
的原子都有一定的明线光谱,每种原子只能发出具有本身特征的某些波长的光,因此,
明线光谱的谱线叫做原子的\textbf{特征谱线}。利用原子的特征谱线
可以鉴别物质和研究原子的结构。

\subsection{吸收光谱}
高温物体发出的白光(其中包含连续分布的
一切波长的光)通过物质时,某些波长的光被物质吸收后产生
的光谱,叫做\textbf{吸收光谱}。例如,让弧光灯发出的白光通过温度
较低的钠气(在酒精灯的灯心上放一些食盐,食盐受热分解就
会产生钠气),然后用分光镜来观察,就会看到在连续光谱的
背景中有两条挨得很近的暗线。这就是钠原子的吸收光谱,值
得注意的是,各种原子的吸收光谱中的每一条暗线都跟该种
原子的发射光谱中的一条明线相对应,这表明,低温气体原
子吸收的光,恰好就是这种原子在高温时发出的光。因此,吸
收光谱中的谱线(暗线),也是原子的特征谱线,只是通常在吸
收光谱中看到的特征谱线比明线光谱中的少。

\subsection{光谱分析}

由于每种原子都有自己的特征谱线,因此可
以根据光谱来鉴别物质和确定它的化学组成。这种方法叫做
光谱分析,做光谱分析时,可以利用发射光谱,也可以利用吸
收光谱。这种方法的优点是非常灵敏而且迅速。某种元素在
物质中的含量达$10^{-10}$克,就可以从光谱中发现它的特征谱
线,因而能够把它检查出来。光谱分析在科学技术中有广泛
的应用。例如,在检查半导体材料硅和锗是不是达到了高纯
度的要求时,就要用到光谱分析,在历史上,光谱分析还帮助
人们发现了许多新元素,例如,铷和铯就是从光谱中看到了
以前所不知道的特征谱线而被发现的。光谱分析对于研究天
体的化学组成也很有用。十九世纪初,在研究太阳光谱时,发
现它的连续光谱中有许多暗线,最初不知道这些暗线是怎样形成的,后来人们了解
了吸收光谱的成因,才知道这是太阳内部发出的强光经过温
度比较低的太阳大气层时产生的吸收光谱。仔细分析这些暗
线,把它跟各种原子的特征谱线对照,人们就知道了太阳大气
层中含有氢、氦、氮、碳、氧、铁、镁、硅、钙、钠等几十种元素。

\section*{复习题}
\begin{enumerate}
\item 十七世纪的光的波动说和微粒说各有什么成功之处
和不足之处?
\item 什么是光的干涉?什么是相干光源?
\item 薄膜干涉现象是怎样产生的?举例说明薄膜干涉现
象的应用。
\item 什么是光的行射?日常生活中为什么不容易观察到
光的衍射现象?
\item 什么是光的偏振?光的偏振表明了什么?
\item 确立光的电磁说,根据是什么?
\item 无线电波、红外线、可见光、紫外线和伦琴射线,它们
的波长(或频率)有什么不同?
\item 发射光谱是怎样产生的?吸收光谱是怎样产生的?什
么叫特征谱线?光谱分析的原理是什么?
\end{enumerate}

\section*{习题}

\begin{enumerate}
    \item 你用哪些现象或实验来说明:
    \begin{enumerate}
        \item 光是一种波;
        \item 光波的波长非常短;
        \item 绿光的波长比红光的短。
    \end{enumerate}
    \item 水对真空中波长为0.656微米的红光的折射率为$n_1
    =1.33$,而对真空中波长为0.405微米的紫光的折射率为
    $n_2=1.343$。求这两种波在水中的传播速度和波长.
    \item 波长为5890埃的黄光照在一双缝上,在距双缝为1
    米的观察屏上,测得20个亮条纹的间距共宽2.4厘米,求双
    缝间的距离。
    \item 用波长为0.75微米的红光做双缝于涉实验,双缝间
    的距离是0.05毫米,缝到屏的距离是1米,相邻两条暗纹间
    的距离是多大?
\end{enumerate}









\chapter{光的粒子性}
光的电磁说使光的波动理论发展到相当完美的地步,取
得了巨大成就.但是,这个学说并不能完美地解释所有的光
现象,还在赫兹用实验证实光的电磁说的时候,就已经发现了
后来叫做光电效应的现象,这个现象使光的电磁说遇到了无
法克服的困难.二十世纪初,人们在新的事实的基础上建立了
关于光的新的学说——光子说,对光的本性的认识又前进了
一步.


\section{光电效应}
1887年,赫兹在进行电磁波的实验时发现了一个奇妙的
现象:当紫外线照射到圆环接收器间隙的电极上时,火花放电
变得容易了,后来其他科学家又发现,用紫外线照射与验电
器相连的带负电的锌板时,验电器的金箔便立即闭合.用紫
外线照射与验电器相连的不带电的锌板时,验电器上带正电.
通过这些现象人们认识到,用紫外线照射金属时,电子会从金
属表面飞出来.在紫外线照射下,赫兹实验中的电极变得容
易放电,带负电的锌板失去电荷,不带电的锌板带上了正电,
都是由子电子从金属表面飞出来了的缘故.
\begin{figure}[htp]\centering
    \includegraphics[scale=.6]{fig/7-1.png}
    \caption{紫外线照射锌板使验电器带正电}
    \end{figure}

在光的照射下物体中发射电子的现象,叫做\textbf{光电效应}.光
电效应中发射出来的电子叫做\textbf{光电子}.实验表明,不仅紫外
线能产生光电效应,对于碱金属,例如锂、钠、钾、铯等,用可见
光照射也能产生光电效应.
\begin{figure}[htp]\centering
    \includegraphics[scale=1.2]{fig/7-2.pdf}
    \caption{研究光电效应的装置简图}
    \end{figure}

图7.2是研究光电效应规律的实验装置简图,其中$S$是
抽成真空的容器,$C$是石英窗口,紫外光和可见光都可以通过
它射到容器里的金属板$K$上.在$K$的对面有另一金属板$A$,$K$
和$A$组成一对电极.象图中那样把$K$跟电池组的负极相连,
$A$正极相连,电路接好后,电流表指示没有电流,因为极板
$K$和$A$之间是断开的,用一定强度的光照射极板$K$,它发射
出的光电子运动到极板$A$时,电流表的指针就发生偏转,指出
电路中产生了电流,这电流是由光电子产生的,叫做光电流.

逐渐增加极板$K$和$A$间的正向电压(使$A$板电势比$K$板
高),电路里的光电流也逐渐增大.当正向电压增到几伏时,
光电流就达到最大值——饱和值.再增加电压,光电流也不
再增大了,这表明$K$板发射的光电子已全部被$A$板吸去.这
时增加入射光的强度,光电流值就继续上升.入射光的强度
增加一倍,光电流的饱和值也增加一倍,这表明,\textit{在单位时间
里从极板$K$发射出的光电子数跟入射光的强度成正比}.

逐渐减小极板$K$和$A$间的正向电压,当这电压减小为零
时,电路里仍有相当强的光电流.可见光电子从极板$K$飞出
时具有某种初动能,没有正向电场的作用,也有相当数量的光
电子到达极板$A$.要完全阻止光电流,必须在$KA$间加一定
的反向电压(使$A$板电势低于$K$板电势),使那些具有最大初
动能的光电子不足以克服反向电场的阻力而到达极板$A$,就
没有光电流了.这时,光电子的最大初动能
$\dfrac{1}{2}mv^2_m$
与反向电压$U$
之间有下面的关系:
\[\dfrac{1}{2}mv^2_m=eU \]
式中的$e$表示电子的电量.
因此,从实验中测出使光电流减小到零时的反向电压$U$的值,
就可以求出光电子的最大初动能.实验表明,\textit{光电子的最大初
动能与入射光的强度无关,只随着入射光频率的增大而增大}.

换用不同材料的极板$K$重做实验,结果表明,光电子的最
大初动能还跟极板$K$的材料有关.\textit{对于任何一种金属,入射光
的频率必须大于某一极限频率才能产生光电效应;低于这个
极限频率的光,无论强度如何,照射时间多长,也不能产生光
电效应}.
下表是几种金属的极限频率$\nu_0$和波长$\lambda_0$.
\begin{center}
    \begin{tabular}{cccccc}
        \hline
        金属& $\lambda_0$(微米)& $\nu_0$(赫)\\
        \hline
        铯&0.66&$4.545\x 10^{14}$ \\
        钠&0.50&$6.000\x 10^{14}$\\
        锌&0.37&$8.065\x 10^{14}$\\
        银&0.26&$1.153\x 10^{15}$\\
        铂& 0.20&$1.529\x 10^{15}$\\
        \hline
    \end{tabular}
\end{center}



\section{爱因斯坦对光电效应的解释}
\subsection{光子}
光电效应的规律无法用经典的波动理论来解释.按照波
动理论,光的能量是由光的强度决定的,而光的强度又是由光
波的振幅决定的,跟频率无关.因此,不论光的频率如何,只
要光的强度足够大或照射时间足够长,都应该有足够的能量
产生光电效应,然而这跟实验结果是直接矛盾的.极限频率
的存在,即频率低于某一数值的光不论强度如何都不能产生
光电效应,这是波动理论不能解释的.

对光电效应的解释是1905年爱因斯坦(1879—1955)在
普朗克(1858—1947)的量子论的基础上作出的.1900年德国
物理学家普朗克在研究电磁辐射的规律时发现,只有假设电
磁辐射的能量是不连续的,而是一份一份地进行的,每一份的
能量是$h\nu$,其中$\nu$是辐射的频率,是一个普适恒量,理论计
算的结果才能跟实验事实很好地符合.这个$h$后来就叫做\textbf{普
朗克恒量},实验测出$h=6.63\x10^{-34}{\rm J}\cdot {\rm s}$,在普朗克的量
子说的启发下,爱因斯坦天才地预见到,为了解释光电效应必
须假设光也是不连续的,而是一份一份的,每一份光叫做一个
\textbf{光子},光子的能量跟它的频率成正比,即$E=h\nu$,其中的$h$是
普朗克恒量.这就是爱因斯坦在1905年提出的光子说.爱
因斯坦是在实验事实还不很充分的情况下做出这一假设的,
但是从这个假设得出的一切结论都跟后来的实验结果相符.

光子说很好地解释了光电效应,当光子照射到金属上
时,它的能量可以被属中的某个电子全部吸收,电子吸收
光子的能量后,动能就增加了,如果电子的动能足够大,能够克
服内部原子对它的引力,就可以离开金属表面逃逸出来,成为
光电子,这就是光电效应,电子吸收光子的能量后可能向各
个方向运动,有的向金属内部运动,并不出来,向金属表面运
动的电子,经过的路程不同,途中损失的能量也不同,因此从
表面出来时的初动能也不同,只有直接从金属表面出来的光
电子才具有最大初动能,这些光电子克服金属原子的引力
所做的功叫做\textbf{逸出功},根据能量守恒定律,光电子的最大初
动能跟入射光子的能量$h\nu$和逸出功$W$之间有下面的关系:
\[\frac{1}{2}mv^2_m=h\nu-W \]
这个方程叫做爱因斯坦\textbf{光电效应方程}.

对于一定的金属来说,逸出功$W$的值是一定的,所以,入
射光子的频率$\nu$越大,光电子的最大初动能也越大,如果入
射光子的频率比较低,它的能量小于金属的逸出功,就不能产
生光电效应了,这就是存在极限频率的原因.极限频率$\nu_0$可
由下式求出:
\[h\nu_0=W\quad \Rightarrow\quad \nu_0=\frac{W}{h} \]

不同金属的逸出功不同,所以它们的极限频率也不同,如
果入射光比较强,即单位时间内入射光子的数目多,产生的光
电子也多,所以光电流的饱和值也大.

\subsection*{练习一}

\begin{enumerate}
    \item 使锌板产生光电效应的光子的最长波长是0.37微
米,这种光子的能量是多少电子伏?锌的逸出功是多少?
\item 可见光的光子,能量范围是多大(用电子伏表示)?为
什么用可见光不能使锌板产生光电效应?
\item 铯的逸出功是$3.0\x10^{-19}$焦,用波长是0.59微米的
黄光照射铯,电子从绝表面飞出的最大初动能是多大?
\item 钨的逸出功是4.52电子伏,使钨产生光电效应的最
长波长是多少?这种波长是可见光吗?
\end{enumerate}

\section{光电效应的应用}

利用光电效应可以把光信号转变为电信号,动作迅速灵
敏,因此利用光电效应制作的光电器件在工农业生产,科学技
术和文化生活领域内得到了广泛的应用,光电管就是应用最
普遍的一种光电器件.

光电管的类型很多.图7.3甲是其中的一种,玻璃泡里
的空气已经抽出,有的管里充有少量的惰性气体(如氩、氖、氦
等),管的内半壁涂有逸出功小的碱金属作为阴极$K$,管内
另有一阳极$A$.使用时照图7.3乙那样把它连在电路里,当
光照射到光电管的阴极$K$时,阴极发射电子,电路里就产生电
流,光电管不能受强光照射,否则容易老化失效.光电管产生
的电流很弱,应用时可以用放大器把它放大.
\begin{figure}[htp]\centering
    \includegraphics[scale=.8]{fig/7-3.png}
    \caption{光电管}
    \end{figure}

下面举例说明光电管的应用.

\subsection{光控继电器}

工业生产中的大部分光电控制设备都用光
控继电器.图7.4是光控继电器的示意图,它由电源、光电
管、放大器、电磁继电器几部分组成.当光照射光电管时,光
放大器电管电路中便产生电流,经放大器放大后,使电磁铁$M$磁化,
把衔铁$N$吸住;没有光照射光电管时,电路中没有电流,衔铁
$N$在弹簧的作用下就自动离开$M$.如果把衔铁$N$跟控制机构
相连,就可以达到自动控制的目的.
\begin{figure}[htp]\centering
    \includegraphics[scale=1.2]{fig/7-4.PDF}
    \caption{光控继电器示意图}
    \end{figure}


光控继电器在工业上可以用于产品的自动计数、安全生
产等方面,用于自动计数时,可以把产品放在传送带上,光源
和光电管分别放在传送带的两侧,每当传送带上输送过去一
个产品时,光线被挡住一次,光控继电器就放开衔铁一次,由
衔铁控制的计数器的数字就加一.工人在冲床、钻床、锻压机
械上劳动时,如有不慎,容易出事故.为保证安全,可以在这些
机床上安装光控继电器,当工人不慎将手伸人危险部位时,
由于遮住了光线,光控继电器就立即动作,使机床停下来,避
免事故的发生.


\subsection{有声电影}

最早的电影是没有声音的,后来虽然有了声
音,但那是靠留声机来配合影片播放的,声和影配合不好时,
效果当然不好,我们现在能够看到声和影完全配合一致的有
声电影,还是多亏了光电管.

影片摄制完后,要进行录音,录音时通过专门的设备使声
音的变化转变成光的变化,从而把声音的“像”摄制在影片的
边缘上,形成宽窄变化的暗条纹,这就是影片边上的\textbf{音道}.

放映电影时,利用光电管把“声音的照片”还原成声音.方
法是:在电影放映机中用强度不变的极窄的光束照射音道,
由于影片上各处的音道宽窄不同,所以在影片移动的过程中,
通过音道的光的强度也就不断变化;变化的光射向光电管时,
在电路中产生变化的电流,把电流放大后,通过喇叭就可以把
声音放出来.

\section{光的波粒二象性}
光的干涉、衍射和偏振等现象无可争辩地表明光具有波
动性,而光电效应又无可争辩地表明光是具有能量$E=h\nu$的
光子流,也就是说光具有粒子性.这样,已经退出历史舞台的
光的微粒说,在二十世纪初又以新的形式被重新提了出来.当
然人们现在对光的粒子性的认识比起十七世纪牛顿提出微粒
说时已经大不相同.人类对光的本性的认识经过曲折的发展
过程已经越来越深入了.现在,人们认识到,光既具有波动
性,又具有粒子性,也就是说,\textit{光具有波粒二象性}.

十七世纪的微粒说和波动说是互相对立的两种学说,都
企图用一种观点去说明光的本性,这是受了传统观念的影响.
传统观念是我们在观察周围世界的宏观现象中形成的,波动
性和粒子性在宏观现象中是互相对立的、矛盾的,没有任何宏
观物体既有波动性、又有粒子性.对于宏观物体来说,波粒二
象性是不可想像的.

但是,对于光子这样的微观粒子,却只有从波粒二象性出
发,才能说明它的各种行为.实际上,光子说并没有否定光的
电磁说,光子的能量$E=h\nu$,其中的频率$\nu$表示的仍是波的特
征.此外,从光子说和电磁说还往往得到一致的结论.例如,
光子说和电磁说都可以推导出光具有动量,并且为实验所证
实,光子说的结论是光子的动量$p=h\nu/c$,
电磁说的结论是辐射
能$E$具有的动量是$p=E/c$.由于光子的能量$E=h\nu$,所以从
这两个学说得到的结论是一致的,由于$c=\lambda\nu$,光子的动量
也可以写成
$$p=\frac{h}{\lambda}$$
式中的波长$\lambda$表示的也是波的特征.可
见,对于宏观物体来说不可想像的波粒二象性,在微观世界却
是不可避免地必须予以承认的现实,接受光的波粒二象性,
就要求我们既不可把光当成宏观观念中的波,也不可把光当
成宏观观念中的粒子.

那么,在微观世界中,波和粒子又是怎样统一起来的呢?
物理学家做的下述实验可以帮助我们理解这个问题.在光的
双缝干涉实验中,在像屏处放上照相底片,并设法减弱光流的
强度,由于每个光子的能量$h\nu$可以从频率$\nu$算出,因此进一
步从光流的能量可以算出所含光子的数目.这样就可以使光
流减弱到使光子只能一个一个地通过狭缝,实验结果表明,
如果曝光时间不太长,底片上只出现一些无规则分布的点子,
那些点子是光子打在底片上形成的,表现出光的粒子性,这
些点子的分布是无规则的,可见光子的运动跟我们在研究宏
观现象时假设的质点的运动不同,没有一定的轨道,如果嘲光
时间足够长,底片上就出现了规则的干涉条纹,就象用强光
经短时间曝光后产生的一样,可见,光的波动性是大量光子
表现出来的现象.在干涉条纹中,那些光波强度大的地方,也
就是光子到达机会多的地方,或者说,是光子到达的几率大的
地方;光波强度小的地方,是光子到达的几率小的地方,所以
从这种意义上,可以把光的波动性看做是表明大量光子运动
规律的一种\textbf{几率波}.

一般说来,大量光子产生的效果往往显示出波动性,个别
光子产生的效果往往显示出粒子性,让我们稍稍详细地说明
一下.

无线电波的频率较低,波长较长,这种电磁波的“光子”的
能量很低,以频率为1兆赫的无线电波来说,它的“光子”的能
量只有$4\x10^{-9}$电子伏,能量这样低,只有非常大量的“光
子”才能使接收装置发生反应.较好的接收机大约要每秒收
到$10^{10}$个这样的“光子”才起作用,所以,这部分电磁波的波
动性很容易观察到,要观察这部分电磁波的粒子性,觉察个别
“光子”的作用,却是非常不容易的.

可见光的频率范围大致是$4\x10^{14}$—$8\x10^{14}$赫,这种光
子的能量大约是几个电子伏.人造的仪器设备既可以比较容
易地探测到大量的这种光子的作用,也可以比较容易地探测
到少数这种光子的作用,因此这种电磁波的波、动性和粒子性
都能够比较容易地观察到.

随着电磁波频率的增大,波长越来越短,波动性就越来越
不显著,而粒子性却越来越明显了.伦琴射线的光子的能量
大约是几千电子伏,$\gamma$射线的光子的能量在几兆电子伏以上.
个别$\gamma$射线的光子很容易探测出来,而要看到它们的干涉、衍
射现象却很困难了.伦琴射线只有用晶体作衍射光栅才能看
到衍射图样,因为晶体内粒子间的距离恰好是$10^{-10}$
米左右.至于$\gamma$射线,连用晶体作衍射光栅也不行,因为晶体里粒子间
的距离,比它的波长大得不可比拟.

总之,要理解各种频率的电磁波,我们必须综合运用波动
的观点和粒子的观点,而且要注意到,这里的波动并不等同
于宏观世界里的机械波,这里的粒子也不等同于宏观世界里
的质点.

\section{物质波}
在光具有波粒二象性的启发下,法国物理学家德布罗意
(1892——1987)在1924年提出一个假说,指出波粒二象性不只是光
子才有,一切微观粒子,包括电子和质子、中子,都有波粒二象
性.他把光子的动量与波长的关系式$p=h/\lambda$
推广到一切微观
粒子上,指出:具有质量$m$和速度$v$的运动粒子也具有波动
性,这种波的波长等于普朗克恒量$h$跟粒子动量$mv$的比,即
\[\lambda=\frac{h}{mv} \]
这个关系式后来就叫做德布罗意公式.

从德布罗意公式很容易算出运动粒子的波长.例如,电
子的电荷是$1.6\x10^{-19}$
库,质量是$0.91\x10^{-30}$千克,经过200
伏电势差加速的电子获得的能量
\[E=Ue=200\x1.6\x10^{-19}=3.2\x10^{-17}{\rm J}\]
这个能量就是电子的动能,即
\[\dfrac{1}{2}mv^2=3.2\x10^{-17}{\rm J} \]
因此
\[v=\sqrt{\frac{2\x 3.2\x10^{-17}}{0.91\x 10^{-30}}}=8.4\x10^6\ms \]
于是,按照德布罗意公式这运动电子的波长是
\[\begin{split}
    \lambda=\dfrac{h}{mv}&=\frac{6.63\x 10^{-34}}{0.91\x10^{-30}\x8.4\x 10^6}{\rm m} \\
&=8.7\x10^{-11}{\rm m}=0.87\text{\AA}
\end{split}\]
我们看到,这个波长与伦琴射线的波长相仿.前面讲过,这样
短的波长,只有用晶体做衍射光栅才能观察到衍射现象.后
来人们的确用这种办法观察到了电子的衍射,从而证明了德
布罗意假说的正确性.
\begin{figure}[htp]\centering
    \includegraphics[scale=.8]{fig/7-5.png}
    \caption{电子衍射实验示意图}
    \end{figure}

图7.5是一种电子衍射实验的示意图,从灯丝$K$发射出
的电子经过电势差为$U$的加速电场,然后通过一组栏板$D$的
小孔,成为很细的电子束射到铝箔$M$上,在铝箔的后面放一
张照相底片$P$,于是就得到图7.6所示的照片,在中央斑点
的周围出现了环形的明暗相间的花纹.这个衍射图样跟伦琴
射线穿过同一铝箔后产生的衍射图样(图7.7)非常相似,这
是电子具有波动性的证明.根据铝箔原子间的距离、加速电
势差和衍射条纹的几何图形,还可以算出电子衍射时的波长,
实测结果跟德布罗意公式相符.
\begin{figure}[htp]
    \centering
    \begin{minipage}[t]{0.48\textwidth}
    \centering
    \includegraphics[scale=.8]{fig/7-6.jpg}
    \caption{电子衍射图样}
    \end{minipage}
    \begin{minipage}[t]{0.48\textwidth}
    \centering
    \includegraphics[scale=.8]{fig/7-7.jpg}
    \caption{伦琴射线衍射图样}
    \end{minipage}
    \end{figure}


后来人们又用原子射线和分子射线做类似的实验,同样
得到了衍射图样,质子和中子的衍射实验也做成功了,这就
证明了一切运动的微观粒子都具有波粒二象性,其波长与动
量的关系都符合德布罗意公式,于是人们就把这种波叫做德
布罗意波或\textbf{物质波}.

那么,物质波是一种什么波呢?我们知道,机械波是周期
性的振动在媒质内的传播,电磁波是周期变化的电磁场的传
播,物质波既不是机械波,也不是电磁波,在德布罗意提出
物质波以后,人们曾经对它提出过各种各样的解释,到1926
年,德国物理学家玻恩(1882—1970)提出了符合实验事实的
后来为大家公认的统计解释:物质波在某一地方的强度跟在
该处找到它所代表的粒子的几率成正比,按玻恩的解释,\textit{物
质波乃是一种几率波}.

牛顿力学完全不能解释电子等微观粒子的衍射现象,用
物质波的统计解释却很容易说明这种现象,在图7.5的实验
里,电子流通过金属箔片以后,物质波发生衍射,有的地方由
于波的叠加而使物质波的强度增大,电子到达这里的几率就
大,因而到达这里的电子数较多,有的地方由于波的叠加而
使物质波的强度减小甚至等于零,电子到达这里的几率就较
小甚至等于零,因面到达这里的电子数很少甚至没有.

发现了电子、质子等微观粒子的波动性以后,我们对微观
世界的认识统一起来了,不但原来认为是电磁波的光具有粒
子性,而且原来认为是粒子的电子、质子等也具有波动性.当
然,应该指出,虽然所有的微观粒子都具有波粒二象性,但光
子跟电子、质子等粒子还是有很基本的区别的,光子没有静
质量,电子、质子等都有静质量.光子的运动速度永远是$c$,
电子、质子等却可以有低于光速$c$的各种不同的运动速度.

\section*{复习题}
\begin{enumerate}
\item 什么是光电效应?光电效应有哪些重要的规律?这些
规律中哪些是波动说无法解释的?
\item 什么是光子说?光子说是怎样解释光电效应的?写出
光电效应方程
\item 说明光电管的结构和原理,并举例说明它的应用.
\item 为什么说光具有波粒二象性?应该怎样理解光的波
粒二象性?
\item 德布罗意提出的物质波的含义是什么?什么实验证
明了德布罗意物质波假设的正确性?
\end{enumerate}

\section*{习题}
\begin{enumerate}
    \item 功率为1瓦的手电筒灯泡大约有5\%的电能转化为
可见光,试估算它1秒钟能释放出多少个可见光的光子.
\item 使铜产生光电效应的最低频率是$1.1\x10^{15}$
赫,用频率为$1.5\x10^{15}$赫的紫外线照射铜时,它发射出的光电子的
最大速度是多大?
\item 一个质量为$4\x10^{-4}$克的尘埃颗粒,以$1{\rm cm}/{\rm s}$
的速度在空气中下落,计算它的德布罗意波长.
\item 计算速度为$10^3\ms$的中子的德布罗意波长.这
个波长跟$\gamma$射线的波长相比如何?中子的质量是$1.67\x10^{-27}$
kg.
\end{enumerate}























\chapter{原子结构}

一百多年前,化学家们从实验中知道,物质是由分子组
成的,分子是由原子组成的,由于在化学反应中原子的种类和
数目不变,于是形成了原子是组成物质的最小微粒的概念.直
到十九世纪末,人们一直认为原子是不可再分的,但是随着物
理研究的深入和实验技术的提高,十九世纪末发现了一些新
的事实,证明原子是由更基本的微粒组成的.二十世纪以来,
物理学家们对原子和原子核的研究日益深人,建立了原子物
理和核物理的科学理论,在实际应用原子能方面也取得了很
大的进展,这一章和下一章我们就来学习原子结构和原子核
的初步知识.


\section{电子的发现}
\subsection{电子的发现}


原子不是不可再分的,而是由更小的微粒
组成的,这一认识是从发现电子开始的.

十九世纪后半期,科学家们在研究稀薄气体的放电时发
现,当玻璃管内的气体足够稀薄时,阴极就发出一种射线,这
种射线能使对着阴极的玻璃管壁发出荧光,叫做阴极射线.

英国科学家汤姆生(1856—1940)对阴极射线进行了一系
列的实验研究.1897年,他确认阴极射线是带负电的粒子.
汤姆生研究了阴极射线在电场和磁场中的偏转,根据测得的
数据计算出了这种带电粒子的荷质比$e/m$.
这种测定荷质比的
原理,我们在第一章中已经讲过,汤姆生发现,不同物质做成
的阴极发出的射线都有相同的
$e/m$
值.这表明不同物质都能发
射这种带电粒子,它是构成各种物质的共有成分.

汤姆生测得的阴极射线粒子的荷质比,大约是当时已知
的氢离子的荷质比的二千倍,汤姆生认为,这可能是由于阴
极射线粒子的电荷$e$很大,或者是它的质量$m$很小.后来汤
姆生测量了氢离子和阴极射线粒子的电荷,虽然测量不很准
确,但是足以证明阴极射线粒子的电荷与氢离子的电荷大小
基本上是相同的.由此得出结论,阴极射线粒子的质量比氢离
子的质量小得多:后来人们逐渐把这种粒子叫做电子.汤姆
生对证实电子的存在有很大的功劳,因而公认他是电子的发
现者.以后,美国科学家密立根又精确地测定了电子的电量,
这样由电子的荷质比和电量就可以算出电子的质量.

氢原子是当时已知的质量最小的原子,由电子的质量比
氢离子的质量小得多,汤姆生认为,电子可能是组成原子的基
本部分.

汤姆生发现电子,是物理学史上的重要事件.由于电子
的发现,人们认识到原子不是组成物质的最小微粒,原子本身
也具有结构,此后,围绕着原子结构的问题,原子物理以飞跃
的速度发展,人们对物质结构的认识进入了一个新时代.

\subsection{汤姆生的原子模型}

我们知道,物质在通常情况下是不带电的,因此,原子应该是电中性的,而电子是带负电的,如
果电子是原子的组成部分,那么原子里一定还有带正电的部
分.电子的质量很小,所以原子的质量应该主要集中在带正
电的部分.原子中带正电的部分和带负电的电子是怎样组成
原子的呢?这是物理学家们关心的一个问题.

在二十世纪的前十年里,科学家们提出了几种原子模型,
其中最有影响的是汤姆生提出的原子模型.在这个模型里,
原子被认为是一个球体,正电荷均匀分布在整个球内,电子则
象枣糕里的枣子那样镶嵌在球
内(图8.1),原子受到激发以后,电子开始振动发光,产生原子光谱.汤姆生的原子模型能
够解释一些实验事实,但是没过几年就被卢瑟福发现的新的实验事实否定了.
\begin{figure}[htp]
    \centering
    \includegraphics{fig/8-1.pdf}
    \caption{汤姆生的原子模型}
\end{figure}

\section{原子的核式结构的发现}
\subsection{$\alpha$粒子的散射和卢瑟福的原子核式结构模型} 

为了探测
原子内电荷的分布,在二十世纪的头十年里,发展了一种实
验方法:用各种粒子——X射线、电子和$\alpha$粒子轰击很薄的物
质层,通过观察这些粒子穿过物质层后的偏转情况,获得原
子结构的信息.这种实验叫做散射实验.那时候人们已经知
道,$\alpha$粒子是一种带正电荷的重粒子,它的电量是电子电量的
二倍,它的质量大约是电子质量的7300倍.某些放射性元素
放出的$\alpha$粒子具有很大的动能,可以当作轰击粒子,1909年
到1911年,在英国物理学家卢瑟福(1871—1937)指导下,
他的合作者们做了用$\alpha$粒子轰击金箔的实验,获得了重要的
发现.

实验的做法如下:在一个小铅盒里放有少量的放射性元
素钋,它发出的$\alpha$粒子从铅盒的小孔射出,形成很细的一束射
线射到金箔上.$\alpha$粒子穿过金箔后,打到荧光屏上产生一个
个的闪光,这些闪光可以用显微镜观察到.整个装置放在一
个抽成真空的容器里,荧光屏和显微镜能够围绕金箔在一个
圆周上转动(图8.2),从而可以观察到穿过金箔后偏转角度
不同的$\alpha$粒子.
\begin{figure}[htp]
    \centering
    \includegraphics{fig/8-2.pdf}
    \caption{$\alpha$粒子散射实验装置示意图}
\end{figure}

实验表明,绝大多数$\alpha$粒子穿过金箔后仍沿原来的方向
前进,但是有少数$\alpha$粒子却发生了较大的偏转,并且有极少数
$\alpha$粒子的偏转超过了$90^{\circ}$,有的甚至几乎达到$180^{\circ}$,象是被金
箔弹了回来,这就是$\alpha$粒子的散射实验.

$\alpha$粒子的大角度散射现象,是出人预料的.因为根据汤
姆生原子模型的计算,$\alpha$粒子穿过金箔后的偏转最大不超过
零点几度,这是因为电子的质量很小,比$\alpha$粒子的质量小得
多,$\alpha$粒子碰到金箔原子内的电子,就象飞行的子弹碰到尘埃
一样,运动方向不会发生明显的改变,正电荷在原子内又是
均匀分布的,x粒子穿过原子时,它受到两侧正电荷的斥力有
相当大一部分互相抵消,因而使$\alpha$粒子偏转的力不会很大.

卢瑟福对$\alpha$粒子散射实验的结果进行了分析,得出结论,
除非原子的几乎全部质量和正电荷都集中在原子中心的一个
很小的核上,否则,$\alpha$粒子的大角度散射是不可能的.由此,
卢瑟福提出了他的原子核式结构模型:\textit{在原子的中心有一个
很小的核,叫做原子核,原子的全部正电荷和几乎全部质量都
集中在原子核里,带负电的电子在核外空间里绕着核旋转}.

按照这个学说,$\alpha$粒子穿过原子时,电子对$\alpha$粒子运动的
影响很小,影响$\alpha$粒子运动的主要是原子核.如果离核较远,
受到的库仑斥力就很小,运动方向也就改变很小.只有当$\alpha$粒
子与核十分接近时,才会受到很大的库仑斥力,发生大角度的
偏转(图8.3).由于原子核很小,$\alpha$粒子十分接近它的机会
很少,所以绝大多数$\alpha$粒子基本上仍按直线方向前进,只有极
少数发生大角度的偏转.
\begin{figure}[htp]
    \centering
    \includegraphics{fig/8-3.pdf}
    \caption{$\alpha$粒子十分接近原子核时才会发生大角度偏转}
\end{figure}


\subsection{原子核的电荷和大小} 

根据卢瑟福的原子核式结构模
型,可以推导出$\alpha$粒子的散射公式,利用这个公式和不同元素
对$\alpha$粒子散射的实验数据,可以计算出各种元素原子核的电
荷$Q$.当时测得碳的电荷$Q=6e$,铝的$Q=13e$或$14e$,金的
$Q=78e$或$79e$,其中$e$是基本电荷.原子核的电荷是研究原
子的重要资料,因为知道了某一元素原子核的电荷数,根据原
子是电中性的,就可以推算出这种原子内含有的电子数.当时
人们还注意到,测出的各种元素的原子核的电荷数,或者说元
素原子内的电子数,非常接近于它们的原子序数.这为深入
理解元素周期表提供了重要的线索,它说明元素周期表是按
原子中的电子数来排列元素的.

从$\alpha$粒子散射实验的数据还可以估计出原子核的大小.
因为知道了$\alpha$粒子的初动能和核电荷,就可以计算出$\alpha$粒子
跟原子核发生对心碰撞时可能达到的离原子核的最小距离
(这时$\alpha$粒子的动能完全转化为核电场中的电势能).这个最
小距离可以达到$10^{-14}$米.可见原子核的大小应在$10^{-14}$米
以下,原子半径大约是$10^{-10}$米,所以原子核的半径只相当于
原子半径的万分之一,原子核的体积只相当于原子体积的万
亿分之一.如果把原子比作直径等于100米的大球,那么原
子核只相当于直径为1厘米的一个小石子.原子核虽小,却
集中着原子的几乎全部质量,因此原子核的密度非常大,如
果1立方厘米的体积中完全充满原子核,那么它的质量将有
$10^7$吨!

\subsection*{练习一}
\begin{enumerate}
    \item $\alpha$粒子被原子散射的原因是什么?
    \item 卢瑟福的原子模型与汤姆生的原子模型,主要区别
    是什么?
    \item $\alpha$粒子的质量大约是电子质量的7300倍.如果$\alpha$
    粒子以速度$v$跟电子发生弹性正碰(假定电子原来是静止
    的),求碰撞后$\alpha$粒子的速度变化了多少,并由此说明,为什
    么原子中的电子不能使$\alpha$粒子发生明显的偏转.
    \item 已知氢原子的半径是$0.53\x10^{-10}$米,电子不致被吸
    引到核上,按照卢瑟福的原子模型,电子绕核做匀速圆周运动
    的速度和频率各是多大?
    \item 为什么在计算电子与核之间的引力作用时,可以不
    考虑万有引力?

\end{enumerate}

\section{玻尔的原子理论}
卢瑟福的原子核式结构学说很好地解释了$\alpha$粒子的散射
实验,初步建立了原子结构的正确图景,但跟经典的电磁理论
发生了矛盾.原来,电子没有被库仑力吸引到核上,它一定是
以很大的速度绕核运动,这种绕核运动是有加速度的.按照
经典电磁理论,做加速运动的电荷应该辐射出电磁波,因此它
的能量要逐渐减少.随着能量的减少,电子绕核运行的轨道
半径也要减小,于是电子将沿着螺旋线的轨道落入原子核,就
象绕地球运行的人造卫星受到上层大气阻力不断损失能量后
要落到地面上一样.这样看来,原子应当是不稳定的,然而实
际上并不是这样.同时,按照经典电磁理论,电子绕核运行时
辐射电磁波的频率应该等于电子绕核运行的频率,随着运行
轨道半径的不断变化,电子绕核运行的频率要不断变化,因此
原子辐射电磁波的频率也要不断变化.这样,大量原子发光
的光谱就应该是包含一切频率的连续光谱,然而实际上原子
光谱是由一些不连续的亮线组成的明线光谱.

这些矛盾表明,从宏观现象总结出来的经典电磁理论不
适用于原子这样小的物体产生的微观现象.

\subsection{玻尔的原子理论} 

为了解决上述矛盾,1913年丹麦青年
物理学家玻尔(1885—1962)在卢瑟福学说的基础上,把普朗
克的量子理论运用到原子系统上,提出了新的原子理论,在
原子物理的研究上迈出了重要的一步.玻尔原子理论的主要
内容是如下的假设:

一、原子只能处于一系列不连续的能量状态中,在这些
状态中原子是稳定的,电子虽然做加速运动,但并不向外辐射
能量、这些状态叫做\textbf{定态}.

二、原子从一种定态(设能量为$E_2$)跃迁到另一种定态
(设能量为$E_1$)时,它辐射或吸收一定频率的光子,光子的能
量由这两种定态的能量差决定,即
\[h\nu=E_2-E_1\]

三、原子的不同能量状态对应于电子的不同运行轨道.
由于原子的能量状态是不连续的,因此电子的可能轨道也是
不连续的,即电子不能在任意半径的轨道上运行.只有满足下
列条件的轨道才是可能的:轨道半径$r$跟电子的动量$mv$的
乘积等于 $h/2\pi$的整数倍,即
\[mvr=n\frac{h}{2\pi},\qquad n=1,2,3,\ldots\]
式中的$n$是正整数,叫做\textbf{量子数}.这种现象叫做\textbf{轨道的量
子化}.

玻尔把量子观念引入原子理论中,这是一个创举.根据
玻尔的假设,电子只能在某些可能的轨道上运动,电子在这些
道上运动时不辐射能量,处于定态,只有电子从一条轨道跃
迁到另一条轨道上时才辐射能量,辐射的能量是一份一份的,
等于这两个定态的能量差.这些就是玻尔理论的主要内容.

\subsection{氢原子的大小和能级}

玻尔在上述假设的基础上,利用
经典电磁理论和牛顿力学,对结构最简单的氢原子(只有一个
电子)进行了计算,算出了氢的电子的各条可能轨道的半径和
电子在各条轨道上运动时的能量(包括动能和电势能).玻尔
的计算结果可概括为两个公式:
\[\begin{cases}
    r_n=n^2 r_1\\
    E_n=\dfrac{1}{n^2}E_1
\end{cases}\qquad n=1,2,3,\ldots\]
式中的$r_1$代表第一条(即离核最近的一条)可能轨道的半径,
$E_1$代表电子在第一条轨道上运动时的能量,$r_n$、$E_n$分别代表
第$n$条可能轨道的半径和电子在第$n$条轨道上运动时的能
量,$n$是量子数.玻尔计算出$r_1$的值为$0.53\x10^{-10}$米,与过
去用其他方法计算出的氢原子的半径非常符合,玻尔计算出
$E_1$的值为$-13.6$电子伏.(计算中取离核无限远处的电势能为
零,电子带负电,在正电荷的场中电势能为负值;电子的动能
等于电势能绝对值的一半,所以总能量为负值.)

氢原子的各个定态的能量值,叫做它的\textbf{能级}.上面的计
算$E_n$的式子就是氢原子的能级公式,已知$E_1=-13.6$电子
伏,把$E_1$的值代入能级公式中,可以计算出氢原子各能级的
值:$n=2$时,$E_2=-3.4$电子伏;$n=3$时,$E_3=-1.51$电子伏;
$n=4$时,$E_4=-0.85$电子伏……可以看出,氢原子各个能级
的能量是不连续的,这种现象叫做\textbf{能量的量子化}.

在正常状态下,原子处于最低能级,这时电子在离核最近
的轨道上运动,这种定态叫做基态,给物体加热或有光照射
物体时,物体中的某些原子能够从相互碰撞或从人射光子中
吸收一定的能量,从基态跃迁到较高能级,这时电子在离核较
远的轨道上运动,这些定态叫做\textbf{激发态}.原子从基态向激发
态跃迁的过程,是吸收能量的过程,原子从较高的激发态面
较低的激发态或基态跃迁的过程,是辐射能量的过程,这个能
量以光子的形式辐射出去,这就是原子发光现象.原子无论
吸收能量或辐射能量,这个能量都不是任意的,而是等于原子
发生跃迁的两个能级间的能量差.


\subsection*{练习二}

\begin{enumerate}
    \item 利用公式$r_n=n^2r_1$和$E_n=E_1/n^2$,
计算氢原子的第2、3、4轨道的半径和电子在这些轨道上的能量.
\item 根据上题算出的结果,说明要把基态的氢原子激发
到$n=2$的能级上去,需要供给电子多大的能量.如果用电磁
波来供给这个能量,需要用波长多长的电磁波?这个波长属于
哪个波段?
\end{enumerate}

\section{玻尔原子理论对氢光谱的解释}

玻尔理论的成功主要表现在对氢光谱规律的解释上.

\subsection{氢光谱的规律}

人们很早就发现每种元素都发出自己独
特的光谱,各种元素的每条光谱线的频率都是固定不变的,在
所有的光谱中,人们对氢光谱研究得最清楚,氢光谱在可见光
区内有四条谱线,这四条谱线叫做$H_{\alpha}$、$H_{\beta}$、$H_{\gamma}$、$H_{\delta}$.它们的波
长分别是
\begin{center}
    \begin{tabular}{cc}
        $H_{\alpha}$  &0.6562微米\\
        $H_{\beta}$  &0.4861微米\\
        $H_{\gamma}$  &0.4340微米\\
        $H_{\delta}$  &0.4101微米\\
    \end{tabular}
\end{center}

1885年瑞士的中学教师巴耳末(1825—1898)研究了这
些波长之间的关系,发现了它们之间的关系可以用一个公式
来表示.如果利用波长的倒数$1/\lambda$,
巴耳末的公式可写作
\[\frac{1}{\lambda}=R\qty(\frac{1}{2^2}-\frac{1}{n^2}),\qquad n=3,4,5,\ldots\]
式中的$R$是一个常数,叫做\textbf{里德伯恒量},实验测得$R$的值为
$1.096776\x10^7{\rm m^{-1}}$.

上面的公式叫做\textbf{巴耳末公式}.当$n=3,4,5,6$时,用这个公
式计算出的四条光谱线的波长跟上面从实验测得的$H_{\alpha}$、$H_{\beta}$、$H_{\gamma}$、$H_{\delta}$四条谱线的波长符合得非常好,于是人们把氢光谱的
这一系列谱线叫做巴耳末系.巴耳末公式反映了氢光谱这一
系列谱线的规律性.

\subsection{玻尔原子理论对氢光谱规律的解释} 

按照玻尔原子理
论,原子从较高能级 $E_2$跃迁到较低能级$E_1$时,辐射出的光子
能量为$k\nu =E_2-E_1$.因此,氢原子的电子从能量较高的轨道$n$
跃迁到能量较低的轨道2时,辐射出的光子能量应为$h\nu=E_n-E_2$.
利用第三节中氢原子的能级公式,可得
\[E_n=\frac{E_1}{n^2},\qquad E_2=\frac{E_1}{2^2} \]
由此可得\[h\nu=-E_1\qty( \frac{1}{2^2}-\frac{1}{n^2} )\]
由于$\nu=c/\lambda$,
所以上式可写作
\[\frac{1}{\lambda}=\frac{-E_1}{hc}\qty( \frac{1}{2^2}-\frac{1}{n^2})\]
把这个式子与前面的巴耳末公式相比较,可以看出它们的形
式是完全一样的,并且$R=-E_1/(hc)$
计算出$-E_1/(hc)$
的值为$1.097373\x10^7{\rm m^{-1}}$,与前面给出的的实验值符合得很好.这就是说,
根据玻尔理论,不但可以推导出表示氢光谱的规律性的公式,
而且还可以从理论上来计算里德伯恒量的值.

由此可知,氢光谱的巴耳末系是电子从$n=3,4,5,6$等能
级跃迁到$n=2$的能级时辐射出来的.

玻尔理论不但成功地解释了氢光谱的巴耳末系,而且对
当时已发现的氢光谱的另一线系——帕邢系(在红外区)也能
很好地解释.它是电子从$n=4,5,6$等能级向$n=3$的能级跃
迁时辐射出来的.此外,玻尔理论还预言了当时尚未发现的
氢原子的其他光谱线系,这些线系后来相继被发现,也都跟玻
尔理论的预言相符.其中赖曼系在紫外区,是电子从$n=2,
3,4$等能级向$n=1$的能级跃迁时发出的;布喇开系在远红外
区,是电子从$n=5,6,7$等能级向$n=4$的能级跃迁时发出的.
图8.4和8.5分别是用氢原子的能级图和轨道图表示的各线
系的情况.
\begin{figure}[htp]
    \centering
\includegraphics[scale=1]{FIG/8-4.pdf}
    \caption{氢原子的能级图}
\end{figure}

\begin{figure}
    \centering
    \begin{tikzpicture}[scale=1.6, >=stealth]
        \draw (0,0) circle (5pt);
        \draw (0,0) circle (15pt);
        \draw (0,0) circle (30pt);
        \draw (0,0) circle (50pt);
        \draw (0,0) circle (75pt);
        \draw (0,0) circle (100pt);
    \fill (0,0) circle (1.5pt);

    \node at (0,-5pt-1.4) {$n=1$};
    \node at (0,-15pt-1.4) {$n=2$};
    \node at (0,-30pt-1.4) {$n=3$};
    \node at (0,-50pt-1.4) {$n=4$};
    \node at (0,-75pt-1.4) {$n=5$};
    \node at (0,-100pt-1.4) {$n=6$};
    
    
    \node at (50:50pt) {赖曼系(紫外线)};
    \node at (0:50pt) {巴耳末系};
    \node at (-25:70pt) {帕邢系(红外线)};
    \node at (-45:85pt) {布喇开系};
    \node at (-60:95pt) {逢德系};
    
    
    
    
    \draw[->] (60:100pt)--(60:5pt);
    \draw[->] (55:75pt)--(55:5pt);
    \draw[->] (50:50pt)--(50:5pt);
    \draw[->] (45:30pt)--(45:5pt);
    \draw[->] (40:15pt)--(40:5pt);
    
    \draw[->] (10:100pt)--(10:15pt);
    \draw[->] (5:75pt)--(5:15pt);
    \draw[->] (0:50pt)--(0:15pt);
    \draw[->] (-5:30pt)--(-5:15pt);
    
    \draw[->] (-20:100pt)--(-20:30pt);
    \draw[->] (-25:75pt)--(-25:30pt);
    \draw[->] (-30:50pt)--(-30:30pt);
    
    \draw[->] (-45:100pt)--(-45:50pt);
    \draw[->] (-50:75pt)--(-50:50pt);
    
    \draw[->] (-60:100pt)--(-60:75pt);
    
    
    \end{tikzpicture}
    \caption{氢原子的轨道图}
\end{figure}

    \section*{阅读材料:定态存在的实验证明——夫兰克-赫兹实验}
    玻尔原子理论的一个重要假设是原子存在着某些分立的
    能态,各能态之间有一定的间隔.玻尔的这一假设能从实验
    中观察到吗?1914年,也就是玻尔原子理论提出后的第二年,
    由德国物理学家夫兰克(1882—1964)和赫兹所做的著名实
    验,证实了原子分立能态的存在,为玻尔假设提供了有力的证
    据.

    夫兰克-赫兹实验的方法是,使电子通过压强很低的汞蒸
    气,测量电子与汞原子碰撞前后损失的能量,同时测定汞原子
    在这些碰撞中获得的能量.夫兰克和赫兹发现,当电子以较
    小的动能碰撞汞原子时,电子通过汞蒸气后的能量几乎完全
    不变.这个结果可以这样来解释:汞原子的质量是电子质量
    的几十万倍,当电子的动能较小时,跟汞原子的碰撞是弹性碰
    撞,汞原子只吸收电子的极小一部分动能,电子几乎不损失动
    能.

    但是当电子的动能增加到5电子伏时,实验结果发生明
    显的变化,这时电子与汞原子碰撞时,几乎准确地损失4.9电
    子伏的能量.当电子的动能增加到6电子伏时,电子与汞原
    子碰撞也仍然只损失4.9电子伏的能量.这表明,汞原子不
    能吸收小于4.9电子伏的能量;当提供的能量比4.9电子伏稍
    微多一点时,它也仍然只接受4.9电子伏.由此可以认为,汞原
    子有一个比它的最低能态大4.9电子伏能量的定态.在这个
    能态和最低能态之间不存在其他的能态.

    按照玻尔原子理论,汞原子在吸收了4.9电子伏的能量
    后,将从最低能态跃迁到较高能态;当电子从较高能态跃迁回
    低能态时,应辐射出光子,而且辐射出的光子的能量应等于原
    来跃迁到较高能态时吸收的能量,即4.9电子伏.夫兰克和
    赫兹在实验中果然找到了被激发的汞原子辐射出的这一光谱
    线,其波长为2537埃,它的能量恰好是4.9电子伏.这表明,
    汞原子和电子碰撞时,确实获得了4.9电子伏的能量.

    夫兰克和赫兹在以后的实验中还发现,被电子碰撞的汞
    原子也能获得其他确定的能量,例如6.7电子伏、10.4电子
伏,这相当于把汞原子激发到更高的能态.在每一种情况下,
汞原子都辐射出相应能量的光谱线,这样就用实验证明了玻
尔关于原子存在着不连续能态的假设是正确的.


\subsection*{练习三}
\begin{enumerate}
    \item 根据玻尔理论,$H_{\alpha}$、$H_{\beta}$谱线光子的能量应该是多少
电子伏?根据实验测得的$H_{\alpha}$、$H_{\beta}$的波长算得的光子的能量是
多少电子伏?二者是否一致?
\item 计算氢原子从$n=4$,$n=5$能级分别跃迁到$n=3$能
级时辐射出的光子的波长.这两条谱线在哪个波段?它们属
于哪个线系?
\item 怎样用玻尔原子理论解释原子吸收光谱的规律?
\end{enumerate}


\section{玻尔原子理论的困难和量子力学}

玻尔原子理论在解释核外只有一个电子的氢光谱上很成
功;但是用来解释具有两个以上电子的比较复杂的原子光谱
时却遇到了困难,理论推导出来的结论跟实验事实出入很大.
玻尔和其他物理学家研究了这个问题,终于明白这个理论成
功之处在于它引入了量子观念,失败之处在于它保留了过多
的经典物理理论.

到本世纪二十年代,大约在玻尔原子理论建立十年之后,
海森堡(1901—1976)、薛定谔(1887—1961)、玻尔、玻恩(1882
—1969)、狄拉克(1902—1984)等物理学家在量子观念的基础上
建立了量子力学.开始,海森堡和薛定谔互相独立地提出了
数学表达形式不同的理论,后来薛定谔很快就证明了这两种
理论是完全等价的.是对同一事物的两种描述方式,薛定谔的
理论更接近德布罗意的物质波的观念,也常常称为波动力学.
薛定谔力图用数学形式来描述物质的波粒二象性.他从麦克
斯韦的光的电磁学说得到启发,认为电子的德布罗意波也
可以用类似于光波的方式来描述,于是写出了描述物质波的
方程,这就是著名的薛定谔方程.这个方程既描述了电子的
物质波的行为,又含有电子的粒子性的特征.可惜由于我们
的数学知识不足,在这里还不能写出这个方程并予以讨论.
从解薛定谔方程得到的结果,不但成功地解释了玻尔原子理
论所能解释的现象,而且能够解释大量的玻尔理论所不能解
释的现象.我们在前面提到的玻尔理论的基本假设,在量子
力学里变成了从理论上推导出来的必然结果,而不再是人为
的假设了.原来,在薛定谔方程中,只有原子中的电子具有某
些不连续的能量值时,这个方程才有解.由薛定谔方程的解
中得出的氢原子中电子能量的可能值,正好就是由玻尔原子
理论给出的值.

建立在量子力学基础上的原子理论,核外的电子并没有
确定的轨道,从薛定谔方程的解我们只能知道,核外电子在原
子内各处出现的几率.氢原子在基态时,它的电子经常出现
的几率最大的区域是以原子核为中心的一个球壳,这个球壳
的半径为$0.53\x10^{-10}$米,这个数值跟用玻尔原子理论计算的
氢原子基态的轨道半径相同,可见,玻尔的电子轨道,只不
过是电子出现几率最大的地方.电子在核外的运动情况,通
常用“电子云”来形象地描述.这就是用小黑点的稠密与稀疏
来代表电子在核外各处出现的几率的大小,这样我们就可以
画出原子的电子云图,图中原子核好象是被一层云雾笼罩着,
云雾浓度大的地方,电子出现
的几率大,云雾浓度小的地方,
电子出现的几率小.于是电子
云的概念就代替了玻尔理论中
电子轨道的概念,图8.6是氢
原子基态的电子云.


量子力学出现以后,在说明原子结构方面迅速取得了巨
大的成功,很快地被物理学家所接受.现在它的应用已远远
超出原子结构的范围,成为物理学家研究微观世界的基本理
论工具.但是,量子力学虽然是原子世界的很成功的数学“模
型”,却不能给我们提供一个直观的物理图景.量子力学告诉
我们,既不可以把电子等微观粒子当成牛顿力学中的质点,也
不可以把物质波当成我们熟悉的机械波.由于在这里研究的
是比日常物体小得很多很多的微观客体,我们在日常经验中
所形成的一些概念在这里不再适用应该是很自然的事情.


\section{原子的受激辐射~~激光}
\subsection{自发辐射和受激辐射}

原子发光有两种情形,一种是自
发辐射,一种是受激辐射.处于激发态的原子是不稳定的,只
能停留很短的时间,通常约为$10^{-8}$秒,就自发地跃迁到较低
能级去,同时辐射出一个光子,光子的能量$h\nu =E_2-E_1$,其中
$E_2$和$E_1$分别代表原子处于高能级和低能级时的能量.这种辐
射叫做\textbf{自发辐射}.原子发生自发辐射时,各个原子发出的光
子是向四面八方辐射的,它们的频率、初相和偏振方向互不相
同,而且每个原子每次发光持续的时间很短,约$10^{-9}$秒,下
一次发光又会发出跟前一次不同的光子,因此这些光叠加时
不会产生稳定的干涉花样,看到的只是大量光产生的一种平
均效果,这种光就是自然光,这就是普通光源发光的情形.

原子发光还有一种情形,就是当原子处于激发态$E_2$时,
如果恰好有能量$h\nu =E_2-E_1$的光子从附近通过,在入射光子
的电磁场的影响下,原子会发出一个同样的光子而跃迁到低
能级$E_1$去,这种辐射叫做\textbf{受激辐射}.原子发生受激辐射时,发
出的光子的频率、发射方向、初相和偏振方向等,都跟入射光
子完全一样,也就是说,受激辐射的光子跟入射光子没有任何
区别.这样,一个入射光子由于引起受激辐射就变成了两个
同样的光子.如果这两个光子在媒质中传播时再引起其他原
子发生受激辐射,就会产生越来越多的相同的光子,使光得到
加强,这就是激光,也就是说,由于受激辐射而得到加强的光
就是\textbf{激光}.

但是,要实际产生激光并不容易,从1917年爱因斯坦从
理论上指出受激辐射到1960年世界上制成第一台激光器,经
过了四十多年的时间,这是因为,能量$h\nu =E_2-E_1$的光子从媒
质中通过时,既能引起处于$E_2$能级的原子发生受激辐射,使
光增强(也叫做光放大),也能使处于$E_1$能级的原子被激发而
跃迁到能级$E_2$,这时光子被吸收,使光减弱.在通常情况下,
处于低能级$E_1$的原子数大于处于高能级$E_2$的原子数,光吸
收过程胜过光放大过程,因而得不到激光.

应用量子力学研究了各种原子的能级后发现,某些原子
有一些特殊的能级,原子处在这些特殊能级上时,虽然也是激
发态,却能停留较长的时间,比处在其他激发态的时间可长约
十万倍,因而不易发生自发跃迁.这种激发态叫做\textbf{亚稳态}.因
此,如果原子有两个激发态$E_2$和$E_1$,其中较高能级$E_2$是亚稳
态,而较低能级$E_1$是寿命很短的激发态,我们设法把处于基
态的原子大量 激发到亚稳态$E_2$,处于高能级$E_2$的原子数就
可以大大超过处于低能级$E_1$的原子数,这种状态叫做\textbf{粒子数
反转}.这时如果有能量等于$E_2-E_1$的光子从原子附近通过,
就会产生大量的受激辐射,于是光放大过程大于光吸收过程,
得到较强的激光.

能产生激光的装置,叫做激光器.现在激光器的种类已
达几百种,有固体激光器、液体激光器、气体激光器、半导体激
光器等多种类型,适用于不同的用途.

\subsection{激光的特点及应用}

激光的主要特点是亮度高、方向性
好、单色性好、相干性好、红宝石激光器产生的激光亮度比太
阳光的亮度要高几百亿倍.激光光束几乎是完全不发散的平
行光,方向性非常好,同一个激光器产生的激光的波长范围很
窄,接近于单一频率,单色性比其他光源发出的光都好,原子
发生受激辐射时发出的光子的频率、初相和偏振方向等都跟
人射光子完全一样,所以激光的相干性很好,正是由于激光
有这些突出的特点,所以在各个领域得到了广泛的应用.

激光的亮度高、能量集中,可以使物体的被照部分在不到
千分之一秒的时间内产生几千万度的高温.因此,工业上可
以利用激光束在硬质、难熔的材料上进行打孔、切割,还可以
用来焊接金属,不但提高工作效率,而且加工质量得到保证.
医学上用激光束作“手术刀”进行手术,出血量少,刀口愈合得
快.激光束可以聚焦到比针尖还小的范围内,手术的时间极
短,特别适宜于眼科手术.利用激光焊接脱落的视网膜、切除
虹膜等都非常成功.

激光的方向性好,能照射到很远的地方,因此用它来测量
距离是非常理想的,可以达到很高的精度,对准目标发出一
个极短时间的激光脉冲,然后测出激光从发出时刻到反射回
发射点经过的时间$t$,就可以按公式$\ell=\frac{1}{2}ct$求出激光从发
射点到被测目标的距离,式中的$c$是光速.用这种方法测量
月球上某点到地球上某点的距离,误差不超过几厘米.实际
上,按照这种原理设计制造的激光测距仪就是一种激光雷达.
多用途的激光雷达不仅可以测量距离,而且能够测定被测目
标的方位、运动速度、运动轨迹,甚至能描绘出目标的形状,进
行识别和自动跟踪.

利用激光单色性好、相干性好的特点,可以用光的干涉方
法来精确地测量长度,检查加工零件的质量,进行全息照相.

激光还可以用来辐照种子,培育良利,控制化学反应,用
于舞台布景等.

激光是二十世纪的重大发明之一,自1960年出现以来,
目前还处在迅速发展时期,激光在科学技术各个领域的应用,
有力地促进了这些领域的发展,并产生了一些新的边缘科学,
例如激光化学、激光生物学等.今后随着激光技术的发展,将
会促进一些重大应用的实,例如光纤激光通讯,激光核聚
变、激光分离同位素等,并将会出现一个新兴的工业部门-
激光工业.在未来的科技发展中,激光起着重要的作用.

\section*{复习题}

\begin{enumerate}
    \item 电子是怎样发现的?电子的发现对认识原子结构起了什么作用?
    \item 原子核是怎样发现的?卢瑟福的原子模型是什么样
的?怎样用卢瑟福的原子模型来解释$\alpha$粒子的大角度散射?
\item 卢瑟福原子模型的困难是什么?玻尔是怎样解决这
一困难的?根据玻尔的原子理论,原子的模型又是怎样的?
\item 写出氢原子的轨道半径公式和能级公式,说明式中
各量的意义.
\item 玻尔理论怎样成功地解释了氮原子光谱的规律?
\item 什么叫做受激辐射?怎样才能产生大量的受激辐
射,从而得到激光?激光有哪些主要的特点?举例说明激光的
应用.
\end{enumerate}



































\chapter{原子核}
\section{天然放射现象}

\subsection{天然放射现象}

人类认识原子核的结构和它的变化规
律,是从发现天然放射现象开始的。

1896年,法国物理学家贝克勒耳(1852—1908)发现,铀和
含铀的矿物能发出某种看不见的射线,这种射线可以穿透黑
纸使照相底片感光。物质发射这种射线的性质,叫做\textbf{放射性};
具有放射性的元素,叫做\textbf{放射性元素}。

在贝克勒耳的建议下,玛丽·居里(1867—1934)和她的丈
夫皮埃尔·居里(1859—1906)对铀和铀的各种矿石进行了深
入的研究,并且发现了两种放射性更强的新元素,玛丽·居里
为了纪念她的祖国波兰,把其中一种元素命名为钋(读作“坡”,
元素符号是Po),另一种命名为镭。

放射性并不是少数几种元素才有的,实际上原子序数大
于83的所有天然存在的元素都具有放射性。这种能自发地
放出射线的现象叫做\textbf{天然放射现象}。

铀、钋和镭放出的射线到底是什么呢?科学家们利用电
场或磁场来研究放射线的性质,确定了放射线的组成。图9.1
表示利用电场进行研究的实验:把放射性样品放在铅块的窄
孔底上,在孔的对面放着照相底片,没有电场时,在显影后的
照相底片上可以发现正对着窄孔有一个暗斑,说明射线顺着
窄孔一直射到底片上,使它感光,在铅块和底片之间放上一
对电极,使电场的方向跟射线的方向垂直时,在显影后的底片
上出现三个暗斑,说明在电场作用下,射线分成三束(图9.1)。
其中有两束向相反方向偏转,表明这两束射线是由带电粒子
组成的,而且两种粒子带有异种电荷;另外那束不发生偏转,
表明这束射线是中性的,从暗
斑的位置知道,带正电的射线
偏转较小,带负电的射线偏转
较大,通常把带正电的射线叫
做$\alpha$射线,带负电的射线叫做
$\beta$射线,不发生偏转的射线叫
做$\gamma$射线。

\begin{figure}[htp]
    \centering
    \includegraphics{fig/9-1.pdf}
    \caption{}
\end{figure}

卢瑟福对这些射线进一步研究,直接用实验证实$\alpha$粒子
带有两个单位的正电荷,质量数是4\footnote{在原子核物理中,把基本电荷取作电荷的单位;把碳原子质量
的1/12取作质量的单位,原子核的质量数通常非常接近整数,因此习
惯上都用整数表示。}。$\alpha$粒子就是氦原子核。
$\alpha$粒子射出时的速度约为光速的十分之一,$\alpha$射线贯穿物质
的本领很小,在空气中只能飞行几厘米,一张薄铝箔或一张薄
纸就能把它挡住;但是它有很强的电离作用,很容易使空气电
离,使照相底片感光的作用也很强。贝克勒耳用实验证实了$\beta$
射线是电子流,速度很接近光速。$\beta$射线的贯穿本领很大,很
容易穿透黑纸,甚至能穿透几毫米厚的铝板,但它的电离作用
比较弱。$\alpha$射线是一种波长很短的电磁波,它的贯穿本领最
强,甚至能穿透几厘米厚的铅板,但它的电离作用却很小。

这三种射线都是从原子核里放射出来的,实验指出,当
放射性物质连续发生衰变时,各种原子核中有的放射$\alpha$射线,
有的放射$\beta$射线,同时伴随,射线,这时在放射线中就会同时
有$\alpha$、$\beta$、$\gamma$三种射线,放射线的发现揭示了原子核结构的复
杂性,从而促进了人类对微观结构的认识。

\subsection{放射性元素的衰变}

某元素的原子核,例如铀核,放出一
个$\alpha$粒子后,就变成了新的原子核。我们把原子核由于放出
某种粒子而转变为新核的变化叫做原子核的衰变。在衰变中
电荷数和质量数都是守恒的,我们用$\atom{U}{238}{92}$代表铀原子核,上
标“238”表示核的质量数,下标“92”表示核的电荷数(可以省
去下标,简写为$\atom{U}{238}{}$,还可以简写为铀238或U238)。同样
地,用$\atom{He}{4}{2}$代表氦原子核(即$\alpha$粒子),用$\atom{Th}{234}{90}$代表钍原子
核。于是,铀238核放出$\alpha$粒子变成钍234核的衰变可用下
面的方程来表示:
\[\atom{U}{238}{92}\longrightarrow \atom{Th}{234}{90}+\atom{He}{4}{2}\]
从这个方程可以看出,方程两边的质量数和电荷数都是相同
的。这种放出$\alpha$粒子的衰变叫做$\alpha$衰变。$\alpha$衰变的规律是:
新核的质量数比原来核的质量数减少4,新核的电荷数比原
来核的电荷数减少2,因此新核在元素周期表中的位置要向
前移两位,如果用$M$表示核的质量数,$Z$表示核的电荷数,则
$\alpha$衰变的规律可以用下式表示(式中$X$、$Y$表示不同元素):
\[\atom{X}{M}{Z}\longrightarrow \atom{Y}{M-4}{Z-2}+\atom{He}{4}{2}   \]

$\atom{U}{238}{92}$发生$\alpha$衰变产生的$\atom{Th}{234}{90}$也具有放射性,它能放出
一个$\beta$粒子而变成$\atom{Pa}{234}{91}$(镤)。由于$\beta$粒子就是电子,电子的
质量比核的质量小得多,一个原子核放出一个$\beta$粒子后,它的
质量数不变,因此,可以认为电子的质量数是零,电荷数是$-1$,
于是我们用$\atom{e}{0}{-1}$来表示电子(即$\beta$粒子)。上述的衰变可表示为:
\[\atom{Th}{234}{90}\longrightarrow \atom{Pa}{234}{91}+ \atom{e}{0}{-1}\]
这个方程两边的质量数和电荷数也是相同的,这种放出$\beta$粒
子的衰变叫做$\beta$衰变,$\beta$衰变的规律是:新核的质量数不变,
电荷数增加1,新核在元素周期表中的位置要向后移一位,$\beta$
衰变的规律可以用下式表示:
\[\atom{X}{M}{Z}\longrightarrow \atom{Y}{M}{Z+1}+ \atom{e}{0}{-1}\]



\subsection{半衰期}
放射性元素的衰变有一定的速率,例如,氡222
经过$\alpha$衰变变为钋218,如果隔一定时间测定一次剩余的氡的
数量,就会发现,大约每过3.8天,就有一半的氡发生了衰变。
也就是说,经过第一个3.8天以后,剩有一半的氡,经过第二
个3.8天以后,剩有四分之一的氡,再经过3.8天以后,就只剩
有八分之一的氡了。因此,我们可以用\textbf{半衰期}来表示放射性
元素衰变的速率:\textit{半衰期是放射性元素的原子核有半数发生
衰变需要的时间}。每一种放射性元素都有一定的半衰期,不
同的放射性元素,半衰期不同,甚至差别非常大。例如前面说
的氡222变为钋218的半衰期是3.8天,而镭226变为氡222
的半衰期是1620年,铀238变为钍234的半衰期竟长达$4.5
\x10^9$年!

放射性元素衰变的速率是由核内部本身的因素决定的,
而跟原子所处的物理状态或化学状态无关。例如,一种放射
性元素,不管它是成单质存在或是成化合物存在,或者对它施
加压力,或者增高它的温度,都不能改变它的半衰期。

\subsection*{练习一}
\begin{enumerate}
    \item 钍230是$\alpha$放射性的,它放出一个$\alpha$粒子后变成了
什么?写出衰变方程。
\item 钫223是$\beta$放射性的,它放出一个$\beta$粒子后变成了
什么?写出衰变方程。
\item 钍232经过六次$\alpha$衰变和四次$\beta$衰变后变成一种稳
定的元素。这种元素是什么?它的原子量是多少?它的原子序
数是多少?
\item 
$\atom{U}{238}{92}$变成$\atom{Pb}{206}{82}$,要经过几次$\alpha$衰变和几次$\beta$衰变?
\item 
$\atom{Bi}{210}{83}$的半衰期是5天。10克的铋210经过20天后
还剩下多少?
\item 放射性元素$\atom{Na}{24}{11}$经过6小时后只剩下1/8没有衰
变,它的半衰期是多少?
\end{enumerate}

\section{探测放射线的方法}
放射性元素放射出的$\alpha$射线、$\beta$射线和$\gamma$射线都是看不
见的,需要根据它们跟其他物质作用产生的各种效应,用适当
的仪器来探测,下面简单介绍三种方法。

\subsection{云室}
我们知道,水蒸气遇冷凝结,会形成很小的雾珠,
这时它需要有凝结的核心,悬浮在空气中的尘埃微粒或气体
离子都可以成为这种凝结核心。云和雾就是这样形成的。如
果空气中没有任何尘埃或离子,水蒸气就是达到过饱和状态,
也不能马上凝结。但是,如果这时由于某种原因在空气中产
生了离子,那么过饱和的水蒸气就会以这些离子为核心立即
凝结成雾珠。离子是看不见的,可是雾珠是看得见的,因此可
以根据出现的雾珠来推测产生离子的情形。云室就是根据这
个原理制成的。

\begin{figure}[htp]
    \centering
    \includegraphics[scale=.35]{fig/9-2.pdf}
    \caption{云室}
\end{figure}

云室(图9.2)的主要部分是一个塑料或玻璃制的容器,
它的下底是在小范围内可以上下移动的活塞,上盖是透明的,
可以通过它来观察室内发生的现象或进行照相,一小块放射
性物质(放射源)放在室内侧壁附近(或放在室外,让放射线从
侧壁的窗射入)。实验时,先往云室里加一些酒精或乙醚
(可以洒在云室下底上的黑绒布上),使室内充满酒精的饱和
蒸气,然后,使活塞突然迅速向下移动,室内气体由于迅速膨
胀而降低温度,于是酒精蒸气达到过饱和。这时如果有射线
粒子从室内气体中飞过,使沿途的气体分子电离,过饱和的酒
精蒸气就会以这些离子为核心凝结成一条雾迹,这种云室是
英国物理学家威耳逊(1869—1959)于1911年发明的,通常叫
做威耳逊云室。

\begin{figure}[htp]
    \centering
    \includegraphics[scale=1.2]{fig/9-3.pdf}
    \caption{云室中的径迹}
\end{figure}

用云室可以清楚地看出$\alpha$粒子和$\beta$粒子的径迹(图9.3)。
$\alpha$粒子质量较大,在气体中行进时不易改变方向,它的电离
本领大,在每厘米的路程中能使气体分子产生10000对离子,
所以它的径迹直而粗。$\beta$粒子质量很小,跟气体分子的电子
碰撞时容易改变方向,而且电离本领小,在每厘米的路程中只
能产生几百对离子,所以它的径迹比较细而且有时发生弯曲。
$\gamma$粒子电离本领更小,有时能产生一些细碎的雾迹。



\subsection{计数器}
\begin{figure}[htp]\centering
\includegraphics[scale=.75]{fig/9-4.png}
\caption{计数管}
\end{figure}

计数器的主要部分是计数管,它是一支玻璃管,
里面有一个导电的圆筒(或在管壁上涂一层导电薄膜)作阴
极,一根通过圆筒轴心的金属丝作阳极(图9.4),管里装入
惰性气体(如氩、氖等)和少量的乙醇汽或溴汽,气压大约是
$1.33\x10^4$—$2.66\x10^4$帕。在两极加上大约800—1500伏的直
流电压,这个电压略低于管内气体的击穿电压。当有射线粒子
飞进管内,使管内气体电离时,产生的电子在电场作用下向阳
极加速运动。电子在运动中能量越来越大,达到一定值时,跟
气体分子碰撞,又可使气体分子电离,再产生电子,于是经过
一段很短时间,就会产生大量电子,这些电子到达阳极,正离
子到达阴极(正离子由于质量大,运动较慢,在运动中不会再
使气体分子电离),就使计数管发生一次短暂的放电,从而得
到一个脉冲电流。这个脉冲电流可以用电子设备录下来。

这种计数器适合于对$\beta$粒子和$\gamma$粒子进行计数、$\alpha$粒
子的贯穿本领很小,要对它计数,需要在计数管上装一个很薄
的窗口,或者制成其他式样。

\subsection{乳胶照相}

放射线能够使照相底片感光,放射线中的粒
子经过照相底片上的乳胶时,使乳胶中的溴化银分解,经显影
后,就有一连串的黑点显示出粒子的径迹。这些径迹可用显
微镜来进行观察与测量,根据径迹的长短和形状,可以判断入
射粒子的性质、种类和能量。乳胶的密度较大,粒子在乳胶中
的射程约为空气中的千分之一,因此容易看到径迹的全部。乳
胶照相的主要优点是能够连续地工作,能够将入射粒子每个
时刻的径迹记录下来。

随着科学技术的发展,探测射线的手段不断改进,近年
来,由于探测仪器大都和电子计算机直接联结,实现了对实验
全过程电子计算机控制、计算、数据处理,已经使实验方法高
度自动化。

\section{原子核的人工转变~~原子核的组成}
放射现象的发现,使人们认识到原子核仍然具有内部结
构,并且是能够发生变化的,但是,能不能用人工的方法使原
子核发生变化呢?原子核是由什么组成的呢?

\subsection{质子的发现}
\begin{figure}[htp]\centering
\includegraphics[scale=.75]{fig/9-5.png}
\caption{}
\end{figure}

1919年,卢瑟福做了用$\alpha$粒子轰击氮原子
核的实验。实验装置如图9.5所示。容器$C$里放有放射性物
质$A$,从$A$射出的$\alpha$粒子射到一个铝箔$F$上,适当选取铝箔
的厚度;使$\alpha$粒子恰好被它完全吸收,而不能透过,在$F$的后
面放一荧光屏$S$,用显微镜$M$来观察荧光屏上是否出现闪光。
通过阀门$T$往容器$C$里通入氮气后,卢瑟福从荧光屏$S$上观
察到了闪光,把氮气换成氧气或二氧化碳,又观察不到闪光。
这个实验表明,闪光一定是$\alpha$粒子击中氮核后产生的新粒子
透过铝箔引起的。

后来,把这种粒子引进电场和磁场中,根据它在电场和磁
场中的偏转,测出了它的质量和电量,确定它就是氢原子核,
又叫做\textbf{质子},通常用符号$\atom{H}{1}{1}$或$\atom{p}{1}{1}$表示。
\begin{figure}[htp]\centering
\begin{tikzpicture}[>=latex]
\draw [->](-1,-.5)node[above]{$\alpha$}--(-0.025,-0.025);
\draw [->](0,0)--node [below]{反冲核}(2,-.2);
\draw [->](0,0)--(.2,2)node[above]{$\alpha$};
\draw [->](0,0)--(1.25,1.25)node[below]{p};
\draw (0,0) [fill=gray] circle (2.5pt);
\node at (0,-1){甲};
\end{tikzpicture}\qquad\qquad \begin{tikzpicture}[>=latex]
    \draw [->](-1,-.5)node[above]{$\alpha$}--(-0.025,-0.025);
    \draw [->](0,0)--node [below]{反冲核}(2,-.2);    \draw [->](0,0)--(1.25,1.25)node[below]{p};
    \draw (0,0) [fill=gray] circle (2.5pt);
    \node at (0,-1){乙};

\end{tikzpicture}
\caption{}
\end{figure}

这个质子是$\alpha$粒子直接从氮核中打出的,还是$\alpha$粒子打
进氮核后形成的复核发生衰变时放出的呢?为了弄清这个问
题,英国物理学家布拉凯特又在充氮的云室里做了这个实验。
如果质子是$\alpha$粒子直接从氮核中打出的,那么在云室里就会
看到四条径迹:入射$\alpha$粒子的径迹,碰撞后$\alpha$粒子的径迹,质
子p的径迹,抛出质子后的核的反冲径迹(图9.6甲)。如果
$\alpha$粒子打进氮核后形成一个复核,这复核立即发生衰变放出
一个质子,那么在云室里就只能看到三条径迹;入射$\alpha$粒子
的径迹,质子p的径迹,核的反冲径迹(图9.6乙),布拉凯特
拍摄了两万多张云室照片,终于从四十多万条$\alpha$粒子径迹的
照片中,发现有八条产生了分叉(图9.7),分叉的情况表明,
上述的第二种设想是正确的,从质量数守恒和电荷数守恒可
以知道,这个新核是质量数等于17的氧。这个变化过程可以
用下面的核反应方程来表示:
\[\atom{N}{14}{7}+\atom{He}{4}{2}\longrightarrow \atom{O}{17}{8}+\atom{H}{1}{1} \]
\begin{figure}[htp]\centering
\includegraphics[scale=.75]{fig/9-7.png}
\caption{}
\end{figure}

在云室的照片中,分叉后细而长的是质子的径迹;短而粗
的是反冲氧核的径迹。

后来,人们用同样的方法使氟、钠、铝等核发生了类似的
转变,并且都产生了质子,由于从各种原子核里都能打出质
子来,可见质子是原子核的组成部分。

\subsection{中子的发现}

卢瑟福用$\alpha$粒子轰击氮核发现质子后,有
人提出原子核可能是由带正电的质子组成的。但这种设想遇
到的困难是:除了氢原子外,所有元素的原子核的电荷数并
不等于原子核的质量数,例如,氮核的质量数是4,电荷数是
2;铀238的质量数是238,电荷数是92。那么原子核里除了
质子外还有什么呢?

1920年,卢瑟福曾预言:可能有一种质量与质子相近的
不带电的中性粒子存在,他把它叫做\textbf{中子}。

1930年发现,用由钋放出的$\alpha$射线轰击铍(Be)时产生一
种射线,这种射线的贯穿能力极强,它能够穿透几厘米厚的
铅。当时,由被轰击物质产生的各种射线中,唯一能够贯穿铅
层的是$\gamma$射线,所以当时认为这种射线可能是$\gamma$射线。

1932年,约里奥·居里(1900—1958)和伊丽芙·居里(1897
—1956)夫妇发现,如果用来自铍的这种射线去轰击石蜡(含
有大量氢原子),竟能从石蜡中打出质子(图9.8)。但从来也
没有发现过$\gamma$射线具有这样的性质,约里奥·居里夫妇想不
出这种射线是什么。
\begin{figure}[htp]\centering
\begin{tikzpicture}[>=latex] 
\draw (0,0) [fill=white] circle(2pt);
\foreach \x in{36,72,...,360}
{
    \draw[dashed,->](\x:.25)--(\x: 1);
}
\node at (0,.3){Po};
\node at (0,-1.2){$\alpha$-粒子};
\draw [pattern=north east lines] (1.3,-.3) rectangle (1.5,.3);
\node at (1.4,.6){Be};
\node at (3.4,.8){未知射线};

\foreach \x in{6,3,0,...,-6}
{
    \draw[dashed,->](\x:1.7)--(\x: 5);
}
\draw  (5.3,-.3) rectangle (5.5,.3);

\foreach \x in{6,3,0,...,-6}
{
    \draw[dashed,->](\x:1.7)--(\x: 5);
}

\draw [dashed,->] (5.6, 0)--(6.6, 0);
\draw [dashed,->] (5.6, 0.1)--(6.6, .3);
\draw [dashed,->] (5.6, -.1)--(6.6, -.3);
\node at (5.4,.6){石蜡};
\node at (6.1,-.6){质子};


\end{tikzpicture}
\caption{}
\end{figure}

1932年英国物理学家查德威克(1891—1974)仔细地研究
了这种射线,发现这种射线在磁场中不发生偏转,可见它是
中性粒子流。测出这种射线的速度不到光速的十分之一,因
此排除了它是$\gamma$射线的可能。查德威克用这种射线轰击氢原
子和氮原子。结果打出了一些氢核(质子)和氮核。他测量了
被打出的氢核和氮核的速度,并由此推算出这种射线粒子的
质量。

被打出的氢核的速度是不同的,查德威克认为速度最大
的氢核是由于未知射线中的粒子与它正碰的结果,其他速度
较小的是由于斜碰的结果,设$m$是未知粒子的质量,$v$是它
的速度,$m_{\rm H}$是氢核的质量,$v'_{\rm H}$是被打出的氢核的最大速度。
假定它们间的碰撞是弹性碰撞,氢核在未被打出前可以认为
是静止的,根据高中一年级学过的弹性碰撞知识,我们知道,
\[ v'_{\rm H}=\frac{2m}{m+m_{\rm H}}v \]

对于打出氮核的实验,设$m_{\rm N}$是氮核的质量,是被打出
的氮核的最大速度,我们同样可以得到,
\[ v'_{\rm N}=\frac{2m}{m+m_{\rm N}}v\]

我们知道,氮核的质量$m_{\rm N}$
是氢核质量$m_{\rm H}$的14倍。把
上述两式相除以消去未知的$v$,并用$14m_{\rm H}$来代替$m_{\rm N}$,可得
\[ \frac{v'_{\rm H}}{v'_{\rm N}} =\frac{m+14m_{\rm H}}{m+m_{\rm H}}\]

查德威克在实验中测得的氢核的最大速度是$3.3\x10^9{\rm cm}/{\rm s}$,氮核的最大速度是$4.7\x10^8{\rm cm}/{\rm s}$。把测得的数值代
入上式进行计算,他得出$m=1.15m_{\rm H}$。

查德威克还用别的物质来代替氢和氮重做这个实验,得
到的结果都是这种未知粒子的质量差不多等于氢核的质量。
这样,查德威克就发现了一种新的与氢核(质子)的质量差不
多的粒子,由于这种粒子不带电,所以叫做中子。

后来的更精确的实验测出,中子的质量非常接近于质子
的质量,只比后者大千分之一多(中子的质量是$1.674920\x
10^{-24}$克,质子的质量是$1.672614\x10^{-24}$克)。

在原子物理学中用$\atom{n}{1}{0}$表示中子,即中子的质量数是1,
电荷数是0。发现中子的核反应方程是
\[\atom{Be}{9}{4}+\atom{He}{4}{2}\longrightarrow \atom{C}{12}{6}+\atom{n}{1}{0}  \]

实验证实,从许多原子核里都能打出中子来,可见中子也
是原子核的组成部分。

中子的发现是物理学史上的一件大事。中子不带电荷,
它与各种物质粒子不发生静电作用,很容易接近甚至打进原
子核。中子发现后,不少科学家用中子轰击原子核,进一步揭
示了物质的微观结构,对近代物理的发展起了很大的作用。

\subsection{原子核的组成}

中子发现以后,如果认为原子核是由质
子和中子组成的,以前在原子核结构理论中遇到的问题就可
以解决了,于是原子核是由质子和中子组成的看法,很快就
得到了公认。

质子和中子统称为\textbf{核子},由于质子带一个单位的正电荷,
中子不带电,质子和中子的质量几乎相等,都等于一个质量单
位,所以原子核的电荷数就等于它的质子数,原子核的质量数
就等于它的质子数与中子数的和。具有相同质子数的原子,它
们核外的电子数也相同,因而有相同的化学性质,属于同一种
元素。但它们的中子数可以是不同的,这些具有相同的质子
数和不同的中子数的原子互称\textbf{同位素}。

在放射性元素的原子核中,2个质子和2个中子结合在
一起从核里发射出来,这就是$\alpha$衰变。原子核里虽然没有电
子,但中子可以转化成质子和电子,这时产生的电子从核里
发射出来,这就是$\beta$衰变。这一点,在后面第八节中还要说
明。至于$\gamma$射线,是因为原子核中具有多余的能量而处于激发
状态时,放出的射线。


\section{放射性同位素及其应用}
1934年,约里奥·居里和伊丽芙·居里夫妇在用$\alpha$粒子轰
击铝箔时,除探测到预料中的中子外,还探测到了正电子。正电
子是物理学家在1932年发现的,它的质量跟电子的相同,带
一个单位的正电荷,跟电子正好相反,更意外的是,拿走$\alpha$放
射源以后,铝箔虽不再发射中子,但仍继续发射正电子,而且
这种放射性随时间衰减的规律跟天然放射性一样,也有一定
的半衰期。原来,铝核被$\alpha$粒子击中后发生了下面的反应:
\[\atom{Al}{27}{13}+\atom{He}{4}{2}\longrightarrow\atom{P}{30}{15}+\atom{n}{1}{0}  \]
反应生成物$\atom{P}{30}{15}$是磷的一种同位素,它有放射性,象天然放射
性元素一样发生衰变,衰变时放出正电子。我们用符号$\atom{e}{0}{1}$表
示正电子,于是$\atom{P}{30}{15}$的衰变反应可写为:
\[\atom{P}{30}{15}\longrightarrow \atom{Si}{30}{14}+\atom{e}{0}{1} \]

这种具有放射性的同位素,叫做\textbf{放射性同位素}。用人工
方法得到放射性同位素,这是一个很重要的发现。后来用质
子、氘核、中子和$\gamma$光子轰击原子核,也得到了放射性同位
素。这样就进一步认识了原子核的性质,并知道了制造放射
性同位素的方法,天然放射性同位素只不过四十几种,而今
天人工制造的放射性同位素已
达一千多种,每种元素都有了
放射性同位素。于是放射性同位素在工业、农业、医疗卫生和
科学研究的许多方面得到了广
泛的应用。

放射性同位素的应用主要
分为两类,

\subsection*{1. 利用它的射线}

例如利用钴60放出的很强的$\gamma$射
线来检查金属内部有没有砂眼或裂纹,这叫做$\gamma$射线探伤
(图9.9),用$\gamma$射线比用X射线好,用X射线只能检查2—
3厘米厚的钢板,并且X射线装置很复杂,使用也不方便。用
$\gamma$射线可以检查30厘米厚的钢铁部件,放射性同位素还可以
放进器件的内部,操作很方便。
\begin{figure}[htp]\centering
\includegraphics[scale=.75]{fig/9-9.png}
\caption{$\gamma$射线探伤的示意图}
\end{figure}

利用放射线的贯穿本领跟物体的厚度和密度的关系,可
以用放射性同位素来检查各种产品的厚度、密封容器中的液
面高度,从而自动控制生产过程。图9.10是轧钢机上钢板厚
度自动控制装置原理图。让放射线穿过钢板射到探测器上,钢
板的厚度发生变化时,透过钢板的射线的强度也随着变化,探
测器把它转变为电信号输入到厚度指示装置和厚度控制装
置,于是厚度指示装置就显示出厚度的变化,同时厚度控制
装置自动地调整轧钢机上两轧辊的距离,使钢板的厚度恢复
正常,从而保证钢板的厚度不超出公差的范围。
\begin{figure}[htp]\centering
\includegraphics[scale=.75]{fig/9-10.png}
\caption{}
\end{figure}

在化纤、纺织等工业生产中,由于摩擦、分离等原因,织物
和纤维上常聚集有害的静电,将放射源放在容易产生静电的
地方,放射性物质放出的射线可以使空气分子电离变成导电
气体,这样可以把静电荷泄出。

用剂量不大的$\gamma$射线照射植物(棉花、白菜、萝卜等)的
种子能使产量显著增加。利用射线可以防治害虫,射线照射
能使幼虫失去发育能力,大剂量的照射能直接杀死害虫。射
线照射还能引起生物遗传特性发生突变以培育良种,在医疗
上射线可以使癌细胞受到抑制或死亡,因此常利用钴60的$\gamma$
射线来治疗肺癌、食道癌等。射线还可以消毒灭菌,处理医院
排除的污水,杀死各种病原体,保护环境卫生。

\subsection*{2.做为示踪原子}

放射性同位素跟同种元素的非放射
性位素的化学性质完全一样,如果在某种元素里搀进一些
放射性同位素,那么,无论这种元素走到哪里,它的放射性同
位素也经历同样的过程,由于放射性同位素会不停地放出放
射线,用适当的探测仪器探测这些放射线,就会知道这种元素
通过什么路径,运动到哪里了。人们把作这种用途的放射性
同位素叫做\textbf{示踪原子}。

示踪原子的应用是多方面的。在内燃机工作时,活塞上
的活塞环由于摩擦而磨损;如果使用带有放射性同位素铁59
的活塞环,这时具有放射性的铁59被磨掉而混入润滑油中,
测出油中的放射性就可以了解活塞环的磨损情况,而不必拆
开内燃机去检查。在农业施肥中,在肥料中加一些放射性同
位素,就会知道哪种农作物在什么季节最能吸收含哪种元素
的肥料。

在生物科学研究方面,同位素示踪技术也起着十分重要
的作用,我国科学家首先用人工方法合成了牛胰岛素,这是
我国科学战线上的一项重大成就,在这项工作中需要证明人
工合成的牛胰岛素结晶跟天然牛胰岛素的结晶是同一种物
质,因此,在合成过程中搀入放射性碳14作示踪原子,然后
把用碳14标记的人工合成的牛胰岛素与天然牛胰岛素混合
到一起,经过多次重新结晶后,得到了放射性碳14分布均匀
的牛胰岛素结晶,这就证明了人工合成的牛胰岛素与天然牛
胰岛素完全融为一体,它们是同一种物质,从而为我国在国际
上首先合成牛胰岛素提供了有力的证据。

放射线对人体组织是有伤害作用的,在使用放射性同位
素时必须注意安全,要防止放射性物质对水源、空气、用具、
工作场所的污染,并且要防止射线过多地照射人体。


\subsection*{练习二}

\begin{enumerate}
    \item 用$\alpha$粒子轰击氮核使它发生转变。从云室的照片中
    为什么可以确定细而长的径迹是质子产生的,粗而短的径迹
    是反冲氧核产生的。
    \item 用$\alpha$粒子轰击氩40,复核衰变时产生一个中子和一
    个反冲核,这反冲核是什么?写出核反应方程。
    \item 用$\alpha$粒子轰击硼10,产生一个中子和一个具有放射
    性的核,它是什么?这个核能放出正电子,它衰变后变成什
    么?写出核反应方程。
    \item 用中子轰击氮14,产生碳14,碳14具有$\beta$放射性,
    它放出一个$\beta$粒子后衰变成什么?写出核反应方程。
    \item 用中子轰击铝27,产生钠24,写出核反应方程。钠
    24是具有放射性的,衰变后变成镁24,写出核反应方程。
    \item 带电的验电器在放射线照射下电荷会很快消失,说
    明原因。
\end{enumerate}


\section{原子核的结合能}
\subsection{核力} 

原子核的半径很小,其中的质子之间的库仑斥力
是很大的,然而通常的原子核却是很稳定的。这表明,在原
子核里,除了质子间的库仑力,还有另一种力,它把各种核子
紧紧地拉在一起。这种力叫做\textbf{核力}。从实验知道,核力是一
种很强的力,它在质子和质子间、质子和中子间、中子和中子
间都存在,并且只在$2.0\x10^{-15}$米的短距离内起作用,超过
了这个距离,核力就迅速减小到零,质子和中子的半径大约
是$0.8\x10^{-15}$米,因此每个核子(质子或中子)只跟它相邻的
核子间才有核力的作用。核力只在很短的距离内发生作用,
因此它既不是电磁力,也不是万有引力。关于核力的本质问
题现在仍在深入研究中。

\subsection{结合能}

由于核子间存在强大的核力,所以原子核是
一个坚固的集合体,要把原子核拆散成核子,需要克服核力
做巨大的功,或者说需要巨大的能量。例如,用强大的$\gamma$光
子照射氘核(它是由1个质子和一个中子组成的),可以使它
分解为质子和中子,这时的核反应方程是:
\[\gamma +\atom{H}{2}{1}\longrightarrow \atom{H}{1}{1}+\atom{n}{1}{0} \]
从实验知道,当光子能量小于2.22MeV时,这个反应并
不发生;只有光子的能量等于或大于2.22MeV时,这个
反应才会发生。相反的过程,例如一个中子和一个质子结合
成氘核,要放出2.22MeV的能量,这个能量以$\gamma$光子的
形式辐射出去。这时的核反应方程是:
\[\atom{n}{1}{0}+ \atom{H}{1}{1}\longrightarrow \atom{H}{2}{1} +\gamma  \]

这表明,核子结合成原子核时要放出一定的能量;原子核
分解成核子时,要吸收同样多的能量。这个能量叫做原子核
的\textbf{结合能}。

\subsection{质能方程~~质量亏损}

怎样才能求出原子核的结合能
呢?虽然核力的本质还在研究之中,但是物理学家却有办法
求出结合能。这要归功于大科学家爱因斯坦,他从相对论得
出质量和能量间有下述关系:
\[E=mc^2\]
这个方程叫做爱因斯坦质能联系方程,简称\textbf{质能方程},式中$c$
是真空中的光速,$m$是物体的质量,$E$是物体的能量,这个方
程表明,物体的质量跟它的能量有一定的联系:物体的能量
跟它的质量成正比,如果物体的能量增加了$\Delta E$,物体的质量
也相应地增加$\Delta m$,反过来也一样。$\Delta E$和$\Delta m$之间的关系符
合爱因斯坦的质能方程
\[\Delta E=\Delta m\cdot c^2\]

核子结合成原子核时要放出结合能,原子核的能量要比
组成核的核子的能量小,所以原子核的质量也要比组成核的
核子的质量小。我们把组成原子核的核子的质量与原子核的
质量之差叫做核的质量亏损。如果知道核的质量亏损,根据
质能方程就可以求出核的结合能。

例如,氦核是由2个质子和2个中子组成的。每个质子
的质量$m_p$是1.007277u,每个中子的质量$m_n$是1.008665u。氦
核的质量$m_{\alpha}$是4.001509u。这里u表示原子质量单位,$1u=
1.660566\x10^{-27}{\rm kg}$。氦核的质量亏损$\Delta m$可以计算如下:
\[\begin{split}
    2m_p&=2\x1.007277{\rm u}=2.014554{\rm u}\\
    2m_n&=2\x1.008665{\rm u}=2.017330{\rm u}\\
    m_{\alpha}&=4.001509{\rm u}\\
    Δm&=2m_p+2m_n-m_{\alpha}=0.030375{\rm u}\\
\end{split}\]

在原子核物理学中,核的结合能是用MeV来表示的。
按照$E=mc^2$可以求出,$1u=931.5{\rm MeV}$.因此氦核的结
合能是
\[0.030375\x 931.5{\rm MeV}=28.3{\rm MeV}\]

\subsection{平均结合能} 

如果用核子数去除核的结合能,就得到每
个核子的平均结合能。对于氦来说,就是$28.3{\rm MeV}/4=
7.1{\rm MeV}$。用同样的方法,可以求出其他原子核中每个核
子的平均结合能,平均结合能是核子结合成原子核时每个核
子平均放出的能量,也是把原子核分解成核子时每个核子平
均吸收的能量。平均结合能越大,原子核就越难拆开。可见
平均结合能的大小能够反映核的稳定程度。图9.11表示核
子的平均结合能随原子核的质量数的变化规律。图中的横坐
标表示核的质量数,纵坐标表示核子的平均结合能。从图中
可以看出,质量数较小的轻核和质量数较大的重核,平均结合
能都比较小,中等质量数的原子核,平均结合能大。质量数
为50—60的原子核,平均结合能最大,约为8.6MeV。

\begin{figure}[htp]
\centering
\includegraphics[scale=1.3]{fig/9-11.pdf}
\caption{平均结合能曲线}
\end{figure}


\subsection*{练习三}

\begin{enumerate}
    \item 氘核的质量是2.013553u,根据质量亏损,计算氘核的结合能。
    \item 碳原子的质量是12.000000u,可以看做是由6个氢
原子(质量是1.007825u)和6个中子组成的。求碳原子核的
结合能。(在计算中可以用碳原子的质量代替碳原子核的质
量,用氢原子的质量代替质子的质量,因为电子的质量可以在
相减过程中消去。)
\item  在$\atom{He}{4}{2}$,$\atom{Kr}{82}{36}$,$\atom{U}{238}{92}$等原子核中核子的平均结合能个最大?哪个最小?原子核的结合能哪个最大?哪个最小?(根据平均结合能曲线进行比较)
\item 如果要把$\atom{O}{16}{8}$分成8个质子和8个中子,要给它多
少能量?要把它分成$\atom{He}{4}{2}$,要给它多少能量?已知$\atom{O}{16}{8}$的核
子平均结合能是7.98MeV,$\atom{He}{4}{2}$的核子平均结合能是
7.07MeV。
\item 在一次核反应中,铀核$\atom{U}{235}{92}$变成了氙核$\atom{Xe}{136}{54}$和锶
核$\atom{Sr}{90}{38}$(同时放出了若干中子)。铀核的核子平均结合能约为
7.6MeV,氙核的核子平均结合能约为8.4MeV,锶核
的核子平均结合能约为8.7MeV。
\begin{enumerate}
    \item 把U235分解为核子,要吸收多少能量?
    \item 再使相应的核子分别结合成Xe136和St90,要放出多少能量?
    \item 在这个核反应中是吸收还是放出能量?这个能量大
约是多大?
\end{enumerate}
\end{enumerate}


\section{重核的裂变}
重核的核子平均结合能比中等质量的核的核子平均结合
能小。因此,重核分裂成中等质量的核时,会有一部分结合能
释放出来。这是释放原子核能——原子能的一种重要方法。
这种核反应叫做\textbf{裂变}。

\subsection{铀核的裂变}

重核的裂变是在本世纪三十年代末期用中
子轰击铀核时发现的。铀核裂变的产物是多种多样的,有时裂
变为氙(Xe)和锶(Sr),有时裂变为钡(Ba)和氮(Kr)或锑(Sb)
和铌(Nb),同时放出2—3个中子,铀核还可能分裂成三部分
或四部分,不过这种情形比较少见。

铀核的裂变过程,可以用下面的示意图来说明(图9.12)。
\begin{figure}[htp]\centering
\includegraphics[scale=.75]{fig/9-12.png}
\caption{}
\end{figure}

当中子打进铀235后,就
形成处于激发状态的复核,复
核中的核子由于激烈运动,使
核变成不规则形状,核子间的
距离增大。由于核力只在极短
距离内发生作用,当核子间距
离增大时,核力迅速减小,因而
不能克服质子间的库仑斥力使
核恢复原状,核就分裂成两部
分,同时放出几个中子。

轴核裂变的许多可能的核
反应中的一个是:
\[\atom{U}{235}{92}+\atom{n}{1}{0}\longleftarrow \atom{Ba}{141}{56}+\atom{Kr}{92}{36}+3\atom{n}{1}{0}\]
在这个反应中释放的能量可以计算如下:
裂变以前:
\[235.0439{\rm u}+1.0087{\rm u}=236.0526{\rm u} \]
裂变以后:
\[140.9139{\rm u}+91.8973{\rm u}+3.0261{\rm u}=235.8373{\rm u} \]
反应过程中质量减少$\Delta m=0.2153{\rm u}$,反应中释放的能量
\[\Delta E=\Delta m\cdot c^2=201{\rm MeV}\]
在不同的反应中,铀核释放的能量也不同。如果按照一个铀
核裂变时放出200MeV的能量来估算,1千克铀全部裂
变时放出的原子能就相当于2500吨优质煤完全燃烧时放出
的化学能。

\subsection{链式反应} 

铀核裂变时,同时放出2—3个中子,如果这
些中子再引起其他铀核裂变,就可使裂变反应不断地进行下
去,这种反应叫做链式反应(图9.13)。

\begin{figure}[htp]
    \centering
    \includegraphics[scale=.7]{fig/9-13.pdf}
    \caption{链式反应}
\end{figure}


在天然铀中,主要有两种同位素,其中99.3\%是铀238,
0.7\%是铀235。这两种铀跟中子的作用很不相同。铀235俘
获各种能量的中子都会发生裂变,而且俘获能量低的中子发
生裂变的几率较大,铀238只有俘获能量大于1MeV的
中子才可能发生裂变,并且几率很小,它俘获能量低于1MeV的中子时不发生裂变,而变成铀239。能量低于1eV的中子跟铀238基本上只发生弹性碰撞,不引起核反应。因
此,为了使裂变的链式反应容易发生,最好是利用纯铀235。

铀块的体积对于产生链式反应也是一个重要因素。因为
原子核非常小,如果铀块的体积不够大,中子从铀块中通过
时,可能还没有碰到铀核就跑到铀块外面去了,能够发生链
式反应的铀块的最小体积叫做它的临界体积。

如果铀235的体积超过了它的临界体积,只要有中子进
入铀块,会立即引起铀核的链式反应,在极短时间内就会释放
出大量的核能,发生猛烈的爆炸,原子弹就是根据这个原理
制成的。

\subsection{核反应堆}

核反应堆是用人工控制链式反应的装置。原
子弹爆炸时链式反应的速度是无法控制的,为了用人工方法
控制链式反应的速度,使核能比较平缓地释放出来,人们制成
了核反应堆。
\begin{figure}[htp]
    \centering
    \includegraphics[scale=1.2]{fig/9-14.pdf}
    \caption{原子反应堆示意图}
\end{figure}

图9.14是原子反应堆的示意图,反应堆里用的铀棒是天
然铀或浓缩铀(其中铀235的含量比天然铀高),由于裂变产
生的是速度很大的快中子,很容易被铀238俘获而不发生裂
变,所以必须设法使中子在碰上铀238前降低速度。为此在
铀棒的周围放上原子量比较小、又不吸收或很少吸收中子的
物质,如石墨、重水或普通水(普通水吸收中子的几率较大,但
可用在用浓缩铀做原料的反应堆中),快中子跟这些物质的原
子核碰撞后,能量减小,变成慢中子。这种用来使中子减速的
物质叫做\textbf{减速剂}。速度跟热运动速度差不多的慢中子(能量约
为1/40电子伏)叫做热中子。热中子碰到铀238时会弹射回
来,却容易被铀235俘获而引起裂变,为了调节中子数目以
控制反应速度,还需要在铀棒之间插进一些镉棒,镉吸收中
子的能力很强,当反应过于激烈时,使镉棒插入深一些,让它
多吸收一些中子,链式反应的速度就会慢一些;当反应过于缓
慢,达不到所需功率时,使镉棒插入浅一些,让它少吸收一些
中子,链式反应速度就可以增大,这种镉棒叫做\textbf{控制棒}。用电
子仪器自动地调节控制棒的升降,就能使反应堆保持一定的
功率安全地工作。

反应堆工作时,核燃料裂变释放出的核能转变为热能,使
反应堆的温度升高,为了控制反应堆的温度,使它能正常工
作,需要用水、液体金属钠或空气等流体作冷却剂,在反应堆
内外循环流动,不断地带走热能。这就是反应堆的冷却系统,
它同时可以用来输出热能。

为了防止铀核裂变物放出的各种射线对人体的危害,在
反应堆的外面需要修建很厚的水泥防护层,用来屏蔽射线,不
让它们透射出来,对放射性的废料,也要装入特制的容器,埋
入深地层进行处理。

利用反应堆工作时释放出的热能使水汽化以推动汽轮发
电机发电,这就是核电站。图9.15是核电站示意图。核电站
消耗的“燃料”很少。一座一百万千瓦的核电站,每年只消耗
30吨浓缩铀,而同样功率的火力发电站,每年却要消耗250万
吨煤。目前,核能发电的经济效益跟火电站大体相同,到1983
年底、核发电已占世界发电总量的12\%。为了适应我国现代
化建设对能源日益增长的需要,在广东、浙江、江苏、辽宁等省
正在建造核电站。

\begin{figure}[htp]
    \centering
    \includegraphics[scale=.8]{fig/9-15.pdf}
    \caption{核电站示意图}
\end{figure}

原子能反应堆不仅可以提供强大的原子能,而且它产生
的大量中子还可以用来进行各种原子核物理实验,制造各种
放射性同位素。

利用原子能反应堆还可以生产新的核燃料,从实验知道,
快中子被铀238俘获后,变成铀239,铀239是不稳定的,经过
两次$\beta$衰变后变成钚239。钍232与中子作用后,经过两次$\beta$
衰变后变成铀233,钚239和铀233的性质跟铀235一样,很
容易俘获中子而发生裂变,因此也可以作为供裂变用的核燃
料。因此,如果在反应堆中装入铀238或钍232,并设法使每
一次核裂变能够产生一个以上的钚239或铀233,那么,我们
就可以使新产生的核燃料多于消耗的核燃料,使铀238和钍
232也可以得到利用,这种反应堆叫做增殖反应堆,地球上
的铀238和钍232的总量大约是铀235的800倍,建造增殖
反应堆可以利用铀238和钍232,更有效地利用核资源。增
殖反应堆虽然处于试验阶段,但从长远来看是很有前途的。

\section{轻核的聚变}
\subsection{聚变}
轻核的结合能更小,某些轻核结合成质量较大的
核时,能释放出更多的结合能。例如:一个氘核和一个氚核结
合成一个氨核时,释放出17.6MeV的能量,平均每个核
子放出的能量在3MeV以上,这时的核反应方程是
\[\atom{H}{2}{1}+\atom{H}{3}{1}\longrightarrow \atom{He}{4}{2}+\atom{n}{1}{0} 
    \]
轻核结合成质量较大的核叫做\textbf{聚变}。

使核发生聚变,必须使它们接近到$10^{-15}$米,也就是接近
到核力能够发生作用的范围。由于原子核都是带正电的,要
使它们接近到这种程度,必须克服电荷之间的很大的斥力作
用。这就要使核具有很大的动能。用什么办法能使大量的轻
核获得足够的动能来产生聚变呢?有一种办法,就是把它们
加热到很高的温度。从理论分析知道,物质达到几百万度以
上的高温时,原子的核外电子已经完全和原子脱离,这时小部
分原子核就具有足够的动能,能够克服相互间的库仑斥力,在
互相碰撞中接近到可以发生聚变的程度。因此,这种反应又
叫做热核反应,怎样产生这样高的温度呢?我们知道,原子
弹爆炸时能产生这样高的温度,所以可以用原子弹来引起热
核反应,氢弹就是这样制造出来的。

热核反应在宇宙中是很普遍的现象。在太阳内部和许多
恒星内部,温度都高达1千万度以上,在那里热核反应激烈地
进行着,太阳每秒钟辐射出来的能量约为$3.8\x10^{26}$焦,就是
从热核反应中产生的,地球只接受了其中的二十亿分之一,就
使地面温暖,产生风云雨露,河川流动,生物生长。

\subsection{可控热核反应}

原子弹、氢弹虽然能够引起热核反应释
放出巨大能量,但能量是瞬时释放出来的。和平利用核能则
需要聚变能缓慢而稳定的释放,释放的速率应当能够被人控
制,即发生可控热核反应,这种反应就是用人工的办法,有控
制地使氘核产生聚变反应,从而释放出能量。热核反应需要的
原料——氘,在世界上的储量是非常丰富的。1升海水中大约
有0.03克的氘,如果用来发生热核反应,它放出的能量就和
燃烧三百升汽油相当,因此海水中的氘就是异常丰富的能源。

热核反应除了原料丰富外,还有以下几个特点:它释放
出的能量,就每一个核子平均来说,比裂变反应要大好几倍。
而且裂变反应会产生带有强放射性的物质,对环境造成放射
性污染;热核反应对环境的污染要轻得多,也比较容易处理。
从热核反应中还可以得到大量有用的中子。可控热核反应是
核能利用的一条途径,受到普遍重视。

目前,世界上许多国家,都在积极研究可控热核反应的理
论和技术。我国自行设计和制造的可控核聚变实验装置“中
国环流器一号”已于1984年9月顺利启动,它标志着我国研
究可控热核聚变的实验手段有了新的发展和提高,必将为人
类探求新能源作出贡献。

\section{基本粒子}

直到十九世纪末,人们都认为原子是组成物质的最小的
不可再分的微粒,后来发现了电子、质子和中子,并且知道了
质子和中子组成了原子核,原子核和电子组成了原子,这时
许多人又认为电子、质子和中子是组成物质的最基本的粒子,
把它们叫做\textbf{基本粒子}。

随着科学技术的发展,从二十世纪三十年代以来,人们不
断地从宇宙射线和原子核物理实验中发现了大量的基本粒
子。

宇宙射线是从宇宙空间射来的高能粒子,其中主要是质
子,还有少量的$\alpha$粒子和其他粒子。这些粒子的能量很高,大
部分高达$10^9$—$10^{10}$eV,少数粒子具有更高的能量,宇宙
射线进入地球大气层后,跟大气中的原子核碰撞,会引起很多
种核反应,产生各种核反应产物,自从1911年发现宇宙射线
以后,就开始了对宇宙射线的观测,在宇宙射线的研究中,陆
续发现了一些新的基本粒子:1932年发现正电子,1937年发
现$\mu$介子(后来称为$\mu$子),1947年又发现K介子和$\pi$介子.
这些介子的质量是介于质子和电子之间的,因此叫做介子,后
来又发现了质量比质子大的粒子,名叫\textbf{超子}\footnote{六十年代以后,又发现了质量比质子大的介子,因此介子、超
子这些名称只具有历史上的意义。}。

1932年发明了回旋加速器,后来建成了各种加速器。在
用加速器进行的实验中,发现了更多的基本粒子。并且发现,
许多粒子都有和它的质量相同而电荷相反的粒子,叫做\textbf{反粒
子}。例如,电子的反粒子就是正电子,正$\pi$介子的反粒子就是
负$\pi$介子,质子的反粒子叫做反质子,是1955年发现的,它带
有单位负电荷,现在发现的基本粒子已达几百种。

按照基本粒子之间的相互作用,可以把它们分为三类:

1. \textbf{强子:} 核子之间的核力是一种比电磁作用大得多的
相互作用,叫做\textbf{强相互作用}。凡是参与强相互作用的粒子,都
叫做强子,目前发现的基本粒子,绝大多数是强子,质子是最
早发现的强子,强子又分重子(中子、质子、超子)和介子两类。

2. \textbf{轻子:} 都不参与强相互作用,只发现几种。电子是最
早发现的轻子。$\mu$子从它的许多性质来看属于轻子,1975年,
又发现了一种质量很大的轻子,称为$\tau$子,也叫重轻子。

3. \textbf{媒介子:} 是传递粒子间相互作用的粒子,例如光子就
是其中的一种,是传递电磁相互作用的。

绝大多数基本粒子都是不稳定的,在很短时间内就发生
衰变,并且能相互转变,例如,正负$\pi$介子的平均寿命约为
$2.6\x10^{-8}$秒,它衰变为$\mu$子,同时产生一种质量非常小(与电
子质量相比,可以认为质量为零)的中性粒子,叫做\textbf{中微子},它
属于轻子,用$\nu$表示。$\mu$子也是不稳定的,平均寿命约为$2.2\x
10^{-8}$秒,衰变为电子和正反两个中微子。

中子也可以转化,自由状态的中子的平均寿命约为16分,
它衰变为一个质子、一个电子和一个反中微子;
\[\atom{n}{1}{0}\longrightarrow \atom{p}{1}{1}+\atom{e}{0}{-1}+\nu   \]
放射性原子核的$\beta$衰变,实际发生的就是这种反应,由中子转
化成的质子仍留在原子核内,同时产生的电子(即$\beta$粒子)和
反中微子则放射出去。

质子在自由状态是稳定的,但在原子核内也会转化为一
个中子,同时放出一个正电子和一个中微子:
\[\atom{p}{1}{1}\longrightarrow \atom{n}{1}{0}+\atom{e}{0}{1}+\nu   \]
放射性同位素磷30的衰变,实际发生的就是这种反应。

一对正反粒子相遇时,会同时消失而转化为别种粒子,这
种现象叫做\textbf{湮灭},例如,一个电子和一个正电子相遇会发生
湮灭而转化为一对光子:
\[\atom{e}{0}{-1}+\atom{e}{0}{1}\longrightarrow \gamma+\gamma  \]

相反的过程也能够发生。例如,能量超过1.02MeV
的$\gamma$光子穿过铅板时,会同时产生电子和正电子,通常把这一
对正负电子叫做电子-正电子偶。图9.16是在云室中看到的
它们的径迹,由于电子和正电子所带的电荷相反,它们在磁场
中向相反的方向偏转,这个反应可表示为:
\[\gamma\longrightarrow \atom{e}{0}{-1}+\atom{e}{0}{1} \]
这些事实表明,光子和电子虽然有基本的区别,但是仍然有着
深刻的联系。
\begin{figure}[htp]\centering
\includegraphics[scale=.75]{fig/9-16.png}
\caption{电子-正电子偶的径迹}
\end{figure}



基本粒子的种类这样多,并且能够相互转化,这就促使人
们进一步去研究基本粒子的结构。许多实验事实表明,强子
是有内部结构的,因此许多物理学家倾向于不再使用“基本
粒子”这个名称,而改称为“粒子”。为了探索强子的内部结构,
先后提出了多种模型,其中比较成功的是\textbf{夸克模型},认为强子
是由夸克(我国也叫层子)组成的。目前,这个模型里有六类
(共十八种)夸克,还有同样数目的反夸克,它们所带的电荷是
基本电荷的$\pm1/3$或$\pm 2/3$。
重子是由三个夸克组成的,反重子
是由三个反夸克组成的,介子是由一个夸克和一个反夸克组
成的。从夸克理论得出的许多结果都跟实验符合得很好,但
在实验中还没有发现自由夸克。关于夸克模型的理论正在进
一步发展中。

物质世界,无论从宏观方面看,还是从微观方面看,都是
无穷尽的,人类对物质世界的认识将一步一步地深人,永远不
会终止。关于基本粒子的物理学——高能物理正在蓬勃发展
中,人们终将认识基本粒子的结构和变化规律。


\section*{复习题}
\begin{enumerate}
    \item 什么是放射性元素?$\alpha$射线、$\beta$射线、$\gamma$射线的本质是什么?三种射线各有什么特性?

    什么是$\alpha$衰变?$\alpha$衰变的规律是什么?什么是$\beta$衰变?$\beta$
    衰变的规律是什么?什么是半衰期?
    \item 说明用云室、计数器、乳胶照相探测射线的基本原
    理。
    \item 什么是质子?质子是怎样发现的?什么是中子?中子
    是怎样发现的?原子核是怎样组成的?什么是同位素?
    \item 简述放射性同位素有哪些应用。
    \item 什么是原子核的结合能?什么是爱因斯坦的质能方
    程?怎样根据质能方程计算结合能?什么是平均结合能?平均
    结合能的大小反映核的什么性质?
    \item 什么是重核的裂变?什么是链式反应?在核反应堆中
    怎样控制裂变的速度?
    \item 什么是轻核的聚变?产生聚变的条件是什么?研究
    可控热核反应有什么意义?
    \item 简述基本粒子的种类。
\end{enumerate}


\section*{习题}
\begin{enumerate}
    \item 铀238的半衰期是$4.5\x 10^9$年,假设一块矿石中含
有1千克的铀238,经过45亿年(相当于地球的年龄)以后,
还剩有多少铀238?假设发生衰变的铀238都变成了铅206,
矿石中会有多少铅?这时铀铅的比例是多少?再经过45亿年,
矿石中的铀铅比例将变成多少? 根据这种铀铅比例能不能判
断出矿石的年龄?
\item 镭核在$\alpha$衰变中放出能量为4.78MeV的$\alpha$粒
子和能量为0.19MeV的$\gamma$粒子。如果1克镭每秒钟有
$3.7\x10^{10}$个原子核发生$\alpha$衰变,算出它每秒钟释放多少能量。
\item 静止状态的放射性原子核镭($\atom{Ra}{226}{88}$)进行$\alpha$衰变。为
了测量$\alpha$粒子的动能$E$,让$\alpha$粒子垂直飞进$B=1$特的匀强磁
场,测得$\alpha$粒子的轨道半径$r=0.2$米。
\begin{enumerate}
    \item 写出Ra的$\alpha$衰变方程。
    \item 试计算$\alpha$粒子的动能$E$。
\end{enumerate}

\item 在某些恒星内,三个$\alpha$粒子结合成一个$\atom{C}{12}{6}$核。$\atom{C}{12}{6}$
的质量是12.0000u,$\atom{He}{4}{2}$的质量是4.0026u。这个反应中放出
多少能量?
\item 已知$\atom{Ra}{226}{88}$,$\atom{Rn}{222}{86}$,$\atom{He}{4}{2}$的原子量分别是226.0254,
222.0175,4.0026。求在$\atom{Ra}{226}{88}$衰变
\[\atom{Ra}{226}{88}\longrightarrow \atom{Rn}{222}{86}+\atom{He}{4}{2}\]
中放出的能量是多少电子伏?如果这些能量都以Rn核和He
核的动能形式释放出来,放出的$\alpha$粒子的速度有多大?

提示:能量和动量都守恒。

\item 在一原子反应堆中,用石墨(碳)作减速剂使快中子
减速。已知碳核的质量是中子的12倍,假设把中子与碳核的
每次碰撞都看作是弹性正碰,而且认为碰撞前碳核都是静止
的。
\begin{enumerate}
    \item 设碰撞前中子的动能是$E_0$,经过一次碰撞,中子损失的能量是多少?
    \item 至少经过多少次碰撞,中子的动能才能小于$10^{-6}E_0$?
\end{enumerate}

\end{enumerate}



































\chapter{学生实验}

\section{观察磁铁对电流的作用}

这个实验我们先用左手定则来判断磁场对电流的作用力
的方向,然后再用实验验证.

如图10.1所示,将矩形线圈悬挂在支架上,线圈的$AB$边
悬在蹄形磁铁的两极间.按照图10.1把电路连接好.
\begin{figure}[htp]\centering
    \includegraphics[scale=.6]{fig/10-1.png}
    \caption{}
    \end{figure}

根据电键闭合后$AB$边中的电流方向和$AB$边所在处的磁
场方向,用左手定则判断出磁场对$AB$边的作用力的方向,合
上电键$K$,观察$AB$边向哪个方向运动,与你判断的是否一
致,注意:只要看出了$AB$边的运动方向就要立即断开电键,
以免电池或线圈烧坏,下面的实验也要这样.

把与电池正、负极连接的两条导线交换一下,改变$AB$边
中的电流方向,用左手定则判断出磁场对$AB$边的作用力的
方向,合上电键$K$,观察$AB$边向哪个方向运动,与你判断的
是否一致.$AB$边的受力方向与上一次比较改变没有?

把磁铁的$N$、$S$极调换一下,改变磁场的方向,先用左手
定则判断出磁场对$AB$边的作用力的方向,然后再做实验,看
一看与你判断的是否一致,$AB$边的受力方向与上一次比较
改变了没有?

同学们自己设计一个表格,在每次实验前,先将$AB$边中
的电流方向以及磁铁的磁场方向填入表中,再根据实验时$AB$
边的运动方向,把磁铁对电流的作用力的方向填入表中.

\section{研究电磁感应现象}

关于感生电流产生的条件,我们已经随同老师一起做过
实验,现在用图2.3所示的装置来研究怎样判断感生电流的
方向,为了加强实验效果,这里用的螺线管$A$带有铁心.

首先查明电流表指针的偏转方向和螺线管中电流方向的
关系,然后把蓄电池(或其他低压电源)、开关和螺线管$A$串联
成一个电路,把电流表和螺线管$B$串联成另一个电路.

接通电源,给螺线管$A$通电,然后把它插入螺线管$B$中,
停留一会儿再取出来.同时注意,当螺线管$A$在插入或取出
时,跟螺线管$B$相连的电流表的偏转方向,并记下$B$中感生电
流的方向.

改变螺线管$A$中的电流方向,重做上面的实验,观察跟螺线
管$B$相连的电流表的偏转方向,并记下$B$中感生电流的方向.

把螺线管$A$放在螺线管$B$中不动,观察在给螺线管$A$通
电和断电的瞬间,跟螺线管$B$相连的电流表的偏转方向,并记
下$B$中感生电流的方向,改变螺线管$A$的通电方向,再观察在
通电和断电的瞬间,跟螺线管$B$相连的电流表的偏转方向,并
记下$B$中感生电流的方向.

分析你的实验并回答下列问题:
\begin{enumerate}
    \item 螺线管$B$内部的磁场增强时,$B$中感生电流的磁场
方向怎样?螺线管$B$内部的磁场减弱时,$B$中感生电流的磁
场方向怎样?
\item 归纳出判断感生电流方向的规律.
\end{enumerate}

\section{用示波器观察交流电的波形}
这个实验是用示波器来观察交流电的波形,我们讲过,
示波器自己能发出正弦交流信号,我们就先观察示波器自己
的正弦交流信号,然后再观察从信号源输入的交流信号.

开机前,先把辉度调节旋钮反时针转到底,衰减旋钮置于
正弦符号“\tikz \draw[x=.7ex,y=1ex] (0,0) sin (1,1) cos (2,0) sin (3,-1) cos (4,0)--(0,0);”档,Y增益旋钮顺时针转到底,扫描范围置于10—100赫兹档,其余各个旋钮置下中间位置.打开电源开关,
经预热后,顺时针旋转辉度调节旋钮,可以看到不稳定的不一
定清晰的波形,调整聚焦调节和辅助聚焦旋钮,使图线清晰.
调整垂直位移和水平位移旋钮,使图象位置适中,调整X增
益旋钮,使图象大小合适.作好这些准备工作后,把同步开关
置于“$+$”位置,把扫描微调旋钮先顺时针转到底再慢慢地反
时针旋转,当转到某一位置时,可以看到一个稳定的完整的正
弦波形(图10.2甲),屏上还同时出现的水平亮线是扫描的回
扫线,把同步开关置于“$-$”位置,又可看到波形改变半个周期
(图10.2乙).如果扫描频率恰好是信号频率的$1/n$($n$为整
数),将可看到稳定的$n$个完整波形,继续慢慢地反时针旋转
扫描微调旋钮,减小扫描频率,就可以看到屏上出现稳定的两
个、三个正弦波形.

\begin{figure}[htp]\centering
\begin{minipage}[t]{0.48\textwidth}
\centering
\begin{tikzpicture}[>=latex, scale=.7]
\draw  (0,0) sin (1,1) cos (2,0) sin (3,-1) cos (4,0)--(0,0);
\draw (2,0) circle (2.2);


\end{tikzpicture}
\caption*{甲}
\end{minipage}
\begin{minipage}[t]{0.48\textwidth}
\centering
\begin{circuitikz}[>=latex, scale=.7]
\draw  (0,0) sin (1,-1) cos (2,0) sin (3,1) cos (4,0)--(0,0);
\draw (2,0) circle (2.2);

\end{circuitikz}
\caption*{乙}
\end{minipage}
    \caption{}
    \end{figure}

    \begin{figure}[htp]\centering
        \includegraphics[scale=.6]{fig/10-3.png}
        \caption{J2465型学生信号源
        的面板}
        \end{figure}

现在再来观察从信号源输入的交流信号.我们用的信号
源是J2465型学生信号源,这种信号源可以输出低频、高频正
弦交流信号和高频调幅信号(在第四章将讲到什么是调幅信
号).J2465型学生信号源的面板如图10.3所示,下面先介
绍它的使用方法.

低频正弦交流信号从右边的两个低频输出接线柱输出.
它们的上边是低频增幅旋钮,顺时针旋转它,低频输出的电压
连续增大.中间的频率选择旋钮用来改变低频输出的频率;
它有五档:500,1000,1500,2000和2500赫.

高频信号从左边的两个高频输出接线柱输出.它们的上
边是高频增幅旋钮,顺时针旋转它,高频输出的电压连续增
大.上边的频率调节旋钮用来连续改变高频输出的频率,当
右边的选择开关在位置“I”时,频率改变范围是500—1700千
赫;在位置“II”时,是400—580千赫.

需要高频正弦交流信号时,左边的选择开关应置于“等
幅”位置.当这个选择开关置于“调幅”位置时,从高频输出接
线柱输出的是高频调幅信号.调幅度的大小用低频增幅旋钮
来调节,调制信号的频率用频率选择旋钮来选择.

让我们观察从信号源输入的低频正弦信号,为此,先用
导线把信号源的两个低频输出接线柱跟示波器的“Y输入”和
“地”两个接线柱连接上;把信号源的低频增幅旋钮转到中间
位置;把示波器的衰减旋钮从正弦符号挡转到最高挡,打开它
们的电源开关,示波器的准备工作跟前面相同,逐步减小衰减
挡并调整Y增益,使图象的竖直幅度适宜.然后根据输入的
信号频率选择适当的扫描范围并调整扫描微调,就可以看到
稳定的整数个完整的波形,再慢慢调整扫描微调,可以看到
波形数发生改变,调整X增益,可以看到波形的水平幅度的
改变.把同步开关从“$+$”位置扳到“$-$”位置,可以看到波形
改变半个周期.旋转信号源的低频增幅旋钮,观察波形的竖
直幅度的改变.在观察过程中,必要时应随时调整辉度调节、
聚焦调节和辅助聚焦,使图象亮度适宜,图线清晰.

利用示波器还可以测出输入的交流信号的电压的最大值
并进而算出它的有效值.测量时应注意把Y增益旋钮顺时针
旋转到底,还应根据衰减乘以相应的倍数.你在实验中,当低
频增幅旋钮转到最大时,信号电压的最大值和有效值各是多
少?

改变输入信号的频率,再次进行各项调整和观察.

如果有多余的时间,还可以继续观察高频正弦信号和高
频调幅信号,各项调整方法跟观察低频正弦信号时基本相
同.

\section{用示波器观察交流电的整流和滤波}

这个实验通过示波器观察波形来了解整流滤波电路的作
用.

图10.4是一个带$\pi$型滤波器的半波整流路.交流电源
的电压,滤波电容器$C_1$、$C_2$的电容,电阻$R$的阻值,都是根据
负载$R_{\text{负}}$的要求选定的.$C_1$、$C_2$越大,滤波效果越好.$R$大些,
滤波效果好些,但电压损失也大些,图中的数据可作参考.
这个电路可提供5—6伏特的直流电.
\begin{figure}[htp]\centering
    \includegraphics[scale=.6]{fig/10-4.png}
    \caption{}
    \end{figure}

在一张适当大小的白纸上,画出图中所示的电路,并标出
各个元件的规格.把这张电路图平铺在一块铁板上,选取你
需要的实验元件(图10.5所示的是专为实验用的几种元件).
将各个元件放在电路图标出的位置上(应注意二极管和电解
电容器的正负极不要接错),用导线将各元件按照电路图连
接起来,为了观察整流前后的波形,$P$点先不要接通.

\begin{figure}[htp]\centering
    \includegraphics[scale=.6]{fig/10-5.png}

    元件装在透明塑料盒内,盖上画着元件的符号,并有弹簧式的接
线头,盒底有块磁铁,使元件能平稳地安放在实验用的铁板上.
    \caption{实验用的元件}
    \end{figure}

电路检查无误后,即可接上6伏的交流电源,用示波器观
察波形.先把负载$R_{\text{负}}$改接在$A$点和“地”(即$BB'$导线)之间,
把$A$点接示波器的Y输入,$BB'$导线接示波器的“地”,观察整
流前的交流电压波形,再把负载$R_{\text{负}}$改接在$A$点和“地”之
间,把$A'$点接示波器的Y输入,$BB'$导线仍然接示波器的
“地”,观察整流后的电压波形,然后接通$P$点,把负载$R_{\text{负}}$接
在图10.4所示的位置,把$A$点接示波器的Y输入,$BB'$导线
仍然接示波器的“地”,观察滤波后的电压波形,在观察波形
时,在坐标纸上把三种波形记录下来,以便比较.

有兴趣的同学还可以改变电容器$C_1$、$C_2$和电阻$R$的数
值,观察$A''$点对“地”的电压波形的变化,研究它们对滤波效
果的影响.

\section{研究变压器的作用}
这个实验利用变压器模型来研究变压器的作用.

这个变压器模型有三个线圈,线圈I为120匝,线圈II为
240匝,线圈III为60匝.

现在拿线圈I作原线圈,线圈III作
副线圈,把它们套在一起,将硅钢片插入原、副线圈中,这样就
装成了一个降压变压器.

记下原线圈的匝数$n_1$和副线圈的匝数$n_2$.将原线圈的
两端接低压交流电源,副线圈的两端跟小电灯泡相连,用交
流电压表分别测量原、副线圈两端的电压$U_1$和$U_2$,记下测得
的数据,看看它们的电压比
$U_1/U_2$跟匝数比$n_1/n_2$
有什么关系.

用交流电流表分别测出原、副线圈中的电流$I_1$和$I_2$,记
下测得的数据,看看原、副线圈中的电流比$I_1/I_2$
跟它们的匝数比$n_1/n_2$有什么关系.

根据测得的原、副线圈的电压和电流的数值,算出变压器
的输入功率$U_1I_1$和输出功率$U_2I_2$,你这个变压器的效率是
多少?

在输出电路中再并联几个小灯泡.负载增加后,副线圈
中的电流增大了,原线圈中的电流怎样变化?输出功率和输
入功率怎样变化?

拆开变压器,换上线圈II作副线圈,然后再将硅钢片插
入,装成一个升压变压器,按上面的步骤重做一遍实验.

根据你的实验,回答下面的问题:
\begin{enumerate}
    \item 变压器原、副线圈两端的电压跟它们的匝数有什么关系?
    \item 变压器原、副线圈中的电流跟它们的匝数有什么关系?
    \item 变压器的输出电流改变时,输入电将怎样改变?输
出功率改变时,输入功率将怎样改变?输出功率和输入功率
是否相等?为什么?
\end{enumerate}

\section{安装简单的收音机}
我们用方框图已经讲过简单晶体管收音机的工作原理.
现在来连接这种收音机的电路,学习它的安装和调试方法.

图10.6是一个两管收音机的线路图,它包括调谐、高频
放大、检波、低频放大四个部分,图中已经用虚线隔开,图中
的$R_1^*$和$R_2^*$叫做偏流电阻.
\begin{figure}[htp]\centering
    \includegraphics[scale=.6]{fig/10-6.png}
    \caption{简单收音机的线路图}
    \end{figure}

按照图10.6连接电路,连接的方法跟实验四相同.先在
白纸上画出电路图,然后根据规格选取需要的元件,放在纸上
摆好,再用导线把它们连接起来.偏流电阻$R_1^*$和$R_2^*$的阻值
待定,可以先空着,等调试后再连入.

连接完毕,要按照电路图全面仔细检查一遍,确认无误
后,即可接上电源进行调试.

对本机的调试,就是调整两个三极管的偏流电阻,把毫
安表或万用电表的毫安档接入三极管的集电极电路(图中画
“$\times$”号处),用20千欧的固定电阻和470千欧的电位器串联
起来暂代偏流电阻连入电路,调节电位器,使高频三极管集电
极电流为1毫安左右,低频三极管集电极电流为2毫安左右.
然后取下串联的固定电阻和电位器,用万用电表分别测出它
们的总电阻,用阻值相同的固定电阻连入电路,调试工作就完
成了.

调试好以后,即可调节可变电容器来试听电台的广播,本
机的灵敏度不高,在广播电台较远时可能收听不到.遇到这
种情况,可以用信号源发出高频调幅信号来代替广播电台,这
时从耳机里可以听到嗡嗡的调制信号的音频交流声.

\section{测定玻璃的折射率}

在这个实验里,我们用两面平行的玻璃砖来测定玻璃的
折射率,从例题可知,当光线斜射入两面平行的玻璃板
时,从玻璃板射出的光线传播方向不变,出射光线跟入射光线
相比,只有一定的侧移.只要我们找出跟某一入射光线对应的
出射光线,就能求出在玻璃中对应的折射光线,从而求出折射
角,再根据折射定律,就可以求出玻璃的折射率.
\begin{figure}[htp]\centering
    \includegraphics[scale=.6]{fig/10-7.png}
    \caption{}
    \end{figure}

实验的具体做法如下:照图10.7那样,先在一张白纸上
画直线$aa'$作为玻璃砖的一个界面,过$aa'$上的一点$O$画界面
的法线$NN'$,再画直线$AO$作为入射光线,把长方形玻璃砖
放在白纸上,使它的长边跟$aa'$对齐,画出玻璃砖的另外一边
$bb'$.在直线$AO$上竖直插上两枚大头针$P_1$、$P_2$,透过玻璃砖观
察大头针$P_1$、$P_2$的像,移动视线的方向,直到$P_1$的像被$P_2$的
像挡住.再在观察的这一侧插两枚大头针$P_3$、$P_4$,使$P_3$挡住
$P_1$、$P_2$的像,$P_4$挡住$P_3$和$P_1$、$P_2$的像.


记下$P_3$、$P_4$的位置.移去玻璃砖和大头针,过$P_3$、$P_4$引直
线$EB$,与$bb'$交于$E$,$EB$就表示沿$AO$方向入射的光线通过
玻璃砖后传播的方向.连接$OE$,$OE$就是玻璃砖内折射光线
的方向,入射角$i=\angle AON$,折射角$r=\angle EON'$.

用量角器量出入射角和折射角,从三角函数表中查出它
们的正弦值,把这些数据记在自己设计的表格里.

用上面的方法分别求出入射角是30$^\circ$、45$^\circ$、60$^\circ$时的折射
角,查出它们的正弦值,把得到的数据记在表格里.

算出不同入射角时
$\sin i/\sin r$
的比值,比较一下,看它们是否接
近于一个常数.求出几次测得的
$\sin i/\sin r$
的平均值,作为测得的
玻璃的折射率$n$.

\section{测定凸透镜的焦距}
测定凸透镜的焦距有各种不同的方法,在这个实验里我
们用三种比较简单的方法来测定.

\subsection{平行光聚焦法}

平行于凸透镜主轴的光线,经凸透
镜折射后将会聚于焦点,利用凸透镜的这一特性,可以测出它
的焦距.方法是把凸透镜对着远处的光源(例如太阳),在透
镜的另一侧放一个光屏(或一张白纸),调节透镜和白纸间的
距离,直到屏上出现的光斑最亮最小为止,这个光斑就是透
镜的焦点,用直尺量出这时透镜到光斑间的距离,就得到凸
透镜的焦距.

\subsection{利用透镜成像公式}

把点燃的蜡烛、凸透镜、光屏照
图5.38那样放在光具座上,调整它们的高度,使烛焰和光屏的
中心位于凸透镜的主轴上(共轴).调节蜡烛和光屏到透镜的
距离,使光屏上呈现出烛焰的清晰的像,量出这时的物距和像
距,填入自己设计的表格中.

改变蜡烛到凸透镜的距离,按照上段的要求再做两次,把
测得的数据也填入表格中.

根据测得的三组数据,利用凸透镜的成像公式
\[\frac{1}{u}+\frac{1}{v}=\frac{1}{f} \]
算出三次得到的$f$值,求出它们的平均值,作为测得的凸透
镜的焦距.

\subsection{利用公式$f=\dfrac{L^2-d^2}{4L}$}

从例题知道,如果保
持物体和光屏之间的距离$L$不变,在物体和光屏之间移动凸透
镜,使物体在光屏上先后两次成像,测出凸透镜的两个位置间
的距离$d$,那么凸透镜的焦距
\[f=\dfrac{L^2-d^2}{4L} \]
需要注意的是,用这
种方法测焦距时,一定要使$L>4f$,才能在光屏上得到物体的
像.(有兴趣的同学可以自己考虑并证明一下这个问题.提示:
利用关系$L=u+v$,
从
$\dfrac{1}{u}+\dfrac{1}{v}=\dfrac{1}{f} $
中解出$v$,可得只有在
$L\ge 4f$时,$v$才有实数解)

实验时,可先用平行光聚焦法粗测出凸透镜的焦距$f$,然
后再利用三中的装置来做,使蜡烛到光屏的距离$L>4f$,把
凸透镜从蜡烛附近逐渐向光屏移动,同时注意观察光屏上烛
焰的像.当第一次出现清晰的像时,在光具座上记下凸透镜的
位置1.继续向光屏移动凸透镜,当光屏上第二次出现烛焰
的清晰的像时,在光具座上记下凸透镜的位置2.量出1、2
两个位置间的距离$d$,把测得的数据记录在自己设的表格
中.

改变$L$,再重做两次,记录下测得的数据.用公式$f=\dfrac{L^2-d^2}{4L}$
求出凸透镜的焦距.算出三次求得的$f$的平均值,作
为凸透镜的焦距.

\section{组成显微镜模型}
在这个实验里,我们用两个焦距为$f_1$和$f_2$的凸透镜在光
具座上组成显微镜模型.

实验装置如图10.8所示.把一块玻璃竖立在光具座的
一端,玻璃上粘一个小物体(例如一个小昆虫等).用焦距$f_1$
较短的凸透镜$L_1$做物镜,焦距$f_2$较长的凸透镜$L_2$做目镜,把
它们安装在光具座上,在$L_1$和$L_2$之间放一个光屏,调整$L_1$、
$L_2$的高度,使它们共轴.同时调整玻璃和光屏的高度,使玻璃
上的小物体和光屏的中心在$L_1$和$L_2$的主轴上.
\begin{figure}[htp]\centering
    \includegraphics[scale=.6]{fig/10-8.png}
    \caption{}
    \end{figure}

用手电筒(或其他光源)照亮玻璃上的小物体,移动物镜
$L_1$,使它到物体的距离$2f_1>u_1>f_1$,同时移动光屏,直到在光
屏上出现小物体清晰的倒立放大的实像.移动目镜$L_2$,同时
通过$L_2$观察光屏上的像,待看到清晰的放大虚像时,移去光
屏,再通过$L_2$观察,这时看到的放大虚像跟有光屏时是一样
的.这表明这个虚像确实是以$L_1$所成的实像为物而产生的.

\section{利用双缝干涉测定光波的波长}

这个实验是利用双缝干涉条纹来测定单色光的波长.实
验装置如图10.9所示,光源发出的光经滤光片成为单色光.
单色光通过单缝后,经双缝产生干涉.干涉条纹可从屏上观
察到.
\begin{figure}[htp]\centering
    \includegraphics[scale=.6]{fig/10-9.png}
    \caption{}
    \end{figure}

双缝间的距离$d$是已知的,双缝到屏的距离$\ell$和相邻两
条干涉条纹间的距离$\Delta x$能够测量,因此可以利用下面的公
式求出波长
\[\lambda=\frac{d\Delta x}{\ell} \]

现在来做实验.把直径约10厘米、长约1米的遮光筒水
平放在光具座上,筒的一端装有双缝,另一端装有毛玻璃屏.
先取下双缝,打开光源,调节它的高度,使它发出的一束光能
够沿着遮光筒的轴线把屏照亮.然后放好单缝和双缝,单缝
和双缝间的距离为5—10厘米,使缝互相平行,中心大致位于
遮光筒的轴线上.这时,在屏上就会看到双缝的干涉条纹,你
看到的干涉条纹是不是彩色的?

放上滤光片,注意观察亮条纹间的距离是否相等,测出$n$
条干涉条纹间的距离$a$,那么相邻两条干涉条纹间的距离
\[\Delta x=\frac{a}{n-1} \]
再用米尺测出双缝到屏的距离$\ell$.换用另外颜
色的滤光片,观察干涉条纹间的距离有什么变化.

根据已知的双缝间的距离$d$和实验所得的数据,代入公
式求出单色光的波长.

你求出的单色光的波长是多少?利用书中所附的连续光
谱图查出这种波长的光的颜色,跟滤光片的颜色相比较,看着
它们是否一致?

\section{观察光的衍射现象}

\subsection{观察单缝衍射}

用一个具有直长灯丝的白炽灯泡作
线状光源(用距离较远的日光灯管也可以).调节游标卡尺
两脚间的距离,形成一个0.5毫米宽的狭缝.在距灯丝几米远
处,使卡尺的狭缝与直灯丝平行,眼睛紧靠狭缝,通过狭缝观
察线光源,这时可以看到光通过狭缝后产生的衍射现象:光
束变宽,并有许多条彩色条纹.

调节狭缝的宽度,使它变宽,观察衍射条纹有什么变化;
使狭缝变窄,再观察衍射条纹有什么变化.

\subsection{观察小孔衍射}

用手电筒的小灯泡作点光源,在铝
箔(或胶片)上打出尺寸不同的小孔.在距点光源1—2米处,
使眼睛紧靠小孔,观察光通过小孔的衍射现象,可以看到彩色
的圆环,孔越小,圆环面积越大,



\begin{appendix}
	
\chapter{课外实验活动}
\section{自制指南针}
如图A.1所示,用硬纸板、大头针、按扣、缝衣针自制一
个指南针。

用磁铁的一端在缝衣针上
朝一个方向擦几下,缝衣针就
有了磁性。为了使缝衣针能顺
利地穿过按扣(取按扣中较薄
的一扇)的两个小孔,可用钳子
把按扣的边缘向下夹一下,当
自制的指南针静下来后,记住针的哪一端指北。

\begin{figure}[htp]\centering
    \includegraphics[scale=.6]{fig/10-10.png}
    \caption{}
    \end{figure}


\section{验证环形电流的磁场}
这个实验是用自制的指南针来验证环形电流的磁场方向
(图A.2),在一个瓶子(或硬纸筒)上用漆包线绕一个10至15
匝的线圈,把绕好的线圈从瓶子上取下来,再用胶布把线圆竖
直固定在一块六板上,将你自制的指南针放在图A.2所示
的位置,转动木板使磁针处在线圈平面内,用学过的环形电
流磁场的知识判断一下,如果线圈的两端接上电池,指南针将
怎样偏转,然后再给线圈通电,看一看实验结果跟你的判断
是否一致。
\begin{figure}[htp]\centering
    \includegraphics[scale=.6]{fig/10-11.png}
    \caption{}
    \end{figure}

    \section{验证通电螺线管的南北极}
把漆包线绕在一支铅笔上,然后抽出铅笔,做成一个螺线
管。用学过的通电螺线管磁场的知识判断一下,如果给螺线
管通电,通电螺线管哪端是南极,哪端是北极。然后把自制的
指南针放在螺线管的两端,给螺线管通电,看看实验结果跟你
的判断是否一致。

\section{观察磁化现象}
取一个条形磁铁和一个大铁钉,把铁钉插入铁屑,并把
条形磁铁的一个磁极靠近钉子头,然后同时提起磁铁和铁钉,
你将看到一些铁屑粘到钉子上。将磁铁移去,铁钉上的大部
分铁屑将掉下来,但仍有一部分铁屑粘在钉子上。再用磁铁
的另一个磁极靠近钉子头,剩在钉子上的铁屑就会掉下来。

解释上述现象。

\section{判断指南针的偏转方向}
在一个铅笔刀或一个大些的铁钉上,用漆包线绕上两个
线圈$A$和$B$,将线圈$B$的两端接在一起,并把$CD$段漆包线放
在静止的自制指南针的上方(图A.3)。试判断当用于电池
给线圈$A$通电的一瞬间,指南针偏转的方向。做这个实验,看
一看你判断的指南针偏转方向与实验是否一致。
\begin{figure}[htp]\centering
    \includegraphics[scale=.6]{fig/10-12.png}
    \caption{}
    \end{figure}

\section{自制测电笔}
准备一个小氖灯,一个小弹簧,再找一个装中药片的小
玻璃瓶,两个瓶盖,两个铁钉,一个0.25瓦、2—5兆欧的电阻。
在稍粗糙的水泥砖上把玻璃瓶底磨掉,做成一个玻璃圆筒,让
铁钉穿过瓶盖,盖上瓶盖后使钉帽在瓶里,把电阻的两根引
线齐根去掉,并把电阻两端的绝缘漆去掉。照图A.4那样
把上述器材安装起来,就做成了一个测电笔。
\begin{figure}[htp]\centering
    \includegraphics[scale=.6]{fig/10-13.png}
    \caption{自制测电笔}
    \end{figure}

用这个自制的测电笔可以辨别照明电路的火线和地线。
用拇指和食指拿住玻璃瓶,前面的钉子接触待辨别的导线,后
面的钉子接触手。当前面的钉子接触的是火线时,小氖灯发
光;接触的是地线时,小氖灯不发光。这样就可以辨别出火线
和地线。

要注意:\textit{手的任何部位都不要接触前面的钉子},因为它
接触可能是火线,会使人触电。

\section{测定水的折射率}
找一个广口瓶,在瓶内盛
满水,照图A.5那样把直尺$AB$
紧挨着瓶口的$C$点竖直插入瓶
内,从尺的对面一点$P$观察水
面,可以同时看到直尺在水中
的部分和露出水面的部分在水
中的像。读出你看到的直尺水
下部分最低点的刻度$S_1$,以及
跟这个刻度相重合的、水上部
分刻度$S_2$的像$S'_2$。记下$CS_1$和
$CS_2$的长度,量出广口瓶瓶口的内径$d$,就能算出水的折射率。
你用这种方法求出的水的折射率为多少?
 
如果你能同时读出直尺在水下的两个刻度$S_1$和$S_3$,以
及跟它们相重合的、两个水上刻度$S_2$和$S_4$在水中的像$S'_2$和
$S'_4$,就可以不必测量瓶口的内径,直接用从直尺上读出的两
组数据求出水的折射率来。比较这两种方法测量的结果,看
哪种方法测得的折射率更准确?

用后一种方法进行测量,瓶中的水不一定非盛满不可,竖
直插入水中的直尺也不一定要紧挨瓶口,做起来更简便。

\section{测定凹透镜的焦距}
凹透镜所成的虚像不能在像屏上显示出来,因此它的焦
距不可能象凸透镜那样直接利用焦点或成像方法来测量,
面介绍一种测量凹透镜焦距的简便方法。

\begin{figure}[htp]
\centering
\begin{minipage}[t]{0.48\textwidth}
\centering
    \includegraphics[scale=.6]{fig/10-14.png}
\caption{}
\end{minipage}
\begin{minipage}[t]{0.48\textwidth}
\centering
\begin{tikzpicture}[yscale=1]
    \draw[>-<, very thick ] (0,1.5)--(0,-1.5);
\draw[thick] (0,1)--(2,1);
\draw[thick] (0,-1)--(2,-1);
\draw[>=latex, -<] (0,1)--(1,1);
\draw[>=latex, -<]  (0,-1)--(1,-1);

\draw[>=latex, ->] (0,1)--(-1.5/2,1.5);
\draw[>=latex, ->]  (0,-1)--(-1.5/2,-1.5);
\draw[dashed] (1.5,0)node[right]{$F$}--(0,1);
\draw[dashed] (1.5,0)--(0,-1);
\draw[thick] (0,1)--(-1.5,2);
\draw[thick] (0,-1)--(-1.5,-2);
\draw[thick] (-1.5,-2.5)--(-1.5,2.5);
\draw [ultra thick, color=red] (-1.5,2)--node[right]{$B$}(-1.5,-2);
\draw [ultra thick, color=blue] (0,1)--node[right]{$A$}(0,-1);
\end{tikzpicture}
\caption{}
\end{minipage}
\end{figure}



在凹透镜的中心贴一个半径为$R$的黑色圆纸片$A$,另取
一张白纸$B$,在$B$上画一个半径为$2R$的圆。把白纸和凹透镜
平行地放在太阳光下(图A.6),让透镜对着太阳,调节透镜
和白纸间的距离,使黑色圆纸片的影恰好跟白纸上的圆圈重
合。这时透镜和白纸间的距离就等于凹透镜的焦距。想想看,
为什么?做这个实验,并将测得的焦距跟已知的焦距相比较,
看相差多少。















































































\chapter{常用电磁学量的国际制单位}

电磁学的单位制是一个比较复杂的问题,长期以来存在
着多种单位制,本书采用的是国际单位制。在国际单位制中,
所有的电磁学量,都是由长度、质量、时间、电流强度这四个基
本量导出的。因此,米、千克、秒、安培是电磁学里的基本单位。
下表列出了常用的电磁学量的国际制单位。

\begin{center}
    \begin{tabular}{cc|cc|c}
  \hline
\multicolumn{2}{c|}{物理量} & \multicolumn{2}{c|}{单位} & 量纲式\\
名称 & 符号 & 名称 & 国际符号 \\
  \hline
电流   &  $I$ & 安培 & A & $[I]$\\
电量    &  $Q$  &  库仑  & C   & $[TI]$   \\
电场强度    &$E$    &  伏特每米  & V/m   &  $[LMT^{-3}I^{-1}]$  \\
电势、电势差、电压& $U$ ($V$)   &  伏特  &  V  &  $[L^2MT^{-3}I^{-1}]$  \\
电容    &  $C$  &  法拉  & F   &  $[L^{-2}M^{-1}T^{4}I^{2}]$   \\
电阻    & $R$  & 欧姆  &  $\Omega$ & $[L^2MT^{-3}I^{-2}]$  \\
电阻率    & $\rho$  & 欧姆米  &  $\Omega\cdot {\rm m}$ & $[L^3MT^{-3}I^{-2}]$  \\
磁感应强度    & $B$  & 特斯拉  & T  &  $[MT^{-2}I^{-1}]$ \\
磁通量    & $\phi$  & 韦伯  & Wb  & $[L^2MT^{-2}I^{-1}]$  \\
电感    & $L$  & 亨利  & H  &  $[L^2MT^{-2}I^{-2}]$ \\
  \hline      
    \end{tabular}
\end{center}

\chapter{常用的物理恒量}
\begin{center}
    \begin{tabular}{ll}
        万有引力恒量&    $G=6.67\x10^{-11}\; {\rm N}\cdot {\rm m^2}/{\rm kg}^2$\\
        摩尔气体恒量&    $R=8.31\;  {\rm J}/({\rm mol}\cdot {\rm K})$\\
        阿伏伽德罗常数&    $N=6.02\x 10^{23}\; {\rm mol}^{-1}$\\
        静电力恒量&    $k=9.0\x10^9\;  {\rm N}\cdot {\rm m^2}/{\rm C}$\\
        法拉第恒量&    $F=9.65\x10^4\;  {\rm C}/{\rm mol}$\\
        基本电荷&    $e=1.60\x10^{-19}\;  {\rm C}$\\
        电子的质量&    $m_e=0.91\x10^{-30}\; {\rm kg}$\\
        质子的质量&    $m_p=1.67\x10^{-27}\;   {\rm kg}$\\
        中子的质量&    $m_n=1.67\x10^{-27} \;   {\rm kg}$\\
        $\alpha$粒子的质量&    $m_{\alpha}=6.64\x10^{-27} \;  {\rm kg}$\\
        原子质量单位&    $1{\rm u}=1.66\x10^{-27} \;   {\rm kg}$\\
        真空中光速&    $c=3.00\x10^8\; \ms$\\
        电子的荷质比&    $e/m=1.76\x10^{11}\; {\rm C}/{\rm kg}$ \\
        氢原子的半径&    $a_0=0.53\x10^{-10}\; {\rm m}$   \\
        普朗克恒量&    $h=6.63\x10^{-34}\;  {\rm J}\cdot {\rm s}$   \\
        里德伯恒量&    $R=1.097\x10^7\;  {\rm m}^{-1}$\\        
    \end{tabular}
\end{center}

\chapter{用于构成十进倍数和分数单位的词头}

\begin{center}
    \begin{tabular}{cccc}
        \hline
所表示的因数&词头名称(中文)&词头名称(英文)&词头符号\\
\hline
$10^{18}$ &艾[可萨]&exa-&E\\
$10^{15}$ &拍[它]&peta-&P\\
$10^{12}$ &太[拉]&tera-&T\\
$10^9$ &吉[咖] &giga-&G\\
$10^6$ &兆 &mega-&M\\
$10^3$ & 千 &kilo-& k\\
$10^2$ & 百 &hecto-&h\\
$10^1$ & 十 &deca-&da\\
$10^{-1}$ & 分 &deci-&d\\
$10^{-2}$ &厘 &centi-&c\\
$10^{-3}$ &毫 &milli-&m\\
$10^{-6}$&微&micro-&$\mu$\\
$10^{-9}$&纳[诺]  &nano-&n\\
$10^{-12}$ &皮[可]&pico-&p\\
$10^{-15}$&飞[母托]&femto-&f\\
$10^{-18}$&阿[托]&atto-&a\\
\hline
    \end{tabular}
\end{center}
















































\end{appendix}






\end{document}